\documentclass[aps,prl,twocolumn,superscriptaddress,showpacs]{revtex4-2}

\usepackage{graphicx}
\usepackage{amsmath}
\usepackage{amssymb}
\usepackage{hyperref}
\usepackage{bm}

\newcommand{\Tr}{\mathrm{Tr}}
\renewcommand{\vec}[1]{\boldsymbol{#1}}

\begin{document}

\title{Hamiltonian of Mean Force Beyond the Commuting Gaussian Benchmark}
\author{Author Name}
\affiliation{Affiliation}
\date{\today}

\begin{abstract}
We study the operator representability problem for the Hamiltonian of mean force
(HMF) in Gaussian linearly coupled environments beyond the commuting limit.
Building on the quenched-density construction, we derive an exact
imaginary-time reformulation in which bath statistics enter as scalar kernel
moments while operator growth is generated by the adjoint chain
$\{\mathrm{ad}_{H_Q}^{n}(f)\}_{n\ge 0}$. This yields a structural criterion:
a finite closed-form HMF exists only when the adjoint-generated operator
algebra closes inside the target ansatz. We then instantiate the criterion in a
Pauli-closed qubit model and obtain an explicit closed-form mean-force
Hamiltonian with temperature- and bath-dependence carried by mode-resolved
kernel coefficients. The manuscript is framed as the noncommuting Gaussian
follow-up to the commuting benchmark analysis of arXiv:2602.13146 and sets up
the non-Gaussian extension program.
\end{abstract}

\maketitle

\input{sections/01_introduction_v5}
% \input{sections/02_model_v2} % Dropped as redundant cruft
\section{Quenched representation and influence functional\label{sec:quenched}\label{sec:model}}
We begin by recapitulating the quenched representation introduced in Ref.~\cite{mccaulMeanForceHamiltoniansInfluence2026}.
To proceed directly from Sec.~\ref{sec:intro}, we explicit the composite model.
We denote the bare system Hamiltonian by $H_Q$ and write
\begin{equation}
    H_{\mathrm{tot}} = H_Q + H_X + H_{\mathrm{int}},
    \label{eq:Htot_split_sec3}
\end{equation}
where $H_X$ is the bath Hamiltonian. We assume a factorizable interaction
\begin{equation}
    H_{\mathrm{int}} = f \otimes B,
    \label{eq:Hint_general_f_B}
\end{equation}
where $f$ acts on the system and $B$ is a bath operator. We can define the reduced equilibrium operator (up to normalisation) by $\bar{\rho}_Q(\beta)\equiv \Tr_X e^{-\beta H_{\mathrm{tot}}}$. As shown in Ref.~\cite{mccaulMeanForceHamiltoniansInfluence2026}, this can be represented as a \emph{quenched density}. This is an average over a stochastic propagator, given by:
\begin{align}
    \bar{\rho}_Q(\beta) &= \mathbb E_\xi\big[U_\xi(\beta)\big],
    \label{eq:quenched_identity_main}\\
    U_\xi(\beta) &\equiv \mathcal T_\tau \exp\!\left[-\int_0^\beta d\tau\,\big(H_Q+\xi(\tau)f\big)\right],
    \label{eq:quenched_propagator_sec3}
\end{align}
where $\xi(\tau)$ is a stochastic process whose statistics encode the bath correlations, and $ \mathcal T_\tau$ denotes time-ordering in imaginary time. Regardless of the precise form of bath and coupling, the reduced density must always be describable in this form. 

This result can be understood intuitively by first observing that since the environment influence can only enter through the system coupling $f$, its influence can be captured by attaching a $\tau$-dependent driving field to $f$~\cite{huQuantumBrownianMotion1992,mccaulHowWinFriends2021b}. If this were a single deterministic field however, it would correspond to the bath exerting the \emph{same} back-action history for every microscopic bath configuration. But $\Tr_X$ averages over many bath microstates in the thermal ensemble, and hence over many back-action histories. In this sense the partial trace is necessarily an average over histories in imaginary time, and the quenched representation simply makes this averaging explicit. For each realisation $\xi(\tau)$ the system evolves under an imaginary-time Hamiltonian $H_Q+\xi(\tau)f$; the bath is then recovered by averaging over $\xi(\tau)$ with a law chosen to reproduce the bath-induced correlations. In this sense $\xi(\tau)$ is not a physical external control field but an efficient parametrisation of the bath history $B(\tau)$ as seen through the coupling channel. 

We may understand what formal properties are demanded of $\xi$ by considering the influence functional it is required to match. Working in imaginary time, introduce the bath interaction picture: 
\begin{equation}
    B(\tau) \equiv e^{\tau H_X} B e^{-\tau H_X},\qquad \tau\in[0,\beta],
\end{equation}
and define the bath thermal state $\rho_X \equiv e^{-\beta H_X}/Z_X$.
For an arbitrary c-number source $j(\tau)$ coupled linearly to $B(\tau)$, the bath generates a
(time-ordered) functional
\begin{equation}
    \mathcal Z_X[j]
    \equiv \Tr_X\!\left[\mathcal T_\tau \exp\!\left(-\int_0^\beta d\tau\, j(\tau)\,B(\tau)\right)\rho_X\right].
    \label{eq:bath_generating_functional}
\end{equation}
Here $j(\tau)$ is introduced as an external c-number source used to generate ordered bath correlators by functional differentiation.  More generally, $j(\tau)$ may be any object commuting with the bath algebra (e.g. a system operator tensored with $\mathbb I_X$). In the influence-functional derivation it is ultimately supplied by the system history, which becomes a c-number function in the path-integral representation. In the influence-functional approach, tracing out the bath produces precisely such a functional, evaluated on the system history through the coupling channel. 

From this, the bath contribution to the effective Euclidean action can be written as \cite{feynmanTheoryGeneralQuantum1963a,grabertQuantumBrownianMotion1988,caldeiraQuantumTunnellingDissipative1983a}
\begin{equation}
    \mathcal F[f] \equiv \mathcal Z_X[f],
    \qquad
    \Phi[f] \equiv \log \mathcal F[f].
    \label{eq:influence_functional_definition}
\end{equation}
A key structural fact is that $\Phi[f]=\log\mathcal F[f]$ is the \emph{cumulant generating functional} of the bath operator $B(\tau)$ with respect to the thermal state. Concretely, the generalised
(time-ordered) cumulant theorem implies the connected expansion
\cite{Kubo1962,BreuerMaPetruccione2002}
\begin{equation}
\begin{split}
\Phi[f]
&=
\sum_{n\ge 1}\frac{(-1)^n}{n!}\!\!\int_0^\beta\!\! d\tau_1\cdots d\tau_n\, \\
&\quad\times K^{(n)}(\tau_1,\ldots,\tau_n)\,
f(\tau_1)\cdots f(\tau_n),
\end{split}
\label{eq:influence_cumulant_expansion}
\end{equation}
where the kernels $K^{(n)}$ are the \emph{connected} (cumulant) bath correlators, defined via \cite{DasThermalFieldTheory}
\begin{equation}
    K^{(n)}(\tau_1,\ldots,\tau_n)
    \equiv
    \big\langle \mathcal T_\tau B(\tau_1)\cdots B(\tau_n)\big\rangle_c.
    \label{eq:Kn_as_connected_bath_correlators}
\end{equation}
The explicit connection back to Eq.~\eqref{eq:bath_generating_functional} is given via its functional differentiation:
\begin{equation}
    \big\langle \mathcal T_\tau B(\tau_1)\cdots B(\tau_n)\big\rangle_c = (-1)^n \left. \frac{\delta^n \log \mathcal Z_X[j]}{\delta j(\tau_1)\cdots\delta j(\tau_n)} \right|_{j=0}.
    \label{eq:connected_correlator_oneliner}
\end{equation}
The bath influence is then completely characterised by the hierarchy $\{K^{(n)}\}_{n\ge1}$. 

To connect the influence functional back to the quenched density, we use the fact that the bath
influence depends on the system history only through the linear functional $\int_0^\beta d\tau\, f(\tau)B(\tau)$. One may therefore represent $\mathcal F[f]$ as the (generalised) characteristic functional of an
auxiliary field $\xi(\tau)$ \cite{hubbardCalculationPartitionFunctions1959a,stratonovich1957QDistro,stockburgerExactNumberRepresentation2002}:
\begin{equation}
\mathcal F[f]
=
\mathbb E_\xi\!\left[\exp\!\left(-\int_0^\beta d\tau\,\xi(\tau)\,f(\tau)\right)\right],
\label{eq:noise_characteristic_functional}
\end{equation}
where $\mathbb E_\xi[\cdot]$ denotes averaging with respect to a (possibly complex) measure on
$\xi$-histories chosen such that \eqref{eq:noise_characteristic_functional} holds.

Since $\xi(\tau)$ is a commuting $c$-number field, its $n$-point moments are symmetric under
permutations of the time arguments. The influence kernels $K^{(n)}$ fix the \emph{cumulants} of
$\xi$ via
\begin{equation}
\big\langle \xi(\tau_1)\cdots \xi(\tau_n)\big\rangle_c
=
K^{(n)}(\tau_1,\ldots,\tau_n).
\label{eq:xi_cumulants_equal_K}
\end{equation}
Consequently, the ordinary correlation functions of the noise are obtained from
$\{K^{(m)}\}$ by the standard moment-cumulant relations:
\begin{equation}
\big\langle \xi(\tau_1)\cdots \xi(\tau_n)\big\rangle
=
\sum_{\pi\in\mathcal P_n}\;
\prod_{C\in\pi}
K^{(|C|)}\!\big(\{\tau_i\}_{i\in C}\big),
\label{eq:moment_cumulant_partition}
\end{equation}
where $\mathcal P_n$ denotes the set of all partitions of the index set $\{1,\ldots,n\}$, and $C$ are the disjoint blocks of a given partition $\pi \in \mathcal P_n$. After shifting the mean so that $K^{(1)}(\tau)=\langle\xi(\tau)\rangle=0$, one has (for example)
\begin{align}
\big\langle \xi(\tau_1)\xi(\tau_2)\big\rangle
&= K^{(2)}(\tau_1,\tau_2),\\
\big\langle \xi(\tau_1)\xi(\tau_2)\xi(\tau_3)\big\rangle
&= K^{(3)}(\tau_1,\tau_2,\tau_3),\\
\begin{split}
\big\langle \xi(\tau_1)\xi(\tau_2)&\xi(\tau_3)\xi(\tau_4)\big\rangle \\
&= K^{(4)}(\tau_1,\tau_2,\tau_3,\tau_4) \\
& \quad + K^{(2)}(\tau_1,\tau_2)K^{(2)}(\tau_3,\tau_4) \\
& \quad + K^{(2)}(\tau_1,\tau_3)K^{(2)}(\tau_2,\tau_4) \\
& \quad + K^{(2)}(\tau_1,\tau_4)K^{(2)}(\tau_2,\tau_3).
\end{split}
\end{align}

\section{Quenched density and the Hamiltonian of mean force \label{sec:hmf}}
The Hamiltonian of mean force is, by definition, the operator whose Gibbs form reproduces the
system's reduced equilibrium state. Equivalently, it is the \emph{operator logarithm} of the
unnormalised reduced equilibrium operator
\begin{equation}
\bar{\rho}_Q(\beta)\equiv \Tr_X e^{-\beta H_{\mathrm{tot}}},
\qquad
H_{\mathrm{MF}}(\beta)\;\equiv\;-\frac{1}{\beta}\log \bar{\rho}_Q(\beta),
\label{eq:rhobar_and_HMF_tilde}
\end{equation}
defined up to an additive multiple of the identity (fixed only when normalising the state).

The quenched representation in Eq.~\eqref{eq:quenched_identity_main} supplies an exact stochastic
parametrisation of the unnormalised mean-force Gibbs operator $\bar{\rho}_Q(\beta)$ by making the
bath trace an explicit average over imaginary-time back-action histories. Combining
Eq.~\eqref{eq:quenched_identity_main} with Eq.~\eqref{eq:rhobar_and_HMF_tilde} yields
\begin{equation}
H_{\mathrm{MF}}(\beta)
=
-\frac{1}{\beta}\log \mathbb E_\xi\!\left[U_\xi(\beta)\right].
\end{equation}
Thus, constructing the mean-force Hamiltonian reduces to evaluating a stochastic average and then
compressing the result via an operator logarithm. The conditions for being able to perform this average exactly is the focus of the present work. 

 To make the handling of this problem more concrete, we shall specialise the environment to the Caldeira-Leggett model, where the bath is a collection of harmonic oscillators ($H_X=\sum_k \omega_k b_k^\dagger b_k$) and the coupling is linear in bath coordinates ($B=\sum_k c_k x_k$). None of the results that follow are essentially dependent on this choice, and a generalisation to anharmonic environments is (relatively) straightforward.  In the interests of comprehensibility however, we restrict our scope to quadratic environments. In this setting, the auxiliary field $\xi(\tau)$ becomes a stationary zero-mean Gaussian process completely characterized by its covariance
\begin{equation}
    \mathbb E_\xi[\xi(\tau)\xi(\tau')] = K(\tau-\tau').
\end{equation}
The kernel $K(\tau)$ is determined by the bath spectral density $J(\omega)=\frac{\pi}{2}\sum_k \frac{c_k^2}{m_k\omega_k}\delta(\omega-\omega_k)$ via the relation~\cite{weissQuantumDissipativeSystems2012}
\begin{equation}
    K(\tau)=\frac{1}{\pi}\int_0^\infty d\omega\,J(\omega)\,
    \frac{\cosh\!\big(\omega(\beta/2-|\tau|)\big)}{\sinh(\beta\omega/2)}.
    \label{eq:K_explicit_cosh}
\end{equation}
A key property of this kernel is its integrated strength. Integrating Eq.~\eqref{eq:K_explicit_cosh} yields
\begin{align}
    \int_0^\beta d\tau\,K(\tau) 
    &= \frac{1}{\pi}\int_0^\infty \!\!d\omega\,J(\omega) \int_0^\beta d\tau\, \frac{\cosh\!\big(\omega(\beta/2-|\tau|)\big)}{\sinh(\beta\omega/2)} \nonumber\\
    &= \frac{1}{\pi}\int_0^\infty \!\!d\omega\,J(\omega)\,\frac{2}{\omega} \nonumber\\
    &= 2\lambda,
\end{align}
where $\lambda$ is the explicit reorganisation energy. Consequently, the total variance of the integrated noise field $\Xi = \int_0^\beta d\tau\,\xi(\tau)$ grows linearly with inverse temperature:
\begin{equation}
    \mathbb E_\xi[\Xi^2] = \int_0^\beta \!\!d\tau\!\int_0^\beta \!\!d\tau'\,K(\tau-\tau') = 2\beta\lambda.
    \label{eq:integrated_variance_linear}
\end{equation}

In the case that the system and its coupling commute, $[H_Q,f]= 0$ and $f$ is $\tau$-independent in imaginary time. In this instance time ordering drops out, and the average is given by \cite{mccaulMeanForceHamiltoniansInfluence2026}:
\begin{equation}
    \bar{\rho}_Q(\beta)=
    \exp\!\left[-\beta\left(H_Q-\frac{\kappa_0(\beta)}{2}f^2\right)\right],
    \label{eq:rho_commuting_closed_sec3}
\end{equation}
which in turn yields
\begin{equation}
    H_{\mathrm{MF}}(\beta)=
    H_Q-\frac{\kappa_0(\beta)}{2}f^2+\frac{1}{\beta}\log Z_X(\beta)\,\mathbb I.
    \label{eq:Heff_commuting_limit}
\end{equation}

Notably this correction to  $H_{\mathrm{MF}}(\beta)$ is entirely \emph{classical} \cite{mccaulMeanForceHamiltoniansInfluence2026}. This is hardly surprising, but emphasises that truly quantum effects \emph{always} stem from non-commutativity. Truly quantum thermodynamic effects are therefore only present when $[H_Q,f]\neq 0$. In this case however, the noise enters through a noncommuting operator inside $\mathcal T_\tau$, rendering the question of averaging highly non-trivial. In the next section, we attack this problem directly, deriving conditions under which $H_{\mathrm{MF}}(\beta)$ possesses a closed form. 


\section{Closure of the Hamiltonian of mean force \label{sec:closure}}
The obstruction to writing $H_{\mathrm{MF}}(\beta)$ in a compact operator form is not the
existence of the mean-force object (it is defined by a logarithm), but the \emph{representability} of that logarithm inside a restricted operator family (few-body, local, Pauli strings, etc.). In the harmonic case this representability question reduces to a precise closure problem. To show this, we first write the quenched propagator in the imaginary-time interaction picture with respect to $H_Q$:
\begin{equation}
    \begin{split}
        U_\xi(\beta) & = e^{-\beta H_Q} W_\xi(\beta),                                                             \\
        W_\xi(\beta) & \equiv \mathcal T_\tau \exp\!\left[-\int_0^\beta d\tau\, \xi(\tau)\,\tilde f(\tau)\right],
    \end{split}
    \label{eq:Wxi_def}
\end{equation}
where $\tilde f(\tau)\equiv e^{\tau H_Q} f e^{-\tau H_Q}$.

Because the noise is classical Gaussian with zero mean, the average of the
time-ordered exponential resums exactly in terms of the second cumulant,
\begin{equation}
    \begin{split}
        \bar W(\beta) & \equiv \langle W_\xi(\beta)\rangle_\xi                                                      \\
                      & =\mathcal T_\tau \exp\bigg[\frac12\int_0^\beta d\tau\int_0^\beta d\tau'                     \\
                      & \quad \times K(\tau-\tau')\,\mathcal T_\tau\!\big(\tilde f(\tau)\tilde f(\tau')\big)\bigg].
    \end{split}
    \label{eq:Wbar_cumulant_T}
\end{equation}
For two operators,
\(
\mathcal T_\tau\!\big(\tilde f(\tau)\tilde f(\tau')\big)
=\theta(\tau-\tau')\tilde f(\tau)\tilde f(\tau')
+\theta(\tau'-\tau)\tilde f(\tau')\tilde f(\tau)
\),
so using the kernel symmetry $K(\tau-\tau')=K(\tau'-\tau)$ on $[0,\beta]$ the square
domain reduces to a single ordered triangle with an \emph{anticommutator}:
\begin{equation}
    \frac12\!\int_0^\beta\! d\tau\!\int_0^\beta\! d\tau'\,
    K(\tau-\tau')\,\mathcal T_\tau\!\big(\tilde f(\tau)\tilde f(\tau')\big)
    =\int_0^\beta\! d\tau\!\int_0^\tau\! d\tau'\,
    K(\tau-\tau')\,\frac12\{\tilde f(\tau),\tilde f(\tau')\}.
    \label{eq:Delta_anticommutator}
\end{equation}
Thus the influence exponent may be taken as the manifestly Hermitian operator
\begin{equation}
    \Delta(\beta)\equiv \int_0^\beta d\tau\int_0^\tau d\tau'\,
    K(\tau-\tau')\,\frac12\{\tilde f(\tau),\tilde f(\tau')\},
    \qquad
    \bar W(\beta)=\mathcal T_\tau e^{\Delta(\beta)}.
    \label{eq:Delta_def_hermitian}
\end{equation}
Expanding $\tilde f(\tau)=\sum_{n\ge0}\tau^n f_n/n!$ with $f_n=\mathrm{ad}_{H_Q}^n(f)$ then gives
\begin{equation}
    \Delta(\beta)=\sum_{n,m\ge0} C_{nm}(\beta)\,\frac12\{f_n,f_m\},
    \qquad
    C_{nm}(\beta)=\int_0^\beta d\tau\int_0^\tau d\tau'\,
    K(\tau-\tau')\,\frac{\tau^n\tau'^m}{n!m!}.
    \label{eq:Delta_Cnm_Jordan}
\end{equation}
The stochastic term can now be averaged exactly. In the harmonic case we have $\langle \xi(\tau)\xi(\tau')\rangle = K(\tau-\tau')$, which gives the exact Euclidean influence-functional form
\begin{equation}
    \bar W(\beta) \equiv \langle W_\xi(\beta)\rangle_\xi =\mathcal T_\tau e^{\Delta(\beta)}
    \label{eq:Wbar_influence}
\end{equation}
where the exponent $\Delta$ is given by
\begin{equation}
    \Delta(\beta)\equiv
    \frac12\iint_0^\beta d\tau d\tau'\,
    K(\tau-\tau')\,\tilde f(\tau)\,\tilde f(\tau').
    \label{eq:Delta_def}
\end{equation}

\begin{equation}
    \bar W(\beta)=\exp\!\big(\Delta(\beta)\big),\qquad
    \bar\rho_Q(\beta)=\langle U_\xi(\beta)\rangle_\xi = e^{-\beta H_Q}\,e^{\Delta(\beta)}.
    \label{eq:rho_bar_product_form}
\end{equation}

To progress, we introduce the adjoint chain generated by $H_Q$ acting on the coupling,
\begin{equation}
    f_n \equiv \mathrm{ad}_{H_Q}^n(f),\qquad \mathrm{ad}_{H_Q}(A)\equiv[H_Q,A],
    \label{eq:adjoint_chain_def}
\end{equation}
so that the interaction-picture operator has the exact series
\begin{equation}
    \tilde f(\tau)=e^{\tau H_Q} f e^{-\tau H_Q}=\sum_{n\ge 0}\frac{\tau^n}{n!}\, f_n.
    \label{eq:ftilde_ad_series}
\end{equation}
Substituting \eqref{eq:ftilde_ad_series} into \eqref{eq:Delta_def_ordered_triangle} yields
\begin{equation}
    \Delta(\beta)
    =\sum_{n,m\ge 0} C_{nm}(\beta)\, f_n f_m,
    \label{eq:Delta_Cnm_fnfm_no_matsubara}
\end{equation}
with the \emph{kernel-moment matrix}
\begin{equation}
    C_{nm}(\beta)\equiv
    \int_0^\beta d\tau\int_0^\tau d\tau'\,
    K(\tau-\tau')\,\frac{\tau^n}{n!}\,\frac{\tau'^m}{m!}.
    \label{eq:Cnm_def_triangle}
\end{equation}
Thus all bath/temperature dependence enters $\Delta(\beta)$ only through the scalar moments
$C_{nm}(\beta)$, while all operator structure is carried by the adjoint chain $\{f_n\}$.


The final step is to re-express the averaged propagator in a \emph{single} exponential. Using
\eqref{eq:rho_bar_product_form} we write
\begin{equation}
    \bar\rho_Q(\beta)=e^{-\beta H_Q}\,e^{\Delta(\beta)} \equiv e^{-\beta H_{\mathrm{MF}}(\beta)},
    \label{eq:HMF_BCH_def_clean_repeat}
\end{equation}
where the mean-force Hamiltonian satisfies
\begin{equation}
    -\beta H_{\mathrm{MF}}(\beta)=\log\!\left(e^{-\beta H_Q}e^{\Delta(\beta)}\right).
\end{equation}
Let $A\equiv-\beta H_Q$ and $B\equiv \Delta(\beta)$. The Baker--Campbell--Hausdorff (BCH) series gives
\begin{equation}
    \label{eq:BCH_AB_general}
    \begin{split}
        \log(e^{A}e^{B}) & = A+B+\frac{1}{2}[A,B]                                      \\
                         & \quad +\frac{1}{12}[A,[A,B]]+\frac{1}{12}[B,[B,A]]+\cdots .
    \end{split}
\end{equation}
Dividing by $-\beta$ and using $A=-\beta H_Q$ yields the expansion
\begin{equation}
    \label{eq:HMF_BCH_explicit}
    \begin{split}
        H_{\mathrm{MF}}(\beta) & = H_Q-\frac{1}{\beta}\Delta(\beta)
        +\frac{1}{2}[H_Q,\Delta(\beta)]                                                 \\
                               & \quad -\frac{\beta}{12}[H_Q,[H_Q,\Delta(\beta)]]       \\
                               & \quad +\frac{1}{12}[\Delta(\beta),[\Delta(\beta),H_Q]]
        +\cdots .
    \end{split}
\end{equation}
This makes the closure content transparent: all terms are built from repeated commutators with $H_Q$
acting on $\Delta$, together with higher BCH terms involving commutators among those objects.

To see the explicit generator, introduce the adjoint superoperator $\mathrm{ad}_{H_Q}(\cdot)\equiv[H_Q,\cdot]$.
Then \eqref{eq:HMF_BCH_explicit} becomes
\begin{equation}
    \label{eq:HMF_ad_form}
    \begin{split}
        H_{\mathrm{MF}}(\beta) & = H_Q-\frac{1}{\beta}\Delta
        +\frac{1}{2}\,\mathrm{ad}_{H_Q}\Delta                                       \\
                               & \quad -\frac{\beta}{12}\,\mathrm{ad}_{H_Q}^2\Delta
        +\frac{1}{12}\,[\Delta,\mathrm{ad}_{H_Q}\Delta]
        +\cdots .
    \end{split}
\end{equation}

Finally, because $\Delta(\beta)$ is quadratic in the adjoint chain,
\begin{equation}
    \label{eq:Delta_quadratic_chain_repeat}
    \Delta(\beta)=\frac12\sum_{n,m\ge0}C_{nm}(\beta)\,f_n f_m,
    \qquad f_n\equiv\mathrm{ad}_{H_Q}^n(f),
\end{equation}
each commutator with $H_Q$ simply shifts indices:
\begin{equation}
    \label{eq:adHQ_on_products}
    \mathrm{ad}_{H_Q}(f_n f_m)=[H_Q,f_n f_m]=f_{n+1}f_m+f_n f_{m+1}.
\end{equation}
Hence the first commutator term takes the explicit form
\begin{equation}
    \label{eq:adHQ_Delta_explicit}
    [H_Q,\Delta(\beta)]
    =\frac12\sum_{n,m\ge0}C_{nm}(\beta)\,\big(f_{n+1}f_m+f_n f_{m+1}\big),
\end{equation}
and higher $\mathrm{ad}_{H_Q}^k\Delta$ generate the same family of products $f_a f_b$ with shifted indices.
The remaining BCH terms (e.g.\ $[\Delta,\mathrm{ad}_{H_Q}\Delta]$) introduce commutators among these quadratic
elements, and are the sole source of additional operator growth beyond the quadratic span.
The only remaining question is whether the operator family generated by \eqref{eq:Delta_Cnm_fnfm} can be brought to a closed form, and re-expressed in a single exponential.

\paragraph{Resummation of the BCH series linear in $\Delta$.}
The preceding identities imply that the entire BCH commutator tower \emph{linear} in $\Delta$ remains
within the quadratic span $\mathrm{span}\{f_nf_m\}$. In fact, these terms admit a closed resummation.
Let $A\equiv-\beta H_Q$ and $B\equiv \Delta(\beta)$. Then the BCH logarithm satisfies the standard
linear-in-$B$ identity
\begin{equation}
    \label{eq:BCH_linear_resum}
    \begin{split}
        \log\!\left(e^{A}e^{B}\right) & = A+\Phi\!\big(\mathrm{ad}_{A}\big)\,B+O(B^2), \\
        \Phi(x)                       & \equiv \frac{x}{1-e^{-x}}.
    \end{split}
\end{equation}
Substituting $A=-\beta H_Q$ and dividing by $-\beta$ gives the mean-force Hamiltonian
to first order in $\Delta$:
\begin{equation}
    \label{eq:HMF_linear_resum}
    \begin{split}
        H_{\mathrm{MF}}(\beta) & = H_Q                                                                                        \\
                               & \quad -\frac{1}{\beta}\,\Phi\!\big(\beta\,\mathrm{ad}_{H_Q}\big)\,\Delta(\beta)+O(\Delta^2).
    \end{split}
\end{equation}
Expanding $\Phi$ yields the explicit commutator series with Bernoulli numbers $B_k$,
\begin{equation}
    \label{eq:HMF_linear_series}
    \begin{split}
        H_{\mathrm{MF}}(\beta) & = H_Q - \frac{1}{\beta}\sum_{k\ge 0}\frac{B_k}{k!}(-\beta)^k\,\mathrm{ad}_{H_Q}^k\Delta(\beta) + O(\Delta^2), \\
        \Phi(x)                & = \sum_{k\ge 0}\frac{B_k}{k!}(-x)^k.
    \end{split}
\end{equation}
Using \eqref{eq:adHQ_on_products} repeatedly, the iterated commutators admit the closed binomial form
\begin{equation}
    \label{eq:adHQk_on_products}
    \mathrm{ad}_{H_Q}^{k}(f_n f_m)
    =\sum_{j=0}^{k}\binom{k}{j}\, f_{n+j}\, f_{m+k-j},
\end{equation}
so every term in \eqref{eq:HMF_linear_series} is explicitly a linear combination of products $f_af_b$
with scalar coefficients determined solely by $C_{nm}(\beta)$ and universal combinatorics.

Because $\Delta(\beta)$ is already a quadratic form in the adjoint-chain products,
the entire BCH contribution \emph{linear} in $\Delta$ can be written explicitly in the same basis
$\{f_af_b\}$, with no remaining reference to $\Delta$ itself. To do so, we introduce the discrete shift operator $\mathsf D$ acting on the moment matrix $C$ by
\begin{equation}
    \label{eq:D_operator_on_C}
    \begin{split}
        (\mathsf D C)_{ab} & \equiv C_{a-1,b}+C_{a,b-1},                    \\
        C_{rs}             & \equiv 0\quad \text{if any index is negative}.
    \end{split}
\end{equation}
Then \eqref{eq:adHQ_on_products} implies, after reindexing,
\begin{equation}
    \label{eq:adHQ_Delta_as_D_on_C}
    \mathrm{ad}_{H_Q}\Delta(\beta)
    =\frac12\sum_{a,b\ge0}(\mathsf D C(\beta))_{ab}\, f_a f_b,
\end{equation}
and by iteration,
\begin{equation}
    \label{eq:adHQk_Delta_as_Dk_on_C}
    \mathrm{ad}_{H_Q}^{k}\Delta(\beta)
    =\frac12\sum_{a,b\ge0}(\mathsf D^{k} C(\beta))_{ab}\, f_a f_b.
\end{equation}
Equivalently, $(\mathsf D^{k}C)_{ab}=\sum_{j=0}^{k}\binom{k}{j}\,C_{a-j,\;b-(k-j)}$, which is the binomial
index-shift structure implied by \eqref{eq:adHQk_on_products}. From this we obtain the following form for $H_{\mathrm{MF}}(\beta)$ to linear order in $\Delta$:
\begin{equation}
    \label{eq:HMF_linear_quadratic_final}
    H_{\mathrm{MF}}(\beta) = H_Q-\frac{1}{2\beta}\sum_{a,b\ge0}\,\widetilde C_{ab}(\beta)\, f_a f_b \;+\; O(\Delta^2),
\end{equation}
where the \emph{renormalised moment matrix} $\widetilde C(\beta)$ is the universal transform
\begin{equation}
    \label{eq:Ctilde_def}
    \widetilde C(\beta) \equiv \Phi\!\big(\beta\,\mathsf D\big)\,C(\beta)
    = \sum_{k\ge0}\frac{B_k}{k!}(-\beta)^k\,\mathsf D^{k}C(\beta).
\end{equation}
Thus, to linear order in the influence operator, the mean-force Hamiltonian is \emph{exactly} a quadratic
form in the products $f_af_b$, with all bath/temperature dependence entering only through the scalar matrix
$\widetilde C_{ab}(\beta)$.
Any operator growth beyond the quadratic span is confined to the nonlinear BCH sector $O(\Delta^2)$ (e.g.\
$[\Delta,\mathrm{ad}_{H_Q}\Delta]$).

With this form in hand, we can now state the \emph{closure criterion} required for $H_{\mathrm{MF}}(\beta)$ to be representable in a given operator family $\mathcal A$.
\subsection{Closure criterion}

Fix a target operator family $\mathcal A$ (e.g.\ $k$-local Pauli strings, a Lie algebra plus identity,
or a finite-dimensional associative operator algebra) in which we seek to represent $H_{\mathrm{MF}}(\beta)$.
The Gaussian construction above shows that representability is controlled by two closure conditions:

\paragraph{(C1) Adjoint closure (no operator growth under commutation).}
The subspace generated by repeated commutators of $f$ with $H_Q$ must remain inside $\mathcal A$:
\begin{equation}
    \begin{split}
        f & \in\mathcal A,\qquad \mathrm{ad}_{H_Q}(\mathcal A)\subseteq \mathcal A,     \\
          & \text{equivalently}\quad \mathrm{span}\{f_n\}_{n\ge 0}\subseteq \mathcal A.
    \end{split}
    \label{eq:C1_adjoint_closure}
\end{equation}

\paragraph{(C2) Quadratic closure (no operator growth in the Gaussian sector).}
Because the exact influence operator $\Delta(\beta)$ is quadratic in the adjoint chain,
the quadratic span generated by $\{f_n\}$ must also lie in $\mathcal A$:
\begin{equation}
    f_n f_m \in \mathcal A\quad \text{for all indices that contribute in $\Delta(\beta)$,}
    \label{eq:C2_product_closure}
\end{equation}
so that $\Delta(\beta)\in\mathcal A$. A sufficient (and common) condition is that $\mathcal A$ is an
associative algebra containing the identity and closed under multiplication.

\paragraph{(C3) BCH closure (no growth from commutators among quadratic elements).}
To ensure that the \emph{full} BCH logarithm $\log(e^{-\beta H_Q}e^{\Delta})$ remains in $\mathcal A$,
the commutators generated among the quadratic elements must also close in $\mathcal A$; in particular
\begin{equation}
    [\Delta(\beta),\,\mathrm{ad}_{H_Q}^k\Delta(\beta)]\in\mathcal A\qquad \text{for all $k\ge 0$,}
    \label{eq:C3_BCH_closure}
\end{equation}
and similarly for the higher nested commutators appearing in the BCH series.
A sufficient (and common) condition is again that $\mathcal A$ is a finite-dimensional associative algebra
(or a Lie algebra containing $\Delta$ and closed under commutators), in which case all BCH commutators remain
in $\mathcal A$ by construction.

\paragraph{Consequence (closed-form mean-force Hamiltonian).}
If (C1)--(C3) hold, then $\Delta(\beta)\in\mathcal A$ and all BCH commutators generated by
\eqref{eq:HMF_BCH_def_final} remain in $\mathcal A$. Since $H_Q\in\mathcal A$ by assumption, it follows that
\begin{equation}
    \label{eq:HMF_BCH_def_final}
    \begin{split}
        \bar\rho_Q(\beta)             & = e^{-\beta H_{\mathrm{MF}}(\beta)},                  \\
        -\beta H_{\mathrm{MF}}(\beta) & = \log\!\left(e^{-\beta H_Q}e^{\Delta(\beta)}\right),
    \end{split}
\end{equation}
defines an $H_{\mathrm{MF}}(\beta)\in\mathcal A$ and hence a closed-form representation of
$H_{\mathrm{MF}}(\beta)$ (up to the usual additive scalar fixed by $Z_X$).

If either (C1) fails (adjoint-chain growth) or (C2)--(C3) fail (growth in the quadratic/BCH sector),
then $\Delta(\beta)$ and/or the BCH commutator tower generates operators outside $\mathcal A$; any
representation within $\mathcal A$ is necessarily truncated or projected.

This algebraic perspective offers a unifying view of strong-coupling approximations. Numerical schemes such as the polaron transformation~\cite{weissQuantumDissipativeSystems2012}, hierarchical equations of motion~\cite{tanimuraReducedHierarchicalEquations2014}, and chain-mapping/DMRG techniques can be understood as distinct choices of the target operator family $\mathcal{A}$. The polaron frame corresponds to dressing $\mathcal{A}$ with coherent displacements; HEOM truncates the memory depth of the influence functional (dual to the operator degree in the adjoint chain), and chain mappings truncate the spatial extent of the harmonic bath. The `closure' problem is thus equivalent to identifying a low-dimensional subalgebra that approximately contains the logarithm of the quenched propagator.

When $\mathcal A$ does not close, \eqref{eq:HMF_BCH_def_final} still yields a controlled organisational
principle: the operator content is generated by the adjoint chain and its quadratic products, while all bath
dependence remains scalar through $C_{nm}(\beta)$. One may therefore truncate (in commutator depth, locality
class, or operator weight) without introducing any approximation on the bath side.

\section{Exact Solution of the Spin-Boson Model}

We now turn to applying the results developed in the previous section. A particularly instructive example concerns spins, as the $\mathfrak{su}(2)$ algebra provides a particularly compact analytic solution. We demonstrate this by applying it to a transverse-coupling spin-boson model. Following  Ref.~\cite{cresserWeakUltrastrongCoupling2021a}, let
%
\begin{equation}
    H_Q = \frac{\omega_q}{2}\sigma_z, 
    \qquad 
    f = \cos\theta\,\sigma_z - \sin\theta\,\sigma_x,
    \label{eq:qubit_setup}
\end{equation}
%
with $c \equiv \cos\theta$, $s \equiv \sin\theta$ throughout. The 
coupling mixes a commuting part $c\sigma_z$ with a transverse part 
$-s\sigma_x$.

Using $[H_Q,\sigma_x] = i\omega_q\sigma_y$ and 
$[H_Q,\sigma_y] = -i\omega_q\sigma_x$, the $\sigma_z$ component of 
$f$ is annihilated by $\mathrm{ad}_{H_Q}$ while the transverse part 
precesses. A direct induction gives
%
\begin{equation}
    f_n \equiv \mathrm{ad}_{H_Q}^n(f) = \begin{cases}
        c\,\sigma_z - s\,\sigma_x & n = 0, \\[4pt]
        -s\,\omega_q^n\,\sigma_x  & n \geq 1,\; n\;\text{even}, \\[4pt]
        -is\,\omega_q^n\,\sigma_y & n \geq 1,\; n\;\text{odd}.
    \end{cases}
    \label{eq:fn_qubit}
\end{equation}
%
The adjoint chain therefore closes on $\mathrm{span}\{\sigma_x,\sigma_y,
\sigma_z\}$, satisfying condition (C1).

We use $f_n$ to evaluate the influence exponent $\Delta(\beta)$. To streamline the algebra, we treat each $f_n$ as a Pauli vector:
\begin{equation}
    f_n = \mathbf v_n \cdot \boldsymbol{\sigma},
    \qquad 
    \boldsymbol{\sigma}=(\sigma_x,\sigma_y,\sigma_z),
    \label{eq:vn_def}
\end{equation}
with
\begin{equation}
    \mathbf v_n =
    \begin{cases}
        \mathbf v_0 = (-s,\,0,\,c), & n=0, \\
        (-s\,\omega_q^{n},\,0,\,0), & n\ge 1,\; n\ \text{even},\\[2pt]
        (0,\,-is\,\omega_q^{n},\,0), & n\ge 1,\; n\ \text{odd}.
    \end{cases}
    \label{eq:vn_qubit}
\end{equation}
The Pauli multiplication rule then becomes the single identity
\begin{equation}
    (\mathbf a\cdot\boldsymbol{\sigma})(\mathbf b\cdot\boldsymbol{\sigma})
    = (\mathbf a\cdot\mathbf b)\,\mathbb I + i(\mathbf a\times\mathbf b)\cdot\boldsymbol{\sigma},
    \label{eq:pauli_vector_product}
\end{equation}
so that
\begin{equation}
    f_n f_m
    = (\mathbf v_n\cdot\mathbf v_m)\,\mathbb I
    + i(\mathbf v_n\times\mathbf v_m)\cdot\boldsymbol{\sigma}.
    \label{eq:fnfm_vector}
\end{equation}
The scalar part contributes only to the overall normalization (free-energy shift) and may be discarded when determining the operator structure.  The non-trivial Pauli sector is therefore controlled entirely by the cross products $\mathbf v_n\times \mathbf v_m$.

Given this, we introduce the symmetric and antisymmetric parts of the ordered kernel moments:
\begin{equation}
    S_{nm}:=\frac{1}{2}\big(C^{>}_{nm}+C^{>}_{mn}\big),
    \quad 
    A_{nm}:=\frac{1}{2}\big(C^{>}_{nm}-C^{>}_{mn}\big),
    \label{eq:Snm_Anm_def}
\end{equation}
so that $C^{>}_{nm}=S_{nm}+A_{nm}$ with $S_{nm}=S_{mn}$ and $A_{nm}=-A_{mn}$.
Since $\mathbf v_n\times\mathbf v_m$ is antisymmetric under $n\leftrightarrow m$, only the antisymmetric sector of the coefficients contributes to the Pauli part of $\Delta$. We may therefore restrict the sum to ordered indices $n>m$ to avoid double counting:
\begin{equation}
\begin{split}
    \Delta(\beta) 
    &\cong i\sum_{n,m\ge 0} A_{nm}\,(\mathbf v_n\times\mathbf v_m)\cdot\boldsymbol{\sigma} \\
    &= 2i\sum_{n>m} A_{nm}\,(\mathbf v_n\times\mathbf v_m)\cdot\boldsymbol{\sigma},
\end{split}
\label{eq:Delta_Pauli_Anm}
\end{equation}
where $\cong$ indicates equality up to an irrelevant $\mathbb I$ contribution. This makes explicit that the non-commuting structure in $\Delta$ is entirely inherited from the ordered nature of $C^{>}_{nm}$.

For $n,m\ge 1$, only even-odd index pairs contribute and the cross product is always parallel to $\hat{\mathbf z}$. This yields a purely $\sigma_z$ contribution from the $n,m\ge 1$ block. The special $n=0$ layer contributes additional terms. In particular, for $\ell\ge 0$,
\begin{equation}
    \begin{split}
    \mathbf v_0\times \mathbf v_{2\ell+1} &= \big(i c s\,\omega_q^{2\ell+1},\,0,\, i s^2\,\omega_q^{2\ell+1}\big), \\
    \mathbf v_0\times \mathbf v_{2\ell} &= \big(0,\,-c s\,\omega_q^{2\ell},\,0\big),
    \end{split}
    \label{eq:v0_cross_parity}
\end{equation}
so that both $\sigma_x$ and $\sigma_z$ arise from the $0$--odd sector, while a $\sigma_y$ sector is generated by the $0$--even sector whenever the ordered coefficients have a non-vanishing antisymmetric part $A_{0,2\ell}$.
Collecting these contributions, the Pauli sector of $\Delta$ may be written in the general Bloch form
\begin{equation}
    \Delta(\beta)\cong \Delta_x(\beta)\,\sigma_x+\Delta_y(\beta)\,\sigma_y+\Delta_z(\beta)\,\sigma_z,
    \label{eq:Delta_xyz_form}
\end{equation}

\begin{align}
    \Delta_x(\beta)
    &= -2cs\sum_{\ell\ge 0} A_{2\ell+1,\, 0}(\beta)\,\omega_q^{2\ell+1},
    \label{eq:Deltax_series_corr}\\[4pt]
    \Delta_y(\beta)
    &= -2ics\sum_{\ell\ge 1} A_{2\ell,\, 0}(\beta)\,\omega_q^{2\ell},
    \label{eq:Deltay_series_corr}\\[4pt]
    \Delta_z(\beta)
    &= -2s^2\sum_{k\ge 1,\,\ell\ge 0} A_{2k,\,2\ell+1}(\beta)\,\omega_q^{2k+2\ell+1} \notag\\
    &\quad\, -2s^2\sum_{\ell\ge 0} A_{2\ell+1,\, 0}(\beta)\,\omega_q^{2\ell+1}.
    \label{eq:Deltaz_series_corr}
\end{align}
The first term in \eqref{eq:Deltaz_series_corr} comes from the $n,m\ge 1$ even--odd sector, and the remaining terms from the $n=0$ cross layer.

\subsection{Generating function for the influence exponent}

While technically complete, this series representation of $\Delta$ is not particularly illuminating. A more transparent expression for the components - with a correspondingly more robust physical interpretation - is obtained by recognising that the ordered moments $C^{>}_{nm}$ can be interpreted as Laplace transforms of the imaginary-time kernel. To this end, we introduce the bivariate generating function
\begin{equation}
    \mathcal{G}^>(x,y)
    =
    \int_0^\beta d\tau \int_0^\tau d\tau'\,
    K(\tau-\tau')\,e^{x\tau+y\tau'},
    \label{eq:Gxy_def}
\end{equation}
and its anti-ordered counterpart $\mathcal{G}^<(x,y) = \mathcal{G}^>(y,x)$, where the equality follows from $K(-u)=K(u)$. Structurally, these objects are the time-ordered and anti-ordered Green's functions of the bath projected onto the Laplace domain. They directly sample the bath's correlations at the frequencies $x$ and $y$ dictated by the system's eigenoperators. This identification immediately suggests an implicit fluctuation theorem, readily uncovered by application of the KMS condition $K(\beta-u)=K(u)$:
\begin{equation}
    \mathcal{G}^>(x,y) = e^{\beta(x+y)}\,\mathcal{G}^>(-y,-x).
    \label{eq:KMS_Gxy}
\end{equation}
Changing variables to $u=\tau-\tau'$, $v=\tau'$ and using the KMS symmetry yields the closed form
\begin{equation}
    \mathcal{G}^>(x,y) = \frac{e^{x\beta}\tilde{K}(y) - \tilde{K}(x)}{x+y},
    \label{eq:Gxy_closed}
\end{equation}
where we have defined the bare Laplace transform of the kernel as
\begin{equation}
    \tilde{K}(\omega) \equiv \int_0^\beta du\, K(u)\,e^{\omega u}.
    \label{eq:Ktilde_def}
\end{equation}
The resonant case $x+y \to 0$ is recovered as the limit $\mathcal{G}^>(x,-x) = e^{x\beta}\tilde{K}'(-x)$, where pure derivatives with respect to frequency appear:
\begin{equation}
    \tilde{K}'(-x) \equiv \partial_{\omega}\tilde{K}(\omega)\big|_{\omega=-x} = \int_0^\beta du\, u\, K(u)\, e^{-xu}.
\end{equation}
Using the KMS symmetry, the resonant limit takes the explicit integral form
\begin{equation}
    \mathcal{G}^>(x,-x) = \int_0^\beta du\,(\beta-u)\,K(u)\,e^{xu}.
    \label{eq:Gxy_resonant_integral}
\end{equation}
These resonant terms introduce linear-in-$\beta$ (secular) and exponential-in-$\beta$ behaviors, which upon recombination generate the hyperbolic structure observed in the $\Delta_z$ component equations.

The effective Hamiltonian is determined by the antisymmetric combination
\begin{equation}
\begin{split}
    \mathcal{G}^-(x,y) 
    &\equiv \mathcal{G}^>(x,y) - \mathcal{G}^<(x,y) \\
    &= \frac{e^{x\beta}\tilde{K}(y) - e^{y\beta}\tilde{K}(x)
           + \tilde{K}(x) - \tilde{K}(y)}{x+y},
\end{split}
\label{eq:Gminus_closed}
\end{equation}
while the symmetric part $\mathcal{G}^+(x,y) \equiv \mathcal{G}^>(x,y) + \mathcal{G}^<(x,y)$ contributes only to the partition function normalisation. This splitting is physically natural: the Pauli algebra maps the antisymmetric operator product to the vector sector $\boldsymbol{\sigma}$, and the symmetric product to the scalar identity.

Because the qubit adjoint chain \eqref{eq:fn_qubit} has a strict even/odd parity structure, only the values $x,y\in\{0,\pm\omega_q\}$ enter the sum. Evaluating \eqref{eq:Gminus_closed} at these points and introducing the even and odd parts of $\tilde{K}$ at the Bohr frequency,
\begin{equation}
    \tilde{K}^\pm \equiv \tilde{K}(\omega_q) \pm \tilde{K}(-\omega_q),
    \label{eq:Ktilde_pm}
\end{equation}
the influence components reduce to
\begin{align}
    \Delta_x(\beta)
        &= \frac{cs}{\omega_q}\,\tilde{K}^-,
    \label{eq:Deltax_final}\\[4pt]
    \Delta_y(\beta)
        &= \frac{ics}{\omega_q}\,\tilde{K}^+,
    \label{eq:Deltay_final}\\[4pt]
    \Delta_z(\beta)
        &= \frac{s^2}{\omega_q}\,\tilde{K}^-
           + \frac{s^2}{2\omega_q}\Bigl[
               \tilde{K}^- \cosh(\omega_q\beta) - \tilde{K}^+ \sinh(\omega_q\beta)
             \Bigr].
    \label{eq:Deltaz_final}
\end{align}
The explicit dependence on the transition frequency $\omega_q$ identifies this as an `on-shell' renormalization: the bath's infinite complexity is distilled into its response at the system's natural energy scale.
This structure establishes a direct map to the standard non-equilibrium Green's function formalism. The odd component $\tilde{K}^-$ plays the role of a dissipative self-energy (analogous to the imaginary part of a retarded Green's function), driving the coherent renormalization of the Hamiltonian parameters $\Delta_x$ and $\Delta_z$. 
Conversely, the even component $\tilde{K}^+$ represents the integrated fluctuation spectrum (analogous to the Keldysh or 'greater' Green's function), explicitly capturing the thermal noise responsible for the $\Delta_y$ component. 
The precise balance between these terms is dictated by the KMS condition, which functions here as an on-shell Fluctuation-Dissipation Theorem, ensuring that the interplay of dissipation and fluctuations correctly targets the canonical equilibrium state.

\subsubsection*{Recombination}
The reduced equilibrium operator is $\bar\rho_Q(\beta)=e^{-\beta H_Q}e^{\Delta(\beta)}$, and the Hamiltonian of mean force is defined (up to an additive constant) by
\begin{equation}
    H_{\mathrm{MF}}(\beta) = -\beta^{-1}\log\bar\rho_Q(\beta).
    \label{eq:HMF_def_qubit}
\end{equation}
Because $\bar\rho_Q(\beta)$ is Hermitian and positive, $H_{\mathrm{MF}}(\beta)$ is itself Hermitian. Since the influence operator $\Delta(\beta)$ lies in the span of $\{\sigma_x,\sigma_z\}$ (after absorbing the $\Delta_y$ phase or rotating), so does $H_{\mathrm{MF}}(\beta)$. Thus one may generally write
\begin{equation}
    H_{\mathrm{MF}}(\beta) \equiv \mathrm{const}\cdot \mathbb I + h_x(\beta)\,\sigma_x + h_z(\beta)\,\sigma_z,
    \label{eq:HMF_xz_form}
\end{equation}
with the effective fields $h_{x,z}(\beta)$ obtained exactly by evaluating the $2\times 2$ logarithm in \eqref{eq:HMF_def_qubit}. 
A subsequent rotation about $\sigma_z$ can be used to align the transverse component purely along $\sigma_x$, yielding a manifestly real representation if desired.
In the following subsection we carry out this final step explicitly and discuss the resulting renormalisation of the qubit field direction as a function of temperature and bath spectrum.


\subsection{Numerical Demonstrations}
\label{sec:numerical_validation}

To validate the analytic derivation of the mean-force Hamiltonian $H_\mathrm{MF}$, we compare our results against exact numerical diagonalization (ED) of the full system-bath Hamiltonian. We model the bath as a set of discrete harmonic oscillators with an Ohmic spectral density, using $N=4$ modes and a Fock space cutoff of $N_{cut}=6$ per mode to ensure convergence. The full Hamiltonian is diagonalized to obtain the exact thermal state $\rho_{tot} = e^{-\beta H_{tot}}/Z$, from which the reduced system state $\rho_S = \mathrm{Tr}_B[\rho_{tot}]$ is computed.

In Fig.~\ref{fig:hmf_v4_validation}, we show the trace distance $D(\rho_{ex}, \rho_{HMF}) = \frac{1}{2}\mathrm{Tr}|\rho_{ex} - \rho_{HMF}|$ between the exact reduced state and the state generated by our analytic mean-force Hamiltonian. The agreement is excellent, with the distance remaining below $10^{-5}$ across the entire range of coupling strengths $\lambda \in [0, 4]$.

Furthermore, we extract the effective fields $h_x, h_z$ from the numerical state by projecting onto the Pauli basis. As predicted by our theory, the effective field vector rotates non-trivially in the $xz$-plane as the coupling increases. The analytic predictions (solid lines) perfectly match the numerical data (dashed lines), confirming that the simple expression for $\mathcal{K}(\beta)$ captures the full non-perturbative influence of the environment.

\begin{figure}[t]
    \centering
    % \includegraphics[width=\columnwidth]{figures/hmf_v4_validation_fig.png}
    \caption{Validation of the analytic mean-force Hamiltonian against exact diagonalization. (Left) The trace distance between the exact reduced state and the analytic prediction remains negligible ($< 10^{-5}$) for all coupling strengths $\lambda$. (Center) The components of the effective field $\vec{h}$ in the rotated frame show perfect agreement between theory (solid) and numerics (dashed). (Right) The unrotated $y$-component is non-zero in the lab frame but vanishes in the rotated frame as predicted.}
    \label{fig:hmf_v4_validation}
\end{figure}

% \input{sections/05_designability_alltoall_v2}
% \input{sections/05_numerical_validation_v2}
\section{Discussion}
\label{sec:discussion}

This manuscript is positioned as the direct sequel to
Ref.~\cite{mccaulMeanForceHamiltoniansInfluence2026}. The earlier paper
established the exact commuting Gaussian benchmark; the present one isolates the
operator-algebraic obstruction that appears once $[H_Q,f]\neq 0$.

We have derived an exact operator reformulation of the reduced equilibrium
object for a Gaussian bath, expressed both as a bilocal imaginary-time
influence functional and as a quenched Gaussian-field average. The nonlocal
structure in imaginary time is traced entirely to noncommutativity between
$H_Q$ and the coupling operator $f$, which forces the interaction-picture
operator $\tilde{f}(\tau)$ to appear in time-ordered products. By expanding
$\tilde{f}(\tau)$ in adjoint actions and isolating kernel moments, we obtained a
purely algebraic representation that yields an exact closure criterion for
locality of $H_{\mathrm{MF}}$.

When the adjoint-generated operator algebra closes, the mean-force Hamiltonian
is a finite operator polynomial and can be constructed via Magnus/BCH. When it
does not, any local representation necessarily involves truncation or
projection; no other approximation is introduced. Broader implications of these
results are deferred to the next stage of the programme: non-Gaussian cumulant
extensions and noncommuting finite-bath benchmarks.

The new designability section (Sec.~\ref{sec:alltoall_designability}) makes the
main constructive consequence explicit for a free-spin core: with a complete
pair-resolved channel basis, finite-$N$ all-to-all 2-local coefficient maps are
full rank and directly invertible. The numerical component there separates two
roles: (i) reconstruction/rank diagnostics for the full all-to-all basis, and
(ii) stochastic free-model benchmarking in a dense $ZZ$ sector, where
reweighting remains controlled enough to compare against exact thermal ED while
also exposing scaling against an MPO/DMRG baseline.

The retained qubit benchmark (Sec.~\ref{sec:numerical_validation}) continues to
serve a different purpose: continuity with the weak/ultrastrong asymptotics of
Ref.~\cite{cresserWeakUltrastrongCoupling2021a}. Together, the two numerical
sections split validation into ``known asymptotic physics recovery'' (qubit) and
``constructive many-body designability from free trajectories'' (all-to-all
construction).

Beyond this, the present formulation makes  an informational interpretation of $H_{\mathrm{MF}}$ unavoidable. From this perspective, the logarithm acts as a \emph{compression map}. The
quenched operator $\bar{\rho}_Q(\beta)=\mathbb E_\xi[U_\xi(\beta)]$ is a mixture over history-conditioned propagators, whereas $\tilde H_{\mathrm{MF}}(\beta)=-(1/\beta)\log\bar{\rho}_Q(\beta)$ is the unique local object whose Gibbs form reproduces that mixture. The gap between ``log of average'' and ``average
of log'' is therefore a measure of \emph{how much information about the bath history is lost} when one insists on a single effective Hamiltonian. This is the same structural gap that distinguishes quenched and annealed free energies in disordered statistical mechanics, where one compares $\langle \log Z \rangle$ to $\log\langle Z\rangle$. In the commuting case this gap disappears entirely. 



\appendix
\setcounter{secnumdepth}{2}
%\section{Quenched Density and Imaginary-Time Evolution}
\label{sec:appendix_quenched}

This appendix makes explicit the operator identity underlying the quenched
imaginary-time formulation used in Sec.~\ref{sec:quenched}. Define a
time-ordered exponential with a (generally) $\tau$-dependent operator
$H(\tau)$,
\begin{equation}
    \begin{split}
        \rho(\tau) &\equiv \mathcal{T}_\tau
        \exp\left(-\int_0^\tau d\tau' \, H(\tau')\right), \\
        \rho(0) &= \mathbb{I}.
    \end{split}
    \label{eq:appendix_rho_def}
\end{equation}
Standard differentiation identities for ordered exponentials imply
\begin{equation}
    -\partial_\tau \rho(\tau) = H(\tau)\rho(\tau),
    \label{eq:appendix_imag_time_eq}
\end{equation}
with the ordering built into $\rho(\tau)$; see, e.g.,
Refs.~\cite{wilcoxExponentialOperatorsParameter1967a,magnusExponentialSolutionDifferential1954a,blanesMagnusExpansionIts2009}
for operator calculus and time-ordered exponentials.

In the present context one takes
\begin{equation}
    H(\tau) = H_Q + \xi(\tau) f,
    \label{eq:appendix_quenched_H}
\end{equation}
where the Gaussian field satisfies
\begin{equation}
    \langle \xi(\tau)\xi(\tau')\rangle = K(\tau-\tau').
    \label{eq:appendix_xi_cov}
\end{equation}
The reduced equilibrium operator can be written as
\begin{equation}
    \bar{\rho}_S =
    e^{-\beta H_Q}\left\langle \rho(\beta) \right\rangle_\xi,
    \label{eq:appendix_rho_beta}
\end{equation}
which is the compact operator form of the quenched Gaussian representation
used in the main text.

To connect Eq.~\eqref{eq:appendix_rho_beta} to an imaginary-time path integral,
one discretizes $\tau \in [0,\beta]$, applies a Trotter (or Zassenhaus)
factorization, and inserts resolutions of identity in the system coordinate
basis. This yields the standard Euclidean path-integral expression for the
canonical density operator, with the $\tau$-dependent potential induced by the
auxiliary field, as in the influence-functional derivation for quadratic baths
and their stochastic unravellings
\cite{feynmanTheoryGeneralQuantum1963a,caldeiraQuantumTunnellingDissipative1983a,grabertQuantumBrownianMotion1988,moixEquilibriumreducedDensityMatrix2012,chenRigorousStochasticMatrix2014}.

\section{Kernel Symmetry and Moment Relations}
\label{sec:appendix_kernel}

The bath kernel appearing in the bilocal influence functional is
\begin{equation}
    \begin{split}
        K(\tau-\tau') &=
        \mathrm{Tr}_B\!\left[\mathcal{T}_\tau \tilde{B}(\tau)\tilde{B}(\tau')\rho_B\right], \\
        \rho_B &= \frac{e^{-\beta H_X}}{Z_B}.
    \end{split}
    \label{eq:appendix_kernel_def}
\end{equation}
For equilibrium baths, $K$ depends only on the imaginary-time difference and is
even under exchange of its arguments, implying
\begin{equation}
    K(\tau-\tau') = K(\tau'-\tau), \qquad
    \mu_{nm} = \mu_{mn},
    \label{eq:appendix_kernel_sym}
\end{equation}
\begin{equation}
    \mu_{nm} = \frac{1}{n!m!} \int_0^\beta d\tau \int_0^\beta d\tau' \tau^n (\tau')^m K(\tau-\tau').
    \label{eq:kernel_moments_def}
\end{equation}
In common quadratic-bath models the kernel is
explicitly constructed from the bath spectral density and satisfies the
Kubo-Martin-Schwinger periodicity in imaginary time, which can be used to
re-express moment integrals in equivalent forms; see
Refs.~\cite{grabertQuantumBrownianMotion1988,tanimuraReducedHierarchicalEquations2014,songCalculationCorrelatedInitial2015}
for explicit constructions.

The moment expansion used in Sec.~\ref{sec:quenched} requires only these symmetry
properties and the existence of the integrals defining $\mu_{nm}$. No further
approximation is introduced at this stage.

\section{Derivation of the Influence Functional\label{app:influence_derivation}}

In this appendix, we construct the influence functional formalism used in Sec.~\ref{sec:quenched}. Our goal is to derive the exact form of the reduced density operator $\bar{\rho}_S(\beta)$ by explicitly integrating out the harmonic bath, and to demonstrate that this leads directly to the stochastic unravelling employed in the main text.

\subsection{Euclidean Path Integral Setup}
We begin with the definition of the unnormalized reduced state,
\begin{equation}
    \bar{\rho}_S(\beta) = \Tr_X \left[ e^{-\beta H_{\mathrm{tot}}} \right].
\end{equation}
This trace can be represented as a Euclidean path integral. Let $|q\rangle$ and $|x\rangle = |x_1, x_2, \dots\rangle$ denote the position bases for the system and bath, respectively. The matrix element $\langle q | \bar{\rho}_S | q' \rangle$ involves a sum over all periodic paths $x(\tau)$ (where $x(0)=x(\beta)$) and open paths $q(\tau)$ (where $q(0)=q'$ and $q(\beta)=q$):
\begin{equation}
\begin{split}
    \langle q | \bar{\rho}_S | q' \rangle = \int_{q(0)=q'}^{q(\beta)=q} &\mathcal{D}q(\tau) e^{-S_Q[q]/\hbar} \\
    &\times \prod_k Z_k[q],
\end{split}
    \label{eq:app_path_integral_start}
\end{equation}
where $S_Q$ is the Euclidean action of the isolated system, and $Z_k[q]$ is the partition function of the $k$-th oscillator in the presence of the external driving force $J_k(\tau) = -c_k f(q(\tau))$:
\begin{equation}
\begin{split}
    Z_k[q] = \oint & \mathcal{D}x_k(\tau) \exp\bigg( -\frac{1}{\hbar} \int_0^\beta d\tau \\
    &\times \left[ \frac{m_k}{2} \dot{x}_k^2 + \frac{m_k \omega_k^2}{2} x_k^2 + c_k x_k f(q(\tau)) \right] \bigg).
\end{split}
\end{equation}
Note that the periodicity of the trace implies periodic boundary conditions for the bath paths $x_k(\tau)$.

\subsection{Gaussian Integration}
The functional integral for $Z_k[q]$ is Gaussian and can be evaluated exactly. It corresponds to the partition function of a forced harmonic oscillator. The result is expressible as the product of the free oscillator partition function, $Z_X^{(k)} = (2\sinh(\beta\hbar\omega_k/2))^{-1}$, and an exponential "influence phase" depending quadratically on the drive~\cite{feynmanTheoryGeneralQuantum1963a,weissQuantumDissipativeSystems2012}:
\begin{equation}
\begin{split}
    Z_k[q] = Z_X^{(k)} \exp\bigg( &\frac{1}{2\hbar} \int_0^\beta d\tau \int_0^\beta d\tau' \\
    &\times K_k(\tau-\tau') f(q(\tau)) f(q(\tau')) \bigg).
\end{split}
\end{equation}
The kernel $K_k(\tau)$ is the equilibrium autocorrelation function of the coordinate $x_k$:
\begin{equation}
    K_k(\tau-\tau') = c_k^2 \langle \mathcal{T}_\tau x_k(\tau) x_k(\tau') \rangle_0.
\end{equation}
Summing over all modes $k$, the total influence functional is $\prod_k Z_k[q] = Z_X \exp( \Phi_{inf}[q] )$, with
\begin{equation}
\begin{split}
    \Phi_{inf}[q] = \frac{1}{2\hbar} \int_0^\beta d\tau &\int_0^\beta d\tau' K(\tau-\tau') \\
    &\times f(q(\tau)) f(q(\tau')),
\end{split}
\end{equation}
where $K(\tau) = \sum_k K_k(\tau)$ is the total force autocorrelation function.

\subsection{From Non-local Action to Stochastic Average}
Substituting this back into Eq.~\eqref{eq:app_path_integral_start}, the reduced density matrix becomes
\begin{equation}
    \bar{\rho}_S = Z_X \int \mathcal{D}q \, e^{-S_Q[q]/\hbar} \exp\left( \Phi_{inf}[q] \right).
\end{equation}
The term $\Phi_{inf}[q]$ is non-local in imaginary time, representing a self-interaction of the system mediated by the bath. To disentangle this, we use the Hubbard-Stratonovich transformation (the continuous analog of the Gaussian identity $e^{\frac{1}{2} A^2} \sim \int d\xi e^{-\frac{1}{2}\xi^2 + \xi A}$). We introduce a real, auxiliary stochastic field $\xi(\tau)$ with zero mean and covariance
\begin{equation}
    \langle \xi(\tau) \xi(\tau') \rangle_\xi = K(\tau-\tau').
\end{equation}
Using this field, we can rewrite the influence exponential as a stochastic average:
\begin{equation}
\begin{split}
    \exp\left( \Phi_{inf}[q] \right) = \bigg\langle \exp\bigg( &\frac{1}{\hbar} \int_0^\beta d\tau \\
    &\times \xi(\tau) f(q(\tau)) \bigg) \bigg\rangle_\xi.
\end{split}
\end{equation}
Inserting this identity into the path integral for $\bar{\rho}_S$, we can swap the order of the path integration over $q$ and the stochastic average over $\xi$:
\begin{equation}
    \bar{\rho}_S = Z_X \left\langle \int \mathcal{D}q \, \exp\left[-\frac{1}{\hbar} S_Q[q] + \frac{1}{\hbar} \int_0^\beta \xi f \right] \right\rangle_\xi.
\end{equation}
The term in the angle brackets is exactly the path integral for a system evolving under the time-dependent Hamiltonian $H(\tau) = H_Q - \xi(\tau)f$. Thus, in operator language, we arrive at the exact stochastic representation:
\begin{equation}
\begin{split}
    \bar{\rho}_S(\beta) = Z_X \bigg\langle \mathcal{T}_\tau \exp\bigg( &-\int_0^\beta d\tau \\
    &\times [H_Q - \xi(\tau)f] \bigg) \bigg\rangle_\xi.
\end{split}
    \label{eq:app_stochastic_final}
\end{equation}
This confirms that the HMF can be constructed by averaging the non-unitary (imaginary-time) evolution of the system driven by colored Gaussian noise.

% \section{Universal reduction of the moment matrix $C_{nm}(\beta)$}
\label{app:Cnm_universal}

This appendix derives the universal representation \eqref{eq:Cnm_universal_main} and gives explicit
closed forms for the thermal form factors $S_r(\omega,\beta)$.

\subsection{Step 1: swap Matsubara sum and spectral integral}

Start from
\begin{equation}
C_{nm}(\beta)=\frac{1}{\beta}\sum_{\ell\in\mathbb Z}\kappa_\ell\,I_n(\nu_\ell)\,I_m(\nu_\ell)^*,
\qquad \nu_\ell=\frac{2\pi\ell}{\beta},
\end{equation}
and insert the harmonic-bath spectral form \eqref{eq:kappa_l_J}:
\begin{equation}
\kappa_\ell=\frac{2}{\pi}\int_0^\infty d\omega\,\frac{\omega J(\omega)}{\omega^2+\nu_\ell^2}.
\end{equation}
Assuming the standard UV regularity on $J(\omega)$ that ensures the Gaussian influence functional is
well-defined, we may exchange the sum and integral to obtain
\begin{equation}
\label{eq:Cnm_swap}
\begin{split}
C_{nm}(\beta)&=\frac{2}{\pi}\int_0^\infty d\omega\,\omega J(\omega)\,
\Phi_{nm}(\omega,\beta), \\
\Phi_{nm}(\omega,\beta)&\equiv
\frac{1}{\beta}\sum_{\ell\in\mathbb Z}\frac{I_n(\nu_\ell)I_m(\nu_\ell)^*}{\omega^2+\nu_\ell^2}.
\end{split}
\end{equation}
Thus all bath dependence is isolated as the scalar weight $\omega J(\omega)$.

\subsection{Step 2: finite inverse-frequency expansion of $I_n(\nu_\ell)$}

For $\ell\neq 0$, bosonic Matsubara frequencies satisfy $e^{i\nu_\ell\beta}=1$.
The integral \eqref{eq:Ftilde_In} can be evaluated in closed form for $\nu\neq 0$ as
\begin{equation}
\label{eq:In_closed_form}
I_n(\nu)
=\frac{1}{(-i\nu)^{n+1}}
\left[
1-e^{i\nu\beta}\sum_{k=0}^{n}\frac{(-i\nu\beta)^k}{k!}
\right],
\end{equation}
which for $\nu_\ell$ simplifies to
\begin{equation}
\label{eq:In_poly_app}
\begin{split}
I_n(\nu_\ell)
&=\frac{1}{(-i\nu_\ell)^{n+1}}
\left[
1-\sum_{k=0}^{n}\frac{(-i\nu_\ell\beta)^k}{k!}
\right] \\
&=-\sum_{p=1}^{n}\frac{\beta^{\,n+1-p}}{(n+1-p)!}\,\frac{1}{(-i\nu_\ell)^{p}},
\qquad \ell\neq 0.
\end{split}
\end{equation}
The $\ell=0$ value is obtained by continuity,
\begin{equation}
\label{eq:In0_app}
I_n(0)=\frac{\beta^{n+1}}{(n+1)!}.
\end{equation}
Define coefficients
\begin{equation}
\label{eq:Anp_def_app}
A_{n,p}\equiv \frac{\beta^{\,n+1-p}}{(n+1-p)!},\qquad p=1,\dots,n,
\end{equation}
so that $I_n(\nu_\ell)=-\sum_{p=1}^{n}A_{n,p}(-i\nu_\ell)^{-p}$ for $\ell\neq 0$.

\subsection{Step 3: reduction to even-power Matsubara sums}

Split $\Phi_{nm}$ into the zero mode and the nonzero modes:
\begin{equation}
\label{eq:Phi_split_app}
\Phi_{nm}(\omega,\beta)=\frac{1}{\beta}\frac{I_n(0)I_m(0)}{\omega^2}
+\frac{1}{\beta}\sum_{\ell\neq 0}\frac{I_n(\nu_\ell)I_m(\nu_\ell)^*}{\omega^2+\nu_\ell^2}.
\end{equation}
Insert the finite expansion \eqref{eq:In_poly_app} into the $\ell\neq 0$ term:
\begin{equation}
I_n(\nu_\ell)I_m(\nu_\ell)^*
=\sum_{p=1}^{n}\sum_{q=1}^{m}A_{n,p}A_{m,q}\,
\frac{1}{(-i\nu_\ell)^{p}}\frac{1}{(+i\nu_\ell)^{q}}.
\end{equation}
Under $\ell\mapsto -\ell$, $\nu_\ell\mapsto -\nu_\ell$ while $\omega^2+\nu_\ell^2$ is invariant. It follows
that contributions with odd $(p+q)$ cancel in the $\ell$-sum, leaving only even total powers
$p+q=2r$. Collecting the surviving terms yields the finite reduction
\begin{equation}
\label{eq:Phi_reduce_app}
\Phi_{nm}(\omega,\beta)
=\frac{1}{\beta}\frac{I_n(0)I_m(0)}{\omega^2}
+\sum_{r=1}^{\lfloor (n+m)/2\rfloor}\alpha^{(nm)}_{r}(\beta)\,S_r(\omega,\beta),
\end{equation}
where the universal form factors are
\begin{equation}
\label{eq:Sr_def_app}
S_r(\omega,\beta)\equiv \frac{1}{\beta}\sum_{\ell\neq 0}\frac{1}{\nu_\ell^{2r}}\frac{1}{\omega^2+\nu_\ell^2},
\qquad r=0,1,2,\dots,
\end{equation}
and the coefficients are the explicit finite combinatorial contractions
\begin{equation}
\label{eq:alpha_def_app}
\alpha^{(nm)}_{r}(\beta)\equiv
\sum_{\substack{p=1,\dots,n\\ q=1,\dots,m\\ p+q=2r}}
(-1)^{p+r}\,A_{n,p}A_{m,q}.
\end{equation}
Substituting \eqref{eq:Phi_reduce_app} into \eqref{eq:Cnm_swap} gives the universal representation
\eqref{eq:Cnm_universal_main}.

\subsection{Step 4: closed form for $S_r(\omega,\beta)$}

The base sum is standard:
\begin{equation}
\label{eq:S0_app}
S_0(\omega,\beta)=\frac{1}{\beta}\sum_{\ell\neq 0}\frac{1}{\omega^2+\nu_\ell^2}
=\frac{1}{2\omega}\coth\!\left(\frac{\beta\omega}{2}\right)-\frac{1}{\beta\omega^2}.
\end{equation}
Define also the zeta moments
\begin{equation}
\label{eq:Zr_app}
Z_r(\beta)\equiv \frac{1}{\beta}\sum_{\ell\neq 0}\frac{1}{\nu_\ell^{2r}}
=\beta^{2r-1}\frac{\zeta(2r)}{(2\pi)^{2r}},\qquad r\ge 1.
\end{equation}
Using the identity
\begin{equation}
\frac{1}{\nu^{2r}(\omega^2+\nu^2)}
=\frac{1}{\omega^2}\left(\frac{1}{\nu^{2r}}-\frac{1}{\nu^{2r-2}(\omega^2+\nu^2)}\right),
\end{equation}
one obtains the recursion
\begin{equation}
\label{eq:Sr_rec_app}
S_r(\omega,\beta)=\frac{1}{\omega^2}\Big(Z_r(\beta)-S_{r-1}(\omega,\beta)\Big),\qquad r\ge 1,
\end{equation}
which may be unrolled to the explicit closed form
\begin{equation}
\label{eq:Sr_closed_app}
S_r(\omega,\beta)=
\sum_{k=0}^{r-1}(-1)^k\frac{Z_{r-k}(\beta)}{\omega^{2(k+1)}}
+(-1)^r\frac{S_0(\omega,\beta)}{\omega^{2r}}.
\end{equation}
Thus every $S_r$ is an explicit combination of a single thermal function $\coth(\beta\omega/2)$ and
a finite set of zeta values $\{\zeta(2),\dots,\zeta(2r)\}$.

\section{Direct $\Delta$ evaluation for the PRL qubit and projector reduction of Eq.~(8)}
\label{app:prl_qubit_delta}

This appendix provides the direct evaluation used in
Sec.~\ref{sec:qubit_prl_recovery}. We take
\begin{equation}
H_S=\frac{\omega_q}{2}\sigma_z,\qquad
X=\cos\theta\,\sigma_z-\sin\theta\,\sigma_x,
\end{equation}
and the coupling scaling $J_\lambda(\omega)=\lambda^2J_0(\omega)$ so that
$\kappa_\ell(\lambda)=\lambda^2\kappa_\ell^{(0)}$.

\subsection{Direct-kernel multiplication in Eq.~\eqref{eq:Delta_direct_prl}}

Using $\tilde X(\tau)=e^{\tau H_S}Xe^{-\tau H_S}$,
\begin{equation}
\tilde X(\tau)=\cos\theta\,\sigma_z
-\sin\theta\!\left[\cosh(\omega_q\tau)\sigma_x+i\sinh(\omega_q\tau)\sigma_y\right].
\end{equation}
Set $c\equiv\cos\theta$, $s\equiv\sin\theta$, and $u\equiv\tau-\tau'$. Direct Pauli multiplication gives
\begin{equation}
\begin{split}
\tilde X(\tau)\tilde X(\tau')
&=\big[c^2+s^2\cosh(\omega_q u)\big]\mathbb I
+s^2\sinh(\omega_q u)\sigma_z \\
&\quad +cs\big[\sinh(\omega_q\tau)-\sinh(\omega_q\tau')\big]\sigma_x \\
&\quad +ics\big[\cosh(\omega_q\tau)-\cosh(\omega_q\tau')\big]\sigma_y.
\end{split}
\end{equation}
In Eq.~\eqref{eq:Delta_direct_prl}, however, the bilinear enters as
$\mathcal T_\tau[\tilde X(\tau)\tilde X(\tau')]$. Use
\begin{equation}
\mathcal T_\tau[AB]=\frac12\{A,B\}+\frac12\,\mathrm{sgn}(\tau-\tau')[A,B].
\end{equation}
Applying this to the product above gives
\begin{equation}
\begin{split}
\mathcal T_\tau[\tilde X(\tau)\tilde X(\tau')]
&=\big[c^2+s^2\cosh(\omega_q u)\big]\mathbb I \\
&\quad +cs\,\mathrm{sgn}(u)\big[\sinh(\omega_q\tau)-\sinh(\omega_q\tau')\big]\sigma_x \\
&\quad +ics\,\mathrm{sgn}(u)\big[\cosh(\omega_q\tau)-\cosh(\omega_q\tau')\big]\sigma_y \\
&\quad +s^2\,\mathrm{sgn}(u)\sinh(\omega_q u)\sigma_z .
\end{split}
\end{equation}
Hence
\begin{equation}
\Delta(\beta,\lambda)=\lambda^2\!\left[\alpha_0\mathbb I
+\delta_x\sigma_x+i\delta_y\sigma_y+\delta_z\sigma_z\right],
\end{equation}
with
\begin{align}
\alpha_0
&=\frac12\!\int_0^\beta d\tau\!\int_0^\beta d\tau'\,
K_0(\tau-\tau')\big[c^2+s^2\cosh(\omega_q(\tau-\tau'))\big],\\
\delta_z
&=\frac{s^2}{2}\!\int_0^\beta d\tau\!\int_0^\beta d\tau'\,
K_0(\tau-\tau')\,\mathrm{sgn}(\tau-\tau')\,\sinh(\omega_q(\tau-\tau')),\\
\delta_x
&=\frac{cs}{2}\!\int_0^\beta d\tau\!\int_0^\beta d\tau'\,
K_0(\tau-\tau')\,\mathrm{sgn}(\tau-\tau')\big[\sinh(\omega_q\tau)-\sinh(\omega_q\tau')\big],\\
\delta_y
&=\frac{cs}{2}\!\int_0^\beta d\tau\!\int_0^\beta d\tau'\,
K_0(\tau-\tau')\,\mathrm{sgn}(\tau-\tau')\big[\cosh(\omega_q\tau)-\cosh(\omega_q\tau')\big].
\end{align}

\subsection{Reduction to one-dimensional direct-kernel moments}

Because $K_0(u)=K_0(-u)$, the coefficients reduce exactly to
\begin{align}
\alpha_0(\beta)
&=\int_{0}^{\beta}\!du\,(\beta-u)\,K_0(u)\big[c^2+s^2\cosh(\omega_q u)\big],\\
\delta_z(\beta)
&=s^2\int_{0}^{\beta}\!du\,(\beta-u)\,K_0(u)\sinh(\omega_q u),\\
\delta_x(\beta)
&=\frac{cs}{\omega_q}\int_{0}^{\beta}\!du\,K_0(u)\!
\left[\cosh(\beta\omega_q)+1-\cosh(\omega_q u)-\cosh\!\big(\omega_q(\beta-u)\big)\right],\\
\delta_y(\beta)
&=\frac{cs}{\omega_q}\int_{0}^{\beta}\!du\,K_0(u)\!
\left[\sinh(\beta\omega_q)-\sinh(\omega_q u)-\sinh\!\big(\omega_q(\beta-u)\big)\right].
\end{align}
These are exactly Eqs.~\eqref{eq:alpha0_direct_kernel}--\eqref{eq:deltaz_direct_kernel}
in the main text.

\subsection{Weak-order correction}

Let $\tau_S=\frac12(\mathbb I-\tanh(\beta\omega_q/2)\sigma_z)$ and
$\Delta=\lambda^2\Delta_0$. Expanding
$\rho\propto e^{-\beta H_S}e^{\Delta}$ to $O(\lambda^2)$ and keeping the Hermitian part gives
\begin{equation}
\delta\rho_{\mathrm{weak}}
=\frac{\delta_x}{2}\sigma_x
+\frac{1-\tanh^2(\beta\omega_q/2)}{2}\,\delta_z\,\sigma_z,
\end{equation}
which is Eq.~\eqref{eq:delta_rho_weak_prl}. The $i\delta_y\sigma_y$ component contributes to the
collapsed finite-coupling product shortcut, but not to this Hermitian weak-order correction.

\subsection{Ordered finite-coupling evaluation used in the benchmark}

The finite comparator in Sec.~\ref{sec:qubit_prl_recovery} is not the collapsed
$e^{-\beta H_S}e^{\Delta}$ product. We evaluate instead
\begin{equation}
\rho_{\mathrm{ord}}(\beta,\lambda)\propto
e^{-\beta H_S}\,\mathcal T_\tau
\exp\!\left[
\lambda^2\!\int_0^\beta d\tau\!\int_0^\tau d\tau'\,
K_0(\tau-\tau')\,\tilde X(\tau)\tilde X(\tau')
\right],
\end{equation}
numerically by deterministic Gaussian quadrature:
\begin{enumerate}
\item discretise $\tau\in[0,\beta]$ and build the covariance matrix
$C_{ij}=K_0(\tau_i-\tau_j)$;
\item apply a KL decomposition $C\approx VV^\top$ with retained rank $r$;
\item write the sampled field as $\xi(\tau_i)=\lambda (V\eta)_i$ with independent
standard-normal components of $\eta$;
\item replace the $\eta$-average by tensor Gauss--Hermite quadrature and evaluate each
time-ordered trajectory exactly using $\exp[-v\tilde X]=\cosh(v)\mathbb I-\sinh(v)\tilde X$,
since $\tilde X(\tau)^2=\mathbb I$.
\end{enumerate}
This yields the ordered finite-coupling state used in the figure and CSV outputs.

\subsection{Projector reduction for the PRL ultrastrong qubit state}

Take the interaction-axis projectors
\begin{equation}
P_\pm=\frac12\left(\mathbb I\pm\sigma_{\hat r}\right),\qquad
\sigma_{\hat r}=\cos\theta\,\sigma_z-\sin\theta\,\sigma_x.
\end{equation}
Using $H_S=(\omega_q/2)\sigma_z$ and $P_\pm \sigma_z P_\pm = \pm\cos\theta\,P_\pm$,
\begin{equation}
\sum_n P_n H_S P_n
=\frac{\omega_q\cos\theta}{2}\left(P_+-P_-\right)
=\frac{\omega_q\cos\theta}{2}\sigma_{\hat r}.
\end{equation}
Substitution into PRL Eq.~(7) gives
\begin{equation}
\rho_{US}
=\frac{e^{-(\beta\omega_q\cos\theta/2)\sigma_{\hat r}}}
{\Tr\!\left[e^{-(\beta\omega_q\cos\theta/2)\sigma_{\hat r}}\right]}
=\frac12\left[\mathbb I-\sigma_{\hat r}\tanh\!\left(\frac{\beta\omega_q\cos\theta}{2}\right)\right],
\end{equation}
which is Eq.~\eqref{eq:rho_us_qubit_prl} in the main text.


\bibliography{../../literature/references_new}

\end{document}
