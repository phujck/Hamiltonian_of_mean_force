\section{Quenched representation of an imaginary-time propagator}
\label{sec:quenched}

\subsection{Definition}

Let $H_Q$ be a system Hamiltonian and let $f$ be an arbitrary system operator.
For a prescribed c-number field $\xi(\tau)$ on $\tau\in[0,\beta]$ we define the
\emph{quenched} imaginary-time propagator
\begin{equation}
    \tilde\rho_Q(\beta;\xi)
    :=
    \mathcal T_\tau
    \exp\!\left[
        -\int_0^\beta d\tau\,
        \big(H_Q+\xi(\tau)\,f\big)
    \right].
    \label{eq:quenched_operator_def}
\end{equation}
The term \emph{quenched} refers only to the status of $\xi(\tau)$ as an external
field held fixed while constructing $\tilde\rho_Q(\beta;\xi)$.

\subsection{Generic phase-space path integral}

We derive a path-integral representation for the kernel
\begin{equation}
    \tilde\rho_Q(q_f,q_i;\beta|\xi)
    := \langle q_f|\tilde\rho_Q(\beta;\xi)|q_i\rangle,
\end{equation}
without assuming that $f$ is diagonal in the $|q\rangle$ basis.

Discretise $[0,\beta]$ into $N$ slices of width $\epsilon=\beta/N$, with
$\tau_n=n\epsilon$ and $\xi_n:=\xi(\tau_n)$. Writing
\begin{equation}
    \tilde\rho_Q(\beta;\xi)
    =
    \lim_{N\to\infty}
    \prod_{n=0}^{N-1}
    \exp\!\left[-\epsilon\big(H_Q+\xi_n f\big)\right],
\end{equation}
insert $N-1$ resolutions of the identity $\mathbf 1=\int dq\,|q\rangle\langle q|$ to obtain
\begin{equation}
    \tilde\rho_Q(q_f,q_i;\beta|\xi)
    =
    \lim_{N\to\infty}
    \int \prod_{n=1}^{N-1} dq_n
    \prod_{n=0}^{N-1}
    \langle q_{n+1}|e^{-\epsilon(H_Q+\xi_n f)}|q_n\rangle,
    \label{eq:quenched_trotter_q}
\end{equation}
with $q_0=q_i$ and $q_N=q_f$.

To obtain a generic expression for the short-time kernel, insert a momentum resolution
$\mathbf 1=\int \frac{dp}{2\pi\hbar}\,|p\rangle\langle p|$ between $|q_n\rangle$ and
$|q_{n+1}\rangle$, and use a mid-point (Weyl) prescription. For sufficiently small
$\epsilon$ one has the operator identity
\begin{equation}
    \langle q'|e^{-\epsilon \hat A}|q\rangle
    =
    \int\frac{dp}{2\pi\hbar}\,
    \exp\!\left[\frac{i}{\hbar}p(q'-q)\right]\,
    \exp\!\left[-\epsilon\,A_W\!\left(\frac{q'+q}{2},p\right)\right]
    +\mathcal O(\epsilon^2),
    \label{eq:short_time_weyl}
\end{equation}
where $A_W(q,p)$ denotes the Weyl symbol of the operator $\hat A$.\footnote{Any other
consistent ordering convention may be used; it only changes the discretisation rule.
The continuum limit is taken with the corresponding prescription fixed.}

Applying \eqref{eq:short_time_weyl} with $\hat A = H_Q+\xi_n f$ gives
\begin{align}
    \langle q_{n+1}|e^{-\epsilon(H_Q+\xi_n f)}|q_n\rangle
    &=
    \int\frac{dp_n}{2\pi\hbar}\,
    \exp\!\left[\frac{i}{\hbar}p_n(q_{n+1}-q_n)\right]\,
    \exp\!\left[
        -\epsilon\,
        \big(H_Q+\xi_n f\big)_W\!\left(\frac{q_{n+1}+q_n}{2},p_n\right)
    \right]
    +\mathcal O(\epsilon^2).
    \label{eq:short_time_quenched}
\end{align}

Substituting \eqref{eq:short_time_quenched} into \eqref{eq:quenched_trotter_q} yields the
discrete phase-space path integral
\begin{align}
    \tilde\rho_Q(q_f,q_i;\beta|\xi)
    &=
    \lim_{N\to\infty}
    \int \prod_{n=1}^{N-1} dq_n
    \int \prod_{n=0}^{N-1}\frac{dp_n}{2\pi\hbar}\,
    \exp\!\left[
        \frac{i}{\hbar}\sum_{n=0}^{N-1} p_n(q_{n+1}-q_n)
        -\epsilon\sum_{n=0}^{N-1}
        \big(H_Q+\xi_n f\big)_W\!\left(\frac{q_{n+1}+q_n}{2},p_n\right)
    \right],
    \label{eq:quenched_discrete_phase_space}
\end{align}
with fixed endpoints $q_0=q_i$, $q_N=q_f$.

In the continuum limit this becomes
\begin{equation}
    \tilde\rho_Q(q_f,q_i;\beta|\xi)
    =
    \int_{q(0)=q_i}^{q(\beta)=q_f}\!\!\mathcal D q
    \int \mathcal D p\,
    \exp\!\left[
        \frac{i}{\hbar}\int_0^\beta d\tau\, p(\tau)\,\dot q(\tau)
        -\int_0^\beta d\tau\, \big(H_Q+\xi(\tau) f\big)_W\!\big(q(\tau),p(\tau)\big)
    \right],
    \label{eq:quenched_phase_space_PI}
\end{equation}
where the subscript $W$ indicates the chosen operator-to-symbol map (here Weyl/mid-point).

Equation \eqref{eq:quenched_phase_space_PI} is the generic path-integral representation
of the quenched $\tau$-ordered propagator \eqref{eq:quenched_operator_def}. No assumption
has been made about diagonal structure of $f$; all operator ordering information is
carried by the fixed discretisation prescription defining the symbol
$\big(H_Q+\xi(\tau)f\big)_W$.
