\subsection{Exact mean-force Hamiltonian for the transverse-coupling 
spin-boson model}

We now turn to applying the results developed in the previous section. A particularly instructive example concerns spins, as the su(2) algebra is sufficiently simple at to permit an exact analytic solution. We demonstrate this by applying it to a transverse-coupling spin-boson model. Following  Ref.~\cite{cresserWeakUltrastrongCoupling2021a}, let
%
\begin{equation}
    H_Q = \frac{\omega_q}{2}\sigma_z, 
    \qquad 
    f = \cos\theta\,\sigma_z - \sin\theta\,\sigma_x,
    \label{eq:qubit_setup}
\end{equation}
%
with $c \equiv \cos\theta$, $s \equiv \sin\theta$ throughout. The 
coupling mixes a commuting part $c\sigma_z$ with a transverse part 
$-s\sigma_x$.

Using $[H_Q,\sigma_x] = i\omega_q\sigma_y$ and 
$[H_Q,\sigma_y] = -i\omega_q\sigma_x$, the $\sigma_z$ component of 
$f$ is annihilated by $\mathrm{ad}_{H_Q}$ while the transverse part 
precesses. A direct induction gives
%
\begin{equation}
    f_n \equiv \mathrm{ad}_{H_Q}^n(f) = \begin{cases}
        c\,\sigma_z - s\,\sigma_x & n = 0, \\[4pt]
        -s\,\omega_q^n\,\sigma_x  & n \geq 1,\; n\;\text{even}, \\[4pt]
        -is\,\omega_q^n\,\sigma_y & n \geq 1,\; n\;\text{odd}.
    \end{cases}
    \label{eq:fn_qubit}
\end{equation}
%
The adjoint chain therefore closes on $\mathrm{span}\{\sigma_x,\sigma_y,
\sigma_z\}$, satisfying condition (C1) We now evaluate each sector of products $f_nf_m$ using the Pauli 
identities $\sigma_j^2 = \mathbb{I}$ and $\sigma_j\sigma_k = 
\delta_{jk}\mathbb{I} + i\epsilon_{jkl}\sigma_l$. Dropping $\mathbb{I}$ terms throughout (they contribute only to the normalisation), the non-trivial contributions to each Pauli component can be summarized by cases. For indices $n,m$ yielding non-vanishing terms, we find:
\begin{equation}
    f_n f_m \cong \begin{cases}
        i s^2 \omega_q^{n+m} \sigma_z & n \in 2\mathbb{N}, m \in 2\mathbb{N}-1, \\
        -i s^2 \omega_q^{n+m} \sigma_z & n \in 2\mathbb{N}-1, m \in 2\mathbb{N}, \\
        c s \omega_q^m \sigma_x - s^2 \omega_q^m \sigma_z & n=0, m \in 2\mathbb{N}-1, \\
        c s \omega_q^n \sigma_x + s^2 \omega_q^n \sigma_z & n \in 2\mathbb{N}-1, m=0.
    \end{cases}
    \label{eq:fnfm_cases}
\end{equation}
Terms generating $\sigma_y$ are omitted as they vanish by the symmetry $C_{nm}=C_{mn}$. The $\sigma_z$ contributions from the $n,m \ge 1$ sector also vanish by this symmetry, as the cases $n$ even, $m$ odd and $n$ odd, $m$ even enter with opposite signs. The only $\sigma_z$ contributions then come from the $n=0$ cross terms. This confirms condition (C2) is also satisfied.

Inserting these into Eq.\eqref{eq:Delta_final}, we now enumerate each component of $\Delta$. For the $\sigma_z$ component, we have
\begin{equation}
    \Delta\big|_{\sigma_z} = -2s^2\sum_{m=0}^\infty
      C_{0,2m+1}\,\omega_q^{2m+1}\,\sigma_z,
\end{equation}
where the symmetry $C_{0m}=C_{m0}$ has again been employed. Similarly for the $\sigma_x$ component, we obtain 
\begin{equation}
    \Delta\big|_{\sigma_x} = 2cs\sum_{m=0}^\infty
      C_{0,2m+1}\,\omega_q^{2m+1}\,\sigma_x.
\end{equation}
From this we find that the non-trivial operator content of $\Delta(\beta)$ is controlled by the single sum
\begin{equation}
    \begin{split}
    &\sum_{m=0}^\infty C_{0,2m+1}\,\omega_q^{2m+1} \\
    &\quad= \frac{1}{2}\int_0^\beta d\tau\int_0^\beta d\tau'\,
      K(\tau-\tau')\sum_{m=0}^\infty
      \frac{(\omega_q\tau')^{2m+1}}{(2m+1)!} \\
    &\quad= \frac{1}{2}\int_0^\beta d\tau\int_0^\beta d\tau'\,
      K(\tau-\tau')\sinh(\omega_q\tau'),
    \end{split}
\end{equation}

From this we may define the \emph{effective kernel}:
\begin{equation}
    \mathcal{K}(\beta) \equiv \int_0^\beta d\tau\int_0^\beta d\tau'\,
    K(\tau-\tau')\sinh(\omega_q\tau').
\end{equation}
Using the periodicity of the kernel $K(\tau)=K(\tau+\beta)$, the integral over the first variable $\tau$ decouples from $\tau'$:
\begin{equation}
    \int_0^\beta d\tau\, K(\tau-\tau') = \int_{-\tau'}^{\beta-\tau'} du\, K(u) = \int_0^\beta du\, K(u).
\end{equation}
The remaining integral over $\sinh(\omega_q\tau')$ can be performed directly. Identifying the integrated kernel strength with the reorganization energy via Eq.~\eqref{eq:int_K_2lambda}, $\int_0^\beta d\tau K(\tau) = 2\lambda$, we obtain:
\begin{equation}
    \mathcal{K}(\beta) = 2\lambda \int_0^\beta d\tau' \sinh(\omega_q\tau') 
    = \frac{4\lambda}{\omega_q}\sinh^2\left(\frac{\beta\omega_q}{2}\right),
    \label{eq:Kscalar}
\end{equation}
which leads to the a compact final form for the influence exponent:
\begin{equation}
    \begin{split}
    \Delta(\beta) &= \mathcal{K}(\beta)
    \bigl(cs\,\sigma_x - s^2\,\sigma_z\bigr) \\
    &= s\,\mathcal{K}(\beta)
    \bigl(c\,\sigma_x - s\,\sigma_z\bigr).
    \end{split}
    \label{eq:Delta_final}
\end{equation}
Amusingly, we see that the form of $\Delta(\beta)$ lies orthogonal to the original coupling $f = c\sigma_z - s\sigma_x$ in the $xz$-plane.

The final step is to recombine $\Delta(\beta)$ with $-\beta H_Q$ to obtain the mean-force Hamiltonian. This generically requires condition (C3), but in this case our task is simplified greatly as both $-\beta H_Q$ and $\Delta(\beta)$ are elements of 
$\mathfrak{su}(2)$. Thus their combined exponential is given by
\begin{equation}
    \begin{split}
    \bar{\rho}_Q(\beta) &= e^{-\beta H_Q}e^{\Delta(\beta)} \\
    &= e^{\vec{a}\cdot\vec{\sigma}}
      e^{\vec{b}\cdot\vec{\sigma}}
    = e^{\vec{\nu}\cdot\vec{\sigma}}
    \end{split}
\end{equation}   
We apply this by setting $\vec{a} = 
-\frac{\beta\omega_q}{2}\hat{z}$ and $\vec{b} \equiv \delta_x\hat{x} + \delta_z\hat{z}$, with $\delta_x = cs\,\mathcal{K}$ and $\delta_z = -s^2\,\mathcal{K}$ This then
points in the fixed direction $c\hat{x} - s\hat{z}$ with magnitude
\begin{equation}
     |\vec{b}| 
    = \sqrt{\delta_x^2+\delta_z^2} 
    = s\,\mathcal{K}(\beta).
    \label{eq:OmegaDelta}
\end{equation}
The composition rule for $\vec{\nu}$ is
\begin{align}
    \cos|\vec{\nu}| &= 
        \cos|\vec{a}|\cos|\vec{b}|
        - \hat{a}\cdot\hat{b}\,
          \sin|\vec{a}|\sin|\vec{b}|,
    \label{eq:su2_cos}\\
    \frac{\vec{\nu}}{|\vec{\nu}|} &= 
    \frac{1}{\sin|\vec{\nu}|}
    \Bigl[
        \sin|\vec{a}|\cos|\vec{b}|\,\hat{a}
        + \cos|\vec{a}|\sin|\vec{b}|\,\hat{b} \notag\\
        &\qquad\qquad + \sin|\vec{a}|\sin|\vec{b}|\,
          (\hat{a}\times\hat{b})
    \Bigr].
    \label{eq:su2_vec}
\end{align}
%
With $\hat{a} = -\hat{z}$, $\hat{b} = c\hat{x}-s\hat{z}$, we have 
$\hat{a}\cdot\hat{b} = s$ and
%
\begin{equation}
    \hat{a}\times\hat{b} = -\hat{z}\times(c\hat{x}-s\hat{z}) 
    = -c\,\hat{y},
\end{equation}
%
so a $\sigma_y$ component is generated by the noncommutativity of 
$H_Q$ and $\Delta$. Substituting $|\vec{a}| = 
\beta\omega_q/2$ and $|\vec{b}| = s\mathcal{K}$, \eqref{eq:su2_cos} gives
%
\begin{equation}
    \cos\Omega \equiv \cos\tfrac{\beta\omega_q}{2}\cos(s\mathcal{K})
    + s\sin\tfrac{\beta\omega_q}{2}\sin(s\mathcal{K}),
    \label{eq:Omega_def}
\end{equation}
%
which defines $\Omega \equiv |\vec{\nu}|$. The resulting vector $\vec{\nu}$ has a nonzero $y$-component \eqref{eq:su2_vec}, indicating that $H_\mathrm{MF}$ contains a term $\propto \sigma_y$. However, the structure of \eqref{eq:su2_vec} allows this to be removed by a simple frame rotation. 

Specifically, the transverse part of $\vec{\nu}$ lies at an angle $-\beta\omega_q/2$ in the $xy$-plane. We may therefore eliminate $\nu_y$ by applying the unitary transformation $U_0 = e^{-i\frac{\beta\omega_q}{4}\sigma_z}$, which rotates the frame by an angle $\phi = \beta\omega_q/2$ about the $z$-axis. In this rotated frame, the mean-force Hamiltonian $H'_\mathrm{MF} = U_0 H_\mathrm{MF} U_0^\dagger$ is given simply by
\begin{equation}
    H_\mathrm{MF}(\beta) = h_x(\beta)\,\sigma_x + h_z(\beta)\,\sigma_z,
    \label{eq:HMF_rotated}
\end{equation}
with components
\begin{align}
    h_x &= c\,\frac{\Omega}{\sin\Omega}\sin(s\mathcal{K}),
    \label{eq:nux_prime}\\
    h_z &= \frac{\Omega}{\sin\Omega}
    \Bigl[
        -\sin\tfrac{\beta\omega_q}{2}\cos(s\mathcal{K})
        -s\cos\tfrac{\beta\omega_q}{2}\sin(s\mathcal{K})
    \Bigr].
    \label{eq:nuz_prime}
\end{align}

It has taken an awfully large amount of work for such an awfully small description, but finally we have it: the complete and exact influence of the environment on the qubit at all coupling strengths and temperatures, captured entirely by the effective kernel $\mathcal{K}(\beta)$ and the mixing angle $\theta$ (via $c,s$). 

This form allows for a direct analysis of the coupling limits. The coupling strength $g$ enters exclusively through the reorganization energy $\lambda \propto g^2$, which in turn scales the effective kernel $\mathcal{K}(\beta) \propto g^2$.
\begin{enumerate}
    \item \textbf{Weak Coupling ($\lambda\to 0$):} In this limit $\mathcal{K}\to 0$, implying $s\mathcal{K} \to 0$. We immediately have $\nu_x \to 0$ and $\nu_z \to -\frac{\beta\omega_q}{2}$. Thus $H_\mathrm{MF} \to \frac{\omega_q}{2}\sigma_z = H_Q$, recovering the bare system Hamiltonian as expected.
    
    \item \textbf{Ultrastrong Coupling ($\lambda\to \infty$):} Here $s\mathcal{K} \propto g^2 \gg 1$. In this regime the bath term dominates the free evolution phase $\beta\omega_q$, so $\Omega \approx s\mathcal{K}$. The vector components simplify to:
    \begin{align}
        \nu_x &\approx c\,\frac{s\mathcal{K}}{\sin(s\mathcal{K})}\sin(s\mathcal{K}) = c\,s\mathcal{K}, \\
        \nu_z &\approx -s\,\frac{s\mathcal{K}}{\sin(s\mathcal{K})}\sin(s\mathcal{K}) = -s^2\mathcal{K}.
    \end{align}
    The mean-force Hamiltonian becomes
    \begin{equation}
        H_\mathrm{MF} \approx -\frac{s\mathcal{K}}{\beta}(c\sigma_x - s\sigma_z) = \frac{s\mathcal{K}}{\beta} f.
    \end{equation}
    This result is physically transparent: at ultrastrong coupling, the mean-force potential is simply the system-bath coupling operator $f$ scaled by the reorganization energy directly.
\end{enumerate}

\subsection{Numerical Demonstrations}

To validate the analytic derivation of the mean-force Hamiltonian $H_\mathrm{MF}$, we compare our results against exact numerical diagonalization (ED) of the full system-bath Hamiltonian. We model the bath as a set of discrete harmonic oscillators with an Ohmic spectral density, using $N=4$ modes and a Fock space cutoff of $N_{cut}=6$ per mode to ensure convergence. The full Hamiltonian is diagonalized to obtain the exact thermal state $\rho_{tot} = e^{-\beta H_{tot}}/Z$, from which the reduced system state $\rho_S = \mathrm{Tr}_B[\rho_{tot}]$ is computed.

In Fig.~\ref{fig:hmf_v4_validation}, we show the trace distance $D(\rho_{ex}, \rho_{HMF}) = \frac{1}{2}\mathrm{Tr}|\rho_{ex} - \rho_{HMF}|$ between the exact reduced state and the state generated by our analytic mean-force Hamiltonian. The agreement is excellent, with the distance remaining below $10^{-5}$ across the entire range of coupling strengths $\lambda \in [0, 4]$.

Furthermore, we extract the effective fields $h_x, h_z$ from the numerical state by projecting onto the Pauli basis. As predicted by our theory, the effective field vector rotates non-trivially in the $xz$-plane as the coupling increases. The analytic predictions (solid lines) perfectly match the numerical data (dashed lines), confirming that the simple expression for $\mathcal{K}(\beta)$ captures the full non-perturbative influence of the environment.

\begin{figure}[t]
    \centering
    \includegraphics[width=\columnwidth]{figures/hmf_v4_validation_fig.png}
    \caption{Validation of the analytic mean-force Hamiltonian against exact diagonalization. (Left) The trace distance between the exact reduced state and the analytic prediction remains negligible ($< 10^{-5}$) for all coupling strengths $\lambda$. (Center) The components of the effective field $\vec{h}$ in the rotated frame show perfect agreement between theory (solid) and numerics (dashed). (Right) The unrotated $y$-component is non-zero in the lab frame but vanishes in the rotated frame as predicted.}
    \label{fig:hmf_v4_validation}
\end{figure}