\section{Exact Solution of the Spin-Boson Model}

We now turn to applying the results developed in the previous section. A particularly instructive example concerns spins, as the $\mathfrak{su}(2)$ algebra provides a particularly compact analytic solution. We demonstrate this by applying it to a transverse-coupling spin-boson model. Following  Ref.~\cite{cresserWeakUltrastrongCoupling2021a}, let
%
\begin{equation}
    H_Q = \frac{\omega_q}{2}\sigma_z, 
    \qquad 
    f = \cos\theta\,\sigma_z - \sin\theta\,\sigma_x,
    \label{eq:qubit_setup}
\end{equation}
%
with $c \equiv \cos\theta$, $s \equiv \sin\theta$ throughout. The 
coupling mixes a commuting part $c\sigma_z$ with a transverse part 
$-s\sigma_x$.

Using $[H_Q,\sigma_x] = i\omega_q\sigma_y$ and 
$[H_Q,\sigma_y] = -i\omega_q\sigma_x$, the $\sigma_z$ component of 
$f$ is annihilated by $\mathrm{ad}_{H_Q}$ while the transverse part 
precesses. A direct induction gives
%
\begin{equation}
    f_n \equiv \mathrm{ad}_{H_Q}^n(f) = \begin{cases}
        c\,\sigma_z - s\,\sigma_x & n = 0, \\[4pt]
        -s\,\omega_q^n\,\sigma_x  & n \geq 1,\; n\;\text{even}, \\[4pt]
        -is\,\omega_q^n\,\sigma_y & n \geq 1,\; n\;\text{odd}.
    \end{cases}
    \label{eq:fn_qubit}
\end{equation}
%
The adjoint chain therefore closes on $\mathrm{span}\{\sigma_x,\sigma_y,
\sigma_z\}$, satisfying condition (C1).

We use $f_n$ to evaluate the influence exponent $\Delta(\beta)$. To streamline the algebra, we treat each $f_n$ as a Pauli vector:
\begin{equation}
    f_n = \mathbf v_n \cdot \boldsymbol{\sigma},
    \qquad 
    \boldsymbol{\sigma}=(\sigma_x,\sigma_y,\sigma_z),
    \label{eq:vn_def}
\end{equation}
with
\begin{equation}
    \mathbf v_n =
    \begin{cases}
        \mathbf v_0 = (-s,\,0,\,c), & n=0, \\
        (-s\,\omega_q^{n},\,0,\,0), & n\ge 1,\; n\ \text{even},\\[2pt]
        (0,\,-is\,\omega_q^{n},\,0), & n\ge 1,\; n\ \text{odd}.
    \end{cases}
    \label{eq:vn_qubit}
\end{equation}
The Pauli multiplication rule then becomes the single identity
\begin{equation}
    (\mathbf a\cdot\boldsymbol{\sigma})(\mathbf b\cdot\boldsymbol{\sigma})
    = (\mathbf a\cdot\mathbf b)\,\mathbb I + i(\mathbf a\times\mathbf b)\cdot\boldsymbol{\sigma},
    \label{eq:pauli_vector_product}
\end{equation}
so that
\begin{equation}
    f_n f_m
    = (\mathbf v_n\cdot\mathbf v_m)\,\mathbb I
    + i(\mathbf v_n\times\mathbf v_m)\cdot\boldsymbol{\sigma}.
    \label{eq:fnfm_vector}
\end{equation}
The scalar part contributes only to the overall normalization (free-energy shift) and may be discarded when determining the operator structure.  The non-trivial Pauli sector is therefore controlled entirely by the cross products $\mathbf v_n\times \mathbf v_m$.

Given this, we introduce the symmetric and antisymmetric parts of the ordered kernel moments:
\begin{equation}
    S_{nm}:=\frac{1}{2}\big(C^{>}_{nm}+C^{>}_{mn}\big),
    \quad 
    A_{nm}:=\frac{1}{2}\big(C^{>}_{nm}-C^{>}_{mn}\big),
    \label{eq:Snm_Anm_def}
\end{equation}
so that $C^{>}_{nm}=S_{nm}+A_{nm}$ with $S_{nm}=S_{mn}$ and $A_{nm}=-A_{mn}$.
Since $\mathbf v_n\times\mathbf v_m$ is antisymmetric under $n\leftrightarrow m$, only the antisymmetric sector of the coefficients contributes to the Pauli part of $\Delta$. We may therefore restrict the sum to ordered indices $n>m$ to avoid double counting:
\begin{equation}
\begin{split}
    \Delta(\beta) 
    &\cong i\sum_{n,m\ge 0} A_{nm}\,(\mathbf v_n\times\mathbf v_m)\cdot\boldsymbol{\sigma} \\
    &= 2i\sum_{n>m} A_{nm}\,(\mathbf v_n\times\mathbf v_m)\cdot\boldsymbol{\sigma},
\end{split}
\label{eq:Delta_Pauli_Anm}
\end{equation}
where $\cong$ indicates equality up to an irrelevant $\mathbb I$ contribution. This makes explicit that the non-commuting structure in $\Delta$ is entirely inherited from the ordered nature of $C^{>}_{nm}$.

For $n,m\ge 1$, only even-odd index pairs contribute and the cross product is always parallel to $\hat{\mathbf z}$. This yields a purely $\sigma_z$ contribution from the $n,m\ge 1$ block. The special $n=0$ layer contributes additional terms. In particular, for $\ell\ge 0$,
\begin{equation}
    \begin{split}
    \mathbf v_0\times \mathbf v_{2\ell+1} &= \big(i c s\,\omega_q^{2\ell+1},\,0,\, i s^2\,\omega_q^{2\ell+1}\big), \\
    \mathbf v_0\times \mathbf v_{2\ell} &= \big(0,\,-c s\,\omega_q^{2\ell},\,0\big),
    \end{split}
    \label{eq:v0_cross_parity}
\end{equation}
so that both $\sigma_x$ and $\sigma_z$ arise from the $0$--odd sector, while a $\sigma_y$ sector is generated by the $0$--even sector whenever the ordered coefficients have a non-vanishing antisymmetric part $A_{0,2\ell}$.
Collecting these contributions, the Pauli sector of $\Delta$ may be written in the general Bloch form
\begin{equation}
    \Delta(\beta)\cong \Delta_x(\beta)\,\sigma_x+\Delta_y(\beta)\,\sigma_y+\Delta_z(\beta)\,\sigma_z,
    \label{eq:Delta_xyz_form}
\end{equation}

\begin{align}
    \Delta_x(\beta)
    &= -2cs\sum_{\ell\ge 0} A_{2\ell+1,\, 0}(\beta)\,\omega_q^{2\ell+1},
    \label{eq:Deltax_series_corr}\\[4pt]
    \Delta_y(\beta)
    &= -2ics\sum_{\ell\ge 1} A_{2\ell,\, 0}(\beta)\,\omega_q^{2\ell},
    \label{eq:Deltay_series_corr}\\[4pt]
    \Delta_z(\beta)
    &= -2s^2\sum_{k\ge 1,\,\ell\ge 0} A_{2k,\,2\ell+1}(\beta)\,\omega_q^{2k+2\ell+1} \notag\\
    &\quad\, -2s^2\sum_{\ell\ge 0} A_{2\ell+1,\, 0}(\beta)\,\omega_q^{2\ell+1}.
    \label{eq:Deltaz_series_corr}
\end{align}
The first term in \eqref{eq:Deltaz_series_corr} comes from the $n,m\ge 1$ even--odd sector, and the remaining terms from the $n=0$ cross layer.

\subsection{Generating function for the influence exponent}

While technically complete, this series representation of $\Delta$ is not particularly illuminating. A more transparent expression for the components - with a correspondingly more robust physical interpretation - is obtained by recognising that the ordered moments $C^{>}_{nm}$ can be interpreted as Laplace transforms of the imaginary-time kernel. To this end, we introduce the bivariate generating function
\begin{equation}
    \mathcal{G}^>(x,y)
    =
    \int_0^\beta d\tau \int_0^\tau d\tau'\,
    K(\tau-\tau')\,e^{x\tau+y\tau'},
    \label{eq:Gxy_def}
\end{equation}
and its anti-ordered counterpart $\mathcal{G}^<(x,y) = \mathcal{G}^>(y,x)$, where the equality follows from $K(-u)=K(u)$. Structurally, these objects are the time-ordered and anti-ordered Green's functions of the bath projected onto the Laplace domain. They directly sample the bath's correlations at the frequencies $x$ and $y$ dictated by the system's eigenoperators. This identification immediately suggests an implicit fluctuation theorem, readily uncovered by application of the KMS condition $K(\beta-u)=K(u)$:
\begin{equation}
    \mathcal{G}^>(x,y) = e^{\beta(x+y)}\,\mathcal{G}^>(-y,-x).
    \label{eq:KMS_Gxy}
\end{equation}
Changing variables to $u=\tau-\tau'$, $v=\tau'$ and using the KMS symmetry yields the closed form
\begin{equation}
    \mathcal{G}^>(x,y) = \frac{e^{x\beta}\tilde{K}(y) - \tilde{K}(x)}{x+y},
    \label{eq:Gxy_closed}
\end{equation}
where we have defined the bare Laplace transform of the kernel as
\begin{equation}
    \tilde{K}(\omega) \equiv \int_0^\beta du\, K(u)\,e^{\omega u}.
    \label{eq:Ktilde_def}
\end{equation}
The resonant case $x+y \to 0$ is recovered as the limit $\mathcal{G}^>(x,-x) = e^{x\beta}\tilde{K}'(-x)$, where pure derivatives with respect to frequency appear:
\begin{equation}
    \tilde{K}'(-x) \equiv \partial_{\omega}\tilde{K}(\omega)\big|_{\omega=-x} = \int_0^\beta du\, u\, K(u)\, e^{-xu}.
\end{equation}
Using the KMS symmetry, the resonant limit takes the explicit integral form
\begin{equation}
    \mathcal{G}^>(x,-x) = \int_0^\beta du\,(\beta-u)\,K(u)\,e^{xu}.
    \label{eq:Gxy_resonant_integral}
\end{equation}
These resonant terms introduce linear-in-$\beta$ (secular) and exponential-in-$\beta$ behaviors, which upon recombination generate the hyperbolic structure observed in the $\Delta_z$ component equations.

The effective Hamiltonian is determined by the antisymmetric combination
\begin{equation}
\begin{split}
    \mathcal{G}^-(x,y) 
    &\equiv \mathcal{G}^>(x,y) - \mathcal{G}^<(x,y) \\
    &= \frac{e^{x\beta}\tilde{K}(y) - e^{y\beta}\tilde{K}(x)
           + \tilde{K}(x) - \tilde{K}(y)}{x+y},
\end{split}
\label{eq:Gminus_closed}
\end{equation}
while the symmetric part $\mathcal{G}^+(x,y) \equiv \mathcal{G}^>(x,y) + \mathcal{G}^<(x,y)$ contributes only to the partition function normalisation. This splitting is physically natural: the Pauli algebra maps the antisymmetric operator product to the vector sector $\boldsymbol{\sigma}$, and the symmetric product to the scalar identity.

Because the qubit adjoint chain \eqref{eq:fn_qubit} has a strict even/odd parity structure, only the values $x,y\in\{0,\pm\omega_q\}$ enter the sum. Evaluating \eqref{eq:Gminus_closed} at these points and introducing the even and odd parts of $\tilde{K}$ at the Bohr frequency,
\begin{equation}
    \tilde{K}^\pm \equiv \tilde{K}(\omega_q) \pm \tilde{K}(-\omega_q),
    \label{eq:Ktilde_pm}
\end{equation}
the influence components reduce to
\begin{align}
    \Delta_x(\beta)
        &= \frac{cs}{\omega_q}\,\tilde{K}^-,
    \label{eq:Deltax_final}\\[4pt]
    \Delta_y(\beta)
        &= \frac{ics}{\omega_q}\,\tilde{K}^+,
    \label{eq:Deltay_final}\\[4pt]
    \Delta_z(\beta)
        &= \frac{s^2}{\omega_q}\,\tilde{K}^-
           + \frac{s^2}{2\omega_q}\Bigl[
               \tilde{K}^- \cosh(\omega_q\beta) - \tilde{K}^+ \sinh(\omega_q\beta)
             \Bigr].
    \label{eq:Deltaz_final}
\end{align}
The explicit dependence on the transition frequency $\omega_q$ identifies this as an `on-shell' renormalization: the bath's infinite complexity is distilled into its response at the system's natural energy scale.
This structure establishes a direct map to the standard non-equilibrium Green's function formalism. The odd component $\tilde{K}^-$ plays the role of a dissipative self-energy (analogous to the imaginary part of a retarded Green's function), driving the coherent renormalization of the Hamiltonian parameters $\Delta_x$ and $\Delta_z$. 
Conversely, the even component $\tilde{K}^+$ represents the integrated fluctuation spectrum (analogous to the Keldysh or 'greater' Green's function), explicitly capturing the thermal noise responsible for the $\Delta_y$ component. 
The precise balance between these terms is dictated by the KMS condition, which functions here as an on-shell Fluctuation-Dissipation Theorem, ensuring that the interplay of dissipation and fluctuations correctly targets the canonical equilibrium state.

\subsubsection*{Recombination}
The reduced equilibrium operator is $\bar\rho_Q(\beta)=e^{-\beta H_Q}e^{\Delta(\beta)}$, and the Hamiltonian of mean force is defined (up to an additive constant) by
\begin{equation}
    H_{\mathrm{MF}}(\beta) = -\beta^{-1}\log\bar\rho_Q(\beta).
    \label{eq:HMF_def_qubit}
\end{equation}
Because $\bar\rho_Q(\beta)$ is Hermitian and positive, $H_{\mathrm{MF}}(\beta)$ is itself Hermitian. Since the influence operator $\Delta(\beta)$ lies in the span of $\{\sigma_x,\sigma_z\}$ (after absorbing the $\Delta_y$ phase or rotating), so does $H_{\mathrm{MF}}(\beta)$. Thus one may generally write
\begin{equation}
    H_{\mathrm{MF}}(\beta) \equiv \mathrm{const}\cdot \mathbb I + h_x(\beta)\,\sigma_x + h_z(\beta)\,\sigma_z,
    \label{eq:HMF_xz_form}
\end{equation}
with the effective fields $h_{x,z}(\beta)$ obtained exactly by evaluating the $2\times 2$ logarithm in \eqref{eq:HMF_def_qubit}. 
A subsequent rotation about $\sigma_z$ can be used to align the transverse component purely along $\sigma_x$, yielding a manifestly real representation if desired.
In the following subsection we carry out this final step explicitly and discuss the resulting renormalisation of the qubit field direction as a function of temperature and bath spectrum.


\subsection{Numerical Demonstrations}
\label{sec:numerical_validation}

To validate the analytic derivation of the mean-force Hamiltonian $H_\mathrm{MF}$, we compare our results against exact numerical diagonalization (ED) of the full system-bath Hamiltonian. We model the bath as a set of discrete harmonic oscillators with an Ohmic spectral density, using $N=4$ modes and a Fock space cutoff of $N_{cut}=6$ per mode to ensure convergence. The full Hamiltonian is diagonalized to obtain the exact thermal state $\rho_{tot} = e^{-\beta H_{tot}}/Z$, from which the reduced system state $\rho_S = \mathrm{Tr}_B[\rho_{tot}]$ is computed.

In Fig.~\ref{fig:hmf_v4_validation}, we show the trace distance $D(\rho_{ex}, \rho_{HMF}) = \frac{1}{2}\mathrm{Tr}|\rho_{ex} - \rho_{HMF}|$ between the exact reduced state and the state generated by our analytic mean-force Hamiltonian. The agreement is excellent, with the distance remaining below $10^{-5}$ across the entire range of coupling strengths $\lambda \in [0, 4]$.

Furthermore, we extract the effective fields $h_x, h_z$ from the numerical state by projecting onto the Pauli basis. As predicted by our theory, the effective field vector rotates non-trivially in the $xz$-plane as the coupling increases. The analytic predictions (solid lines) perfectly match the numerical data (dashed lines), confirming that the simple expression for $\mathcal{K}(\beta)$ captures the full non-perturbative influence of the environment.

\begin{figure}[t]
    \centering
    % \includegraphics[width=\columnwidth]{figures/hmf_v4_validation_fig.png}
    \caption{Validation of the analytic mean-force Hamiltonian against exact diagonalization. (Left) The trace distance between the exact reduced state and the analytic prediction remains negligible ($< 10^{-5}$) for all coupling strengths $\lambda$. (Center) The components of the effective field $\vec{h}$ in the rotated frame show perfect agreement between theory (solid) and numerics (dashed). (Right) The unrotated $y$-component is non-zero in the lab frame but vanishes in the rotated frame as predicted.}
    \label{fig:hmf_v4_validation}
\end{figure}
