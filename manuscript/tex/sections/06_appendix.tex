\section{Quenched Density and Imaginary-Time Evolution}
\label{sec:appendix_quenched}

This appendix makes explicit the operator identity underlying the quenched
imaginary-time formulation used in Sec.~\ref{sec:theory}. Define a
time-ordered exponential with a (generally) $\tau$-dependent operator
$H(\tau)$,
\begin{equation}
    \begin{split}
        \rho(\tau) &\equiv \mathcal{T}_\tau
        \exp\left(-\int_0^\tau d\tau' \, H(\tau')\right), \\
        \rho(0) &= \mathbb{I}.
    \end{split}
    \label{eq:appendix_rho_def}
\end{equation}
Standard differentiation identities for ordered exponentials imply
\begin{equation}
    -\partial_\tau \rho(\tau) = H(\tau)\rho(\tau),
    \label{eq:appendix_imag_time_eq}
\end{equation}
with the ordering built into $\rho(\tau)$; see, e.g.,
Refs.~\cite{wilcoxExponentialOperatorsParameter1967a,magnusExponentialSolutionDifferential1954a,blanesMagnusExpansionIts2009}
for operator calculus and time-ordered exponentials.

In the present context one takes
\begin{equation}
    H(\tau) = H_Q + \xi(\tau) f,
    \label{eq:appendix_quenched_H}
\end{equation}
where the Gaussian field satisfies
\begin{equation}
    \langle \xi(\tau)\xi(\tau')\rangle = K(\tau-\tau').
    \label{eq:appendix_xi_cov}
\end{equation}
The reduced equilibrium operator can be written as
\begin{equation}
    \bar{\rho}_S =
    e^{-\beta H_Q}\left\langle \rho(\beta) \right\rangle_\xi,
    \label{eq:appendix_rho_beta}
\end{equation}
which is the compact operator form of the quenched Gaussian representation
used in the main text.

To connect Eq.~\eqref{eq:appendix_rho_beta} to an imaginary-time path integral,
one discretizes $\tau \in [0,\beta]$, applies a Trotter (or Zassenhaus)
factorization, and inserts resolutions of identity in the system coordinate
basis. This yields the standard Euclidean path-integral expression for the
canonical density operator, with the $\tau$-dependent potential induced by the
auxiliary field, as in the influence-functional derivation for quadratic baths
and their stochastic unravellings
\cite{feynmanTheoryGeneralQuantum1963a,caldeiraQuantumTunnellingDissipative1983a,grabertQuantumBrownianMotion1988,moixEquilibriumreducedDensityMatrix2012,chenRigorousStochasticMatrix2014}.

\section{Kernel Symmetry and Moment Relations}
\label{sec:appendix_kernel}

The bath kernel appearing in the bilocal influence functional is
\begin{equation}
    \begin{split}
        K(\tau-\tau') &=
        \mathrm{Tr}_B\!\left[\mathcal{T}_\tau \tilde{B}(\tau)\tilde{B}(\tau')\rho_B\right], \\
        \rho_B &= \frac{e^{-\beta H_B}}{Z_B}.
    \end{split}
    \label{eq:appendix_kernel_def}
\end{equation}
For equilibrium baths, $K$ depends only on the imaginary-time difference and is
even under exchange of its arguments, implying
\begin{equation}
    K(\tau-\tau') = K(\tau'-\tau), \qquad
    \mu_{nm} = \mu_{mn},
    \label{eq:appendix_kernel_sym}
\end{equation}
where $\mu_{nm}$ are the kernel moments defined in
Eq.~\eqref{eq:kernel_moments}. In common quadratic-bath models the kernel is
explicitly constructed from the bath spectral density and satisfies the
Kubo-Martin-Schwinger periodicity in imaginary time, which can be used to
re-express moment integrals in equivalent forms; see
Refs.~\cite{grabertQuantumBrownianMotion1988,tanimuraReducedHierarchicalEquations2014,songCalculationCorrelatedInitial2015}
for explicit constructions.

The moment expansion used in Sec.~\ref{sec:theory} requires only these symmetry
properties and the existence of the integrals defining $\mu_{nm}$. No further
approximation is introduced at this stage.

