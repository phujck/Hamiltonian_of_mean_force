\section{Quenched Density and Imaginary-Time Evolution}
\label{sec:appendix_quenched}

This appendix makes explicit the operator identity underlying the quenched
imaginary-time formulation used in Sec.~\ref{sec:model}. Define a
time-ordered exponential with a (generally) $\tau$-dependent operator
$H(\tau)$,
\begin{equation}
    \begin{split}
        \rho(\tau) &\equiv \mathcal{T}_\tau
        \exp\left(-\int_0^\tau d\tau' \, H(\tau')\right), \\
        \rho(0) &= \mathbb{I}.
    \end{split}
    \label{eq:appendix_rho_def}
\end{equation}
Standard differentiation identities for ordered exponentials imply
\begin{equation}
    -\partial_\tau \rho(\tau) = H(\tau)\rho(\tau),
    \label{eq:appendix_imag_time_eq}
\end{equation}
with the ordering built into $\rho(\tau)$; see, e.g.,
Refs.~\cite{wilcoxExponentialOperatorsParameter1967a,magnusExponentialSolutionDifferential1954a,blanesMagnusExpansionIts2009}
for operator calculus and time-ordered exponentials.

In the present context one takes
\begin{equation}
    H(\tau) = H_Q + \xi(\tau) f,
    \label{eq:appendix_quenched_H}
\end{equation}
where the Gaussian field satisfies
\begin{equation}
    \langle \xi(\tau)\xi(\tau')\rangle = K(\tau-\tau').
    \label{eq:appendix_xi_cov}
\end{equation}
The reduced equilibrium operator can be written as
\begin{equation}
    \bar{\rho}_S =
    e^{-\beta H_Q}\left\langle \rho(\beta) \right\rangle_\xi,
    \label{eq:appendix_rho_beta}
\end{equation}
which is the compact operator form of the quenched Gaussian representation
used in the main text.

To connect Eq.~\eqref{eq:appendix_rho_beta} to an imaginary-time path integral,
one discretizes $\tau \in [0,\beta]$, applies a Trotter (or Zassenhaus)
factorization, and inserts resolutions of identity in the system coordinate
basis. This yields the standard Euclidean path-integral expression for the
canonical density operator, with the $\tau$-dependent potential induced by the
auxiliary field, as in the influence-functional derivation for quadratic baths
and their stochastic unravellings
\cite{feynmanTheoryGeneralQuantum1963a,caldeiraQuantumTunnellingDissipative1983a,grabertQuantumBrownianMotion1988,moixEquilibriumreducedDensityMatrix2012,chenRigorousStochasticMatrix2014}.

\section{Kernel Symmetry and Moment Relations}
\label{sec:appendix_kernel}

The bath kernel appearing in the bilocal influence functional is
\begin{equation}
    \begin{split}
        K(\tau-\tau') &=
        \mathrm{Tr}_B\!\left[\mathcal{T}_\tau \tilde{B}(\tau)\tilde{B}(\tau')\rho_B\right], \\
        \rho_B &= \frac{e^{-\beta H_X}}{Z_B}.
    \end{split}
    \label{eq:appendix_kernel_def}
\end{equation}
For equilibrium baths, $K$ depends only on the imaginary-time difference and is
even under exchange of its arguments, implying
\begin{equation}
    K(\tau-\tau') = K(\tau'-\tau), \qquad
    \mu_{nm} = \mu_{mn},
    \label{eq:appendix_kernel_sym}
\end{equation}
\begin{equation}
    \mu_{nm} = \frac{1}{n!m!} \int_0^\beta d\tau \int_0^\beta d\tau' \tau^n (\tau')^m K(\tau-\tau').
    \label{eq:kernel_moments_def}
\end{equation}
In common quadratic-bath models the kernel is
explicitly constructed from the bath spectral density and satisfies the
Kubo-Martin-Schwinger periodicity in imaginary time, which can be used to
re-express moment integrals in equivalent forms; see
Refs.~\cite{grabertQuantumBrownianMotion1988,tanimuraReducedHierarchicalEquations2014,songCalculationCorrelatedInitial2015}
for explicit constructions.

The moment expansion used in Sec.~\ref{sec:model} requires only these symmetry
properties and the existence of the integrals defining $\mu_{nm}$. No further
approximation is introduced at this stage.

\section{Derivation of the Influence Functional\label{app:influence_derivation}}

In this appendix, we construct the influence functional formalism used in Sec.~\ref{sec:quenched}. Our goal is to derive the exact form of the reduced density operator $\bar{\rho}_S(\beta)$ by explicitly integrating out the harmonic bath, and to demonstrate that this leads directly to the stochastic unravelling employed in the main text.

\subsection{Euclidean Path Integral Setup}
We begin with the definition of the unnormalized reduced state,
\begin{equation}
    \bar{\rho}_S(\beta) = \Tr_X \left[ e^{-\beta H_{\mathrm{tot}}} \right].
\end{equation}
This trace can be represented as a Euclidean path integral. Let $|q\rangle$ and $|x\rangle = |x_1, x_2, \dots\rangle$ denote the position bases for the system and bath, respectively. The matrix element $\langle q | \bar{\rho}_S | q' \rangle$ involves a sum over all periodic paths $x(\tau)$ (where $x(0)=x(\beta)$) and open paths $q(\tau)$ (where $q(0)=q'$ and $q(\beta)=q$):
\begin{equation}
\begin{split}
    \langle q | \bar{\rho}_S | q' \rangle = \int_{q(0)=q'}^{q(\beta)=q} &\mathcal{D}q(\tau) e^{-S_Q[q]/\hbar} \\
    &\times \prod_k Z_k[q],
\end{split}
    \label{eq:app_path_integral_start}
\end{equation}
where $S_Q$ is the Euclidean action of the isolated system, and $Z_k[q]$ is the partition function of the $k$-th oscillator in the presence of the external driving force $J_k(\tau) = -c_k f(q(\tau))$:
\begin{equation}
\begin{split}
    Z_k[q] = \oint & \mathcal{D}x_k(\tau) \exp\bigg( -\frac{1}{\hbar} \int_0^\beta d\tau \\
    &\times \left[ \frac{m_k}{2} \dot{x}_k^2 + \frac{m_k \omega_k^2}{2} x_k^2 + c_k x_k f(q(\tau)) \right] \bigg).
\end{split}
\end{equation}
Note that the periodicity of the trace implies periodic boundary conditions for the bath paths $x_k(\tau)$.

\subsection{Gaussian Integration}
The functional integral for $Z_k[q]$ is Gaussian and can be evaluated exactly. It corresponds to the partition function of a forced harmonic oscillator. The result is expressible as the product of the free oscillator partition function, $Z_X^{(k)} = (2\sinh(\beta\hbar\omega_k/2))^{-1}$, and an exponential "influence phase" depending quadratically on the drive~\cite{feynmanTheoryGeneralQuantum1963a,weissQuantumDissipativeSystems2012}:
\begin{equation}
\begin{split}
    Z_k[q] = Z_X^{(k)} \exp\bigg( &\frac{1}{2\hbar} \int_0^\beta d\tau \int_0^\beta d\tau' \\
    &\times K_k(\tau-\tau') f(q(\tau)) f(q(\tau')) \bigg).
\end{split}
\end{equation}
The kernel $K_k(\tau)$ is the equilibrium autocorrelation function of the coordinate $x_k$:
\begin{equation}
    K_k(\tau-\tau') = c_k^2 \langle \mathcal{T}_\tau x_k(\tau) x_k(\tau') \rangle_0.
\end{equation}
Summing over all modes $k$, the total influence functional is $\prod_k Z_k[q] = Z_X \exp( \Phi_{inf}[q] )$, with
\begin{equation}
\begin{split}
    \Phi_{inf}[q] = \frac{1}{2\hbar} \int_0^\beta d\tau &\int_0^\beta d\tau' K(\tau-\tau') \\
    &\times f(q(\tau)) f(q(\tau')),
\end{split}
\end{equation}
where $K(\tau) = \sum_k K_k(\tau)$ is the total force autocorrelation function.

\subsection{From Non-local Action to Stochastic Average}
Substituting this back into Eq.~\eqref{eq:app_path_integral_start}, the reduced density matrix becomes
\begin{equation}
    \bar{\rho}_S = Z_X \int \mathcal{D}q \, e^{-S_Q[q]/\hbar} \exp\left( \Phi_{inf}[q] \right).
\end{equation}
The term $\Phi_{inf}[q]$ is non-local in imaginary time, representing a self-interaction of the system mediated by the bath. To disentangle this, we use the Hubbard-Stratonovich transformation (the continuous analog of the Gaussian identity $e^{\frac{1}{2} A^2} \sim \int d\xi e^{-\frac{1}{2}\xi^2 + \xi A}$). We introduce a real, auxiliary stochastic field $\xi(\tau)$ with zero mean and covariance
\begin{equation}
    \langle \xi(\tau) \xi(\tau') \rangle_\xi = K(\tau-\tau').
\end{equation}
Using this field, we can rewrite the influence exponential as a stochastic average:
\begin{equation}
\begin{split}
    \exp\left( \Phi_{inf}[q] \right) = \bigg\langle \exp\bigg( &\frac{1}{\hbar} \int_0^\beta d\tau \\
    &\times \xi(\tau) f(q(\tau)) \bigg) \bigg\rangle_\xi.
\end{split}
\end{equation}
Inserting this identity into the path integral for $\bar{\rho}_S$, we can swap the order of the path integration over $q$ and the stochastic average over $\xi$:
\begin{equation}
    \bar{\rho}_S = Z_X \left\langle \int \mathcal{D}q \, \exp\left[-\frac{1}{\hbar} S_Q[q] + \frac{1}{\hbar} \int_0^\beta \xi f \right] \right\rangle_\xi.
\end{equation}
The term in the angle brackets is exactly the path integral for a system evolving under the time-dependent Hamiltonian $H(\tau) = H_Q - \xi(\tau)f$. Thus, in operator language, we arrive at the exact stochastic representation:
\begin{equation}
\begin{split}
    \bar{\rho}_S(\beta) = Z_X \bigg\langle \mathcal{T}_\tau \exp\bigg( &-\int_0^\beta d\tau \\
    &\times [H_Q - \xi(\tau)f] \bigg) \bigg\rangle_\xi.
\end{split}
    \label{eq:app_stochastic_final}
\end{equation}
This confirms that the HMF can be constructed by averaging the non-unitary (imaginary-time) evolution of the system driven by colored Gaussian noise.

\section{Exact BCH resummation for the qubit: closing the operator tower}
\label{app:bch_qubit}

Section~\ref{sec:closure} establishes that $H_{\mathrm{MF}}(\beta)$ admits a closed-form
representation whenever conditions (C1)--(C3) are satisfied, and notes that the BCH
commutator tower generated by $[{\Delta},\mathrm{ad}_{H_Q}^k\Delta]$ \emph{stays in
the algebra} but defers the demonstration to ``all BCH commutators remain in
$\mathcal A$ by construction.'' This appendix makes that remark explicit for the
qubit case: we compute every nested commutator that appears in the BCH series
directly to show that the su$(2)$ algebra closes the tower at \emph{every} order,
and then resum the entire series exactly using a single Pauli identity.

\subsection*{Decomposition of the influence operator}

Write the influence operator \eqref{eq:Delta_sigma_pm_v5} as
\begin{equation}
    \Delta(\beta) = \Delta_0\,\mathbb{I} + M,
    \qquad
    M \equiv \Delta_z\sigma_z + \Sigma_+\sigma_+ + \Sigma_-\sigma_-,
    \label{eq:app_Delta_decomp}
\end{equation}
separating the scalar (free-energy) part $\Delta_0$ from the traceless
$\mathfrak{su}(2)$ element $M$.  All the physically non-trivial content lives in
$M$.  We have $[H_Q, \mathbb{I}]=0$, so the BCH commutator tower acts only on $M$:
the chain $\mathrm{ad}_{H_Q}^k\Delta = \mathrm{ad}_{H_Q}^k M$ for $k\ge1$, and
$[\Delta,\mathrm{ad}_{H_Q}^k\Delta]=[M,\mathrm{ad}_{H_Q}^k M]$.

\subsection*{Step 1: The nilpotency condition $M^2 = \chi^2\mathbb{I}$}

Using the standard anticommutation and multiplication rules\footnote{%
$\{\sigma_z,\sigma_\pm\}=\sigma_z\sigma_\pm+\sigma_\pm\sigma_z=0$,\;
$\sigma_\pm^2=0$,\;
$\sigma_+\sigma_-+\sigma_-\sigma_+=\mathbb{I}$.}
for $\sigma_z$ and $\sigma_\pm=(\sigma_x\pm i\sigma_y)/2$, the square of $M$ is
\begin{align}
    M^2 &= \Delta_z^2\sigma_z^2
         + \Sigma_+^2\sigma_+^2
         + \Sigma_-^2\sigma_-^2 \notag\\
        &\quad + \Delta_z\Sigma_+\underbrace{\{\sigma_z,\sigma_+\}}_{=\,0}
         + \Delta_z\Sigma_-\underbrace{\{\sigma_z,\sigma_-\}}_{=\,0}
         + \Sigma_+\Sigma_-\underbrace{(\sigma_+\sigma_-+\sigma_-\sigma_+)}_{=\,\mathbb{I}}
         \notag\\[4pt]
        &= \bigl(\Delta_z^2 + \Sigma_+\Sigma_-\bigr)\mathbb{I}
        \;\equiv\; \chi^2\,\mathbb{I}.
    \label{eq:app_Msquared}
\end{align}
Because $M^2 = \chi^2 \mathbb{I}$, all even powers of $M$ are proportional to the
identity and all odd powers are proportional to $M$:
\begin{equation}
    M^{2k} = \chi^{2k}\mathbb{I}, \qquad M^{2k+1} = \chi^{2k}M.
    \label{eq:app_M_powers}
\end{equation}
This is the key algebraic fact from which the entire exact resummation follows.

\subsection*{Step 2: Exact resummation of $e^\Delta$}

Summing the power series for $e^M = \sum_{n=0}^\infty M^n/n!$ and separating even
from odd terms using \eqref{eq:app_M_powers},
\begin{align}
    e^M &= \sum_{k=0}^\infty \frac{M^{2k}}{(2k)!}
          + \sum_{k=0}^\infty \frac{M^{2k+1}}{(2k+1)!}
          = \cosh\chi\,\mathbb{I} + \frac{\sinh\chi}{\chi}\,M,
    \label{eq:app_expM_exact}
\end{align}
and therefore, exactly:
\begin{equation}
    e^\Delta = e^{\Delta_0}\!\left[\cosh\chi\,\mathbb{I}
    + \frac{\sinh\chi}{\chi}\,M\right].
    \label{eq:app_expDelta_exact}
\end{equation}
This is \emph{not} a perturbative approximation. No BCH is required to evaluate
$e^\Delta$; the resummation is closed by \eqref{eq:app_Msquared} alone.

\subsection*{Step 3: Verifying BCH closure order by order}

Returning to the full problem $\bar\rho_Q = e^{A}e^{B}$ with $A=-\beta H_Q$ and
$B=\Delta$, the BCH logarithm generates an infinite tower of nested commutators.
We now verify explicitly that every level of this tower remains in
$\mathcal U(2)\equiv\mathrm{span}\{\mathbb{I},\sigma_z,\sigma_+,\sigma_-\}$.

\paragraph{First-order terms: $\mathrm{ad}_{H_Q}^k M$.}
Since $[H_Q,\sigma_z]=0$ and $[H_Q,\sigma_\pm]=\pm\omega_q\sigma_\pm$, we have
\begin{equation}
    \mathrm{ad}_{H_Q}(M) = \omega_q\!\left(\Sigma_+\sigma_+ - \Sigma_-\sigma_-\right)
    \;\in\;\mathrm{span}\{\sigma_+,\sigma_-\}.
    \label{eq:app_adM_first}
\end{equation}
Iterating: $\mathrm{ad}_{H_Q}^k(M)=\omega_q^k(\Sigma_+\sigma_+ +(-1)^k\Sigma_-\sigma_-)$
for $k\ge 1$ (and $\mathrm{ad}_{H_Q}^k(\Delta_z\sigma_z)=0$ for all $k$).
So the entire linear-in-$M$ tower $\{\mathrm{ad}_{H_Q}^k M\}$ lies in
$\mathrm{span}\{\sigma_z,\sigma_+,\sigma_-\}\subset\mathfrak{su}(2)$.
This is condition (C1) of Sec.~\ref{sec:closure}, verified exactly for the qubit.

\paragraph{Second-order term: $[M,\mathrm{ad}_{H_Q}M]$.}
Substituting \eqref{eq:app_adM_first} and using $[σ_z,σ_\pm]=\pm2\sigma_\pm$,
$[\sigma_+,\sigma_-]=\sigma_z$, $[\sigma_\pm,\sigma_\pm]=0$:
\begin{align}
    \bigl[M,\,\mathrm{ad}_{H_Q}M\bigr]
    &= \omega_q\bigl[M,\,\Sigma_+\sigma_+-\Sigma_-\sigma_-\bigr] \notag\\
    &= \omega_q\Bigl(
        \Delta_z\Sigma_+[\sigma_z,\sigma_+]
       -\Delta_z\Sigma_-[\sigma_z,\sigma_-] \notag\\
    &\qquad\quad
       +\Sigma_+\Sigma_-[\sigma_+,\sigma_-]
       +\Sigma_+\Sigma_-[\sigma_-,\sigma_+]
       \Bigr)              \notag\\[4pt]
    &= \omega_q\Bigl(
        2\Delta_z\Sigma_+\sigma_+
       +2\Delta_z\Sigma_-\sigma_-
       -2\Sigma_+\Sigma_-\sigma_z
       \Bigr)              \notag\\[4pt]
    &= 2\omega_q\!\left(\Delta_z M - \chi^2\sigma_z\right),
    \label{eq:app_MadM}
\end{align}
where in the last line we used $\chi^2=\Delta_z^2+\Sigma_+\Sigma_-$ and regrouped.
This lies in $\mathrm{span}\{\sigma_z,\sigma_+,\sigma_-\}\subset\mathfrak{su}(2)$ ✓.
Moreover, \eqref{eq:app_MadM} takes the memorable form $2\omega_q(\Delta_z M - \chi^2\sigma_z)$:
the commutator is a linear combination of $M$ itself and $\sigma_z$, so the algebra
is genuinely self-referential at this order.

\paragraph{Higher orders: closure by induction.}
Since $M,\mathrm{ad}_{H_Q}^k M\in\mathfrak{su}(2)$ for all $k\ge0$, and
$\mathfrak{su}(2)$ is a \emph{Lie algebra} (closed under commutators), every nested
commutator of the form
\begin{equation}
    \bigl[M,\bigl[M,\cdots\bigl[M,\mathrm{ad}_{H_Q}^k M\bigr]\cdots\bigr]\bigr]
    \;\in\;\mathfrak{su}(2)
    \label{eq:app_general_closure}
\end{equation}
by induction on the nesting depth.  No new operator basis elements are generated
at any order.  This is condition (C3) of Sec.~\ref{sec:closure}, established here
not merely by appeal to the associativity of $\mathcal{A}$ but by explicit Pauli
algebra.

\subsection*{Step 4: The BCH series resums to a $2\times2$ matrix logarithm}

The closure established above implies that $-\beta H_{\mathrm{MF}}=\log(e^Ae^B)$
is an element of $\mathbb{C}\cdot\mathbb{I}\oplus\mathfrak{su}(2)$ at every
truncation order.  But since we can compute $e^A$ and $e^B$ in closed form, we
need not sum the BCH series at all: the product $e^A e^B=e^{-\beta H_Q}e^\Delta$
is a concrete $2\times2$ matrix, and its logarithm is the exact sum of the entire
BCH tower.

Setting $a\equiv\beta\omega_q/2$ and $\varphi\equiv\sinh\chi/\chi$, the matrix
product reads
\begin{equation}
    e^{-\beta H_Q}e^\Delta
    = e^{\Delta_0}
    \begin{pmatrix}
        e^{-a}(\cosh\chi + \varphi\Delta_z) & e^{-a}\varphi\Sigma_+ \\
        e^{a}\varphi\Sigma_-                 & e^{a}(\cosh\chi - \varphi\Delta_z)
    \end{pmatrix}.
    \label{eq:app_product_matrix}
\end{equation}
After normalisation, the Bloch vector of $\rho_Q=\bar\rho_Q/\mathrm{Tr}\,\bar\rho_Q$
is read from the off-diagonal and diagonal entries.  The exact matrix logarithm
then gives $H_{\mathrm{MF}}=-\beta^{-1}\log\rho_Q$.  For any qubit state
$\rho_Q=\tfrac12(\mathbb{I}+\mathbf{r}\cdot\boldsymbol\sigma)$ with Bloch radius
$r=|\mathbf{r}|$, this logarithm is, by elementary $2\times2$ matrix calculus,
\begin{equation}
    H_{\mathrm{MF}} = c_0\,\mathbb{I}
    - \frac{1}{\beta}\frac{\operatorname{arctanh}r}{r}\,\mathbf{r}\cdot\boldsymbol\sigma,
    \label{eq:app_HMF_bloch_log}
\end{equation}
recovering Eq.~\eqref{eq:HMF_bloch_log_v5} of the main text.

\subsection*{Summary: what the BCH tower contributes}

Table~\ref{tab:bch_tower} collects the BCH commutator hierarchy for the qubit.
Every row stays within $\mathcal U(2)$; the resummation of the infinite tower is
achieved by the single matrix computation \eqref{eq:app_product_matrix}.

\begin{table}[h]
\centering
\renewcommand{\arraystretch}{1.4}
\begin{tabular}{lll}
    \hline
    BCH term & Explicit result & Resides in \\
    \hline
    $\mathrm{ad}_{H_Q}^k M$ & $\omega_q^k(\Sigma_+\sigma_+ + (-1)^k\Sigma_-\sigma_-)$ &
        $\mathrm{span}\{\sigma_+,\sigma_-\}$ \\
    $[M,\mathrm{ad}_{H_Q}M]$ & $2\omega_q(\Delta_z M - \chi^2\sigma_z)$ &
        $\mathrm{span}\{\sigma_z,\sigma_+,\sigma_-\}$ \\
    All higher nesting & Lie algebra closure & $\mathfrak{su}(2)$ \\
    Full BCH sum & $2\times2$ matrix log \eqref{eq:app_HMF_bloch_log} &
        $\mathbb{C}\mathbb{I}\oplus\mathfrak{su}(2)$ \\
    \hline
\end{tabular}
\caption{The BCH commutator tower for the qubit.
All terms remain in $\mathcal{U}(2)$; the infinite tower resumms to the exact
Bloch-log formula \eqref{eq:app_HMF_bloch_log}.}
\label{tab:bch_tower}
\end{table}

The connection to the generating-function result of Sec.~\ref{sec:closure} is now
transparent.  The linear-in-$\Delta$ piece of the BCH logarithm is
$-\beta^{-1}\Phi(\beta\,\mathrm{ad}_{H_Q})\Delta$ with $\Phi(x)=x/(1-e^{-x})$,
which, restricted to the qubit's $(0,\pm\omega_q)$ eigenfrequencies, gives
\begin{equation}
    \Phi(\beta\,\mathrm{ad}_{H_Q})M
    = \Delta_z\sigma_z
      + \Phi(\beta\omega_q)\,\Sigma_+\sigma_+
      + \Phi(-\beta\omega_q)\,\Sigma_-\sigma_-.
    \label{eq:app_Phi_on_M}
\end{equation}
The nonlinear BCH corrections ($O(\Delta^2)$ and beyond) shift the effective
arguments of the $\cosh$/$\sinh$ functions in \eqref{eq:app_product_matrix} away
from the linear-in-$\Delta$ values.  In the weak-coupling limit ($g\to0$,
$\chi\to0$), $\cosh\chi\to1$ and $\sinh\chi/\chi\to1$, so the nonlinear
corrections vanish and \eqref{eq:app_Phi_on_M} is exact.  At finite coupling
the full expression \eqref{eq:app_HMF_bloch_log} is required, with $\chi$
depending nonlinearly on the channel amplitudes.

