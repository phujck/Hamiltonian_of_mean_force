\section{Results: Analytic Examples}
\label{sec:results}

We provide three minimal analytic examples that illustrate the closure
criterion and its consequences. These examples are not approximations; they
serve only to show when closure is exact.

\subsection{Commuting coupling}
If $[H_Q,f]=0$, then $\mathrm{ad}_{H_Q}^n(f)=0$ for all $n\ge 1$, so
$\tilde{f}(\tau)=f$ and the time ordering becomes trivial. The bilocal exponent
reduces to a scalar multiple of $f^2$, and the HMF is exactly local. This is the
simplest solvable case and is consistent with known exactly solvable strong
coupling models\cite{campisiTalknerHanggi2009Solvable}.

\subsection{Quadratic/Gaussian system}
Consider a harmonic system with
$H_Q = p^2/2m + (1/2)m\omega^2 q^2$ and linear coupling $f=q$. The adjoint
action closes on the finite set $\{q,p,\mathbb{I}\}$ since
$[H_Q,q] \propto p$ and $[H_Q,p] \propto q$. Consequently, the associative
algebra generated by $\mathcal{A}_f$ is finite dimensional and the HMF is a
quadratic operator. This reproduces the known Gaussian character of the
reduced equilibrium state and its mean-force Hamiltonian in damped harmonic
models\cite{grabertQuantumBrownianMotion1988,hiltHamiltonianMeanForce2011}.

\subsection{Single qubit (Pauli algebra)}
Let $H_Q = (\omega/2)\sigma_z$ and $f=\sigma_x$. Then
$\mathrm{ad}_{H_Q}(f) = \omega i\sigma_y$ and
$\mathrm{ad}_{H_Q}^2(f) = -\omega^2 \sigma_x$, so the adjoint chain closes on the
Pauli algebra $\{\sigma_x,\sigma_y,\sigma_z,\mathbb{I}\}$. Hence the closure
criterion is satisfied and $H_{\mathrm{MF}}$ lies in the same finite operator
class. This is consistent with standard spin-boson constructions\cite{leggettDynamicsDissipativeTwostate1987}.

