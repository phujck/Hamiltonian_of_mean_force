\section{Quenched representation and influence functional\label{sec:quenched}\label{sec:model}}
We begin by recapitulating the quenched representation introduced in Ref.~\cite{mccaulMeanForceHamiltoniansInfluence2026}.
To proceed directly from Sec.~\ref{sec:intro}, we explicit the composite model.
We denote the bare system Hamiltonian by $H_Q$ and write
\begin{equation}
    H_{\mathrm{tot}} = H_Q + H_X + H_{\mathrm{int}},
    \label{eq:Htot_split_sec3}
\end{equation}
where $H_X$ is the bath Hamiltonian. We assume a factorizable interaction
\begin{equation}
    H_{\mathrm{int}} = f \otimes B,
    \label{eq:Hint_general_f_B}
\end{equation}
where $f$ acts on the system and $B$ is a bath operator. We can define the reduced equilibrium operator (up to normalisation) by $\bar{\rho}_Q(\beta)\equiv \Tr_X e^{-\beta H_{\mathrm{tot}}}$. As shown in Ref.~\cite{mccaulMeanForceHamiltoniansInfluence2026}, this can be represented as a \emph{quenched density}. This is an average over a stochastic propagator, given by:
\begin{align}
    \bar{\rho}_Q(\beta) &= \mathbb E_\xi\big[U_\xi(\beta)\big],
    \label{eq:quenched_identity_main}\\
    U_\xi(\beta) &\equiv \mathcal T_\tau \exp\!\left[-\int_0^\beta d\tau\,\big(H_Q+\xi(\tau)f\big)\right],
    \label{eq:quenched_propagator_sec3}
\end{align}
where $\xi(\tau)$ is a stochastic process whose statistics encode the bath correlations, and $ \mathcal T_\tau$ denotes time-ordering in imaginary time. Regardless of the precise form of bath and coupling, the reduced density must always be describable in this form. 

This result can be understood intuitively by first observing that since the environment influence can only enter through the system coupling $f$, its influence can be captured by attaching a $\tau$-dependent driving field to $f$~\cite{huQuantumBrownianMotion1992,mccaulHowWinFriends2021b}. If this were a single deterministic field however, it would correspond to the bath exerting the \emph{same} back-action history for every microscopic bath configuration. But $\Tr_X$ averages over many bath microstates in the thermal ensemble, and hence over many back-action histories. In this sense the partial trace is necessarily an average over histories in imaginary time, and the quenched representation simply makes this averaging explicit. For each realisation $\xi(\tau)$ the system evolves under an imaginary-time Hamiltonian $H_Q+\xi(\tau)f$; the bath is then recovered by averaging over $\xi(\tau)$ with a law chosen to reproduce the bath-induced correlations. In this sense $\xi(\tau)$ is not a physical external control field but an efficient parametrisation of the bath history $B(\tau)$ as seen through the coupling channel. 

We may understand what formal properties are demanded of $\xi$ by considering the influence functional it is required to match. Working in imaginary time, introduce the bath interaction picture: 
\begin{equation}
    B(\tau) \equiv e^{\tau H_X} B e^{-\tau H_X},\qquad \tau\in[0,\beta],
\end{equation}
and define the bath thermal state $\rho_X \equiv e^{-\beta H_X}/Z_X$.
For an arbitrary c-number source $j(\tau)$ coupled linearly to $B(\tau)$, the bath generates a
(time-ordered) functional
\begin{equation}
    \mathcal Z_X[j]
    \equiv \Tr_X\!\left[\mathcal T_\tau \exp\!\left(-\int_0^\beta d\tau\, j(\tau)\,B(\tau)\right)\rho_X\right].
    \label{eq:bath_generating_functional}
\end{equation}
Here $j(\tau)$ is introduced as an external c-number source used to generate ordered bath correlators by functional differentiation.  More generally, $j(\tau)$ may be any object commuting with the bath algebra (e.g. a system operator tensored with $\mathbb I_X$). In the influence-functional derivation it is ultimately supplied by the system history, which becomes a c-number function in the path-integral representation. In the influence-functional approach, tracing out the bath produces precisely such a functional, evaluated on the system history through the coupling channel. 

From this, the bath contribution to the effective Euclidean action can be written as \cite{feynmanTheoryGeneralQuantum1963a,grabertQuantumBrownianMotion1988,caldeiraQuantumTunnellingDissipative1983a}
\begin{equation}
    \mathcal F[f] \equiv \mathcal Z_X[f],
    \qquad
    \Phi[f] \equiv \log \mathcal F[f].
    \label{eq:influence_functional_definition}
\end{equation}
A key structural fact is that $\Phi[f]=\log\mathcal F[f]$ is the \emph{cumulant generating functional} of the bath operator $B(\tau)$ with respect to the thermal state. Concretely, the generalised
(time-ordered) cumulant theorem implies the connected expansion
\cite{Kubo1962,BreuerMaPetruccione2002}
\begin{equation}
\begin{split}
\Phi[f]
&=
\sum_{n\ge 1}\frac{(-1)^n}{n!}\!\!\int_0^\beta\!\! d\tau_1\cdots d\tau_n\, \\
&\quad\times K^{(n)}(\tau_1,\ldots,\tau_n)\,
f(\tau_1)\cdots f(\tau_n),
\end{split}
\label{eq:influence_cumulant_expansion}
\end{equation}
where the kernels $K^{(n)}$ are the \emph{connected} (cumulant) bath correlators, defined via \cite{DasThermalFieldTheory}
\begin{equation}
    K^{(n)}(\tau_1,\ldots,\tau_n)
    \equiv
    \big\langle \mathcal T_\tau B(\tau_1)\cdots B(\tau_n)\big\rangle_c.
    \label{eq:Kn_as_connected_bath_correlators}
\end{equation}
The explicit connection back to Eq.~\eqref{eq:bath_generating_functional} is given via its functional differentiation:
\begin{equation}
    \big\langle \mathcal T_\tau B(\tau_1)\cdots B(\tau_n)\big\rangle_c = (-1)^n \left. \frac{\delta^n \log \mathcal Z_X[j]}{\delta j(\tau_1)\cdots\delta j(\tau_n)} \right|_{j=0}.
    \label{eq:connected_correlator_oneliner}
\end{equation}
The bath influence is then completely characterised by the hierarchy $\{K^{(n)}\}_{n\ge1}$. 

To connect the influence functional back to the quenched density, we use the fact that the bath
influence depends on the system history only through the linear functional $\int_0^\beta d\tau\, f(\tau)B(\tau)$. One may therefore represent $\mathcal F[f]$ as the (generalised) characteristic functional of an
auxiliary field $\xi(\tau)$ \cite{hubbardCalculationPartitionFunctions1959a,stratonovich1957QDistro,stockburgerExactNumberRepresentation2002}:
\begin{equation}
\mathcal F[f]
=
\mathbb E_\xi\!\left[\exp\!\left(-\int_0^\beta d\tau\,\xi(\tau)\,f(\tau)\right)\right],
\label{eq:noise_characteristic_functional}
\end{equation}
where $\mathbb E_\xi[\cdot]$ denotes averaging with respect to a (possibly complex) measure on
$\xi$-histories chosen such that \eqref{eq:noise_characteristic_functional} holds.

Since $\xi(\tau)$ is a commuting $c$-number field, its $n$-point moments are symmetric under
permutations of the time arguments. The influence kernels $K^{(n)}$ fix the \emph{cumulants} of
$\xi$ via
\begin{equation}
\big\langle \xi(\tau_1)\cdots \xi(\tau_n)\big\rangle_c
=
K^{(n)}(\tau_1,\ldots,\tau_n).
\label{eq:xi_cumulants_equal_K}
\end{equation}
Consequently, the ordinary correlation functions of the noise are obtained from
$\{K^{(m)}\}$ by the standard moment-cumulant relations:
\begin{equation}
\big\langle \xi(\tau_1)\cdots \xi(\tau_n)\big\rangle
=
\sum_{\pi\in\mathcal P_n}\;
\prod_{C\in\pi}
K^{(|C|)}\!\big(\{\tau_i\}_{i\in C}\big),
\label{eq:moment_cumulant_partition}
\end{equation}
where $\mathcal P_n$ denotes the set of all partitions of the index set $\{1,\ldots,n\}$, and $C$ are the disjoint blocks of a given partition $\pi \in \mathcal P_n$. After shifting the mean so that $K^{(1)}(\tau)=\langle\xi(\tau)\rangle=0$, one has (for example)
\begin{align}
\big\langle \xi(\tau_1)\xi(\tau_2)\big\rangle
&= K^{(2)}(\tau_1,\tau_2),\\
\big\langle \xi(\tau_1)\xi(\tau_2)\xi(\tau_3)\big\rangle
&= K^{(3)}(\tau_1,\tau_2,\tau_3),\\
\begin{split}
\big\langle \xi(\tau_1)\xi(\tau_2)&\xi(\tau_3)\xi(\tau_4)\big\rangle \\
&= K^{(4)}(\tau_1,\tau_2,\tau_3,\tau_4) \\
& \quad + K^{(2)}(\tau_1,\tau_2)K^{(2)}(\tau_3,\tau_4) \\
& \quad + K^{(2)}(\tau_1,\tau_3)K^{(2)}(\tau_2,\tau_4) \\
& \quad + K^{(2)}(\tau_1,\tau_4)K^{(2)}(\tau_2,\tau_3).
\end{split}
\end{align}
