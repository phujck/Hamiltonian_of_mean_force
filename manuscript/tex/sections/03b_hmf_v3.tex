\section{Quenched density and the Hamiltonian of mean force \label{sec:hmf}}
The Hamiltonian of mean force is, by definition, the operator whose Gibbs form reproduces the
system's reduced equilibrium state. Equivalently, it is the \emph{operator logarithm} of the
unnormalised reduced equilibrium operator
\begin{equation}
\bar{\rho}_Q(\beta)\equiv \Tr_X e^{-\beta H_{\mathrm{tot}}},
\qquad
H_{\mathrm{MF}}(\beta)\;\equiv\;-\frac{1}{\beta}\log \bar{\rho}_Q(\beta),
\label{eq:rhobar_and_HMF_tilde}
\end{equation}
defined up to an additive multiple of the identity (fixed only when normalising the state).

The quenched representation in Eq.~\eqref{eq:quenched_identity_main} supplies an exact stochastic
parametrisation of the unnormalised mean-force Gibbs operator $\bar{\rho}_Q(\beta)$ by making the
bath trace an explicit average over imaginary-time back-action histories. Combining
Eq.~\eqref{eq:quenched_identity_main} with Eq.~\eqref{eq:rhobar_and_HMF_tilde} yields
\begin{equation}
H_{\mathrm{MF}}(\beta)
=
-\frac{1}{\beta}\log \mathbb E_\xi\!\left[U_\xi(\beta)\right].
\end{equation}
Thus, constructing the mean-force Hamiltonian reduces to evaluating a stochastic average and then
compressing the result via an operator logarithm. The conditions for being able to perform this average exactly is the focus of the present work. 

 To make the handling of this problem more concrete, we shall specialise the environment to the Caldeira-Leggett model, where the bath is a collection of harmonic oscillators ($H_X=\sum_k \omega_k b_k^\dagger b_k$) and the coupling is linear in bath coordinates ($B=\sum_k c_k x_k$). None of the results that follow are essentially dependent on this choice, and a generalisation to anharmonic environments is (relatively) straightforward.  In the interests of comprehensibility however, we restrict our scope to quadratic environments. In this setting, the auxiliary field $\xi(\tau)$ becomes a stationary zero-mean Gaussian process completely characterized by its covariance
\begin{equation}
    \mathbb E_\xi[\xi(\tau)\xi(\tau')] = K(\tau-\tau').
\end{equation}
The kernel $K(\tau)$ is determined by the bath spectral density $J(\omega)=\frac{\pi}{2}\sum_k \frac{c_k^2}{m_k\omega_k}\delta(\omega-\omega_k)$ via the relation~\cite{weissQuantumDissipativeSystems2012}
\begin{equation}
    K(\tau)=\frac{1}{\pi}\int_0^\infty d\omega\,J(\omega)\,
    \frac{\cosh\!\big(\omega(\beta/2-|\tau|)\big)}{\sinh(\beta\omega/2)}.
    \label{eq:K_explicit_cosh}
\end{equation}
A key property of this kernel is its integrated strength. Integrating Eq.~\eqref{eq:K_explicit_cosh} yields
\begin{align}
    \int_0^\beta d\tau\,K(\tau) 
    &= \frac{1}{\pi}\int_0^\infty \!\!d\omega\,J(\omega) \int_0^\beta d\tau\, \frac{\cosh\!\big(\omega(\beta/2-|\tau|)\big)}{\sinh(\beta\omega/2)} \nonumber\\
    &= \frac{1}{\pi}\int_0^\infty \!\!d\omega\,J(\omega)\,\frac{2}{\omega} \nonumber\\
    &= 2\lambda,
\end{align}
where $\lambda$ is the explicit reorganisation energy. Consequently, the total variance of the integrated noise field $\Xi = \int_0^\beta d\tau\,\xi(\tau)$ grows linearly with inverse temperature:
\begin{equation}
    \mathbb E_\xi[\Xi^2] = \int_0^\beta \!\!d\tau\!\int_0^\beta \!\!d\tau'\,K(\tau-\tau') = 2\beta\lambda.
    \label{eq:integrated_variance_linear}
\end{equation}

In the case that the system and its coupling commute, $[H_Q,f]= 0$ and $f$ is $\tau$-independent in imaginary time. In this instance time ordering drops out, and the average is given by \cite{mccaulMeanForceHamiltoniansInfluence2026}:
\begin{equation}
    \bar{\rho}_Q(\beta)=
    \exp\!\left[-\beta\left(H_Q-\frac{\kappa_0(\beta)}{2}f^2\right)\right],
    \label{eq:rho_commuting_closed_sec3}
\end{equation}
which in turn yields
\begin{equation}
    H_{\mathrm{MF}}(\beta)=
    H_Q-\frac{\kappa_0(\beta)}{2}f^2+\frac{1}{\beta}\log Z_X(\beta)\,\mathbb I.
    \label{eq:Heff_commuting_limit}
\end{equation}

Notably this correction to  $H_{\mathrm{MF}}(\beta)$ is entirely \emph{classical} \cite{mccaulMeanForceHamiltoniansInfluence2026}. This is hardly surprising, but emphasises that truly quantum effects \emph{always} stem from non-commutativity. Truly quantum thermodynamic effects are therefore only present when $[H_Q,f]\neq 0$. In this case however, the noise enters through a noncommuting operator inside $\mathcal T_\tau$, rendering the question of averaging highly non-trivial. In the next section, we attack this problem directly, deriving conditions under which $H_{\mathrm{MF}}(\beta)$ possesses a closed form. 

