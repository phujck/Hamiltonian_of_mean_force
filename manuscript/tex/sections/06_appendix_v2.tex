\section{Quenched Density and Imaginary-Time Evolution}
\label{sec:appendix_quenched}

This appendix makes explicit the operator identity underlying the quenched
imaginary-time formulation used in Sec.~\ref{sec:quenched}. Define a
time-ordered exponential with a (generally) $\tau$-dependent operator
$H(\tau)$,
\begin{equation}
    \begin{split}
        \rho(\tau) &\equiv \mathcal{T}_\tau
        \exp\left(-\int_0^\tau d\tau' \, H(\tau')\right), \\
        \rho(0) &= \mathbb{I}.
    \end{split}
    \label{eq:appendix_rho_def}
\end{equation}
Standard differentiation identities for ordered exponentials imply
\begin{equation}
    -\partial_\tau \rho(\tau) = H(\tau)\rho(\tau),
    \label{eq:appendix_imag_time_eq}
\end{equation}
with the ordering built into $\rho(\tau)$; see, e.g.,
Refs.~\cite{wilcoxExponentialOperatorsParameter1967a,magnusExponentialSolutionDifferential1954a,blanesMagnusExpansionIts2009}
for operator calculus and time-ordered exponentials.

In the present context one takes
\begin{equation}
    H(\tau) = H_Q + \xi(\tau) f,
    \label{eq:appendix_quenched_H}
\end{equation}
where the Gaussian field satisfies
\begin{equation}
    \langle \xi(\tau)\xi(\tau')\rangle = K(\tau-\tau').
    \label{eq:appendix_xi_cov}
\end{equation}
The reduced equilibrium operator can be written as
\begin{equation}
    \bar{\rho}_S =
    e^{-\beta H_Q}\left\langle \rho(\beta) \right\rangle_\xi,
    \label{eq:appendix_rho_beta}
\end{equation}
which is the compact operator form of the quenched Gaussian representation
used in the main text.

To connect Eq.~\eqref{eq:appendix_rho_beta} to an imaginary-time path integral,
one discretizes $\tau \in [0,\beta]$, applies a Trotter (or Zassenhaus)
factorization, and inserts resolutions of identity in the system coordinate
basis. This yields the standard Euclidean path-integral expression for the
canonical density operator, with the $\tau$-dependent potential induced by the
auxiliary field, as in the influence-functional derivation for quadratic baths
and their stochastic unravellings
\cite{feynmanTheoryGeneralQuantum1963a,caldeiraQuantumTunnellingDissipative1983a,grabertQuantumBrownianMotion1988,moixEquilibriumreducedDensityMatrix2012,chenRigorousStochasticMatrix2014}.

\section{Kernel Symmetry and Moment Relations}
\label{sec:appendix_kernel}

The bath kernel appearing in the bilocal influence functional is
\begin{equation}
    \begin{split}
        K(\tau-\tau') &=
        \mathrm{Tr}_B\!\left[\mathcal{T}_\tau \tilde{B}(\tau)\tilde{B}(\tau')\rho_B\right], \\
        \rho_B &= \frac{e^{-\beta H_X}}{Z_B}.
    \end{split}
    \label{eq:appendix_kernel_def}
\end{equation}
For equilibrium baths, $K$ depends only on the imaginary-time difference and is
even under exchange of its arguments, implying
\begin{equation}
    K(\tau-\tau') = K(\tau'-\tau), \qquad
    \mu_{nm} = \mu_{mn},
    \label{eq:appendix_kernel_sym}
\end{equation}
\begin{equation}
    \mu_{nm} = \frac{1}{n!m!} \int_0^\beta d\tau \int_0^\beta d\tau' \tau^n (\tau')^m K(\tau-\tau').
    \label{eq:kernel_moments_def}
\end{equation}
In common quadratic-bath models the kernel is
explicitly constructed from the bath spectral density and satisfies the
Kubo-Martin-Schwinger periodicity in imaginary time, which can be used to
re-express moment integrals in equivalent forms; see
Refs.~\cite{grabertQuantumBrownianMotion1988,tanimuraReducedHierarchicalEquations2014,songCalculationCorrelatedInitial2015}
for explicit constructions.

The moment expansion used in Sec.~\ref{sec:quenched} requires only these symmetry
properties and the existence of the integrals defining $\mu_{nm}$. No further
approximation is introduced at this stage.

\section{Derivation of the Influence Functional\label{app:influence_derivation}}

In this appendix, we construct the influence functional formalism used in Sec.~\ref{sec:quenched}. Our goal is to derive the exact form of the reduced density operator $\bar{\rho}_S(\beta)$ by explicitly integrating out the harmonic bath, and to demonstrate that this leads directly to the stochastic unravelling employed in the main text.

\subsection{Euclidean Path Integral Setup}
We begin with the definition of the unnormalized reduced state,
\begin{equation}
    \bar{\rho}_S(\beta) = \Tr_X \left[ e^{-\beta H_{\mathrm{tot}}} \right].
\end{equation}
This trace can be represented as a Euclidean path integral. Let $|q\rangle$ and $|x\rangle = |x_1, x_2, \dots\rangle$ denote the position bases for the system and bath, respectively. The matrix element $\langle q | \bar{\rho}_S | q' \rangle$ involves a sum over all periodic paths $x(\tau)$ (where $x(0)=x(\beta)$) and open paths $q(\tau)$ (where $q(0)=q'$ and $q(\beta)=q$):
\begin{equation}
\begin{split}
    \langle q | \bar{\rho}_S | q' \rangle = \int_{q(0)=q'}^{q(\beta)=q} &\mathcal{D}q(\tau) e^{-S_Q[q]/\hbar} \\
    &\times \prod_k Z_k[q],
\end{split}
    \label{eq:app_path_integral_start}
\end{equation}
where $S_Q$ is the Euclidean action of the isolated system, and $Z_k[q]$ is the partition function of the $k$-th oscillator in the presence of the external driving force $J_k(\tau) = -c_k f(q(\tau))$:
\begin{equation}
\begin{split}
    Z_k[q] = \oint & \mathcal{D}x_k(\tau) \exp\bigg( -\frac{1}{\hbar} \int_0^\beta d\tau \\
    &\times \left[ \frac{m_k}{2} \dot{x}_k^2 + \frac{m_k \omega_k^2}{2} x_k^2 + c_k x_k f(q(\tau)) \right] \bigg).
\end{split}
\end{equation}
Note that the periodicity of the trace implies periodic boundary conditions for the bath paths $x_k(\tau)$.

\subsection{Gaussian Integration}
The functional integral for $Z_k[q]$ is Gaussian and can be evaluated exactly. It corresponds to the partition function of a forced harmonic oscillator. The result is expressible as the product of the free oscillator partition function, $Z_X^{(k)} = (2\sinh(\beta\hbar\omega_k/2))^{-1}$, and an exponential "influence phase" depending quadratically on the drive~\cite{feynmanTheoryGeneralQuantum1963a,weissQuantumDissipativeSystems2012}:
\begin{equation}
\begin{split}
    Z_k[q] = Z_X^{(k)} \exp\bigg( &\frac{1}{2\hbar} \int_0^\beta d\tau \int_0^\beta d\tau' \\
    &\times K_k(\tau-\tau') f(q(\tau)) f(q(\tau')) \bigg).
\end{split}
\end{equation}
The kernel $K_k(\tau)$ is the equilibrium autocorrelation function of the coordinate $x_k$:
\begin{equation}
    K_k(\tau-\tau') = c_k^2 \langle \mathcal{T}_\tau x_k(\tau) x_k(\tau') \rangle_0.
\end{equation}
Summing over all modes $k$, the total influence functional is $\prod_k Z_k[q] = Z_X \exp( \Phi_{inf}[q] )$, with
\begin{equation}
\begin{split}
    \Phi_{inf}[q] = \frac{1}{2\hbar} \int_0^\beta d\tau &\int_0^\beta d\tau' K(\tau-\tau') \\
    &\times f(q(\tau)) f(q(\tau')),
\end{split}
\end{equation}
where $K(\tau) = \sum_k K_k(\tau)$ is the total force autocorrelation function.

\subsection{From Non-local Action to Stochastic Average}
Substituting this back into Eq.~\eqref{eq:app_path_integral_start}, the reduced density matrix becomes
\begin{equation}
    \bar{\rho}_S = Z_X \int \mathcal{D}q \, e^{-S_Q[q]/\hbar} \exp\left( \Phi_{inf}[q] \right).
\end{equation}
The term $\Phi_{inf}[q]$ is non-local in imaginary time, representing a self-interaction of the system mediated by the bath. To disentangle this, we use the Hubbard-Stratonovich transformation (the continuous analog of the Gaussian identity $e^{\frac{1}{2} A^2} \sim \int d\xi e^{-\frac{1}{2}\xi^2 + \xi A}$). We introduce a real, auxiliary stochastic field $\xi(\tau)$ with zero mean and covariance
\begin{equation}
    \langle \xi(\tau) \xi(\tau') \rangle_\xi = K(\tau-\tau').
\end{equation}
Using this field, we can rewrite the influence exponential as a stochastic average:
\begin{equation}
\begin{split}
    \exp\left( \Phi_{inf}[q] \right) = \bigg\langle \exp\bigg( &\frac{1}{\hbar} \int_0^\beta d\tau \\
    &\times \xi(\tau) f(q(\tau)) \bigg) \bigg\rangle_\xi.
\end{split}
\end{equation}
Inserting this identity into the path integral for $\bar{\rho}_S$, we can swap the order of the path integration over $q$ and the stochastic average over $\xi$:
\begin{equation}
    \bar{\rho}_S = Z_X \left\langle \int \mathcal{D}q \, \exp\left[-\frac{1}{\hbar} S_Q[q] + \frac{1}{\hbar} \int_0^\beta \xi f \right] \right\rangle_\xi.
\end{equation}
The term in the angle brackets is exactly the path integral for a system evolving under the time-dependent Hamiltonian $H(\tau) = H_Q - \xi(\tau)f$. Thus, in operator language, we arrive at the exact stochastic representation:
\begin{equation}
\begin{split}
    \bar{\rho}_S(\beta) = Z_X \bigg\langle \mathcal{T}_\tau \exp\bigg( &-\int_0^\beta d\tau \\
    &\times [H_Q - \xi(\tau)f] \bigg) \bigg\rangle_\xi.
\end{split}
    \label{eq:app_stochastic_final}
\end{equation}
This confirms that the HMF can be constructed by averaging the non-unitary (imaginary-time) evolution of the system driven by colored Gaussian noise.
