\section{Theoretical Derivation}
\label{sec:theory}

This section derives an exact operator expression for the reduced equilibrium
object and provides a systematic criterion for when it admits a local
Hamiltonian representation. No weak-coupling or Markovian assumptions are
introduced. The only structural assumption is a Gaussian bath with linear
coupling, which makes the influence functional bilocal and exact.

\subsection{Definitions and reduced equilibrium operator}
We consider a composite Hilbert space $\mathcal{H}=\mathcal{H}_S\otimes
\mathcal{H}_B$ and total Hamiltonian
\begin{equation}
    H_{\mathrm{tot}} = H_Q + H_B + H_I, \qquad H_I = f \otimes B,
    \label{eq:Htot_def}
\end{equation}
where $H_Q$ acts on the system, $H_B$ on the bath, and the interaction is
bilinear for clarity. (A sum $\sum_j f_j\otimes B_j$ is treated by the same
steps with a matrix-valued kernel.) The unnormalized reduced equilibrium
operator is
\begin{equation}
    \bar{\rho}_S(\beta)=\mathrm{Tr}_B\,e^{-\beta H_{\mathrm{tot}}},
    \label{eq:rho_unnorm}
\end{equation}
with normalized state
\begin{equation}
    \rho_S=\frac{\bar{\rho}_S}{\mathrm{Tr}_S\bar{\rho}_S}.
    \label{eq:rho_norm}
\end{equation}
The Hamiltonian of mean force (HMF) is defined by
\begin{align}
    e^{-\beta H_{\mathrm{MF}}(\beta)}
    &= \frac{\mathrm{Tr}_B\,e^{-\beta(H_Q+H_B+H_I)}}{Z_B(\beta)},
    \label{eq:HMF_def} \\
    Z_B(\beta) &= \mathrm{Tr}_B e^{-\beta H_B},
    \nonumber
\end{align}
up to an additive scalar fixed by $\mathrm{Tr}_S e^{-\beta H_{\mathrm{MF}}}$.
This operator definition is standard in strong-coupling thermodynamics and is
exact\cite{campisiFluctuationTheoremArbitrary2009,talknerColloquiumStatisticalMechanics2020,trushechkinOpenQuantumSystem2022,seifertFirstSecondLaw2016}.

\subsection{Imaginary-time interaction picture and exact Dyson identity}
Split $H_{\mathrm{tot}}=H_0+H_I$ with $H_0=H_Q+H_B$. Define the imaginary-time
interaction-picture propagator
\begin{equation}
    U(\tau) \equiv e^{\tau H_0} e^{-\tau H_{\mathrm{tot}}}, \qquad U(0)=\mathbb{I}.
    \label{eq:U_tau_def}
\end{equation}
Differentiating and using $H_{\mathrm{tot}}=H_0+H_I$ gives the exact evolution
\begin{equation}
    \frac{d}{d\tau}U(\tau) = -\tilde{H}_I(\tau)\,U(\tau),
    \qquad \tilde{H}_I(\tau)=e^{\tau H_0}H_I e^{-\tau H_0},
    \label{eq:U_tau_eom}
\end{equation}
whose solution is the ordered exponential
\begin{equation}
    U(\beta)=\mathcal{T}_\tau\exp\!\left(-\int_0^\beta d\tau\,\tilde{H}_I(\tau)\right).
    \label{eq:U_tau_solution}
\end{equation}
Therefore,
\begin{equation}
    e^{-\beta H_{\mathrm{tot}}} = e^{-\beta H_0}\,\mathcal{T}_\tau
    \exp\!\left(-\int_0^\beta d\tau\,\tilde{H}_I(\tau)\right),
    \label{eq:dyson_imag}
\end{equation}
which is the standard imaginary-time Dyson identity
\cite{feynmanTheoryGeneralQuantum1963a,caldeiraQuantumTunnellingDissipative1983a,grabertQuantumBrownianMotion1988}.
Since $H_0$ is a sum of commuting system and bath Hamiltonians,
$\tilde{H}_I(\tau)=\tilde{f}(\tau)\otimes\tilde{B}(\tau)$ with
\begin{align}
    \tilde{f}(\tau) &= e^{\tau H_Q} f e^{-\tau H_Q},
    \label{eq:ftilde_def} \\
    \tilde{B}(\tau) &= e^{\tau H_B} B e^{-\tau H_B}.
    \nonumber
\end{align}

\subsection{Gaussian bath trace and bilocal influence functional}
Using $e^{-\beta H_0}=e^{-\beta H_Q}e^{-\beta H_B}$, the bath trace becomes
\begin{equation}
    \bar{\rho}_S = e^{-\beta H_Q}
    \left\langle \mathcal{T}_\tau
    \exp\!\left(-\int_0^\beta d\tau\,\tilde{f}(\tau)\tilde{B}(\tau)\right)
    \right\rangle_B,
    \label{eq:trace_bath_first}
\end{equation}
where $\langle\cdot\rangle_B\equiv\mathrm{Tr}_B(\cdot\,\rho_B)$ and
$\rho_B=Z_B^{-1}e^{-\beta H_B}$. A cumulant expansion for the ordered
exponential gives
\begin{multline}
    \left\langle \mathcal{T}_\tau
    \exp\!\left(-\int d\tau\,\tilde{f}(\tau)\tilde{B}(\tau)\right)\right\rangle_B \\
    = \exp\!\Bigg(\sum_{n=1}^\infty \frac{(-1)^n}{n!}
    \int d\tau_1\cdots d\tau_n\,C_n(\tau_1,\ldots,\tau_n) \\
    \times \prod_{j=1}^n \tilde{f}(\tau_j)\Bigg),
    \label{eq:cumulant_expansion}
\end{multline}
with cumulants
\begin{equation}
    C_n(\tau_1,\ldots,\tau_n) \equiv
    \langle \mathcal{T}_\tau\tilde{B}(\tau_1)\cdots\tilde{B}(\tau_n)\rangle_c.
    \label{eq:cumulant_def}
\end{equation}
For a Gaussian bath (e.g., a harmonic bath with linear coupling), all cumulants
beyond second order vanish, yielding the exact bilocal influence functional
\begin{multline}
    \bar{\rho}_S = e^{-\beta H_Q}\,\mathcal{T}_\tau
    \exp\!\Bigg(-\frac{1}{2}\int_0^\beta d\tau\int_0^\beta d\tau' \\
    \tilde{f}(\tau)\,K(\tau-\tau')\,\tilde{f}(\tau')\Bigg),
    \label{eq:influence_functional}
\end{multline}
where the thermal kernel is
\begin{equation}
    K(\tau-\tau') \equiv
    \langle \mathcal{T}_\tau\tilde{B}(\tau)\tilde{B}(\tau')\rangle_B.
    \label{eq:kernel_def}
\end{equation}
This form is standard for linear coupling to Gaussian baths and is exact under
that assumption\cite{feynmanTheoryGeneralQuantum1963a,caldeiraQuantumTunnellingDissipative1983a,grabertQuantumBrownianMotion1988,tanimuraReducedHierarchicalEquations2014,songCalculationCorrelatedInitial2015}.
The ordering operator remains because the system operators at different
imaginary times need not commute.

\subsection{Gaussian-field (Hubbard--Stratonovich) reformulation}
The bilocal quadratic form in Eq.~\eqref{eq:influence_functional} can be
linearized exactly by a Hubbard--Stratonovich transformation. Introducing a
real Gaussian field $\xi(\tau)$ with covariance
$\langle\xi(\tau)\xi(\tau')\rangle=K(\tau-\tau')$, one obtains
\begin{equation}
    \bar{\rho}_S = e^{-\beta H_Q}
    \left\langle \mathcal{T}_\tau
    \exp\!\left(-\int_0^\beta d\tau\,\xi(\tau)\tilde{f}(\tau)\right)
    \right\rangle_{\xi}.
    \label{eq:hs_rep}
\end{equation}
This ``quenched'' Gaussian average is an exact operator identity and is widely
used in stochastic unravellings and equilibrium reduced-density formulations
\cite{hubbardCalculationPartitionFunctions1959a,stratonovich1957QDistro,stockburgerExactNumberRepresentation2002,moixEquilibriumreducedDensityMatrix2012,chenRigorousStochasticMatrix2014,mccaulPartitionfreeApproachOpen2017c}.
The associated imaginary-time evolution equation for the ordered exponential is
made explicit in Appendix~\ref{sec:appendix_quenched}.

\subsection{Time ordering and interaction-picture dynamics of the coupling operator}
Although $f$ is time independent in the Schr\"odinger picture, the
interaction-picture operator $\tilde{f}(\tau)$ is nontrivial whenever
$[H_Q,f]\neq 0$. Differentiating Eq.~\eqref{eq:ftilde_def} yields
\begin{equation}
    \frac{d}{d\tau}\tilde{f}(\tau) = [H_Q,\tilde{f}(\tau)],
    \qquad \tilde{f}(0)=f,
    \label{eq:ftilde_eom}
\end{equation}
so the ordered product $\mathcal{T}_\tau\tilde{f}(\tau)\tilde{f}(\tau')$ cannot
be reduced to $f^2$ unless $[H_Q,f]=0$. Ordered-exponential calculus and the
resulting commutator hierarchy are standard; see
Refs.~\cite{wilcoxExponentialOperatorsParameter1967a,magnusExponentialSolutionDifferential1954a,blanesMagnusExpansionIts2009}.

\subsection{Adjoint-action expansion and kernel moments}
Define the adjoint action $\mathrm{ad}_{H_Q}(X)=[H_Q,X]$ and its iterates.
The interaction-picture operator admits the exact expansion
\begin{equation}
    \tilde{f}(\tau)=e^{\tau\mathrm{ad}_{H_Q}}(f)
    =\sum_{n=0}^\infty\frac{\tau^n}{n!}\,\mathrm{ad}_{H_Q}^n(f),
    \label{eq:adjoint_expansion}
\end{equation}
which follows from the exponential map for operator adjoint actions
\cite{wilcoxExponentialOperatorsParameter1967a}. Substituting into the bilocal
exponent in Eq.~\eqref{eq:influence_functional} yields the exact algebraic
expansion
\begin{multline}
    \int_0^\beta d\tau\int_0^\beta d\tau'\,K(\tau-\tau')\,
    \tilde{f}(\tau)\tilde{f}(\tau') \\
    = \sum_{n,m=0}^\infty \mu_{nm}\,
    \mathrm{ad}_{H_Q}^n(f)\,\mathrm{ad}_{H_Q}^m(f),
    \label{eq:bilocal_expansion}
\end{multline}
with kernel moments
\begin{equation}
    \mu_{nm} \equiv \frac{1}{n!\,m!}
    \int_0^\beta d\tau\int_0^\beta d\tau'\,\tau^n(\tau')^m K(\tau-\tau').
    \label{eq:kernel_moments}
\end{equation}
This separates bath statistics (in $\mu_{nm}$) from the system operator
algebra. Symmetry properties of $K$ and relations among $\mu_{nm}$ are given in
Appendix~\ref{sec:appendix_kernel}.

\subsection{Exact closure criterion and local construction}
Define the adjoint-generated subspace
\begin{equation}
    \mathcal{A}_f = \mathrm{span}\{\mathrm{ad}_{H_Q}^n(f)\}_{n=0}^\infty.
    \label{eq:Af_def}
\end{equation}
The influence functional in Eq.~\eqref{eq:influence_functional} is an ordered
exponential built from products of elements of $\mathcal{A}_f$. Consequently,
\textbf{an exact local Hamiltonian of mean force exists as a finite operator
polynomial if and only if the associative algebra generated by
$\mathcal{A}_f$ (together with the identity) is finite dimensional and closed
under multiplication}. Equivalently, the Lie algebra generated by
$\mathcal{A}_f$ is finite dimensional, and the corresponding Magnus/BCH series
for the logarithm of the ordered exponential closes within its enveloping
algebra\cite{weiLieAlgebraicSolution1963,magnusExponentialSolutionDifferential1954a,blanesMagnusExpansionIts2009,vanbruntVisser2015BCHClosedForms}.

When the closure condition holds, one may write
\begin{equation}
    \mathcal{T}_\tau\exp\!\left(
        -\frac{1}{2}\int_0^\beta d\tau\int_0^\beta d\tau'\,
        \tilde{f}(\tau)\,K(\tau-\tau')\,\tilde{f}(\tau')
    \right)
    = e^{\Omega},
    \label{eq:Magnus_omega}
\end{equation}
with $\Omega$ a finite linear combination of basis elements in the closed
algebra (Magnus expansion). Combining $e^{-\beta H_Q}$ with $e^{\Omega}$ via the
Baker--Campbell--Hausdorff formula yields
$\bar{\rho}_S=e^{-\beta H_{\mathrm{MF}}}$ with $H_{\mathrm{MF}}$ in the same
operator class. When closure fails, the Magnus/BCH series does not truncate and
any finite local ansatz for $H_{\mathrm{MF}}$ necessarily requires truncation or
projection; this is the only point where approximation enters.






