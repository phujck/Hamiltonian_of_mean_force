\section{Discussion}
\label{sec:discussion}

This manuscript is positioned as the direct sequel to
Ref.~\cite{mccaulMeanForceHamiltoniansInfluence2026}. The earlier paper
established the exact commuting Gaussian benchmark; the present one isolates the
operator-algebraic obstruction that appears once $[H_Q,f]\neq 0$.

We have derived an exact operator reformulation of the reduced equilibrium
object for a Gaussian bath, expressed both as a bilocal imaginary-time
influence functional and as a quenched Gaussian-field average. The nonlocal
structure in imaginary time is traced entirely to noncommutativity between
$H_Q$ and the coupling operator $f$, which forces the interaction-picture
operator $\tilde{f}(\tau)$ to appear in time-ordered products. By expanding
$\tilde{f}(\tau)$ in adjoint actions and isolating kernel moments, we obtained a
purely algebraic representation that yields an exact closure criterion for
locality of $H_{\mathrm{MF}}$.

When the adjoint-generated operator algebra closes, the mean-force Hamiltonian
is a finite operator polynomial and can be constructed via Magnus/BCH. When it
does not, any local representation necessarily involves truncation or
projection; no other approximation is introduced. Broader implications of these
results are deferred to the next stage of the programme: non-Gaussian cumulant
extensions and noncommuting finite-bath benchmarks.

The new designability section (Sec.~\ref{sec:alltoall_designability}) makes the
main constructive consequence explicit for a free-spin core: with a complete
pair-resolved channel basis, finite-$N$ all-to-all 2-local coefficient maps are
full rank and directly invertible. The numerical component there separates two
roles: (i) reconstruction/rank diagnostics for the full all-to-all basis, and
(ii) stochastic free-model benchmarking in a dense $ZZ$ sector, where
reweighting remains controlled enough to compare against exact thermal ED while
also exposing scaling against an MPO/DMRG baseline.

The retained qubit benchmark (Sec.~\ref{sec:numerical_validation}) continues to
serve a different purpose: continuity with the weak/ultrastrong asymptotics of
Ref.~\cite{cresserWeakUltrastrongCoupling2021a}. Together, the two numerical
sections split validation into ``known asymptotic physics recovery'' (qubit) and
``constructive many-body designability from free trajectories'' (all-to-all
construction).

Beyond this, the present formulation makes  an informational interpretation of $H_{\mathrm{MF}}$ unavoidable. From this perspective, the logarithm acts as a \emph{compression map}. The
quenched operator $\bar{\rho}_Q(\beta)=\mathbb E_\xi[U_\xi(\beta)]$ is a mixture over history-conditioned propagators, whereas $\tilde H_{\mathrm{MF}}(\beta)=-(1/\beta)\log\bar{\rho}_Q(\beta)$ is the unique local object whose Gibbs form reproduces that mixture. The gap between ``log of average'' and ``average
of log'' is therefore a measure of \emph{how much information about the bath history is lost} when one insists on a single effective Hamiltonian. This is the same structural gap that distinguishes quenched and annealed free energies in disordered statistical mechanics, where one compares $\langle \log Z \rangle$ to $\log\langle Z\rangle$. In the commuting case this gap disappears entirely. 

