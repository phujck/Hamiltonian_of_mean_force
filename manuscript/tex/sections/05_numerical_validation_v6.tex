% !TEX root = ../main_v2.tex
\section{Numerical Validation of the Exact Theory}
\label{sec:numerical_v6}

Even in this simplest possible case, the physics of the reduced system is startlingly rich. The exact solution of the previous section therefore invites a detailed numerical analysis. With the exact form of the mean-force state, we can precisely characterise the behaviour of the system in different regimes of temperature and bath coupling. 

First, to expose the coupling dependence cleanly, we reintroduce an explicit dimensionless strength $g$ by writing the bath kernel as
\begin{equation}
    K(u)=g^2 K_0(u).
\end{equation}
All response kernels inherit the same quadratic scaling, such that
\begin{equation}
    \Delta_z \to g^2 \Delta_{z},
    \qquad
    \Sigma_\pm \to g^2 \Sigma_{\pm},
    \qquad
    \chi \to g^2 \chi,
\end{equation}
The entire non-perturbative recombination of bare Hamiltonian and influence is therefore controlled by the single dimensionless dressing parameter $\chi$ through
\begin{equation}
    \gamma(\chi)=\frac{\tanh\chi}{\chi}.
    \label{eq:gamma_chi_crossover_v5}
\end{equation}

This apparently innocuous expression encodes the entire non-perturbative recombination of bare Hamiltonian and influence, and functions as an \emph{order} parameter for the crossover between weak and ultrastrong coupling regimes. The reduced equilibrium state transitions between these regimes as $\chi$ passes through unity,
\begin{equation}
    \chi(\beta,g,\theta)\sim 1
    \qquad \Longleftrightarrow \qquad
    g^2 \sim \chi_0(\beta,\theta)^{-1}.
    \label{eq:chi_crossover_condition_v5}
\end{equation}


This crossover condition $\chi(\beta,g,\theta)=1$ therefore
defines a coupling-dependent temperature scale
\begin{equation}
  g_\star(\beta) = \chi_0(\beta,\theta)^{-1/2},
  \label{eq:gstar_def_v6}
\end{equation}
and an equivalent crossover temperature $\beta_\star(g)$.  No approximation has
been made; $g_\star$ is an exact property of the solution.

The function $\gamma(\chi)$ is elementary but worth dwelling on.  It is
monotone decreasing from $\gamma(0)=1$ to $\gamma(\infty)=0$, with curvature
that changes sign near $\chi\approx 1.2$.  In the two limiting regimes:
\begin{align}
  \chi \ll 1:\quad & \gamma \simeq 1 - \tfrac{\chi^2}{3},
  \label{eq:gamma_weak}\\
  \chi \gg 1:\quad & \gamma \simeq \chi^{-1} = (g^2\chi_0)^{-1}.
  \label{eq:gamma_strong}
\end{align}
These two expansions give a quantitative account of the two regimes identified
qualitatively in Ref.~\cite{cresserWeakUltrastrongCoupling2021a}.

\paragraph{Weak-coupling regime, $\chi\ll 1$.}

When $\chi\ll 1$ (small $g$, or high temperature so that $\chi_0(\beta,\theta)$
is small), $\gamma\simeq 1$ and the density matrix
\eqref{eq:rho_matrix_v5} reduces to
\begin{equation}
  \rho_Q \approx \frac{1}{\tilde Z}\begin{pmatrix}
    e^{-a}(1+\Delta_z) & e^{-a}\sqrt{\chi^2-\Delta_z^2}\\
    e^{a}\sqrt{\chi^2-\Delta_z^2} & e^{a}(1-\Delta_z)
  \end{pmatrix} + O(\chi^3),
\end{equation}
with $a=\beta\omega_q/2$.  The off-diagonal entry---the bare-basis coherence---is
$O(g^2)$, arising from $\Sigma_\pm\propto g^2$.  The diagonal elements are
similarly deformed from bare Gibbs at order $g^2$ through $\Delta_z$.  This is
the perturbative dressing familiar from weak-coupling master equations, now
derived exactly with no approximation on the spectral density.

\paragraph{Ultrastrong / low-temperature regime, $\chi\gg 1$.}

When $\chi\gg 1$ (large $g$, or very low temperature where $\chi_0$ grows),
$\tanh\chi\to 1$ and $\gamma\to\chi^{-1}$.  The density matrix is then
dominated by the normalised influence direction $\hat{M}=M/\chi$:
\begin{equation}
  \rho_Q \approx \frac{e^{-\beta H_Q}\!\left(\mathbb{I} + \hat{M}\right)}{Z_Q}.
\end{equation}
The state is sharply polarised along the axis defined by $M$, losing
sensitivity to the bare Gibbs factor beyond what is needed to satisfy the KMS
boundary conditions.  The specific bath spectral weight matters only through the
saturated invariants entering $\chi_0$, not through any detailed frequency
structure.  In this limit the qubit effectively decouples from its own
Hamiltonian and aligns with the interaction direction, the situation described
as the ``ultrastrong'' mean-force fixed point in
Ref.~\cite{cresserWeakUltrastrongCoupling2021a}.

The passage between these regimes is not a phase transition but a smooth
crossover centered at $\chi\approx 1$. For the mixed coupling studied here,
the crossover is marked by two distinct signatures in the system's 
differential response: a peak in the Bloch-angle susceptibility 
$\partial_g\varphi$ at $g_{\mathrm{peak},\varphi}$ and a peak in the 
Bloch-radius gradient $\partial_g r$ at $g_{\mathrm{peak},r}$. 
The location of the observable crossover signatures can be determined 
analytically by differentiating the exact Bloch-vector components 
with respect to $g$. Defining $a = \beta \omega_q / 2$ and the 
dimensionless invariant $\chi = g^2 \chi_0(\beta)$, the components 
$m_x$ and $m_z$ (fixing $\omega_q=1$) are
\begin{align}
  m_x &= \frac{\Sigma \tanh \chi}{\chi \cosh a - \Delta_z \sinh a \tanh \chi}, \\
  m_z &= \frac{\Delta_z \cosh a \tanh \chi - \chi \sinh a}{\chi \cosh a - \Delta_z \sinh a \tanh \chi},
\end{align}
where $\Delta_z = g^2 \Delta_{z0}$ and $\Sigma = g^2 \Sigma_0$. 
The exact observable gradients are then given by
\begin{align}
  \partial_g \varphi &= \frac{2g \Sigma_0 \chi \sinh a\,\mathrm{sech}^2 \chi}{\Sigma^2 \tanh^2 \chi + (\Delta_z \tanh \chi - \chi \tanh a)^2}, 
  \label{eq:dphi_dg_exact} \\
   \partial_g r &= \frac{4g\chi_0 \tanh \chi \,\mathrm{sech}^2 \chi}{r Z_Q^2}.
\end{align}
The exact location of these crossover signatures can be determined by 
setting the derivatives of the gradients to zero. Since the 
$g$-dependence enters solely through the invariant $\chi = (g/g_\star)^2$, 
we express the gradients as functions of $\chi$ and seek their 
maxima. In the high-temperature limit ($a \gg 1$), the condition 
$\frac{d}{dg} (\partial_g \varphi) = 0$ leads to the transcendental 
equation for the susceptibility peak:
\begin{equation}
  1 - 4\chi \tanh \chi + \frac{4\chi c\,\mathrm{sech}^2 \chi}{1 - c\tanh \chi} = 0,
  \label{eq:peak_condition_phi}
\end{equation}
where $c = \Delta_{z0}/\chi_0 \approx 0.7$ for the Ohmic bath at 
$\theta = \pi/4$. Solving this yields $\chi_{\mathrm{peak},\varphi} \approx 0.42$, 
pinning the peak at $g_{\mathrm{peak},\varphi} \approx 0.65\, g_\star(\beta)$. 
Similarly, the purity gradient $\partial_g r$ achieves its maximum when
\begin{equation}
  1 + 2\chi \left[ \coth \chi - 3\tanh \chi + \frac{2c\,\mathrm{sech}^2 \chi}{1 - c\tanh \chi} \right] = 0,
\end{equation}
which is satisfied at $\chi_{\mathrm{peak},r} \approx 0.85$, or 
$g_{\mathrm{peak},r} \approx 0.92\, g_\star(\beta)$. Both conditions confirm that the observable signatures are universal 
multiples of the bath response scale $g_\star(\beta) = \chi_0^{-1/2}$. 
Indeed, by rescaling the coupling as $\xi = g/g_\star(\beta)$, both 
crossover signatures are pinned to fixed universal values of the 
invariant $\chi$, aligning the horizontal peak positions for all 
temperatures (Fig.~\ref{fig:chi_theory}c,d). This demonstrates that 
the mean-force crossover is fundamentally governed by the accumulated 
bath response scale regardless of the environment statistics.

\begin{figure}[t]
  \centering
  \includegraphics[width=0.48\textwidth]{../figures/hmf_fig1_chi_theory.png}
  \caption{Exact $\chi$-crossover theory for the Ohmic bath~\eqref{eq:J_ohmic_v6} ($\theta=\pi/4$).
    \textit{Panel~(a)}: Universal crossover function $\gamma(\chi)=\tanh\chi/\chi$
    between the weak-coupling ($1-\chi^2/3$) and ultrastrong ($1/\chi$) limits.
    \textit{Panel~(b)}: Map of $\chi(\beta,g)=g^2\chi_0(\beta)$ showing the 
    low-T divergence of the crossover scale. The white curve marks 
    the exact center $g_\star(\beta)=\chi_0(\beta)^{-1/2}$.
    \textit{Panels~(c,d)}: Observable signatures of the crossover in the 
    Bloch-angle susceptibility $\partial_g\varphi$ and Bloch-radius 
    gradient $\partial_g r$ plotted against the scaled coupling 
    $g/g_\star(\beta)$. Rescaling the coupling aligns the peak positions 
    for all temperatures at the same horizontal markers. Shaded regions 
    demarcate the weak (blue) and strong (orange) coupling regimes, 
    with filled circles indicating the analytical peak locations.}
  \label{fig:chi_theory}
\end{figure}


To make these predictions concrete we commit to a specific bath that 
permits exact analytic evaluation of all quantities. We choose a windowed Ohmic spectral density
\begin{equation}
  J(\omega) = g^2\,\omega\,e^{-\omega/\omega_c},
  \qquad \omega\in[0, 2\omega_q],
  \label{eq:J_ohmic_v6}
\end{equation}
with cutoff $\omega_c=5\omega_q$. The UV window $[0, 2\omega_q]$ is 
chosen to match typical numerical simulations where the bath state 
is truncated, while the exponential cutoff regularises the high-frequency 
tail. Having specified the bath, we further specialise simulations to $\theta=\pi/4$ and $\omega_q =1$, as all other behaviour follows from the universal scaling. Figure~\ref{fig:chi_theory} shows the exact $\chi$-crossover theory for the Ohmic bath~\eqref{eq:J_ohmic_v6} ($\theta=\pi/4$). With these analytic predictions in hand, we now turn to comparison with numerical approximations. 

