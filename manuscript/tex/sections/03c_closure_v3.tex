% !TEX root = ../main_v2.tex
\section{Closure of the Hamiltonian of mean force \label{sec:closure}}
The obstruction to writing $H_{\mathrm{MF}}(\beta)$ in a compact operator form is not the
existence of the mean-force object (it is defined by a logarithm), but the \emph{representability} of that logarithm inside a restricted operator family (few-body, local, Pauli strings, etc.). In the harmonic case this representability question reduces to a precise closure problem. To show this, we first write the quenched propagator in the imaginary-time interaction picture with respect to $H_Q$:
\begin{equation}
    \begin{split}
        U_\xi(\beta) & = e^{-\beta H_Q} W_\xi(\beta),                                                             \\
        W_\xi(\beta) & \equiv \mathcal T_\tau \exp\!\left[-\int_0^\beta d\tau\, \xi(\tau)\,\tilde f(\tau)\right],
    \end{split}
    \label{eq:Wxi_def}
\end{equation}
where $\tilde f(\tau)\equiv e^{\tau H_Q} f e^{-\tau H_Q}$.

Because the noise is classical Gaussian with zero mean, the average of the
time-ordered exponential resums exactly in terms of the second cumulant. Time ordering adds a non-trivial complication, which is addressed in Appendix~\ref{app:gaussian_averaging}. The result may be succinctly expressed in terms of an \emph{influence operator}
\begin{equation}
    \begin{split}
          \langle W_\xi(\beta)\rangle_\xi &\equiv \bar W(\beta) = \exp\!\big(\Delta(\beta)\big), \\
        \bar\rho_Q(\beta) &= \langle U_\xi(\beta)\rangle_\xi = e^{-\beta H_Q}\,e^{\Delta(\beta)},
    \end{split}
    \label{eq:rho_bar_product_form}
\end{equation}
where the influence exponent $\Delta(\beta)$ is defined in accordance with Eq.\eqref{eq:Wbar_cumulant_T}. 
It is convenient to perform a final massaging of this exponent to remove the explicit time ordering, the details of which we relegate to Appendix~\ref{app:gaussian_averaging}. After this manipulation, $\Delta(\beta)$ takes the form
\begin{equation}
\Delta(\beta)\equiv \int_0^\beta d\tau\int_0^\tau d\tau'\,
K(\tau-\tau')\,\tilde f(\tau)\tilde f(\tau').
\label{eq:Delta_def_hermitian}
\end{equation}

To progress, we introduce the adjoint chain generated by $H_Q$ acting on the coupling,
\begin{equation}
    f_n \equiv \mathrm{ad}_{H_Q}^n(f),\qquad \mathrm{ad}_{H_Q}(A)\equiv[H_Q,A],
    \label{eq:adjoint_chain_def}
\end{equation}
so that the interaction-picture operator has the exact series
\begin{equation}
    \tilde f(\tau)=e^{\tau H_Q} f e^{-\tau H_Q}=\sum_{n\ge 0}\frac{\tau^n}{n!}\, f_n.
    \label{eq:ftilde_ad_series}
\end{equation}
Substituting \eqref{eq:ftilde_ad_series} into \eqref{eq:Delta_def_hermitian} yields
\begin{equation}
    \Delta(\beta)
    =\sum_{n,m\ge 0} C^{>}_{nm}(\beta)f_nf_m,
    \label{eq:Delta_Cnm_fnfm_no_matsubara}
\end{equation}
where we have further defined an \emph{ordered kernel-moment matrix}
\begin{equation}
    C^{>}_{nm}(\beta)\equiv
    \int_0^\beta d\tau\int_0^\tau d\tau'\,
    K(\tau-\tau')\,\frac{\tau^n}{n!}\,\frac{\tau'^m}{m!}.
    \label{eq:Cnm_def_triangle}
\end{equation}
Here the notation $>$ is used to track the triangular domain of integration, where $\tau$ (associated with $n$) is the upper limit of integration for $\tau'$ (associated with $m$).

With this definition, all bath/temperature dependence enters $\Delta(\beta)$ only through the scalar moments $C^{>}_{nm}(\beta)$, while all operator structure is carried by the adjoint chain $\{f_n\}$. The final step is to re-express the averaged propagator in a \emph{single} exponential. Using
\eqref{eq:rho_bar_product_form} we write
\begin{equation}
    \bar\rho_Q(\beta)=e^{-\beta H_Q}\,e^{\Delta(\beta)} \equiv e^{-\beta H_{\mathrm{MF}}(\beta)},
    \label{eq:HMF_BCH_def_clean_repeat}
\end{equation}
where the mean-force Hamiltonian satisfies
\begin{equation}
    -\beta H_{\mathrm{MF}}(\beta)=\log\!\left(e^{-\beta H_Q}e^{\Delta(\beta)}\right).
\end{equation}
Let $A\equiv-\beta H_Q$ and $B\equiv \Delta(\beta)$. The Baker--Campbell--Hausdorff (BCH) series gives
\begin{equation}
    \label{eq:BCH_AB_general}
    \begin{split}
        \log(e^{A}e^{B}) & = A+B+\frac{1}{2}[A,B]                                      \\
                         & \quad +\frac{1}{12}[A,[A,B]]+\frac{1}{12}[B,[B,A]]+\cdots .
    \end{split}
\end{equation}
Dividing by $-\beta$ and using $A=-\beta H_Q$ yields the expansion
\begin{equation}
    \label{eq:HMF_BCH_explicit}
    \begin{split}
        H_{\mathrm{MF}}(\beta) & = H_Q-\frac{1}{\beta}\Delta(\beta)
        +\frac{1}{2}[H_Q,\Delta(\beta)]                                                 \\
                               & \quad -\frac{\beta}{12}[H_Q,[H_Q,\Delta(\beta)]]       \\
                               & \quad +\frac{1}{12}[\Delta(\beta),[\Delta(\beta),H_Q]]
        +\cdots .
    \end{split}
\end{equation}
This expansion is our main result - an exact expression for the mean-force Hamiltonian in terms of the exact influence operator. Expressed in this way, the closure content of $H_{\mathrm{MF}}(\beta)$ is transparent: all terms are built from repeated commutators with $H_Q$
acting on $\Delta$, together with higher BCH terms involving commutators among those objects. 

To see the explicit generator, introduce the adjoint superoperator $\mathrm{ad}_{H_Q}(\cdot)\equiv[H_Q,\cdot]$.
Then \eqref{eq:HMF_BCH_explicit} becomes
\begin{equation}
    \label{eq:HMF_ad_form}
    \begin{split}
        H_{\mathrm{MF}}(\beta) & = H_Q-\frac{1}{\beta}\Delta
        +\frac{1}{2}\,\mathrm{ad}_{H_Q}\Delta                                       \\
                               & \quad -\frac{\beta}{12}\,\mathrm{ad}_{H_Q}^2\Delta
        +\frac{1}{12}\,[\Delta,\mathrm{ad}_{H_Q}\Delta]
        +\cdots .
    \end{split}
\end{equation}

Finally, because $\Delta(\beta)$ is quadratic in the adjoint chain, each commutator with $H_Q$ simply shifts indices:
\begin{equation}
    \label{eq:adHQ_on_products}
    \mathrm{ad}_{H_Q}(f_n f_m)=[H_Q,f_n f_m]=f_{n+1}f_m+f_n f_{m+1}.
\end{equation}
Hence the first commutator term takes the explicit form
\begin{equation}
    \label{eq:adHQ_Delta_explicit}
    [H_Q,\Delta(\beta)]
    =\sum_{n,m\ge0}C^{>}_{nm}(\beta)\,\big(f_{n+1}f_m+f_n f_{m+1}\big),
\end{equation}
and higher $\mathrm{ad}_{H_Q}^k\Delta$ generate the same family of products $f_a f_b$ with shifted indices.
The remaining BCH terms (e.g.\ $[\Delta,\mathrm{ad}_{H_Q}\Delta]$) introduce commutators among these quadratic
elements, and are the sole source of additional operator growth beyond the quadratic span.
The only remaining question is whether the operator family generated by \eqref{eq:HMF_ad_form} can be brought to a closed form.

The preceding identities imply that the entire BCH commutator tower \emph{linear} in $\Delta$ remains within the quadratic span $\mathrm{span}\{f_nf_m\}$. Terms to this order explicitly admit a closed resummation. Let $A\equiv-\beta H_Q$ and $B\equiv \Delta(\beta)$. Then the BCH logarithm satisfies the standard
linear-in-$B$ identity
\begin{equation}
    \label{eq:BCH_linear_resum}
    \begin{split}
        \log\!\left(e^{A}e^{B}\right) & = A+\Phi\!\big(\mathrm{ad}_{A}\big)\,B+O(B^2), \\
        \Phi(x)                       & \equiv \frac{x}{1-e^{-x}}.
    \end{split}
\end{equation}
Substituting $A=-\beta H_Q$ and dividing by $-\beta$ gives the mean-force Hamiltonian
to first order in $\Delta$:
\begin{equation}
    \label{eq:HMF_linear_resum}
    H_{\mathrm{MF}}(\beta) = H_Q - \frac{1}{\beta}\,\Phi\!\big(\beta\,\mathrm{ad}_{H_Q}\big)\,\Delta(\beta)+O(\Delta^2).
\end{equation}
Expanding $\Phi$ yields the explicit commutator series with Bernoulli numbers $B_k$,
\begin{equation}
    \label{eq:HMF_linear_series}
    \begin{split}
        H_{\mathrm{MF}}(\beta) & = H_Q - \frac{1}{\beta}\sum_{k\ge 0}\frac{B_k}{k!}(-\beta)^k\,\mathrm{ad}_{H_Q}^k\Delta(\beta) + O(\Delta^2), \\
        \Phi(x)                & = \sum_{k\ge 0}\frac{B_k}{k!}(-x)^k.
    \end{split}
\end{equation}
Using \eqref{eq:adHQ_on_products} repeatedly, the iterated commutators admit the closed binomial form
\begin{equation}
    \label{eq:adHQk_on_products}
    \mathrm{ad}_{H_Q}^{k}(f_n f_m)
    =\sum_{j=0}^{k}\binom{k}{j}\, f_{n+j}\, f_{m+k-j},
\end{equation}
so every term in \eqref{eq:HMF_linear_series} is explicitly a linear combination of products $f_af_b$
with scalar coefficients determined solely by $C^{>}_{nm}(\beta)$ and universal combinatorics. Because of this, we may bring the first order contribution into a particularly compact form by introducing the discrete shift operator $\mathsf D$. This will act on the moment matrix $C^{>}$ as
\begin{equation}
    \label{eq:D_operator_on_C}
    \begin{split}
        (\mathsf D C^{>})_{ab} & \equiv C^{>}_{a-1,b}+C^{>}_{a,b-1},                    \\
        C^{>}_{rs}             & \equiv 0\quad \text{if any index is negative}.
    \end{split}
\end{equation}
Then \eqref{eq:adHQ_on_products} implies, after reindexing,
\begin{equation}
    \label{eq:adHQ_Delta_as_D_on_C}
    \mathrm{ad}_{H_Q}\Delta(\beta)
    =\sum_{a,b\ge0}(\mathsf D C^{>}(\beta))_{ab}\, f_a f_b,
\end{equation}
and by iteration,
\begin{equation}
    \label{eq:adHQk_Delta_as_Dk_on_C}
    \mathrm{ad}_{H_Q}^{k}\Delta(\beta)
    =\sum_{a,b\ge0}(\mathsf D^{k} C^{>}(\beta))_{ab}\, f_a f_b.
\end{equation}
Equivalently, $(\mathsf D^{k}C^{>})_{ab}=\sum_{j=0}^{k}\binom{k}{j}\,C^{>}_{a-j,\;b-(k-j)}$, which is the binomial
index-shift structure implied by \eqref{eq:adHQk_on_products}. From this we obtain the following form for $H_{\mathrm{MF}}(\beta)$ to linear order in $\Delta$:
\begin{equation}
    \begin{split}
        H_{\mathrm{MF}}(\beta) &= H_Q-\frac{1}{\beta}\sum_{a,b\ge0}\,\widetilde C^{>}_{ab}(\beta)\, f_a f_b \\
        &\quad + O(\Delta^2),
    \end{split}
    \label{eq:HMF_linear_quadratic_final}
\end{equation}
where the \emph{renormalised moment matrix} $\widetilde C^{>}(\beta)$ is the universal transform
\begin{equation}
    \label{eq:Ctilde_def}
    \widetilde C^{>}(\beta) \equiv \Phi\!\big(\beta\,\mathsf D\big)\,C^{>}(\beta)
    = \sum_{k\ge0}\frac{B_k}{k!}(-\beta)^k\,\mathsf D^{k}C^{>}(\beta).
\end{equation}
Thus, to linear order in the influence operator, the mean-force Hamiltonian is \emph{exactly} a quadratic
form in the products $f_af_b$, with all bath/temperature dependence entering only through the scalar matrix
$\widetilde C^{>}_{ab}(\beta)$.
 Any operator growth beyond the quadratic span is confined to the nonlinear BCH sector $O(\Delta^2)$ (e.g.\
$[\Delta,\mathrm{ad}_{H_Q}\Delta]$). While these terms are more difficult to handle as they involve commutators of quadratic forms, they nevertheless admit a precise series expansion via the full BCH formula \eqref{eq:BCH_AB_general}. With this form in hand, we can now state a theorem for the representability of $H_{\mathrm{MF}}(\beta)$ in a given operator family $\mathcal A$ as a \emph{closure criterion}.

\subsection{Theorem: Finite representability of mean force}

Fix a target operator family $\mathcal A$ (e.g.\ $k$-local Pauli strings, a Lie algebra plus identity,
or a finite-dimensional associative operator algebra) in which we seek to represent $H_{\mathrm{MF}}(\beta)$.
The Gaussian construction above allows us to formulate the exact criteria for representability as a closure theorem.
In a physical sense, these conditions determine whether a valid \emph{local generator} can be defined for the system at all;
failure of closure implies that the effective Hamiltonian is fundamentally non-local with respect to the algebra $\mathcal A$.

\vspace{0.5em}
\noindent\textbf{Theorem (Closure of the Mean Force).}
\textit{Let $\mathcal A$ be a target operator space containing the bare system Hamiltonian $H_Q$ and coupling $f$. The Hamiltonian of Mean Force $H_{\mathrm{MF}}(\beta)$ admits a finite representation within $\mathcal A$ if and only if the following three closure conditions are satisfied:}

\paragraph{(C1) Adjoint Closure (No linear operator growth).}
The subspace generated by repeated commutators of the coupling $f$ with $H_Q$ must remain inside $\mathcal A$:
\begin{equation}
    \begin{split}
        f & \in\mathcal A,\qquad \mathrm{ad}_{H_Q}(\mathcal A)\subseteq \mathcal A,     \\
          & \text{equivalently}\quad \mathrm{span}\{f_n\}_{n\ge 0}\subseteq \mathcal A.
    \end{split}
    \label{eq:C1_adjoint_closure}
\end{equation}

\paragraph{(C2) Quadratic Closure (No growth in the Gaussian sector).}
Because the exact influence operator $\Delta(\beta)$ is quadratic in the adjoint chain,
the quadratic span generated by $\{f_n\}$ must also lie in $\mathcal A$:
\begin{equation}
    f_n f_m \in \mathcal A\quad \text{for all indices that contribute in $\Delta(\beta)$,}
    \label{eq:C2_product_closure}
\end{equation}
so that $\Delta(\beta)\in\mathcal A$. A sufficient (and common) condition is that $\mathcal A$ is an
associative algebra containing the identity and closed under multiplication.

\paragraph{(C3) BCH Closure (No growth from commutators among quadratic elements).}
To ensure that the \emph{full} BCH logarithm $\log(e^{-\beta H_Q}e^{\Delta})$ remains in $\mathcal A$,
the commutators generated among the quadratic elements must also close in $\mathcal A$; in particular
\begin{equation}
    [\Delta(\beta),\,\mathrm{ad}_{H_Q}^k\Delta(\beta)]\in\mathcal A\qquad \text{for all $k\ge 0$,}
    \label{eq:C3_BCH_closure}
\end{equation}
and similarly for the higher nested commutators appearing in the BCH series.
A sufficient (and common) condition is again that $\mathcal A$ is a finite-dimensional associative algebra
(or a Lie algebra containing $\Delta$ and closed under commutators), in which case all BCH commutators remain
in $\mathcal A$ by construction.

\paragraph{Consequence (closed-form mean-force Hamiltonian).}
If (C1)--(C3) hold, then $\Delta(\beta)\in\mathcal A$ and all BCH commutators generated by
\eqref{eq:HMF_BCH_def_final} remain in $\mathcal A$. Since $H_Q\in\mathcal A$ by assumption, it follows that
\begin{equation}
    \label{eq:HMF_BCH_def_final}
    \begin{split}
        \bar\rho_Q(\beta)             & = e^{-\beta H_{\mathrm{MF}}(\beta)},                  \\
        -\beta H_{\mathrm{MF}}(\beta) & = \log\!\left(e^{-\beta H_Q}e^{\Delta(\beta)}\right),
    \end{split}
\end{equation}
defines an $H_{\mathrm{MF}}(\beta)\in\mathcal A$ and hence a closed-form representation of
$H_{\mathrm{MF}}(\beta)$ (up to the usual additive scalar fixed by $Z_X$). If either (C1) fails (adjoint-chain growth) or (C2)--(C3) fail (growth in the quadratic/BCH sector),
then $\Delta(\beta)$ and/or the BCH commutator tower generates operators outside $\mathcal A$; any representation within $\mathcal A$ is necessarily truncated or projected.

This statement leads to an important corollary. For continuous variable systems $H_Q = p^2/2m + V(x)$, if the potential $V(x)$ is a polynomial of degree greater than 2, the adjoint chain $\{f_n\}$ generally does not close in any finite-dimensional Lie algebra; instead, it generates the full universal enveloping algebra. Consequently, no finite representation of the mean force exists. This obstruction shares a common origin with the ordering ambiguity in quantisation (Groenewold-Van Hove theorem): the failure to close the operator algebra corresponds to the impossibility of consistently mapping the quantum evolution of non-quadratic potentials to a finite classical phase space flow. This result extends the impossibility: for such potentials, there is not even a \emph{local operator} which can describe the equilibrium state exactly. 

From a practical standpoint, the closure of $\mathcal A$ is not necessary for an arbitrarily accurate approximation of $H_{\mathrm{MF}}(\beta)$, as the BCH logarithm $\log(e^{-\beta H_Q}e^\Delta)$ still yields a controlled expansion. The operator content is generated by the adjoint chain and its quadratic products, while all bath dependence remains scalar through $C^{>}_{nm}(\beta)$. One may therefore truncate (in commutator depth, locality
class, or operator weight) without introducing any approximation on the bath side.

This algebraic perspective offers a unifying view of strong-coupling approximations. Numerical schemes such as the polaron transformation~\cite{weissQuantumDissipativeSystems2012}, hierarchical equations of motion~\cite{tanimuraReducedHierarchicalEquations2014}, and chain-mapping/DMRG techniques can be understood as distinct choices of the target operator family $\mathcal{A}$. The polaron frame corresponds to dressing $\mathcal{A}$ with coherent displacements; HEOM truncates the memory depth of the influence functional (dual to the operator degree in the adjoint chain), and chain mappings truncate the spatial extent of the harmonic bath. The `closure' problem is thus equivalent to identifying a low-dimensional subalgebra that approximately contains the logarithm of the quenched propagator.
