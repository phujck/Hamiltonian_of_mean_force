\section{Closure of the Hamiltonian of mean force \label{sec:closure}}
The obstruction to writing $H_{\mathrm{MF}}(\beta)$ in a compact operator form is not the
existence of the mean-force object (it is defined by a logarithm), but the \emph{representability} of that logarithm inside a restricted operator family (few-body, local, Pauli strings, etc.). In the harmonic case this representability question reduces to a precise closure problem. To show this, we first write the quenched propagator in the imaginary-time interaction picture with respect to $H_Q$:
\begin{equation}
    \begin{split}
        U_\xi(\beta) & = e^{-\beta H_Q} W_\xi(\beta),                                                             \\
        W_\xi(\beta) & \equiv \mathcal T_\tau \exp\!\left[-\int_0^\beta d\tau\, \xi(\tau)\,\tilde f(\tau)\right],
    \end{split}
    \label{eq:Wxi_def}
\end{equation}
where $\tilde f(\tau)\equiv e^{\tau H_Q} f e^{-\tau H_Q}$.

Because the noise is classical Gaussian with zero mean, the average of the
time-ordered exponential resums exactly in terms of the second cumulant,
\begin{equation}
    \begin{split}
        \bar W(\beta) & \equiv \langle W_\xi(\beta)\rangle_\xi                                                      \\
                      & =\mathcal T_\tau \exp\bigg[\frac12\int_0^\beta d\tau\int_0^\beta d\tau'                     \\
                      & \quad \times K(\tau-\tau')\,\mathcal T_\tau\!\big(\tilde f(\tau)\tilde f(\tau')\big)\bigg].
    \end{split}
    \label{eq:Wbar_cumulant_T}
\end{equation}
For two operators,
\(
\mathcal T_\tau\!\big(\tilde f(\tau)\tilde f(\tau')\big)
=\theta(\tau-\tau')\tilde f(\tau)\tilde f(\tau')
+\theta(\tau'-\tau)\tilde f(\tau')\tilde f(\tau)
\),
so using the kernel symmetry $K(\tau-\tau')=K(\tau'-\tau)$ on $[0,\beta]$ the square
domain reduces to a single ordered triangle with an \emph{anticommutator}:
\begin{equation}
    \frac12\!\int_0^\beta\! d\tau\!\int_0^\beta\! d\tau'\,
    K(\tau-\tau')\,\mathcal T_\tau\!\big(\tilde f(\tau)\tilde f(\tau')\big)
    =\int_0^\beta\! d\tau\!\int_0^\tau\! d\tau'\,
    K(\tau-\tau')\,\frac12\{\tilde f(\tau),\tilde f(\tau')\}.
    \label{eq:Delta_anticommutator}
\end{equation}
Thus the influence exponent may be taken as the manifestly Hermitian operator
\begin{equation}
    \Delta(\beta)\equiv \int_0^\beta d\tau\int_0^\tau d\tau'\,
    K(\tau-\tau')\,\frac12\{\tilde f(\tau),\tilde f(\tau')\},
    \qquad
    \bar W(\beta)=\mathcal T_\tau e^{\Delta(\beta)}.
    \label{eq:Delta_def_hermitian}
\end{equation}
Expanding $\tilde f(\tau)=\sum_{n\ge0}\tau^n f_n/n!$ with $f_n=\mathrm{ad}_{H_Q}^n(f)$ then gives
\begin{equation}
    \Delta(\beta)=\sum_{n,m\ge0} C_{nm}(\beta)\,\frac12\{f_n,f_m\},
    \qquad
    C_{nm}(\beta)=\int_0^\beta d\tau\int_0^\tau d\tau'\,
    K(\tau-\tau')\,\frac{\tau^n\tau'^m}{n!m!}.
    \label{eq:Delta_Cnm_Jordan}
\end{equation}
The stochastic term can now be averaged exactly. In the harmonic case we have $\langle \xi(\tau)\xi(\tau')\rangle = K(\tau-\tau')$, which gives the exact Euclidean influence-functional form
\begin{equation}
    \bar W(\beta) \equiv \langle W_\xi(\beta)\rangle_\xi =\mathcal T_\tau e^{\Delta(\beta)}
    \label{eq:Wbar_influence}
\end{equation}
where the exponent $\Delta$ is given by
\begin{equation}
    \Delta(\beta)\equiv
    \frac12\iint_0^\beta d\tau d\tau'\,
    K(\tau-\tau')\,\tilde f(\tau)\,\tilde f(\tau').
    \label{eq:Delta_def}
\end{equation}

\begin{equation}
    \bar W(\beta)=\exp\!\big(\Delta(\beta)\big),\qquad
    \bar\rho_Q(\beta)=\langle U_\xi(\beta)\rangle_\xi = e^{-\beta H_Q}\,e^{\Delta(\beta)}.
    \label{eq:rho_bar_product_form}
\end{equation}

To progress, we introduce the adjoint chain generated by $H_Q$ acting on the coupling,
\begin{equation}
    f_n \equiv \mathrm{ad}_{H_Q}^n(f),\qquad \mathrm{ad}_{H_Q}(A)\equiv[H_Q,A],
    \label{eq:adjoint_chain_def}
\end{equation}
so that the interaction-picture operator has the exact series
\begin{equation}
    \tilde f(\tau)=e^{\tau H_Q} f e^{-\tau H_Q}=\sum_{n\ge 0}\frac{\tau^n}{n!}\, f_n.
    \label{eq:ftilde_ad_series}
\end{equation}
Substituting \eqref{eq:ftilde_ad_series} into \eqref{eq:Delta_def_ordered_triangle} yields
\begin{equation}
    \Delta(\beta)
    =\sum_{n,m\ge 0} C_{nm}(\beta)\, f_n f_m,
    \label{eq:Delta_Cnm_fnfm_no_matsubara}
\end{equation}
with the \emph{kernel-moment matrix}
\begin{equation}
    C_{nm}(\beta)\equiv
    \int_0^\beta d\tau\int_0^\tau d\tau'\,
    K(\tau-\tau')\,\frac{\tau^n}{n!}\,\frac{\tau'^m}{m!}.
    \label{eq:Cnm_def_triangle}
\end{equation}
Thus all bath/temperature dependence enters $\Delta(\beta)$ only through the scalar moments
$C_{nm}(\beta)$, while all operator structure is carried by the adjoint chain $\{f_n\}$.


The final step is to re-express the averaged propagator in a \emph{single} exponential. Using
\eqref{eq:rho_bar_product_form} we write
\begin{equation}
    \bar\rho_Q(\beta)=e^{-\beta H_Q}\,e^{\Delta(\beta)} \equiv e^{-\beta H_{\mathrm{MF}}(\beta)},
    \label{eq:HMF_BCH_def_clean_repeat}
\end{equation}
where the mean-force Hamiltonian satisfies
\begin{equation}
    -\beta H_{\mathrm{MF}}(\beta)=\log\!\left(e^{-\beta H_Q}e^{\Delta(\beta)}\right).
\end{equation}
Let $A\equiv-\beta H_Q$ and $B\equiv \Delta(\beta)$. The Baker--Campbell--Hausdorff (BCH) series gives
\begin{equation}
    \label{eq:BCH_AB_general}
    \begin{split}
        \log(e^{A}e^{B}) & = A+B+\frac{1}{2}[A,B]                                      \\
                         & \quad +\frac{1}{12}[A,[A,B]]+\frac{1}{12}[B,[B,A]]+\cdots .
    \end{split}
\end{equation}
Dividing by $-\beta$ and using $A=-\beta H_Q$ yields the expansion
\begin{equation}
    \label{eq:HMF_BCH_explicit}
    \begin{split}
        H_{\mathrm{MF}}(\beta) & = H_Q-\frac{1}{\beta}\Delta(\beta)
        +\frac{1}{2}[H_Q,\Delta(\beta)]                                                 \\
                               & \quad -\frac{\beta}{12}[H_Q,[H_Q,\Delta(\beta)]]       \\
                               & \quad +\frac{1}{12}[\Delta(\beta),[\Delta(\beta),H_Q]]
        +\cdots .
    \end{split}
\end{equation}
This makes the closure content transparent: all terms are built from repeated commutators with $H_Q$
acting on $\Delta$, together with higher BCH terms involving commutators among those objects.

To see the explicit generator, introduce the adjoint superoperator $\mathrm{ad}_{H_Q}(\cdot)\equiv[H_Q,\cdot]$.
Then \eqref{eq:HMF_BCH_explicit} becomes
\begin{equation}
    \label{eq:HMF_ad_form}
    \begin{split}
        H_{\mathrm{MF}}(\beta) & = H_Q-\frac{1}{\beta}\Delta
        +\frac{1}{2}\,\mathrm{ad}_{H_Q}\Delta                                       \\
                               & \quad -\frac{\beta}{12}\,\mathrm{ad}_{H_Q}^2\Delta
        +\frac{1}{12}\,[\Delta,\mathrm{ad}_{H_Q}\Delta]
        +\cdots .
    \end{split}
\end{equation}

Finally, because $\Delta(\beta)$ is quadratic in the adjoint chain,
\begin{equation}
    \label{eq:Delta_quadratic_chain_repeat}
    \Delta(\beta)=\frac12\sum_{n,m\ge0}C_{nm}(\beta)\,f_n f_m,
    \qquad f_n\equiv\mathrm{ad}_{H_Q}^n(f),
\end{equation}
each commutator with $H_Q$ simply shifts indices:
\begin{equation}
    \label{eq:adHQ_on_products}
    \mathrm{ad}_{H_Q}(f_n f_m)=[H_Q,f_n f_m]=f_{n+1}f_m+f_n f_{m+1}.
\end{equation}
Hence the first commutator term takes the explicit form
\begin{equation}
    \label{eq:adHQ_Delta_explicit}
    [H_Q,\Delta(\beta)]
    =\frac12\sum_{n,m\ge0}C_{nm}(\beta)\,\big(f_{n+1}f_m+f_n f_{m+1}\big),
\end{equation}
and higher $\mathrm{ad}_{H_Q}^k\Delta$ generate the same family of products $f_a f_b$ with shifted indices.
The remaining BCH terms (e.g.\ $[\Delta,\mathrm{ad}_{H_Q}\Delta]$) introduce commutators among these quadratic
elements, and are the sole source of additional operator growth beyond the quadratic span.
The only remaining question is whether the operator family generated by \eqref{eq:Delta_Cnm_fnfm} can be brought to a closed form, and re-expressed in a single exponential.

\paragraph{Resummation of the BCH series linear in $\Delta$.}
The preceding identities imply that the entire BCH commutator tower \emph{linear} in $\Delta$ remains
within the quadratic span $\mathrm{span}\{f_nf_m\}$. In fact, these terms admit a closed resummation.
Let $A\equiv-\beta H_Q$ and $B\equiv \Delta(\beta)$. Then the BCH logarithm satisfies the standard
linear-in-$B$ identity
\begin{equation}
    \label{eq:BCH_linear_resum}
    \begin{split}
        \log\!\left(e^{A}e^{B}\right) & = A+\Phi\!\big(\mathrm{ad}_{A}\big)\,B+O(B^2), \\
        \Phi(x)                       & \equiv \frac{x}{1-e^{-x}}.
    \end{split}
\end{equation}
Substituting $A=-\beta H_Q$ and dividing by $-\beta$ gives the mean-force Hamiltonian
to first order in $\Delta$:
\begin{equation}
    \label{eq:HMF_linear_resum}
    \begin{split}
        H_{\mathrm{MF}}(\beta) & = H_Q                                                                                        \\
                               & \quad -\frac{1}{\beta}\,\Phi\!\big(\beta\,\mathrm{ad}_{H_Q}\big)\,\Delta(\beta)+O(\Delta^2).
    \end{split}
\end{equation}
Expanding $\Phi$ yields the explicit commutator series with Bernoulli numbers $B_k$,
\begin{equation}
    \label{eq:HMF_linear_series}
    \begin{split}
        H_{\mathrm{MF}}(\beta) & = H_Q - \frac{1}{\beta}\sum_{k\ge 0}\frac{B_k}{k!}(-\beta)^k\,\mathrm{ad}_{H_Q}^k\Delta(\beta) + O(\Delta^2), \\
        \Phi(x)                & = \sum_{k\ge 0}\frac{B_k}{k!}(-x)^k.
    \end{split}
\end{equation}
Using \eqref{eq:adHQ_on_products} repeatedly, the iterated commutators admit the closed binomial form
\begin{equation}
    \label{eq:adHQk_on_products}
    \mathrm{ad}_{H_Q}^{k}(f_n f_m)
    =\sum_{j=0}^{k}\binom{k}{j}\, f_{n+j}\, f_{m+k-j},
\end{equation}
so every term in \eqref{eq:HMF_linear_series} is explicitly a linear combination of products $f_af_b$
with scalar coefficients determined solely by $C_{nm}(\beta)$ and universal combinatorics.

Because $\Delta(\beta)$ is already a quadratic form in the adjoint-chain products,
the entire BCH contribution \emph{linear} in $\Delta$ can be written explicitly in the same basis
$\{f_af_b\}$, with no remaining reference to $\Delta$ itself. To do so, we introduce the discrete shift operator $\mathsf D$ acting on the moment matrix $C$ by
\begin{equation}
    \label{eq:D_operator_on_C}
    \begin{split}
        (\mathsf D C)_{ab} & \equiv C_{a-1,b}+C_{a,b-1},                    \\
        C_{rs}             & \equiv 0\quad \text{if any index is negative}.
    \end{split}
\end{equation}
Then \eqref{eq:adHQ_on_products} implies, after reindexing,
\begin{equation}
    \label{eq:adHQ_Delta_as_D_on_C}
    \mathrm{ad}_{H_Q}\Delta(\beta)
    =\frac12\sum_{a,b\ge0}(\mathsf D C(\beta))_{ab}\, f_a f_b,
\end{equation}
and by iteration,
\begin{equation}
    \label{eq:adHQk_Delta_as_Dk_on_C}
    \mathrm{ad}_{H_Q}^{k}\Delta(\beta)
    =\frac12\sum_{a,b\ge0}(\mathsf D^{k} C(\beta))_{ab}\, f_a f_b.
\end{equation}
Equivalently, $(\mathsf D^{k}C)_{ab}=\sum_{j=0}^{k}\binom{k}{j}\,C_{a-j,\;b-(k-j)}$, which is the binomial
index-shift structure implied by \eqref{eq:adHQk_on_products}. From this we obtain the following form for $H_{\mathrm{MF}}(\beta)$ to linear order in $\Delta$:
\begin{equation}
    \label{eq:HMF_linear_quadratic_final}
    H_{\mathrm{MF}}(\beta) = H_Q-\frac{1}{2\beta}\sum_{a,b\ge0}\,\widetilde C_{ab}(\beta)\, f_a f_b \;+\; O(\Delta^2),
\end{equation}
where the \emph{renormalised moment matrix} $\widetilde C(\beta)$ is the universal transform
\begin{equation}
    \label{eq:Ctilde_def}
    \widetilde C(\beta) \equiv \Phi\!\big(\beta\,\mathsf D\big)\,C(\beta)
    = \sum_{k\ge0}\frac{B_k}{k!}(-\beta)^k\,\mathsf D^{k}C(\beta).
\end{equation}
Thus, to linear order in the influence operator, the mean-force Hamiltonian is \emph{exactly} a quadratic
form in the products $f_af_b$, with all bath/temperature dependence entering only through the scalar matrix
$\widetilde C_{ab}(\beta)$.
Any operator growth beyond the quadratic span is confined to the nonlinear BCH sector $O(\Delta^2)$ (e.g.\
$[\Delta,\mathrm{ad}_{H_Q}\Delta]$).

With this form in hand, we can now state the \emph{closure criterion} required for $H_{\mathrm{MF}}(\beta)$ to be representable in a given operator family $\mathcal A$.
\subsection{Closure criterion}

Fix a target operator family $\mathcal A$ (e.g.\ $k$-local Pauli strings, a Lie algebra plus identity,
or a finite-dimensional associative operator algebra) in which we seek to represent $H_{\mathrm{MF}}(\beta)$.
The Gaussian construction above shows that representability is controlled by two closure conditions:

\paragraph{(C1) Adjoint closure (no operator growth under commutation).}
The subspace generated by repeated commutators of $f$ with $H_Q$ must remain inside $\mathcal A$:
\begin{equation}
    \begin{split}
        f & \in\mathcal A,\qquad \mathrm{ad}_{H_Q}(\mathcal A)\subseteq \mathcal A,     \\
          & \text{equivalently}\quad \mathrm{span}\{f_n\}_{n\ge 0}\subseteq \mathcal A.
    \end{split}
    \label{eq:C1_adjoint_closure}
\end{equation}

\paragraph{(C2) Quadratic closure (no operator growth in the Gaussian sector).}
Because the exact influence operator $\Delta(\beta)$ is quadratic in the adjoint chain,
the quadratic span generated by $\{f_n\}$ must also lie in $\mathcal A$:
\begin{equation}
    f_n f_m \in \mathcal A\quad \text{for all indices that contribute in $\Delta(\beta)$,}
    \label{eq:C2_product_closure}
\end{equation}
so that $\Delta(\beta)\in\mathcal A$. A sufficient (and common) condition is that $\mathcal A$ is an
associative algebra containing the identity and closed under multiplication.

\paragraph{(C3) BCH closure (no growth from commutators among quadratic elements).}
To ensure that the \emph{full} BCH logarithm $\log(e^{-\beta H_Q}e^{\Delta})$ remains in $\mathcal A$,
the commutators generated among the quadratic elements must also close in $\mathcal A$; in particular
\begin{equation}
    [\Delta(\beta),\,\mathrm{ad}_{H_Q}^k\Delta(\beta)]\in\mathcal A\qquad \text{for all $k\ge 0$,}
    \label{eq:C3_BCH_closure}
\end{equation}
and similarly for the higher nested commutators appearing in the BCH series.
A sufficient (and common) condition is again that $\mathcal A$ is a finite-dimensional associative algebra
(or a Lie algebra containing $\Delta$ and closed under commutators), in which case all BCH commutators remain
in $\mathcal A$ by construction.

\paragraph{Consequence (closed-form mean-force Hamiltonian).}
If (C1)--(C3) hold, then $\Delta(\beta)\in\mathcal A$ and all BCH commutators generated by
\eqref{eq:HMF_BCH_def_final} remain in $\mathcal A$. Since $H_Q\in\mathcal A$ by assumption, it follows that
\begin{equation}
    \label{eq:HMF_BCH_def_final}
    \begin{split}
        \bar\rho_Q(\beta)             & = e^{-\beta H_{\mathrm{MF}}(\beta)},                  \\
        -\beta H_{\mathrm{MF}}(\beta) & = \log\!\left(e^{-\beta H_Q}e^{\Delta(\beta)}\right),
    \end{split}
\end{equation}
defines an $H_{\mathrm{MF}}(\beta)\in\mathcal A$ and hence a closed-form representation of
$H_{\mathrm{MF}}(\beta)$ (up to the usual additive scalar fixed by $Z_X$).

If either (C1) fails (adjoint-chain growth) or (C2)--(C3) fail (growth in the quadratic/BCH sector),
then $\Delta(\beta)$ and/or the BCH commutator tower generates operators outside $\mathcal A$; any
representation within $\mathcal A$ is necessarily truncated or projected.

This algebraic perspective offers a unifying view of strong-coupling approximations. Numerical schemes such as the polaron transformation~\cite{weissQuantumDissipativeSystems2012}, hierarchical equations of motion~\cite{tanimuraReducedHierarchicalEquations2014}, and chain-mapping/DMRG techniques can be understood as distinct choices of the target operator family $\mathcal{A}$. The polaron frame corresponds to dressing $\mathcal{A}$ with coherent displacements; HEOM truncates the memory depth of the influence functional (dual to the operator degree in the adjoint chain), and chain mappings truncate the spatial extent of the harmonic bath. The `closure' problem is thus equivalent to identifying a low-dimensional subalgebra that approximately contains the logarithm of the quenched propagator.

When $\mathcal A$ does not close, \eqref{eq:HMF_BCH_def_final} still yields a controlled organisational
principle: the operator content is generated by the adjoint chain and its quadratic products, while all bath
dependence remains scalar through $C_{nm}(\beta)$. One may therefore truncate (in commutator depth, locality
class, or operator weight) without introducing any approximation on the bath side.
