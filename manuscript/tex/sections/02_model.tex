\section{Model\label{sec:model}}

How do we model an environment? The Hamiltonian of mean force is defined by
tracing the joint equilibrium operator, so its very definition requires a
model for the composite Hamiltonian. To do so, we describe the
subsystem by a Hamiltonian $H_Q$ with coordinates $q$, coupled to a bath described by $H_X$ (using $x$ as coordinates) via an interaction $H_I$. The total system is then
\begin{equation}
    H_{\mathrm{tot}}=H_Q+H_X+H_I.
    \label{eq:Htot_model}
\end{equation}
Such a form is of course too general to be useful, and it is therefore necessary to specify a functional form for both the bath and the interaction. We choose a Caldeira-Leggett (CL) environment. This is appropriate when the bath is large and close to equilibrium, so its fluctuations are approximately Gaussian and the leading, linear coupling dominates a local expansion of the interaction. The bath is modeled as a collection of harmonic oscillators,
\begin{equation}
    H_X=\sum_k\left(\frac{p_k^2}{2m_k}+\frac{1}{2}m_k\omega_k^2 x_k^2\right).
    \label{eq:HX_model}
\end{equation}
We then specify the interaction as 
\begin{equation}
    H_I=\sum_k c_k f(q) x_k,
    \label{eq:HI_model}
\end{equation}
where the bath coordinate $x_k$ is coupled with strength $c_k$ to a system operator $f\equiv f(q)$. The total Hamiltonian is then
\begin{equation}
    H_{\mathrm{tot}} = H_Q + \sum_k \left( \frac{p_k^2}{2m_k} + \frac{1}{2}m_k\omega_k^2 x_k^2 + c_k f(q) x_k \right).
    \label{eq:Htot_full}
\end{equation}
To enforce translational invariance, CL models often insert a counter-term proportional to $f(q)^2$. This cancels the potential renormalisation induced by the coupling, and may be absorbed into the definition of $H_Q$. This framework generalizes the Jaynes-Cummings model to a multi-mode environment, reducing to it when the system is a two-level atom and a single mode is considered. 

Finally, while the bath is parametrised by the masses $m_k$, frequencies $\omega_k$, and coupling strengths $c_k$, its influence on the system may ultimately be characterised by the spectral density $J(\omega)$, defined by
\begin{equation}
    J(\omega) = \frac{\pi}{2} \sum_k \frac{c_k^2}{m_k \omega_k} \delta(\omega-\omega_k).
    \label{eq:spectral_density}
\end{equation}
This compactly characterises the bath spectral properties and ultimately determines the
form of bath correlations with the system.

\subsection{Hamiltonian of mean force and the obstruction to a direct construction}
\label{sec:hmf_bridge}

With the composite Hamiltonian specified, we turn our attention to the central question of this paper - the construction of the \emph{Hamiltonian of mean force} $H_{\mathrm{MF}}$. This serves as the effective thermodynamic description of the reduced system, and it is defined as the operator whose Gibbs form reproduces the reduced equilibrium object up to a chosen normalisation. If the total system is in equilibrium at inverse temperature $\beta$, it will be described (up to normalisation) by $e^{-\beta H_{\mathrm{tot}}}$. The subsystem, however, does not inherit a Gibbs form generated by $H_Q$ alone. The correct reduced equilibrium
object is instead
\begin{equation}
    \bar{\rho}_S(\beta)\equiv \Tr_X\, e^{-\beta H_{\mathrm{tot}}} ,
    \label{eq:reduced_equilibrium_operator}
\end{equation}
where $\Tr_X$ traces out the bath degrees of freedom. The Hamiltonian of mean
force is defined as the operator whose Gibbs form reproduces this reduced
object up to a chosen normalisation:
\begin{equation}
    e^{-\beta H_{\mathrm{MF}}(\beta)}
    \equiv
    \frac{\Tr_X\, e^{-\beta H_{\mathrm{tot}}}}{Z_X(\beta)},
    \qquad
    Z_X(\beta)\equiv \Tr_X\, e^{-\beta H_X}.
    \label{eq:HMF_def}
\end{equation}
Equivalently,
$H_{\mathrm{MF}}(\beta)=-(1/\beta)\log\!\big[\Tr_X e^{-\beta H_{\mathrm{tot}}}\big]$
up to an additive scalar fixed by the choice of $Z_X$.

At first sight one might hope to evaluate the trace in Eq.~\eqref{eq:HMF_def}
directly. For a harmonic bath linearly coupled through a collective coordinate,
the bath can indeed be eliminated exactly, yielding the Feynman--Vernon
influence functional~\cite{feynmanTheoryGeneralQuantum1963a}. This formalism
has been used to obtain exact reduced descriptions at the level of stochastic
equations of motion in a variety of settings
\cite{grabertQuantumBrownianMotion1988,stockburgerExactNumberRepresentation2002,xuSimulatingNonMarkovianDynamics2026}.
However, this dynamical tractability does not translate into a tractable static
expression for $H_{\mathrm{MF}}$. The obstruction is structural: eliminating the
bath produces a bilocal self-interaction in imaginary time, with a kernel fixed
by the bath two-point structure (equivalently by $J(\omega)$), acting on the
interaction-picture operator associated with the coupling $f$. When $[H_Q,f]\neq
0$, this yields a time-ordered exponential of a memory-kernel self-coupling
\cite{gelinThermodynamicsSubensembleCanonical2009a,campisiFluctuationTheoremArbitrary2009},
rather than a simple exponential $e^{-\beta H_{\mathrm{eff}}}$ of a local
operator. Identifying the Hamiltonian of mean force---the logarithm of this
ordered object---is therefore a non-trivial inverse problem.

The influence functional nevertheless contains exactly the ingredients required
to construct $H_{\mathrm{MF}}(\beta)$ and to state conditions under which it
closes within a restricted operator class. Our strategy is to rewrite the
influence functional so that these ingredients are separated cleanly: bath
statistics enter only through $J(\omega)$ (or the associated kernel), while all
nontrivial operator content is isolated to the algebra generated by $H_Q$ and
$f$. In this form we can obtain a series for $H_{\mathrm{MF}}(\beta)$ organised
by kernel moments and iterated commutators of $f$ with $H_Q$.

