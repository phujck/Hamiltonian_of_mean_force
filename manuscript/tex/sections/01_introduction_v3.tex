\section{Introduction}
\label{sec:intro}

In closed quantum statistical mechanics, equilibrium is generated by a
Hamiltonian: $\rho \propto e^{-\beta H}$. For an open system with finite
coupling, the operationally defined equilibrium state of the subsystem is the
reduced state of the global Gibbs ensemble,
\begin{align}
    \bar{\rho}_S(\beta) &= \mathrm{Tr}_B\, e^{-\beta H_{\mathrm{tot}}},
    \label{eq:intro_rho_bar} \\
    e^{-\beta H_{\mathrm{MF}}(\beta)}
    &\propto \frac{\mathrm{Tr}_B e^{-\beta H_{\mathrm{tot}}}}{Z_B(\beta)},
    \quad Z_B(\beta)=\mathrm{Tr}_B e^{-\beta H_B}.
    \label{eq:intro_hmf_def}
\end{align}
This object is generally not $e^{-\beta H_Q}$ for the bare system Hamiltonian
$H_Q$. The resulting representational question is precise: what operator, if
any, plays the role of an equilibrium generator for the subsystem once the
coupling is non-negligible? The Hamiltonian of mean force (HMF) answers this by
construction and is the standard starting point in strong-coupling
thermodynamics and the mean-force Gibbs-state program
\cite{campisiFluctuationTheoremArbitrary2009,talknerColloquiumStatisticalMechanics2020,trushechkinOpenQuantumSystem2022,seifertFirstSecondLaw2016}.

Why care about a mean-force Hamiltonian at all? First, it provides the exact
reduced equilibrium object that underlies strong-coupling thermodynamic
identities, including free-energy and work relations that remain valid beyond
weak coupling\cite{jarzynskiNonequilibriumWorkTheorem2004,campisiFluctuationTheoremArbitrary2009,seifertFirstSecondLaw2016}.
Second, it furnishes a consistent equilibrium initialization for open-system
dynamics when correlations with the bath are unavoidable, a point emphasized in
the literature on correlated initial states and reduced dynamics
\cite{pechukasReducedDynamicsNeed1994b,trushechkinOpenQuantumSystem2022}.
Related subensemble approaches explicitly treat the subsystem thermodynamics as
inherited from a global canonical ensemble, making the effective reduced
generator central to the formalism\cite{gelinThermodynamicsSubensembleCanonical2009a}.
In short, the HMF is not an optional reinterpretation; it is the exact operator
that encodes the reduced equilibrium state whenever system--bath coupling is
finite.

Historically, open-system theory prioritized weak-coupling and Markovian
regimes, where reduced equilibrium can often be approximated by a Gibbs state of
a renormalized $H_Q$. At finite coupling the reduced state inherits explicit
temperature dependence and interaction-induced operator content that is not
captured by a simple renormalization. Coupling-dependent thermodynamic response
features in quantum Brownian motion and related models highlight this
complexity\cite{hanggiFiniteQuantumDissipation2008,ingoldSpecificHeatAnomalies2009}.
The strong-coupling literature consequently treats the mean-force Gibbs state as
a distinct equilibrium object, with operational ramifications for heat and
energy definitions\cite{espositoNatureHeatStrongly2015,rivasStrongCouplingThermodynamics2020}.

The HMF literature itself is broad but structured. Canonical definitions and
thermodynamic identities are developed in strong-coupling thermodynamics and
fluctuation-relation work\cite{campisiFluctuationTheoremArbitrary2009,jarzynskiNonequilibriumWorkTheorem2004,seifertFirstSecondLaw2016,talknerColloquiumStatisticalMechanics2020}.
A comprehensive review consolidates the ``static'' mean-force Gibbs perspective
with the ``dynamical'' return-to-equilibrium perspective in open quantum systems
\cite{trushechkinOpenQuantumSystem2022}.
Operational questions such as measurability and thermodynamic consistency at
strong coupling have also been pursued\cite{strasbergMeasurabilityNonequilibriumThermodynamics2020,rivasStrongCouplingThermodynamics2020}.

Exact or controlled evaluations exist in special cases. For commuting (QND)
interactions, the operator algebra closes trivially and the HMF can be written
explicitly\cite{campisiTalknerHanggi2009Solvable}. Quadratic/Gaussian models
(e.g., damped harmonic oscillators) are solvable because Gaussianity is
preserved, leading to closed operator forms\cite{caldeiraQuantumTunnellingDissipative1983a,grabertQuantumBrownianMotion1988,hiltHamiltonianMeanForce2011}.
Finite-dimensional closures such as spin-boson or single-qubit models provide
additional controlled benchmarks\cite{leggettDynamicsDissipativeTwostate1987}.
Beyond these cases, the mean-force Gibbs state is often accessed through
systematic limits: Cresser and Anders derive weak- and ultrastrong-coupling
expressions and show that, in the ultrastrong limit, the mean-force Gibbs state
becomes diagonal in the interaction basis rather than the system Hamiltonian
basis\cite{cresserWeakUltrastrongCoupling2021a}. Recent work generalizes the HMF
framework to finite baths by introducing a pair of quantum Hamiltonians of mean
force that incorporate bath feedback\cite{duGeneralizedHamiltonianMeanforce2025a},
and structural studies further analyze the operator content of the HMF in
extended settings\cite{burkeStructureHamiltonianMean2024}.

Outside solvable models, most approaches are perturbative or numerical.
Weak-coupling/high-temperature expansions yield controlled but limited
series\cite{cresserWeakUltrastrongCoupling2021a}. Imaginary-time path-integral
methods, stochastic representations, and hierarchical-equations techniques can
compute $\rho_S(\beta)$ directly but typically do not provide a compact operator
form for $H_{\mathrm{MF}}$\cite{moixEquilibriumreducedDensityMatrix2012,chenRigorousStochasticMatrix2014,makriExploitingClassicalDecoherence2014,tanimuraReducedHierarchicalEquations2014,songCalculationCorrelatedInitial2015}.
Stochastic Liouville and partition-free approaches provide complementary
numerical access to open-system equilibration without yielding closed-form HMFs
\cite{stockburgerSimulatingSpinbosonDynamics2004,mccaulPartitionfreeApproachOpen2017c}.
These methods are indispensable for quantitative predictions but leave open the
representational question: when does a local or otherwise restricted operator
form exist?

The influence-functional formalism is a natural language for this problem.
For Gaussian baths with linear coupling it provides an exact route to
integrating out bath degrees of freedom\cite{feynmanTheoryGeneralQuantum1963a,caldeiraQuantumTunnellingDissipative1983a,grabertQuantumBrownianMotion1988}.
In equilibrium it becomes a Euclidean (imaginary-time) influence functional,
usually bilocal in $\tau$, and admits a Hubbard--Stratonovich rewriting as a
quenched Gaussian-field average\cite{hubbardCalculationPartitionFunctions1959a,stratonovich1957QDistro,stockburgerExactNumberRepresentation2002}.
Related path-integral derivations of quantum Langevin dynamics make explicit the
noise and dissipation structure inherited from the bath\cite{kleinertQuantumLangevinEquation1995,vankampenDerivationQuantumLangevin1997}.
This formulation also clarifies the terminology: ``nonlocal'' initially refers
to imaginary-time nonlocality of the kernel, whereas the HMF locality question
concerns the operator structure on the system Hilbert space.

The existence of $H_{\mathrm{MF}}$ is therefore not the issue---it is defined by
a logarithm of a traced exponential. The real obstruction is representability:
when does $H_{\mathrm{MF}}$ admit a closed-form expression within a restricted
operator family (few-body, spatially local, or a given algebra)? Much of the
literature either (i) solves special models, (ii) expands in controlled limits,
or (iii) computes $\rho_S(\beta)$ numerically and analyzes its properties, but a
general structural criterion is still lacking
\cite{burkeStructureHamiltonianMean2024,duGeneralizedHamiltonianMeanforce2025a,cresserWeakUltrastrongCoupling2021a}.

This paper provides an exact structural reformulation of the reduced equilibrium
operator for Gaussian baths with linear coupling, written both as an
imaginary-time influence functional and as a quenched Gaussian-field average.
We then organize the operator content by the adjoint-action hierarchy and
Magnus/BCH/Lie-factorization language to state an explicit closure criterion: a
closed-form local HMF exists only when the operator algebra generated by
repeated adjoint action of $H_Q$ on the coupling operators closes inside the
target ansatz. The analysis is anchored by minimal solvable examples (commuting
coupling, quadratic/Gaussian models, single qubit), and broader implications are
deferred.
