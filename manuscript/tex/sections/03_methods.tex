\section{Methods}
\label{sec:methods}

This section summarizes a concrete procedure for determining whether the HMF
can be represented within a local operator class and for constructing it when
closure holds.

\paragraph{Step 1: Adjoint chain.}
Compute the adjoint sequence
$\{\mathrm{ad}_{H_Q}^n(f)\}_{n\ge 0}$ until linear dependence is observed.
If the span stabilizes after $N$ steps, record a basis
$\{O_1,\ldots,O_N\}$ for $\mathcal{A}_f$.

\paragraph{Step 2: Algebraic closure.}
Check whether the associative products $O_i O_j$ can be expressed within the
same finite basis (together with the identity). If so, the generated operator
algebra is finite dimensional and closed under multiplication. This is the
exact closure criterion of Sec.~\ref{sec:theory}. Lie-algebraic closure of
$\{O_i\}$ implies that the Magnus/BCH series for the ordered exponential closes
within the finite enveloping algebra
\cite{weiLieAlgebraicSolution1963,magnusExponentialSolutionDifferential1954a,blanesMagnusExpansionIts2009}.

\paragraph{Step 3: Kernel moments.}
Compute the kernel moments $\mu_{nm}$ from Eq.~\eqref{eq:kernel_moments} using the
bath correlation function $K(\tau-\tau')$. For harmonic baths, $K$ is determined
by the spectral density and temperature
\cite{grabertQuantumBrownianMotion1988,tanimuraReducedHierarchicalEquations2014}.

\paragraph{Step 4: Construct $H_{\mathrm{MF}}$.}
With the basis and moments in hand, assemble the bilocal exponent via
Eq.~\eqref{eq:bilocal_expansion}, write the ordered exponential as
$\exp(\Omega)$ using the Magnus expansion, and combine with $e^{-\beta H_Q}$
using BCH. When the algebra closes, $H_{\mathrm{MF}}$ is a finite operator
polynomial. When it does not, any truncation or projection is the sole source
of approximation.

