\documentclass[aps,prl,twocolumn,superscriptaddress,showpacs]{revtex4-2}

\usepackage{graphicx}
\usepackage{amsmath}
\usepackage{amssymb}
\usepackage{hyperref}

\newcommand{\Tr}{\mathrm{Tr}}

\begin{document}

\title{Hamiltonian of Mean Force Beyond the Commuting Gaussian Benchmark}
\author{Author Name}
\affiliation{Affiliation}
\date{\today}

\begin{abstract}
We study the operator representability problem for the Hamiltonian of mean force
(HMF) in Gaussian linearly coupled environments beyond the commuting limit.
Building on the quenched-density construction, we derive an exact
imaginary-time reformulation in which bath statistics enter as scalar kernel
moments while operator growth is generated by the adjoint chain
$\{\mathrm{ad}_{H_Q}^{n}(f)\}_{n\ge 0}$. This yields a structural criterion:
a finite closed-form HMF exists only when the adjoint-generated operator
algebra closes inside the target ansatz. We then instantiate the criterion in a
Pauli-closed qubit model and obtain an explicit closed-form mean-force
Hamiltonian with temperature- and bath-dependence carried by mode-resolved
kernel coefficients. The manuscript is framed as the noncommuting Gaussian
follow-up to the commuting benchmark analysis of arXiv:2602.13146 and sets up
the non-Gaussian extension program.
\end{abstract}

\maketitle

\section{Introduction}
\label{sec:intro}

This manuscript is a direct follow-up to
Ref.~\cite{mccaulMeanForceHamiltoniansInfluence2026} (arXiv:2602.13146).
That paper established the quenched-density representation and delivered an
exact benchmark in the commuting Gaussian sector. The present paper starts from
that benchmark and addresses the next step: how the Hamiltonian of mean force
is organised once the coupling operator does \emph{not} commute with the bare
system Hamiltonian.

In closed quantum statistical mechanics, equilibrium is generated by a
Hamiltonian: $\rho \propto e^{-\beta H}$. For an open system with finite
coupling, the operationally defined equilibrium state of the subsystem is the
reduced state of the global Gibbs ensemble,
\begin{align}
    \bar{\rho}_S(\beta) &= \mathrm{Tr}_B\, e^{-\beta H_{\mathrm{tot}}},
    \label{eq:intro_rho_bar} \\
    e^{-\beta H_{\mathrm{MF}}(\beta)}
    &\propto \frac{\mathrm{Tr}_B e^{-\beta H_{\mathrm{tot}}}}{Z_B(\beta)},
    \quad Z_B(\beta)=\mathrm{Tr}_B e^{-\beta H_X}.
    \label{eq:intro_hmf_def}
\end{align}
This object is generally not $e^{-\beta H_Q}$ for the bare system Hamiltonian
$H_Q$. The resulting representational question is precise: what operator, if
any, plays the role of an equilibrium generator for the subsystem once the
coupling is non-negligible? The Hamiltonian of mean force (HMF) answers this by
construction and is the standard starting point in strong-coupling
thermodynamics and the mean-force Gibbs-state program
\cite{campisiFluctuationTheoremArbitrary2009,talknerColloquiumStatisticalMechanics2020,trushechkinOpenQuantumSystem2022,seifertFirstSecondLaw2016}.

Why care about a mean-force Hamiltonian at all? First, it provides the exact
reduced equilibrium object that underlies strong-coupling thermodynamic
identities, including free-energy and work relations that remain valid beyond
weak coupling\cite{jarzynskiNonequilibriumWorkTheorem2004,campisiFluctuationTheoremArbitrary2009,seifertFirstSecondLaw2016}.
Second, it furnishes a consistent equilibrium initialization for open-system
dynamics when correlations with the bath are unavoidable, a point emphasized in
the literature on correlated initial states and reduced dynamics
\cite{pechukasReducedDynamicsNeed1994b,trushechkinOpenQuantumSystem2022}.
Related subensemble approaches explicitly treat the subsystem thermodynamics as
inherited from a global canonical ensemble, making the effective reduced
generator central to the formalism\cite{gelinThermodynamicsSubensembleCanonical2009a}.
In short, the HMF is not an optional reinterpretation; it is the exact operator
that encodes the reduced equilibrium state whenever system--bath coupling is
finite.

Historically, open-system theory prioritized weak-coupling and Markovian
regimes, where reduced equilibrium can often be approximated by a Gibbs state of
a renormalized $H_Q$. At finite coupling the reduced state inherits explicit
temperature dependence and interaction-induced operator content that is not
captured by a simple renormalization. Coupling-dependent thermodynamic response
features in quantum Brownian motion and related models highlight this
complexity\cite{hanggiFiniteQuantumDissipation2008,ingoldSpecificHeatAnomalies2009}.
The strong-coupling literature consequently treats the mean-force Gibbs state as
a distinct equilibrium object, with operational ramifications for heat and
energy definitions\cite{espositoNatureHeatStrongly2015,rivasStrongCouplingThermodynamics2020}.

The HMF literature itself is broad but structured. Canonical definitions and
thermodynamic identities are developed in strong-coupling thermodynamics and
fluctuation-relation work\cite{campisiFluctuationTheoremArbitrary2009,jarzynskiNonequilibriumWorkTheorem2004,seifertFirstSecondLaw2016,talknerColloquiumStatisticalMechanics2020}.
A comprehensive review consolidates the ``static'' mean-force Gibbs perspective
with the ``dynamical'' return-to-equilibrium perspective in open quantum systems
\cite{trushechkinOpenQuantumSystem2022}.
Operational questions such as measurability and thermodynamic consistency at
strong coupling have also been pursued\cite{strasbergMeasurabilityNonequilibriumThermodynamics2020,rivasStrongCouplingThermodynamics2020}.

Exact or controlled evaluations exist in special cases. For commuting (QND)
interactions, the operator algebra closes trivially and the HMF can be written
explicitly\cite{campisiTalknerHanggi2009Solvable}. Quadratic/Gaussian models
(e.g., damped harmonic oscillators) are solvable because Gaussianity is
preserved, leading to closed operator forms\cite{caldeiraQuantumTunnellingDissipative1983a,grabertQuantumBrownianMotion1988,hiltHamiltonianMeanForce2011}.
Finite-dimensional closures such as spin-boson or single-qubit models provide
additional controlled benchmarks\cite{leggettDynamicsDissipativeTwostate1987}.
Beyond these cases, the mean-force Gibbs state is often accessed through
systematic limits: Cresser and Anders derive weak- and ultrastrong-coupling
expressions and show that, in the ultrastrong limit, the mean-force Gibbs state
becomes diagonal in the interaction basis rather than the system Hamiltonian
basis\cite{cresserWeakUltrastrongCoupling2021a}. Recent work generalizes the HMF
framework to finite baths by introducing a pair of quantum Hamiltonians of mean
force that incorporate bath feedback\cite{duGeneralizedHamiltonianMeanforce2025a},
and structural studies further analyze the operator content of the HMF in
extended settings\cite{burkeStructureHamiltonianMean2024}.

Outside solvable models, most approaches are perturbative or numerical.
Weak-coupling/high-temperature expansions yield controlled but limited
series\cite{cresserWeakUltrastrongCoupling2021a}. Imaginary-time path-integral
methods, stochastic representations, and hierarchical-equations techniques can
compute $\rho_S(\beta)$ directly but typically do not provide a compact operator
form for $H_{\mathrm{MF}}$\cite{moixEquilibriumreducedDensityMatrix2012,chenRigorousStochasticMatrix2014,makriExploitingClassicalDecoherence2014,tanimuraReducedHierarchicalEquations2014,songCalculationCorrelatedInitial2015}.
Stochastic Liouville and partition-free approaches provide complementary
numerical access to open-system equilibration without yielding closed-form HMFs
\cite{stockburgerSimulatingSpinbosonDynamics2004,mccaulPartitionfreeApproachOpen2017c}.
These methods are indispensable for quantitative predictions but leave open the
representational question: when does a local or otherwise restricted operator
form exist?

The influence-functional formalism is a natural language for this problem.
For Gaussian baths with linear coupling it provides an exact route to
integrating out bath degrees of freedom\cite{feynmanTheoryGeneralQuantum1963a,caldeiraQuantumTunnellingDissipative1983a,grabertQuantumBrownianMotion1988}.
In equilibrium it becomes a Euclidean (imaginary-time) influence functional,
usually bilocal in $\tau$, and admits a Hubbard--Stratonovich rewriting as a
quenched Gaussian-field average\cite{hubbardCalculationPartitionFunctions1959a,stratonovich1957QDistro,stockburgerExactNumberRepresentation2002}.
Related path-integral derivations of quantum Langevin dynamics make explicit the
noise and dissipation structure inherited from the bath\cite{kleinertQuantumLangevinEquation1995,vankampenDerivationQuantumLangevin1997}.
This formulation also clarifies the terminology: ``nonlocal'' initially refers
to imaginary-time nonlocality of the kernel, whereas the HMF locality question
concerns the operator structure on the system Hilbert space.

The existence of $H_{\mathrm{MF}}$ is therefore not the issue---it is defined by
a logarithm of a traced exponential. The real obstruction is representability:
when does $H_{\mathrm{MF}}$ admit a closed-form expression within a restricted
operator family (few-body, spatially local, or a given algebra)? Much of the
literature either (i) solves special models, (ii) expands in controlled limits,
or (iii) computes $\rho_S(\beta)$ numerically and analyzes its properties, but a
general structural criterion is still lacking
\cite{burkeStructureHamiltonianMean2024,duGeneralizedHamiltonianMeanforce2025a,cresserWeakUltrastrongCoupling2021a}.

Relative to Ref.~\cite{mccaulMeanForceHamiltoniansInfluence2026}, the present
manuscript has a narrower but deeper scope. We retain the exact Gaussian
influence-functional and quenched-field identities, then use adjoint-action and
Magnus/BCH organisation to study representability in the noncommuting sector.
The central result is an explicit closure criterion: a finite closed-form HMF
exists only when the operator algebra generated by repeated adjoint action of
$H_Q$ on the coupling operators closes inside the target ansatz. We then
instantiate this criterion in a Pauli-closed qubit example, where all bath
dependence is reduced to scalar kernel moments and the mean-force Hamiltonian
is written in exact closed form.

\section{Model\label{sec:model}}

How do we model an environment? The Hamiltonian of mean force is defined by
tracing the joint equilibrium operator, so its very definition requires a
model for the composite Hamiltonian. To do so, we describe the
subsystem by a Hamiltonian $H_Q$ with coordinates $q$, coupled to a bath described by $H_X$ (using $x$ as coordinates) via an interaction $H_I$. The total system is then
\begin{equation}
    H_{\mathrm{tot}}=H_Q+H_X+H_I.
    \label{eq:Htot_model}
\end{equation}
Such a form is of course too general to be useful, and it is therefore necessary to specify a functional form for both the bath and the interaction. We choose a Caldeira-Leggett (CL) environment. This is appropriate when the bath is large and close to equilibrium, so its fluctuations are approximately Gaussian and the leading, linear coupling dominates a local expansion of the interaction. The bath is modeled as a collection of harmonic oscillators,
\begin{equation}
    H_X=\sum_k\left(\frac{p_k^2}{2m_k}+\frac{1}{2}m_k\omega_k^2 x_k^2\right).
    \label{eq:HX_model}
\end{equation}
We then specify the interaction as 
\begin{equation}
    H_I=\sum_k c_k f(q) x_k,
    \label{eq:HI_model}
\end{equation}
where the bath coordinate $x_k$ is coupled with strength $c_k$ to a system operator $f\equiv f(q)$. The total Hamiltonian is then
\begin{equation}
    H_{\mathrm{tot}} = H_Q + \sum_k \left( \frac{p_k^2}{2m_k} + \frac{1}{2}m_k\omega_k^2 x_k^2 + c_k f(q) x_k \right).
    \label{eq:Htot_full}
\end{equation}
To enforce translational invariance, CL models often insert a counter-term proportional to $f(q)^2$. This cancels the potential renormalisation induced by the coupling, and may be absorbed into the definition of $H_Q$. This framework generalizes the Jaynes-Cummings model to a multi-mode environment, reducing to it when the system is a two-level atom and a single mode is considered. 

Finally, while the bath is parametrised by the masses $m_k$, frequencies $\omega_k$, and coupling strengths $c_k$, its influence on the system may ultimately be characterised by the spectral density $J(\omega)$, defined by
\begin{equation}
    J(\omega) = \frac{\pi}{2} \sum_k \frac{c_k^2}{m_k \omega_k} \delta(\omega-\omega_k).
    \label{eq:spectral_density}
\end{equation}
This compactly characterises the bath spectral properties and ultimately determines the
form of bath correlations with the system.

\subsection{Hamiltonian of mean force and the obstruction to a direct construction}
\label{sec:hmf_bridge}

With the composite Hamiltonian specified, we turn our attention to the central question of this paper - the construction of the \emph{Hamiltonian of mean force} $H_{\mathrm{MF}}$. This serves as the effective thermodynamic description of the reduced system, and it is defined as the operator whose Gibbs form reproduces the reduced equilibrium object up to a chosen normalisation. If the total system is in equilibrium at inverse temperature $\beta$, it will be described (up to normalisation) by $e^{-\beta H_{\mathrm{tot}}}$. The subsystem, however, does not inherit a Gibbs form generated by $H_Q$ alone. The correct reduced equilibrium
object is instead
\begin{equation}
    \bar{\rho}_S(\beta)\equiv \Tr_X\, e^{-\beta H_{\mathrm{tot}}} ,
    \label{eq:reduced_equilibrium_operator}
\end{equation}
where $\Tr_X$ traces out the bath degrees of freedom. The Hamiltonian of mean
force is defined as the operator whose Gibbs form reproduces this reduced
object up to a chosen normalisation:
\begin{equation}
    e^{-\beta H_{\mathrm{MF}}(\beta)}
    \equiv
    \frac{\Tr_X\, e^{-\beta H_{\mathrm{tot}}}}{Z_X(\beta)},
    \qquad
    Z_X(\beta)\equiv \Tr_X\, e^{-\beta H_X}.
    \label{eq:HMF_def}
\end{equation}
Equivalently,
$H_{\mathrm{MF}}(\beta)=-(1/\beta)\log\!\big[\Tr_X e^{-\beta H_{\mathrm{tot}}}\big]$
up to an additive scalar fixed by the choice of $Z_X$.

At first sight one might hope to evaluate the trace in Eq.~\eqref{eq:HMF_def}
directly. For a harmonic bath linearly coupled through a collective coordinate,
the bath can indeed be eliminated exactly, yielding the Feynman--Vernon
influence functional~\cite{feynmanTheoryGeneralQuantum1963a}. This formalism
has been used to obtain exact reduced descriptions at the level of stochastic
equations of motion in a variety of settings
\cite{grabertQuantumBrownianMotion1988,stockburgerExactNumberRepresentation2002,xuSimulatingNonMarkovianDynamics2026}.
However, this dynamical tractability does not translate into a tractable static
expression for $H_{\mathrm{MF}}$. The obstruction is structural: eliminating the
bath produces a bilocal self-interaction in imaginary time, with a kernel fixed
by the bath two-point structure (equivalently by $J(\omega)$), acting on the
interaction-picture operator associated with the coupling $f$. When $[H_Q,f]\neq
0$, this yields a time-ordered exponential of a memory-kernel self-coupling
\cite{gelinThermodynamicsSubensembleCanonical2009a,campisiFluctuationTheoremArbitrary2009},
rather than a simple exponential $e^{-\beta H_{\mathrm{eff}}}$ of a local
operator. Identifying the Hamiltonian of mean force---the logarithm of this
ordered object---is therefore a non-trivial inverse problem.

The influence functional nevertheless contains exactly the ingredients required
to construct $H_{\mathrm{MF}}(\beta)$ and to state conditions under which it
closes within a restricted operator class. Our strategy is to rewrite the
influence functional so that these ingredients are separated cleanly: bath
statistics enter only through $J(\omega)$ (or the associated kernel), while all
nontrivial operator content is isolated to the algebra generated by $H_Q$ and
$f$. In this form we can obtain a series for $H_{\mathrm{MF}}(\beta)$ organised
by kernel moments and iterated commutators of $f$ with $H_Q$.


\section{Quenched representation of the reduced equilibrium operator\label{sec:quenched}}

The Hamiltonian of mean force is defined implicitly by the reduced equilibrium
operator $\bar\rho_S(\beta)=\Tr_X e^{-\beta H_{\mathrm{tot}}}$. To obtain a
constructive handle on $H_{\mathrm{MF}}(\beta)$, it is useful to reorganise the
problem in two steps.

First, we treat $\bar\rho_S(\beta)$ as an imaginary-time propagator on the system
Hilbert space: it is a positive operator obtained by evolving over an interval
of length $\beta$, and any such operator may be written in the form
\begin{equation}
    \bar\rho_S(\beta)
    =
    \mathcal T_\tau \exp\!\left[-\int_0^\beta d\tau\, H_{\mathrm{eff}}(\tau)\right],
    \label{eq:bar_rho_as_tau_propagator}
\end{equation}
for some (in general non-unique) generator $H_{\mathrm{eff}}(\tau)$.

Second, we explicitly allow the generator to depend on $\tau$. This is not a technical luxury but a structural necessity: after eliminating the bath, the system acquires a non-local memory-kernel self-coupling. As emphasized in recent partition-free approaches~\cite{mccaulPartitionFreeApproach2018,mccaulHowToWin2021} and the development of stochastic Liouville-von Neumann methods~\cite{stockburgerExactNumberRepresentation2002,stockburgerSimulatingSpinbosonDynamics2004,stockburgerVarianceReduction2016,stockburgerStochasticLiouvillevon2018}, this non-local object can be mapped \emph{exactly} to a local, stochastic evolution via the Hubbard-Stratonovich transformation. This transformation trades the temporal non-locality for an average over an auxiliary Gaussian field, ensuring that the full non-Markovian character of the bath is retained without approximation.

A full derivation of this result is provided in Appendix~\ref{app:influence_derivation}.

For a CL environment the bath can be eliminated exactly, yielding an imaginary-time
influence functional with a memory kernel fixed by the spectral density $J(\omega)$.
Rather than working with the resulting bilocal object directly, we adopt an equivalent
representation in which the bath enters through an auxiliary Gaussian field and the
system evolves with a \emph{local-in-$\tau$} generator. Specifically, we introduce a real stochastic field $\xi(\tau)$ on $\tau\in[0,\beta]$ with
\begin{equation}
    \langle \xi(\tau)\rangle_\xi = 0,
    \qquad
    \langle \xi(\tau)\xi(\tau')\rangle_\xi = K(\tau-\tau'),
    \label{eq:xi_cov_main}
\end{equation}
where $K$ is the Euclidean bath correlation kernel determined by $J(\omega)$. A convenient
explicit form is
\begin{equation}
    K(\tau)=\frac{1}{\pi}\int_0^\infty d\omega\,J(\omega)\,
    \frac{\cosh\!\big(\omega(\beta/2-|\tau|)\big)}{\sinh(\beta\omega/2)} ,
    \qquad \tau\in[-\beta,\beta].
    \label{eq:K_of_tau_main}
\end{equation}

For each realisation $\xi$, we define the quenched effective Hamiltonian
\begin{equation}
    H_{\mathrm{eff}}[\xi](\tau) \equiv H_Q + \xi(\tau)\,f,
    \label{eq:Heff_xi_main}
\end{equation}
and the corresponding quenched propagator
\begin{equation}
    U_\xi(\beta)\equiv \mathcal T_\tau \exp\!\left[-\int_0^\beta d\tau\,H_{\mathrm{eff}}[\xi](\tau)\right].
    \label{eq:Uxi_def_main}
\end{equation}
The reduced equilibrium operator admits the quenched average representation
\begin{equation}
    \bar\rho_S(\beta)\equiv \Tr_X e^{-\beta H_{\mathrm{tot}}}
    \;=\;
    \big\langle U_\xi(\beta)\big\rangle_\xi,
    \label{eq:quenched_identity_main}
\end{equation}
with the Gaussian measure on $\xi$ fixed by Eqs.\eqref{eq:xi_cov_main}--\eqref{eq:K_of_tau_main}.
Consequently, performing the average in \eqref{eq:quenched_identity_main} yields the
Hamiltonian of mean force identically via
\begin{equation}
    H_{\mathrm{MF}}(\beta)
    = -\frac{1}{\beta}\log\!\left[\frac{\bar\rho_S(\beta)}{Z_X(\beta)}\right],
    \qquad Z_X(\beta)=\Tr_X e^{-\beta H_X},
    \label{eq:HMF_from_quenched_identity_main}
\end{equation}
with the scalar fixed by the chosen normalisation. Consequently, calculating $H_{\mathrm{MF}}(\beta)$ is equivalent to determining $\bar\rho_S(\beta)$, which in turn requires performing the average in \eqref{eq:quenched_identity_main} over the Gaussian field $\xi$. 

To perform this trick, it is necessary first to introduce the imaginary-time interaction picture with respect to $H_Q$ by writing
\begin{align}
    U_\xi(\beta)&=e^{-\beta H_Q}\,W_\xi(\beta), \nonumber\\
    W_\xi(\beta)&\equiv \mathcal T_\tau
    \exp\!\left[-\int_0^\beta d\tau\,\xi(\tau)\,\tilde f(\tau)\right],
    \label{eq:Uxi_interaction_picture}
\end{align}
where
\begin{equation}
    \tilde f(\tau)\equiv e^{\tau H_Q}\,f\,e^{-\tau H_Q}.
    \label{eq:f_tilde_def}
\end{equation}
This factors out the $\int H_Q$ piece from the ordered exponential, leaving $\int \xi(\tau)\tilde f(\tau)$ as the remaining term to be averaged over. As a final step before averaging, it is convenient to introduce the adjoint action of $H_Q$ on $f$, defined as $\mathrm{ad}_{H_Q}(f) = [H_Q, f]$ and $\mathrm{ad}_{H_Q}^n(f) = [H_Q, \mathrm{ad}_{H_Q}^{n-1}(f)]$. Then the interaction-picture coupling obeys
\begin{equation}
    \tilde f(\tau)=e^{\tau H_Q} f e^{-\tau H_Q}
    = e^{\tau\,{\rm ad}_{H_Q}}(f)
    = \sum_{n=0}^\infty \frac{\tau^n}{n!}\,{\rm ad}_{H_Q}^n(f).
    \label{eq:adjoint_series}
\end{equation}
All operator growth induced by eliminating the bath is therefore controlled by the nested-commutator chain $\{{\rm ad}_{H_Q}^n(f)\}_{n\ge 0}$.

We now compute the averaged interaction-picture propagator and actor out the free evolution:
\begin{equation}
    U_\xi(\beta)=e^{-\beta H_Q}\,W_\xi(\beta),
    \qquad
    W_\xi(\beta)\equiv
    \mathcal T_\tau
    \exp\!\left[
        -\int_0^\beta d\tau\,\xi(\tau)\,\tilde f(\tau)
    \right].
    \label{eq:Uxi_factor_Wxi}
\end{equation}
Taking the Gaussian average yields
\begin{equation}
    \bar\rho_S(\beta)=e^{-\beta H_Q}\,\bar W(\beta),
    \qquad
    \bar W(\beta)\equiv \big\langle W_\xi(\beta)\big\rangle_\xi.
    \label{eq:bar_rho_as_eH_barW}
\end{equation}
Thus the task is to evaluate the averaged interaction-picture propagator $\bar W(\beta)$. To do so, we expand the time-ordered exponential in Eq.~\eqref{eq:Uxi_factor_Wxi} in the standard Dyson series form,
\begin{equation}
\begin{split}
    W_\xi(\beta)
    =
    \sum_{k=0}^\infty (-1)^k
    \int_{\Delta_k} d\tau_1\cdots d\tau_k\;
    &\xi(\tau_1)\cdots\xi(\tau_k) \\
    &\times \tilde f(\tau_1)\cdots \tilde f(\tau_k),
\end{split}
    \label{eq:Wxi_Dyson_simplex}
\end{equation}
where $\Delta_k$ denotes the time-ordered integration domain
\begin{equation}
    \Delta_k:\qquad
    0\le \tau_k\le\cdots\le \tau_2\le\tau_1\le\beta,
    \label{eq:simplex_domain_def}
\end{equation}
and $d\tau_1\cdots d\tau_k$ is shorthand for the product measure over that domain.

Averaging term-by-term gives
\begin{equation}
\begin{split}
    \bar W(\beta)
    =
    \sum_{k=0}^\infty (-1)^k
    &\int_{\Delta_k} d\tau_1\cdots d\tau_k\,
    \big\langle \xi(\tau_1)\cdots\xi(\tau_k)\big\rangle_\xi \\
    &\times \tilde f(\tau_1)\cdots \tilde f(\tau_k).
\end{split}
    \label{eq:barW_Dyson_avg}
\end{equation}

\paragraph{Gaussian moments and Wick's theorem.}
Since $\xi$ is Gaussian with zero mean, all odd moments vanish:
\begin{equation}
    \big\langle \xi(\tau_1)\cdots\xi(\tau_{2n+1})\big\rangle_\xi = 0.
    \label{eq:odd_moments_vanish}
\end{equation}
For even moments, Wick's theorem (Isserlis' theorem) yields a sum over all pairings:
\begin{equation}
    \big\langle \xi(\tau_1)\cdots\xi(\tau_{2n})\big\rangle_\xi
    =
    \sum_{P\in \mathcal P_{2n}}
    \prod_{(i,j)\in P}
    K(\tau_i-\tau_j),
    \label{eq:wick_pairings_general}
\end{equation}
where $\mathcal P_{2n}$ is the set of perfect matchings (pair partitions) of $\{1,\dots,2n\}$.
Substituting Eq.~\eqref{eq:wick_pairings_general} into Eq.~\eqref{eq:barW_Dyson_avg} gives the explicit
all-orders expansion
\begin{equation}
\begin{split}
    \bar W(\beta)
    =
    \sum_{n=0}^\infty
    \int_{\Delta_{2n}} &d\tau_1\cdots d\tau_{2n}\,
    \left(
        \sum_{P\in \mathcal P_{2n}}
        \prod_{(i,j)\in P}
        K(\tau_i-\tau_j)
    \right) \\
    &\times \tilde f(\tau_1)\cdots \tilde f(\tau_{2n}).
\end{split}
    \label{eq:barW_all_orders_pairings}
\end{equation}
This series is precisely the Wick expansion of a time-ordered exponential
with a quadratic (pairwise) contraction. Since the exponent in Eq.~\eqref{eq:Uxi_factor_Wxi} is linear in the
Gaussian field, the average can be resummed exactly to yield the bilocal influence-functional form
\begin{equation}
    \boxed{
    \bar W(\beta)
    =
    \mathcal T_\tau
    \exp\!\left[
        \frac{1}{2}
        \int_0^\beta d\tau
        \int_0^\beta d\tau'\;
        K(\tau-\tau')\,\tilde f(\tau)\tilde f(\tau')
    \right].
    }
    \label{eq:barW_bilocal_exact}
\end{equation}
For later convenience we define the bilocal quadratic operator
\begin{equation}
    \Delta
    \;\equiv\;
    \frac{1}{2}
    \int_0^\beta d\tau
    \int_0^\beta d\tau'\;
    K(\tau-\tau')\,\tilde f(\tau)\tilde f(\tau'),
    \label{eq:Delta_def_bilocal}
\end{equation}
so that
\begin{equation}
    \bar W(\beta)=\mathcal T_\tau e^{\Delta}.
    \label{eq:barW_as_Texp_Delta}
\end{equation}
At this stage, the Gaussian field has been eliminated exactly, and all bath dependence is contained in the
translation-invariant kernel $K(\tau-\tau')$. In the next step we exploit this translation invariance to diagonalise $K$ in Matsubara modes and rewrite $\Delta$ as a sum of squares of mode-projected operators.

\subsection{Diagonalising the translation-invariant kernel and a sum-of-squares form}
\label{sec:kernel_matsubara_diagonalisation}

We now exploit the key structural property of the Euclidean correlation kernel:
it is translation invariant on the thermal circle, i.e.\ it depends only on the time difference.
Equivalently, $K$ is $\beta$-periodic and admits a bosonic Matsubara Fourier series.
This allows us to rewrite the bilocal quadratic operator $\Delta$ in Eq.~\eqref{eq:Delta_def_bilocal}
in a form that isolates all bath dependence into scalar mode weights.

\paragraph{Matsubara expansion of the kernel.}
On $\tau\in[0,\beta]$, any $\beta$-periodic translation-invariant kernel may be expanded as
\begin{equation}
    K(u)
    =
    \frac{1}{\beta}\sum_{\ell\in\mathbb Z}\kappa_\ell\,e^{i\nu_\ell u},
    \qquad
    \nu_\ell\equiv \frac{2\pi\ell}{\beta},
    \label{eq:K_matsubara_series}
\end{equation}
with Fourier coefficients
\begin{equation}
    \kappa_\ell
    \equiv
    \int_0^\beta du\;K(u)\,e^{-i\nu_\ell u}.
    \label{eq:kappa_l_def}
\end{equation}
For a real, even kernel $K(u)=K(-u)$ one has $\kappa_\ell\in\mathbb R$ and $\kappa_{-\ell}=\kappa_\ell$.
(For the Caldeira--Leggett class one further has $\kappa_\ell\ge 0$.)

Substituting Eq.~\eqref{eq:K_matsubara_series} into Eq.~\eqref{eq:Delta_def_bilocal} gives
\begin{align}
    \Delta
    &=
    \frac{1}{2}\int_0^\beta d\tau\int_0^\beta d\tau'\;
    \left[
        \frac{1}{\beta}\sum_{\ell\in\mathbb Z}\kappa_\ell\,e^{i\nu_\ell(\tau-\tau')}
    \right]
    \tilde f(\tau)\tilde f(\tau')
    \nonumber\\
    &=
    \frac{1}{2\beta}\sum_{\ell\in\mathbb Z}\kappa_\ell\;
    \left(\int_0^\beta d\tau\,e^{i\nu_\ell\tau}\,\tilde f(\tau)\right)
    \left(\int_0^\beta d\tau'\,e^{-i\nu_\ell\tau'}\,\tilde f(\tau')\right).
    \label{eq:Delta_matsubara_inserted}
\end{align}

\paragraph{Mode-projected operators and sum-of-squares form.}
Define the Matsubara-projected interaction-picture operators
\begin{equation}
    \tilde F_\ell
    \;\equiv\;
    \int_0^\beta d\tau\,e^{i\nu_\ell\tau}\,\tilde f(\tau).
    \label{eq:Ftilde_l_def}
\end{equation}
Since $\tilde f(\tau)$ is Hermitian for Hermitian $f$, one has $\tilde F_{-\ell}=\tilde F_\ell^\dagger$.
Equation~\eqref{eq:Delta_matsubara_inserted} then becomes the exact quadratic sum
\begin{equation}
    \boxed{
    \Delta
    =
    \frac{1}{2\beta}\sum_{\ell\in\mathbb Z}\kappa_\ell\,
    \tilde F_\ell\,\tilde F_\ell^\dagger.
    }
    \label{eq:Delta_sum_of_squares_complex}
\end{equation}
This form cleanly separates bath and system structure:
all information about the bath enters through the scalar mode weights $\{\kappa_\ell\}$,
while the system algebra is carried by the family of operators $\{\tilde F_\ell\}$.

\paragraph{Manifestly real form (optional).}
When $K$ is real and even (so $\kappa_\ell=\kappa_{-\ell}\in\mathbb R$), it is often convenient to pair $\pm\ell$
and write $\Delta$ in terms of cosine and sine projections. Using
\begin{equation}
    K(u)=\frac{\kappa_0}{\beta}+\frac{2}{\beta}\sum_{\ell\ge 1}\kappa_\ell\cos(\nu_\ell u),
    \label{eq:K_cos_series}
\end{equation}
define the real mode operators
\begin{align}
    \tilde F_{\ell,c}&\equiv \int_0^\beta d\tau\,\cos(\nu_\ell\tau)\,\tilde f(\tau), \nonumber\\
    \tilde F_{\ell,s}&\equiv \int_0^\beta d\tau\,\sin(\nu_\ell\tau)\,\tilde f(\tau),
    \quad (\ell\ge 1),
    \label{eq:Ftilde_cos_sin_def}
\end{align}
and $\tilde F_{0,c}\equiv \int_0^\beta d\tau\,\tilde f(\tau)$.
Then Eq.~\eqref{eq:Delta_sum_of_squares_complex} is equivalent to
\begin{equation}
    \Delta
    =
    \frac{\kappa_0}{2\beta}\,\tilde F_{0,c}^2
    +
    \frac{1}{2\beta}\sum_{\ell\ge 1}\kappa_\ell\,
    \Big(\tilde F_{\ell,c}^2+\tilde F_{\ell,s}^2\Big),
    \label{eq:Delta_sum_of_squares_real}
\end{equation}
making explicit that the quadratic form is real.

In the next step we express the projected operators $\tilde F_\ell$ (or equivalently $\tilde F_{\ell,c/s}$)
in terms of the nested-commutator chain generated by $H_Q$ acting on $f$. This yields a generating-function
representation for all kernel moments and will allow us to collect the bath dependence into scalar prefactors
multiplying products of $\mathrm{ad}_{H_Q}^n(f)$.

% =========================
% Step 3: Express mode operators in the adjoint chain; generating functions for moments
% (keeps factorial/binomial structure explicit)
% =========================

\subsection{Adjoint-chain expansion and generating functions for Matsubara moments}
\label{sec:adjoint_chain_generating_functions}

We now rewrite the mode-projected operators $\tilde F_\ell$ in Eq.~\eqref{eq:Ftilde_l_def}
in a basis that makes the operator growth induced by the bath completely explicit.
The key observation is that the entire $\tau$-dependence of $\tilde f(\tau)$ is generated by the
adjoint action of $H_Q$.

\paragraph{Adjoint chain and interaction-picture coupling.}
Introduce the adjoint superoperator
\begin{equation}
\begin{split}
    &\mathrm{ad}_{H_Q}(A)\equiv [H_Q,A],
    \quad
    \mathrm{ad}_{H_Q}^{0}(A)\equiv A, \\
    &\mathrm{ad}_{H_Q}^{n}(A)\equiv [H_Q,\mathrm{ad}_{H_Q}^{n-1}(A)].
\end{split}
    \label{eq:ad_def_recalled}
\end{equation}
Define the nested-commutator chain generated from $f$,
\begin{equation}
    f_n \;\equiv\; \mathrm{ad}_{H_Q}^{n}(f),
    \qquad n\ge 0.
    \label{eq:fn_def}
\end{equation}
Then the interaction-picture operator is
\begin{equation}
    \tilde f(\tau)
    =
    e^{\tau H_Q} f e^{-\tau H_Q}
    =
    e^{\tau\,\mathrm{ad}_{H_Q}}(f)
    =
    \sum_{n=0}^\infty \frac{\tau^n}{n!}\,f_n.
    \label{eq:ftilde_adjoint_series_step3}
\end{equation}
Equation~\eqref{eq:ftilde_adjoint_series_step3} cleanly separates
(i) the scalar monomials $\tau^n/n!$ from
(ii) the operator content $f_n$, and the latter is the sole source of operator growth.

\paragraph{Matsubara moments and operator-valued mode expansion.}
Substituting Eq.~\eqref{eq:ftilde_adjoint_series_step3} into the mode definition \eqref{eq:Ftilde_l_def} yields
\begin{equation}
    \tilde F_\ell
    =
    \int_0^\beta d\tau\,e^{i\nu_\ell\tau}\sum_{n=0}^\infty \frac{\tau^n}{n!}\,f_n
    =
    \sum_{n=0}^\infty I_n(\nu_\ell)\,f_n,
    \label{eq:Ftilde_l_as_sum_fn}
\end{equation}
where we have defined the factorial-weighted Matsubara moments
\begin{equation}
    I_n(\nu)
    \;\equiv\;
    \int_0^\beta d\tau\,\frac{\tau^n}{n!}\,e^{i\nu\tau}.
    \label{eq:In_def}
\end{equation}
Since $\nu_{-\ell}=-\nu_\ell$, one has $I_n(\nu_{-\ell})=I_n(\nu_\ell)^\ast$ and hence $\tilde F_{-\ell}=\tilde F_\ell^\dagger$
for Hermitian $f$.

\paragraph{Generating function for all moments $I_n(\nu_\ell)$.}
The full family $\{I_n(\nu_\ell)\}_{n\ge 0}$ is generated by the scalar function
\begin{equation}
\begin{split}
    I(\nu;s)
    \;\equiv\;
    \sum_{n=0}^\infty s^n\,I_n(\nu)
    &=
    \int_0^\beta d\tau\,\exp\!\big[(s+i\nu)\tau\big] \\
    &=
    \frac{e^{(s+i\nu)\beta}-1}{s+i\nu}.
\end{split}
    \label{eq:I_gen_general}
\end{equation}
For bosonic Matsubara frequencies $\nu=\nu_\ell=2\pi\ell/\beta$ one has $e^{i\nu_\ell\beta}=1$, and therefore
\begin{equation}
    I(\nu_\ell;s)
    =
    \frac{e^{s\beta}-1}{s+i\nu_\ell}.
    \label{eq:I_gen_matsubara}
\end{equation}
The individual coefficients are recovered by differentiation at $s=0$:
\begin{equation}
    I_n(\nu_\ell)
    =
    \frac{1}{n!}\,\partial_s^n I(\nu_\ell;s)\Big|_{s=0}.
    \label{eq:In_from_gen}
\end{equation}
Equation~\eqref{eq:I_gen_matsubara} provides a compact representation for all prefactors produced by the time integrals.

\paragraph{Factorisation of the quadratic form.}
Combining Eq.~\eqref{eq:Delta_sum_of_squares_complex} with Eq.~\eqref{eq:Ftilde_l_as_sum_fn} yields
\begin{align}
    \Delta
    &=
    \frac{1}{2\beta}\sum_{\ell\in\mathbb Z}\kappa_\ell\,
    \left(\sum_{n=0}^\infty I_n(\nu_\ell)\,f_n\right)
    \left(\sum_{m=0}^\infty I_m(\nu_\ell)^\ast\,f_m\right)
    \nonumber\\
    &=
    \frac{1}{2}
    \sum_{n,m=0}^\infty
    C_{nm}(\beta)\,f_n f_m,
    \label{eq:Delta_as_Cnm_fnfm}
\end{align}
where the moment matrix is
\begin{equation}
    C_{nm}(\beta)
    \;\equiv\;
    \frac{1}{\beta}\sum_{\ell\in\mathbb Z}\kappa_\ell\,
    I_n(\nu_\ell)\,I_m(\nu_\ell)^\ast.
    \label{eq:Cnm_factorised}
\end{equation}
This makes the desired decoupling explicit: $C_{nm}$ factorises as a sum of outer products over Matsubara modes.
Equivalently, defining mode coefficients
\begin{equation}
    a_{n\ell}\;\equiv\;\sqrt{\frac{\kappa_\ell}{\beta}}\,I_n(\nu_\ell),
    \label{eq:anl_def}
\end{equation}
one has
\begin{equation}
    C_{nm}(\beta)
    =
    \sum_{\ell\in\mathbb Z} a_{n\ell}\,a_{m\ell}^\ast,
    \label{eq:Cnm_outer_product}
\end{equation}
and therefore
\begin{equation}
    \Delta=\frac{1}{2}\sum_{\ell\in\mathbb Z}
    \left(\sum_{n\ge 0} a_{n\ell}\,f_n\right)
    \left(\sum_{m\ge 0} a_{m\ell}^\ast\,f_m\right).
    \label{eq:Delta_as_sum_mode_squares}
\end{equation}
All temperature and spectral-density dependence is confined to the scalar mode weights $\kappa_\ell$
and the generating function \eqref{eq:I_gen_matsubara}, while all operator complexity resides in the adjoint chain $\{f_n\}$.

\medskip
In the next step we return from the interaction picture to the Schr\"odinger picture, combining
$e^{-\beta H_Q}$ with the averaged propagator $\bar W(\beta)=\mathcal T_\tau e^{\Delta}$ to obtain a single-exponent
representation and hence a systematic expansion of the mean-force Hamiltonian in commutator order.

% =========================
% Step 4: Drop time ordering (Delta is tau-independent), de-interaction-picture,
% and collapse to a single exponent via BCH / Bernoulli (systematic commutator order).
% Keep binomial factors explicit.
% =========================

\subsection{De-interaction picture and single-exponent form}
\label{sec:collapse_single_exponent}

At this point the Gaussian average has been performed exactly, and the averaged interaction-picture propagator
is
\begin{align}
    \bar W(\beta)&=\mathcal T_\tau e^{\Delta}, \nonumber\\
    \Delta&=\frac{1}{2\beta}\sum_{\ell\in\mathbb Z}\kappa_\ell\,\tilde F_\ell\tilde F_\ell^\dagger
    =\frac12\sum_{n,m\ge 0} C_{nm}(\beta)\,f_n f_m,
    \label{eq:barW_Texp_Delta_recalled_step4}
\end{align}
with $\tilde F_\ell=\sum_{n\ge0}I_n(\nu_\ell)f_n$ and $C_{nm}$ given by Eq.~\eqref{eq:Cnm_factorised}.

\paragraph{Time ordering becomes redundant after mode reduction.}
Although Eq.~\eqref{eq:barW_Texp_Delta_recalled_step4} is written as a time-ordered exponential,
the operator $\Delta$ itself contains no remaining $\tau$-dependence: all $\tau$-integrals have been carried out
explicitly into scalar coefficients ($\kappa_\ell$, $I_n$, and hence $C_{nm}$), while the operator content is
carried by the time-independent adjoint-chain elements $\{f_n\}$.
Therefore $\mathcal T_\tau$ acts trivially and we may write
\begin{equation}
    \bar W(\beta)=e^{\Delta}.
    \label{eq:barW_equals_expDelta}
\end{equation}
Consequently, the reduced equilibrium operator takes the nonperturbative product form
\begin{equation}
    \bar\rho_S(\beta)=e^{-\beta H_Q}\,e^{\Delta}.
    \label{eq:bar_rho_product_form}
\end{equation}

\paragraph{Definition of the effective (mean-force) Hamiltonian.}
We now define an effective system Hamiltonian $H_{\rm eff}(\beta)$ by the single-exponent representation
\begin{equation}
    \bar\rho_S(\beta) \equiv e^{-\beta H_{\rm eff}(\beta)},
    \label{eq:Heff_def_by_single_exponent}
\end{equation}
up to the overall scalar normalisation fixed by $Z_X(\beta)$ in Eq.~\eqref{eq:HMF_from_quenched_identity_main}.
Combining Eqs.~\eqref{eq:bar_rho_product_form} and \eqref{eq:Heff_def_by_single_exponent} gives
\begin{equation}
    -\beta H_{\rm eff}(\beta) = \log\!\left(e^{-\beta H_Q}e^{\Delta}\right),
    \label{eq:Heff_log_product}
\end{equation}
so $H_{\rm eff}$ is obtained by the Baker--Campbell--Hausdorff (BCH) series with
\begin{equation}
\begin{split}
    &A\equiv -\beta H_Q,
    \qquad
    B\equiv \Delta, \\
    &\log(e^{A}e^{B}) = A + B + \frac12[A,B] \\
    &\quad + \frac{1}{12}[A,[A,B]] - \frac{1}{12}[B,[A,B]] + \cdots.
\end{split}
    \label{eq:BCH_recalled}
\end{equation}

\paragraph{Systematic commutator-order expansion at fixed quadratic content.}
A key advantage of the adjoint-chain representation is that commutators with $H_Q$ simply shift the chain index.
Indeed, by definition $[H_Q,f_n]=f_{n+1}$, and the adjoint action satisfies the Leibniz rule. Iterating yields
the explicit binomial decomposition
\begin{equation}
    \mathrm{ad}_{H_Q}^{r}(f_n f_m)
    =
    \sum_{k=0}^{r}\binom{r}{k}\,f_{n+k}\,f_{m+r-k},
    \qquad r\ge 0,
    \label{eq:ad_r_on_product_binomial}
\end{equation}
which isolates the combinatorics (the binomial factors) from the operator content.

Using Eq.~\eqref{eq:Delta_as_Cnm_fnfm}, the $r$-fold adjoint action of $H_Q$ on $\Delta$ is therefore
\begin{equation}
\begin{split}
    \mathrm{ad}_{H_Q}^{r}(\Delta)
    &=
    \frac12\sum_{n,m\ge 0} C_{nm}(\beta)\;
    \mathrm{ad}_{H_Q}^{r}(f_n f_m) \\
    &=
    \frac12\sum_{n,m\ge 0} C_{nm}(\beta)\;
    \sum_{k=0}^{r}\binom{r}{k}\,f_{n+k}\,f_{m+r-k}.
\end{split}
    \label{eq:ad_r_on_Delta_binomial}
\end{equation}

\paragraph{Linear-in-$\Delta$ resummation (Bernoulli/BCH kernel).}
If we retain only terms \emph{linear} in $\Delta$ in the BCH series (i.e.\ we drop terms of order $\Delta^2$ and higher,
such as $[B,[A,B]]$ in Eq.~\eqref{eq:BCH_recalled}), then the remaining commutator tower
$B+\frac12[A,B]+\frac{1}{12}[A,[A,B]]+\cdots$ can be resummed to all commutator orders via a Bernoulli series.
Specifically,
\begin{align}
    \log(e^{A}e^{B})
    &=
    A + \varphi(\mathrm{ad}_{A})\,B + \mathcal O(B^2), \nonumber\\
    \varphi(x)&\equiv \frac{x}{1-e^{-x}}
    =
    \sum_{r=0}^\infty \frac{B_r}{r!}\,x^r,
    \label{eq:Bernoulli_resummation_phi}
\end{align}
where $B_r$ are Bernoulli numbers with the convention $B_0=1$ and $B_1=+1/2$.
Since $A=-\beta H_Q$, one has $\mathrm{ad}_{A}=-\beta\,\mathrm{ad}_{H_Q}$, and hence
\begin{equation}
    -\beta H_{\rm eff}(\beta)
    =
    -\beta H_Q + \varphi\!\big(-\beta\,\mathrm{ad}_{H_Q}\big)\Delta + \mathcal O(\Delta^2).
    \label{eq:Heff_phi_adHQ}
\end{equation}
Equivalently,
\begin{equation}
    \boxed{
    \begin{aligned}
    H_{\rm eff}(\beta)
    &=
    H_Q
    -\frac{1}{\beta}\,
    \varphi\!\big(-\beta\,\mathrm{ad}_{H_Q}\big)\Delta \\
    &\quad +\mathcal O(\Delta^2).
    \end{aligned}
    }
    \label{eq:Heff_linear_in_Delta}
\end{equation}
Expanding $\varphi$ explicitly gives the commutator-order series
\begin{equation}
    H_{\rm eff}(\beta)
    =
    H_Q
    -\frac{1}{\beta}\sum_{r=0}^\infty \frac{B_r}{r!}\,(-\beta)^r\,
    \mathrm{ad}_{H_Q}^{r}(\Delta)
    +\mathcal O(\Delta^2),
    \label{eq:Heff_Bernoulli_commutator_series}
\end{equation}
and substituting Eq.~\eqref{eq:ad_r_on_Delta_binomial} yields the fully explicit adjoint-chain form
\begin{equation}
\begin{split}
    H_{\rm eff}(\beta)
    =
    H_Q
    &-\frac{1}{2\beta}\sum_{r=0}^\infty \frac{B_r}{r!}\,(-\beta)^r
    \sum_{n,m\ge 0} C_{nm}(\beta) \\
    &\times \sum_{k=0}^{r}\binom{r}{k}\,f_{n+k}\,f_{m+r-k} + \mathcal O(\Delta^2).
\end{split}
    \label{eq:Heff_explicit_binomial}
\end{equation}
All bath dependence is contained in the scalar matrix $C_{nm}(\beta)$ (or equivalently $\kappa_\ell$ and $I_n(\nu_\ell)$),
while the full operator content is controlled by the adjoint chain $\{f_n\}$ and the binomial factors in
Eq.~\eqref{eq:ad_r_on_product_binomial}.

% =========================
% Step 5: Insert an explicit "commuting check / baseline" subsection
% and then append the normalisation to obtain H_MF from H_eff.
% =========================

\subsection{Commuting baseline and commutator expansion of $H_{\rm eff}(\beta)$}
\label{sec:commuting_baseline}

It is useful to isolate the simplest limiting case in which the coupling operator $f$ commutes with the system
Hamiltonian $H_Q$. This provides both a consistency check and a natural ``baseline'' around which to organise
the general noncommuting corrections in terms of nested commutators with $H_Q$.

\paragraph{Commuting check: $[H_Q,f]=0$.}
If $[H_Q,f]=0$, then the interaction-picture coupling is $\tau$-independent:
$\tilde f(\tau)=f$. The Matsubara-projected operators defined in Eq.~\eqref{eq:Ftilde_l_def} reduce to
\begin{equation}
    \tilde F_\ell
    =
    \int_0^\beta d\tau\,e^{i\nu_\ell\tau}\,f
    =
    f\int_0^\beta d\tau\,e^{i\nu_\ell\tau}
    =
    \beta f\,\delta_{\ell 0},
    \label{eq:Ftilde_l_commuting_limit}
\end{equation}
so only the $\ell=0$ mode contributes to the sum-of-squares form \eqref{eq:Delta_sum_of_squares_complex}.
Consequently,
\begin{equation}
    \Delta
    =
    \frac{1}{2\beta}\kappa_0\,(\beta f)(\beta f)
    =
    \frac{\beta\kappa_0}{2}\,f^2.
    \label{eq:Delta_commuting_limit}
\end{equation}
Since $\Delta$ then commutes with $H_Q$, the reduced equilibrium operator
$\bar\rho_S(\beta)=e^{-\beta H_Q}e^{\Delta}$ collapses to a single exponential,
\begin{equation}
    \bar\rho_S(\beta)
    =
    \exp\!\left[-\beta\left(H_Q-\frac{\kappa_0}{2}f^2\right)\right].
    \label{eq:bar_rho_commuting_single_exponent}
\end{equation}
Thus in the commuting limit the effective Hamiltonian acquires a purely local quadratic correction,
\begin{equation}
    H_{\rm eff}(\beta)
    =
    H_Q-\frac{\kappa_0}{2}f^2,
    \qquad\text{when }[H_Q,f]=0,
    \label{eq:Heff_commuting_limit}
\end{equation}
consistent with interpreting $\kappa_0$ as the zero-frequency weight of the Euclidean bath correlations.

\paragraph{General case: organise $H_{\rm eff}$ in $f$ and commutators with $H_Q$.}
Away from the commuting limit, we retain the exact product form
\begin{equation}
    \bar\rho_S(\beta)=e^{-\beta H_Q}e^{\Delta},
    \label{eq:bar_rho_product_recalled_step5}
\end{equation}
with $\Delta$ given explicitly by Eqs.~\eqref{eq:Delta_sum_of_squares_complex}--\eqref{eq:Delta_as_Cnm_fnfm}.
We then define $H_{\rm eff}(\beta)$ by
\begin{equation}
    \bar\rho_S(\beta)\equiv e^{-\beta H_{\rm eff}(\beta)},
    \label{eq:Heff_def_recalled_step5}
\end{equation}
so that $-\beta H_{\rm eff}(\beta)=\log(e^{-\beta H_Q}e^{\Delta})$.

To express $H_{\rm eff}$ in a form that depends only on $f$ and nested commutators with $H_Q$, we use the
adjoint-chain representation $f_n\equiv \mathrm{ad}_{H_Q}^n(f)$ in Eq.~\eqref{eq:fn_def}.
In particular, $[H_Q,f_n]=f_{n+1}$ and for products the $r$-fold adjoint action obeys the binomial identity
\begin{equation}
    \mathrm{ad}_{H_Q}^{r}(f_n f_m)
    =
    \sum_{k=0}^{r}\binom{r}{k}\,f_{n+k}\,f_{m+r-k}.
    \label{eq:ad_r_product_binomial_recalled_step5}
\end{equation}
Using Eq.~\eqref{eq:Delta_as_Cnm_fnfm}, the commutator tower $\mathrm{ad}_{H_Q}^r(\Delta)$ is therefore
\begin{equation}
    \mathrm{ad}_{H_Q}^{r}(\Delta)
    =
    \frac12\sum_{n,m\ge 0} C_{nm}(\beta)
    \sum_{k=0}^{r}\binom{r}{k}\,f_{n+k}\,f_{m+r-k}.
    \label{eq:ad_r_Delta_explicit_step5}
\end{equation}
This exhibits the desired separation: all $\beta$- and bath-dependence resides in the scalar coefficients $C_{nm}(\beta)$,
while the operator basis is generated entirely by $f$ and successive commutators with $H_Q$.

\paragraph{Linear-in-$\Delta$ commutator resummation.}
Retaining only terms linear in $\Delta$ in the BCH series for $\log(e^{-\beta H_Q}e^{\Delta})$,
the commutator tower can be resummed to all orders in $\mathrm{ad}_{H_Q}$ using Bernoulli numbers:
\begin{equation}
    H_{\rm eff}(\beta)
    =
    H_Q
    -\frac{1}{\beta}\sum_{r=0}^\infty \frac{B_r}{r!}\,(-\beta)^r\,
    \mathrm{ad}_{H_Q}^{r}(\Delta)
    +\mathcal O(\Delta^2),
    \label{eq:Heff_Bernoulli_recalled_step5}
\end{equation}
with $B_0=1$ and $B_1=+1/2$.
Substituting Eq.~\eqref{eq:ad_r_Delta_explicit_step5} yields an explicit expansion in products of commutator-chain elements:
\begin{equation}
    \boxed{
    \begin{aligned}
    H_{\rm eff}(\beta)
    =
    H_Q
    &-\frac{1}{2\beta}\sum_{r=0}^\infty \frac{B_r}{r!}\,(-\beta)^r \\
    &\times \sum_{n,m\ge 0} C_{nm}(\beta)
    \sum_{k=0}^{r}\binom{r}{k}\,f_{n+k}\,f_{m+r-k} \\
    &+\mathcal O(\Delta^2).
    \end{aligned}
    }
    \label{eq:Heff_commutator_chain_explicit_step5}
\end{equation}
Equation~\eqref{eq:Heff_commutator_chain_explicit_step5} is a systematic commutator-order construction of the effective
Hamiltonian: for any truncation in the adjoint chain $\{f_n\}$ one obtains a closed operator approximation, with all
scalar prefactors determined exactly from the translation-invariant kernel via Eqs.~\eqref{eq:Cnm_factorised} and
\eqref{eq:I_gen_matsubara}.

\subsection{Normalisation and the Hamiltonian of mean force}
\label{sec:normalisation_HMF}

The Hamiltonian of mean force is defined (up to an additive scalar multiple of the identity) by
\begin{equation}
    H_{\mathrm{MF}}(\beta)
    =
    -\frac{1}{\beta}\log\!\left[\frac{\bar\rho_S(\beta)}{Z_X(\beta)}\right],
    \qquad
    Z_X(\beta)=\Tr_X e^{-\beta H_X}.
    \label{eq:HMF_def_recalled_step5}
\end{equation}
Using the definition of $H_{\rm eff}$ in Eq.~\eqref{eq:Heff_def_recalled_step5},
\begin{equation}
    \bar\rho_S(\beta)=e^{-\beta H_{\rm eff}(\beta)},
    \label{eq:bar_rho_as_Heff_again}
\end{equation}
we obtain the compact relation
\begin{equation}
    \boxed{
    H_{\mathrm{MF}}(\beta)
    =
    H_{\rm eff}(\beta)
    +\frac{1}{\beta}\log Z_X(\beta)\;\mathbb I_S.
    }
    \label{eq:HMF_from_Heff}
\end{equation}
Equivalently, if one chooses the conventional normalisation $\rho_S(\beta)\equiv \bar\rho_S(\beta)/Z_X(\beta)$,
then $\rho_S(\beta)=e^{-\beta H_{\rm MF}(\beta)}$ exactly, and the scalar shift in Eq.~\eqref{eq:HMF_from_Heff}
ensures $\Tr_S \rho_S(\beta)=1$.

\section{Results: Analytic Examples}
\label{sec:results}

We provide three minimal analytic examples that illustrate the closure
criterion and its consequences. These examples are not approximations; they
serve only to show when closure is exact.

\subsection{Commuting coupling}
If $[H_Q,f]=0$, then $\mathrm{ad}_{H_Q}^n(f)=0$ for all $n\ge 1$, so
$\tilde{f}(\tau)=f$ and the time ordering becomes trivial. The bilocal exponent
reduces to a scalar multiple of $f^2$, and the HMF is exactly local. This is the
simplest solvable case and is consistent with known exactly solvable strong
coupling models\cite{campisiTalknerHanggi2009Solvable}.

\subsection{Quadratic/Gaussian system}
Consider a harmonic system with
$H_Q = p^2/2m + (1/2)m\omega^2 q^2$ and linear coupling $f=q$. The adjoint
action closes on the finite set $\{q,p,\mathbb{I}\}$ since
$[H_Q,q] \propto p$ and $[H_Q,p] \propto q$. Consequently, the associative
algebra generated by $\mathcal{A}_f$ is finite dimensional and the HMF is a
quadratic operator. This reproduces the known Gaussian character of the
reduced equilibrium state and its mean-force Hamiltonian in damped harmonic
models\cite{grabertQuantumBrownianMotion1988,hiltHamiltonianMeanForce2011}.

\subsection{Single qubit (Pauli algebra)}
Let $H_Q = (\omega/2)\sigma_z$ and $f=\sigma_x$. Then
$\mathrm{ad}_{H_Q}(f) = \omega i\sigma_y$ and
$\mathrm{ad}_{H_Q}^2(f) = -\omega^2 \sigma_x$, so the adjoint chain closes on the
Pauli algebra $\{\sigma_x,\sigma_y,\sigma_z,\mathbb{I}\}$. Hence the closure
criterion is satisfied and $H_{\mathrm{MF}}$ lies in the same finite operator
class. This is consistent with standard spin-boson constructions\cite{leggettDynamicsDissipativeTwostate1987}.


\section{All-to-All 2-Local Designability from a Free-Spin Core}
\label{sec:alltoall_designability}

The key consequence of the closure framework is constructive control: with a
free-spin system Hamiltonian and a complete pair-resolved coupling basis, the
Gaussian mean-force map can be used to synthesize finite-$N$ all-to-all
2-local Hamiltonians. We make this statement precise and then benchmark it
numerically via stochastic simulation of the free model.

\subsection{Free-spin closure and operator content}

We fix the bare system Hamiltonian to be strictly non-interacting,
\begin{equation}
    H_S=\sum_{i=1}^N\left(\Omega_i^x X_i+\Omega_i^z Z_i\right).
    \label{eq:designability_hs_free}
\end{equation}
For each site, $\mathrm{ad}_{H_S}$ acts only on the local Pauli triplet
$\{X_i,Y_i,Z_i\}$, so the adjoint chain is finite-dimensional and closes
sitewise. For pair-resolved channels
\begin{equation}
    f_{ij}^{\alpha\beta}
    =
    u_{ij}^{\alpha\beta}\sigma_i^\alpha
    +
    v_{ij}^{\alpha\beta}\sigma_j^\beta,
    \qquad i<j,\;\alpha,\beta\in\{x,y,z\},
\end{equation}
products of adjoint-generated operators remain in the span of
0-local, 1-local, and 2-local Pauli strings. Thus, within this channel family,
Gaussian influence terms do not generate operators beyond 2-local order.

\subsection{Constructive all-to-all map}

At fixed finite $N$, define the target coefficient set
\begin{equation}
    H_{\mathrm{tar}}
    =
    \sum_i \mathbf h_i\!\cdot\!\boldsymbol{\sigma}_i
    +
    \sum_{i<j}\sum_{\alpha,\beta}
    J_{ij}^{\alpha\beta}\sigma_i^\alpha \sigma_j^\beta.
\end{equation}
Using the pair-resolved basis above, the coefficient map is linear after
flattening:
\begin{equation}
    \mathbf J = A\,\boldsymbol{\theta},
\end{equation}
where $\mathbf J$ stacks all $J_{ij}^{\alpha\beta}$ and
$\boldsymbol{\theta}$ stacks channel parameters. In the direct pair-resolved
basis, $A$ is full rank (identity up to ordering), so inversion is exact at
finite $N$.

Important scope caveat: ``arbitrary'' means arbitrary within this finite-$N$,
complete-basis setting. It is not a continuum-limit universality claim.

\subsection{Numerical demonstration}

We ran three complementary tests:
\begin{enumerate}
    \item \textbf{Design-map inversion test} (all Pauli-pair channels active):
    random dense all-to-all targets at $N=6,8,10,12$.
    \item \textbf{Physics agreement test}:
    stochastic free-model simulation versus exact thermal ED on
    $N=6,8,10$ (two random instances each), using a dense all-to-all $ZZ$
    sector for stable reweighting.
    \item \textbf{Scaling comparison}:
    stochastic runtime/memory versus quimb MPO+DMRG runtime/memory
    \cite{whiteDensityMatrixFormulation1992,schollwockDensityMatrixRenormalization2011}
    at $N=12,16,20$.
\end{enumerate}
Reported values below correspond to the production run
$(\beta=6,\;N\in\{6,8,10,12,16,20\},\;\text{samples}=192,\;\tau\text{-steps}=96,\;\text{seeds}=2)$.

The reconstruction test shows full-rank maps with machine-precision inversion:
all sampled instances satisfy $\mathrm{rank}(A)=\dim(\boldsymbol{\theta})$ and
the relative reconstruction residual is numerically zero in the reported runs
(Fig.~\ref{fig:designability_reconstruction}).

For stochastic-vs-ED physics checks, the median absolute discrepancies across
the six overlap runs are
$|\Delta e|=1.76\times 10^{-2}$ (energy density),
$|\Delta m_z|=9.76\times 10^{-2}$, and
$|\Delta \overline{\langle Z_iZ_j\rangle}|=3.61\times 10^{-2}$
(Fig.~\ref{fig:designability_ed_agreement}). The energy-density relative error
median is $8.75\%$ at the present sampling/Trotter settings.

The scaling test shows the stochastic solver remains practical across the full
all-to-all term count while keeping the same free-spin trajectory structure.
In this run, stochastic wall time increased from $4.48$ s ($N=6$) to
$53.4$ s ($N=20$), with peak memory from $\sim195$ MB to $\sim741$ MB. The MPO
baseline (ground-state proxy) scaled from $0.45$ s ($N=12$) to $3.51$ s
($N=20$), with memory near $734$--$744$ MB
(Fig.~\ref{fig:designability_scaling}).

\begin{figure*}[t]
    \centering
    \includegraphics[width=0.95\textwidth]{../figures/hmf_designability_alltoall_reconstruction.png}
    \caption{
    All-to-all constructive-map diagnostics: rank($A$) versus parameter-space
    dimension and inverse-map residual across sampled system sizes. The
    pair-resolved channel basis is full rank at finite $N$ and reconstructs
    sampled target couplings to numerical precision.
    }
    \label{fig:designability_reconstruction}
\end{figure*}

\begin{figure*}[t]
    \centering
    \includegraphics[width=0.95\textwidth]{../figures/hmf_designability_alltoall_ed_agreement.png}
    \caption{
    Stochastic free-model benchmark against exact thermal ED on overlap sizes.
    Left: energy density. Middle: average $m_z$. Right: average all-to-all
    $\overline{\langle Z_i Z_j\rangle}$. Error bars show stochastic standard
    errors.
    }
    \label{fig:designability_ed_agreement}
\end{figure*}

\begin{figure*}[t]
    \centering
    \includegraphics[width=0.95\textwidth]{../figures/hmf_designability_alltoall_scaling.png}
    \caption{
    Runtime and peak-memory scaling comparison between the stochastic free-model
    solver and a tensor-network MPO+DMRG baseline.
    }
    \label{fig:designability_scaling}
\end{figure*}

\section{Numerical Validation Against Weak/Ultrastrong Predictions}
\label{sec:numerical_validation}

We benchmark the qubit predictions of Ref.~\cite{cresserWeakUltrastrongCoupling2021a}
against exact finite-bath trace-out calculations over the full coupling range.
The system Hamiltonian and coupling operator are
\begin{equation}
    H_S=\frac{\omega_q}{2}\sigma_z,
    \qquad
    X=\cos\theta\,\sigma_z-\sin\theta\,\sigma_x,
\end{equation}
with $\omega_q=3$ and $\theta=0.25$. The bath is discretised from
$J(\omega)=Q\tau_c\omega e^{-\tau_c\omega}$ using $N_\omega=3$ modes over
$[0.5,8.0]$, per-mode cutoff $n_{\max}=4$, and $(Q,\tau_c)=(10,1)$.

For each $\lambda$ we compute $\rho_S^{\mathrm{exact}}(\lambda)$ from the global
Gibbs state by exact diagonalisation and partial trace. We compare it against:
\begin{enumerate}
    \item the bare Gibbs state $\tau_S\propto e^{-\beta H_S}$;
    \item the ultrastrong projected state $\rho_{US}$ from Eq.~(7) of
    Ref.~\cite{cresserWeakUltrastrongCoupling2021a} (equivalently the qubit form,
    Eq.~(8) there).
\end{enumerate}
The primary diagnostics are
\begin{equation}
    D_\tau=\frac{1}{2}\|\rho_S^{\mathrm{exact}}-\tau_S\|_1,
    \qquad
    D_{US}=\frac{1}{2}\|\rho_S^{\mathrm{exact}}-\rho_{US}\|_1,
\end{equation}
plus the $\ell_1$ coherence in the $H_S$ eigenbasis.

The numerical results support the PRL predictions:
\begin{enumerate}
    \item At $\lambda=0$, $D_\tau$ is numerically zero
    ($\sim 10^{-16}$), as required.
    \item As $\lambda$ increases, $\rho_S^{\mathrm{exact}}$ departs from
    $\tau_S$ and approaches $\rho_{US}$. For $\beta=1$, $D_{US}$ drops from
    $0.1124$ at $\lambda=0$ to $1.99\times10^{-3}$ at $\lambda=8$, while
    $D_\tau$ increases to $0.1104$.
    \item Energetic coherence persists in the $H_S$ basis at strong coupling,
    consistent with Eq.~(8) in Ref.~\cite{cresserWeakUltrastrongCoupling2021a}:
    for $\beta=1$, $C_{H_S}$ increases from $0$ at $\lambda=0$ to $0.218$ at
    $\lambda=8$.
\end{enumerate}

Figure~\ref{fig:prl_qubit_beta1_scan} shows the full coupling scan at $\beta=1$.
Figure~\ref{fig:prl_qubit_beta_compare} shows the temperature dependence
($\beta=0.5,1,2$): in all cases $D_{US}$ is minimised at the largest simulated
coupling and remains near $2\times10^{-3}$ at $\lambda=8$, while the crossover
from Gibbs-like to ultrastrong-like behaviour shifts to smaller $\lambda$ as
$\beta$ increases (approximately $0.6\rightarrow0.4\rightarrow0.2$).

A resolution check (low-resolution run: $N_\omega=2$, $n_{\max}=3$) changes the
$\beta=1$, $\lambda=8$ diagnostics only slightly:
$|D_{US}^{\mathrm{prod}}-D_{US}^{\mathrm{low}}|=6.45\times10^{-4}$ and
$|C_{H_S}^{\mathrm{prod}}-C_{H_S}^{\mathrm{low}}|=1.25\times10^{-3}$, indicating
that the qualitative conclusions are robust to the present discretisation.

\begin{figure*}[t]
    \centering
    \includegraphics[width=0.95\textwidth]{../figures/hmf_prl_qubit_beta1_scan.png}
    \caption{
    Single-qubit benchmark at $\beta=1$: trace-distance crossover, basis-dependent
    coherence, and expectation-value trends versus coupling strength $\lambda$.
    The exact finite-bath state moves away from the bare Gibbs state and towards
    the ultrastrong projected state as coupling grows.
    }
    \label{fig:prl_qubit_beta1_scan}
\end{figure*}

\begin{figure*}[t]
    \centering
    \includegraphics[width=0.95\textwidth]{../figures/hmf_prl_qubit_beta_comparison.png}
    \caption{
    Temperature comparison ($\beta=0.5,1,2$): left panel shows
    $D_\tau$ (solid) and $D_{US}$ (dashed), right panel shows energetic coherence
    in the $H_S$ basis. Across temperatures, increasing coupling drives the exact
    state towards the ultrastrong predicted form while preserving nonzero
    energetic coherence.
    }
    \label{fig:prl_qubit_beta_compare}
\end{figure*}

\section{Discussion}
\label{sec:discussion}

We have derived an exact operator reformulation of the reduced equilibrium
object for a Gaussian bath, expressed both as a bilocal imaginary-time
influence functional and as a quenched Gaussian-field average. The nonlocal
structure in imaginary time is traced entirely to noncommutativity between
$H_Q$ and the coupling operator $f$, which forces the interaction-picture
operator $\tilde{f}(\tau)$ to appear in time-ordered products. By expanding
$\tilde{f}(\tau)$ in adjoint actions and isolating kernel moments, we obtained a
purely algebraic representation that yields an exact closure criterion for
locality of $H_{\mathrm{MF}}$.

When the adjoint-generated operator algebra closes, the mean-force Hamiltonian
is a finite operator polynomial and can be constructed via Magnus/BCH. When it
does not, any local representation necessarily involves truncation or
projection; no other approximation is introduced. Broader implications of these
results are deferred.



\appendix
\section{Quenched Density and Imaginary-Time Evolution}
\label{sec:appendix_quenched}

This appendix makes explicit the operator identity underlying the quenched
imaginary-time formulation used in Sec.~\ref{sec:model}. Define a
time-ordered exponential with a (generally) $\tau$-dependent operator
$H(\tau)$,
\begin{equation}
    \begin{split}
        \rho(\tau) &\equiv \mathcal{T}_\tau
        \exp\left(-\int_0^\tau d\tau' \, H(\tau')\right), \\
        \rho(0) &= \mathbb{I}.
    \end{split}
    \label{eq:appendix_rho_def}
\end{equation}
Standard differentiation identities for ordered exponentials imply
\begin{equation}
    -\partial_\tau \rho(\tau) = H(\tau)\rho(\tau),
    \label{eq:appendix_imag_time_eq}
\end{equation}
with the ordering built into $\rho(\tau)$; see, e.g.,
Refs.~\cite{wilcoxExponentialOperatorsParameter1967a,magnusExponentialSolutionDifferential1954a,blanesMagnusExpansionIts2009}
for operator calculus and time-ordered exponentials.

In the present context one takes
\begin{equation}
    H(\tau) = H_Q + \xi(\tau) f,
    \label{eq:appendix_quenched_H}
\end{equation}
where the Gaussian field satisfies
\begin{equation}
    \langle \xi(\tau)\xi(\tau')\rangle = K(\tau-\tau').
    \label{eq:appendix_xi_cov}
\end{equation}
The reduced equilibrium operator can be written as
\begin{equation}
    \bar{\rho}_S =
    e^{-\beta H_Q}\left\langle \rho(\beta) \right\rangle_\xi,
    \label{eq:appendix_rho_beta}
\end{equation}
which is the compact operator form of the quenched Gaussian representation
used in the main text.

To connect Eq.~\eqref{eq:appendix_rho_beta} to an imaginary-time path integral,
one discretizes $\tau \in [0,\beta]$, applies a Trotter (or Zassenhaus)
factorization, and inserts resolutions of identity in the system coordinate
basis. This yields the standard Euclidean path-integral expression for the
canonical density operator, with the $\tau$-dependent potential induced by the
auxiliary field, as in the influence-functional derivation for quadratic baths
and their stochastic unravellings
\cite{feynmanTheoryGeneralQuantum1963a,caldeiraQuantumTunnellingDissipative1983a,grabertQuantumBrownianMotion1988,moixEquilibriumreducedDensityMatrix2012,chenRigorousStochasticMatrix2014}.

\section{Kernel Symmetry and Moment Relations}
\label{sec:appendix_kernel}

The bath kernel appearing in the bilocal influence functional is
\begin{equation}
    \begin{split}
        K(\tau-\tau') &=
        \mathrm{Tr}_B\!\left[\mathcal{T}_\tau \tilde{B}(\tau)\tilde{B}(\tau')\rho_B\right], \\
        \rho_B &= \frac{e^{-\beta H_X}}{Z_B}.
    \end{split}
    \label{eq:appendix_kernel_def}
\end{equation}
For equilibrium baths, $K$ depends only on the imaginary-time difference and is
even under exchange of its arguments, implying
\begin{equation}
    K(\tau-\tau') = K(\tau'-\tau), \qquad
    \mu_{nm} = \mu_{mn},
    \label{eq:appendix_kernel_sym}
\end{equation}
\begin{equation}
    \mu_{nm} = \frac{1}{n!m!} \int_0^\beta d\tau \int_0^\beta d\tau' \tau^n (\tau')^m K(\tau-\tau').
    \label{eq:kernel_moments_def}
\end{equation}
In common quadratic-bath models the kernel is
explicitly constructed from the bath spectral density and satisfies the
Kubo-Martin-Schwinger periodicity in imaginary time, which can be used to
re-express moment integrals in equivalent forms; see
Refs.~\cite{grabertQuantumBrownianMotion1988,tanimuraReducedHierarchicalEquations2014,songCalculationCorrelatedInitial2015}
for explicit constructions.

The moment expansion used in Sec.~\ref{sec:model} requires only these symmetry
properties and the existence of the integrals defining $\mu_{nm}$. No further
approximation is introduced at this stage.

\section{Derivation of the Influence Functional\label{app:influence_derivation}}

In this appendix, we construct the influence functional formalism used in Sec.~\ref{sec:quenched}. Our goal is to derive the exact form of the reduced density operator $\bar{\rho}_S(\beta)$ by explicitly integrating out the harmonic bath, and to demonstrate that this leads directly to the stochastic unravelling employed in the main text.

\subsection{Euclidean Path Integral Setup}
We begin with the definition of the unnormalized reduced state,
\begin{equation}
    \bar{\rho}_S(\beta) = \Tr_X \left[ e^{-\beta H_{\mathrm{tot}}} \right].
\end{equation}
This trace can be represented as a Euclidean path integral. Let $|q\rangle$ and $|x\rangle = |x_1, x_2, \dots\rangle$ denote the position bases for the system and bath, respectively. The matrix element $\langle q | \bar{\rho}_S | q' \rangle$ involves a sum over all periodic paths $x(\tau)$ (where $x(0)=x(\beta)$) and open paths $q(\tau)$ (where $q(0)=q'$ and $q(\beta)=q$):
\begin{equation}
\begin{split}
    \langle q | \bar{\rho}_S | q' \rangle = \int_{q(0)=q'}^{q(\beta)=q} &\mathcal{D}q(\tau) e^{-S_Q[q]/\hbar} \\
    &\times \prod_k Z_k[q],
\end{split}
    \label{eq:app_path_integral_start}
\end{equation}
where $S_Q$ is the Euclidean action of the isolated system, and $Z_k[q]$ is the partition function of the $k$-th oscillator in the presence of the external driving force $J_k(\tau) = -c_k f(q(\tau))$:
\begin{equation}
\begin{split}
    Z_k[q] = \oint & \mathcal{D}x_k(\tau) \exp\bigg( -\frac{1}{\hbar} \int_0^\beta d\tau \\
    &\times \left[ \frac{m_k}{2} \dot{x}_k^2 + \frac{m_k \omega_k^2}{2} x_k^2 + c_k x_k f(q(\tau)) \right] \bigg).
\end{split}
\end{equation}
Note that the periodicity of the trace implies periodic boundary conditions for the bath paths $x_k(\tau)$.

\subsection{Gaussian Integration}
The functional integral for $Z_k[q]$ is Gaussian and can be evaluated exactly. It corresponds to the partition function of a forced harmonic oscillator. The result is expressible as the product of the free oscillator partition function, $Z_X^{(k)} = (2\sinh(\beta\hbar\omega_k/2))^{-1}$, and an exponential "influence phase" depending quadratically on the drive~\cite{feynmanTheoryGeneralQuantum1963a,weissQuantumDissipativeSystems2012}:
\begin{equation}
\begin{split}
    Z_k[q] = Z_X^{(k)} \exp\bigg( &\frac{1}{2\hbar} \int_0^\beta d\tau \int_0^\beta d\tau' \\
    &\times K_k(\tau-\tau') f(q(\tau)) f(q(\tau')) \bigg).
\end{split}
\end{equation}
The kernel $K_k(\tau)$ is the equilibrium autocorrelation function of the coordinate $x_k$:
\begin{equation}
    K_k(\tau-\tau') = c_k^2 \langle \mathcal{T}_\tau x_k(\tau) x_k(\tau') \rangle_0.
\end{equation}
Summing over all modes $k$, the total influence functional is $\prod_k Z_k[q] = Z_X \exp( \Phi_{inf}[q] )$, with
\begin{equation}
\begin{split}
    \Phi_{inf}[q] = \frac{1}{2\hbar} \int_0^\beta d\tau &\int_0^\beta d\tau' K(\tau-\tau') \\
    &\times f(q(\tau)) f(q(\tau')),
\end{split}
\end{equation}
where $K(\tau) = \sum_k K_k(\tau)$ is the total force autocorrelation function.

\subsection{From Non-local Action to Stochastic Average}
Substituting this back into Eq.~\eqref{eq:app_path_integral_start}, the reduced density matrix becomes
\begin{equation}
    \bar{\rho}_S = Z_X \int \mathcal{D}q \, e^{-S_Q[q]/\hbar} \exp\left( \Phi_{inf}[q] \right).
\end{equation}
The term $\Phi_{inf}[q]$ is non-local in imaginary time, representing a self-interaction of the system mediated by the bath. To disentangle this, we use the Hubbard-Stratonovich transformation (the continuous analog of the Gaussian identity $e^{\frac{1}{2} A^2} \sim \int d\xi e^{-\frac{1}{2}\xi^2 + \xi A}$). We introduce a real, auxiliary stochastic field $\xi(\tau)$ with zero mean and covariance
\begin{equation}
    \langle \xi(\tau) \xi(\tau') \rangle_\xi = K(\tau-\tau').
\end{equation}
Using this field, we can rewrite the influence exponential as a stochastic average:
\begin{equation}
\begin{split}
    \exp\left( \Phi_{inf}[q] \right) = \bigg\langle \exp\bigg( &\frac{1}{\hbar} \int_0^\beta d\tau \\
    &\times \xi(\tau) f(q(\tau)) \bigg) \bigg\rangle_\xi.
\end{split}
\end{equation}
Inserting this identity into the path integral for $\bar{\rho}_S$, we can swap the order of the path integration over $q$ and the stochastic average over $\xi$:
\begin{equation}
    \bar{\rho}_S = Z_X \left\langle \int \mathcal{D}q \, \exp\left[-\frac{1}{\hbar} S_Q[q] + \frac{1}{\hbar} \int_0^\beta \xi f \right] \right\rangle_\xi.
\end{equation}
The term in the angle brackets is exactly the path integral for a system evolving under the time-dependent Hamiltonian $H(\tau) = H_Q - \xi(\tau)f$. Thus, in operator language, we arrive at the exact stochastic representation:
\begin{equation}
\begin{split}
    \bar{\rho}_S(\beta) = Z_X \bigg\langle \mathcal{T}_\tau \exp\bigg( &-\int_0^\beta d\tau \\
    &\times [H_Q - \xi(\tau)f] \bigg) \bigg\rangle_\xi.
\end{split}
    \label{eq:app_stochastic_final}
\end{equation}
This confirms that the HMF can be constructed by averaging the non-unitary (imaginary-time) evolution of the system driven by colored Gaussian noise.

\section{Exact BCH resummation for the qubit: closing the operator tower}
\label{app:bch_qubit}

Section~\ref{sec:closure} establishes that $H_{\mathrm{MF}}(\beta)$ admits a closed-form
representation whenever conditions (C1)--(C3) are satisfied, and notes that the BCH
commutator tower generated by $[{\Delta},\mathrm{ad}_{H_Q}^k\Delta]$ \emph{stays in
the algebra} but defers the demonstration to ``all BCH commutators remain in
$\mathcal A$ by construction.'' This appendix makes that remark explicit for the
qubit case: we compute every nested commutator that appears in the BCH series
directly to show that the su$(2)$ algebra closes the tower at \emph{every} order,
and then resum the entire series exactly using a single Pauli identity.

\subsection*{Decomposition of the influence operator}

Write the influence operator \eqref{eq:Delta_sigma_pm_v5} as
\begin{equation}
    \Delta(\beta) = \Delta_0\,\mathbb{I} + M,
    \qquad
    M \equiv \Delta_z\sigma_z + \Sigma_+\sigma_+ + \Sigma_-\sigma_-,
    \label{eq:app_Delta_decomp}
\end{equation}
separating the scalar (free-energy) part $\Delta_0$ from the traceless
$\mathfrak{su}(2)$ element $M$.  All the physically non-trivial content lives in
$M$.  We have $[H_Q, \mathbb{I}]=0$, so the BCH commutator tower acts only on $M$:
the chain $\mathrm{ad}_{H_Q}^k\Delta = \mathrm{ad}_{H_Q}^k M$ for $k\ge1$, and
$[\Delta,\mathrm{ad}_{H_Q}^k\Delta]=[M,\mathrm{ad}_{H_Q}^k M]$.

\subsection*{Step 1: The nilpotency condition $M^2 = \chi^2\mathbb{I}$}

Using the standard anticommutation and multiplication rules\footnote{%
$\{\sigma_z,\sigma_\pm\}=\sigma_z\sigma_\pm+\sigma_\pm\sigma_z=0$,\;
$\sigma_\pm^2=0$,\;
$\sigma_+\sigma_-+\sigma_-\sigma_+=\mathbb{I}$.}
for $\sigma_z$ and $\sigma_\pm=(\sigma_x\pm i\sigma_y)/2$, the square of $M$ is
\begin{align}
    M^2 &= \Delta_z^2\sigma_z^2
         + \Sigma_+^2\sigma_+^2
         + \Sigma_-^2\sigma_-^2 \notag\\
        &\quad + \Delta_z\Sigma_+\underbrace{\{\sigma_z,\sigma_+\}}_{=\,0}
         + \Delta_z\Sigma_-\underbrace{\{\sigma_z,\sigma_-\}}_{=\,0}
         + \Sigma_+\Sigma_-\underbrace{(\sigma_+\sigma_-+\sigma_-\sigma_+)}_{=\,\mathbb{I}}
         \notag\\[4pt]
        &= \bigl(\Delta_z^2 + \Sigma_+\Sigma_-\bigr)\mathbb{I}
        \;\equiv\; \chi^2\,\mathbb{I}.
    \label{eq:app_Msquared}
\end{align}
Because $M^2 = \chi^2 \mathbb{I}$, all even powers of $M$ are proportional to the
identity and all odd powers are proportional to $M$:
\begin{equation}
    M^{2k} = \chi^{2k}\mathbb{I}, \qquad M^{2k+1} = \chi^{2k}M.
    \label{eq:app_M_powers}
\end{equation}
This is the key algebraic fact from which the entire exact resummation follows.

\subsection*{Step 2: Exact resummation of $e^\Delta$}

Summing the power series for $e^M = \sum_{n=0}^\infty M^n/n!$ and separating even
from odd terms using \eqref{eq:app_M_powers},
\begin{align}
    e^M &= \sum_{k=0}^\infty \frac{M^{2k}}{(2k)!}
          + \sum_{k=0}^\infty \frac{M^{2k+1}}{(2k+1)!}
          = \cosh\chi\,\mathbb{I} + \frac{\sinh\chi}{\chi}\,M,
    \label{eq:app_expM_exact}
\end{align}
and therefore, exactly:
\begin{equation}
    e^\Delta = e^{\Delta_0}\!\left[\cosh\chi\,\mathbb{I}
    + \frac{\sinh\chi}{\chi}\,M\right].
    \label{eq:app_expDelta_exact}
\end{equation}
This is \emph{not} a perturbative approximation. No BCH is required to evaluate
$e^\Delta$; the resummation is closed by \eqref{eq:app_Msquared} alone.

\subsection*{Step 3: Verifying BCH closure order by order}

Returning to the full problem $\bar\rho_Q = e^{A}e^{B}$ with $A=-\beta H_Q$ and
$B=\Delta$, the BCH logarithm generates an infinite tower of nested commutators.
We now verify explicitly that every level of this tower remains in
$\mathcal U(2)\equiv\mathrm{span}\{\mathbb{I},\sigma_z,\sigma_+,\sigma_-\}$.

\paragraph{First-order terms: $\mathrm{ad}_{H_Q}^k M$.}
Since $[H_Q,\sigma_z]=0$ and $[H_Q,\sigma_\pm]=\pm\omega_q\sigma_\pm$, we have
\begin{equation}
    \mathrm{ad}_{H_Q}(M) = \omega_q\!\left(\Sigma_+\sigma_+ - \Sigma_-\sigma_-\right)
    \;\in\;\mathrm{span}\{\sigma_+,\sigma_-\}.
    \label{eq:app_adM_first}
\end{equation}
Iterating: $\mathrm{ad}_{H_Q}^k(M)=\omega_q^k(\Sigma_+\sigma_+ +(-1)^k\Sigma_-\sigma_-)$
for $k\ge 1$ (and $\mathrm{ad}_{H_Q}^k(\Delta_z\sigma_z)=0$ for all $k$).
So the entire linear-in-$M$ tower $\{\mathrm{ad}_{H_Q}^k M\}$ lies in
$\mathrm{span}\{\sigma_z,\sigma_+,\sigma_-\}\subset\mathfrak{su}(2)$.
This is condition (C1) of Sec.~\ref{sec:closure}, verified exactly for the qubit.

\paragraph{Second-order term: $[M,\mathrm{ad}_{H_Q}M]$.}
Substituting \eqref{eq:app_adM_first} and using $[σ_z,σ_\pm]=\pm2\sigma_\pm$,
$[\sigma_+,\sigma_-]=\sigma_z$, $[\sigma_\pm,\sigma_\pm]=0$:
\begin{align}
    \bigl[M,\,\mathrm{ad}_{H_Q}M\bigr]
    &= \omega_q\bigl[M,\,\Sigma_+\sigma_+-\Sigma_-\sigma_-\bigr] \notag\\
    &= \omega_q\Bigl(
        \Delta_z\Sigma_+[\sigma_z,\sigma_+]
       -\Delta_z\Sigma_-[\sigma_z,\sigma_-] \notag\\
    &\qquad\quad
       +\Sigma_+\Sigma_-[\sigma_+,\sigma_-]
       +\Sigma_+\Sigma_-[\sigma_-,\sigma_+]
       \Bigr)              \notag\\[4pt]
    &= \omega_q\Bigl(
        2\Delta_z\Sigma_+\sigma_+
       +2\Delta_z\Sigma_-\sigma_-
       -2\Sigma_+\Sigma_-\sigma_z
       \Bigr)              \notag\\[4pt]
    &= 2\omega_q\!\left(\Delta_z M - \chi^2\sigma_z\right),
    \label{eq:app_MadM}
\end{align}
where in the last line we used $\chi^2=\Delta_z^2+\Sigma_+\Sigma_-$ and regrouped.
This lies in $\mathrm{span}\{\sigma_z,\sigma_+,\sigma_-\}\subset\mathfrak{su}(2)$ ✓.
Moreover, \eqref{eq:app_MadM} takes the memorable form $2\omega_q(\Delta_z M - \chi^2\sigma_z)$:
the commutator is a linear combination of $M$ itself and $\sigma_z$, so the algebra
is genuinely self-referential at this order.

\paragraph{Higher orders: closure by induction.}
Since $M,\mathrm{ad}_{H_Q}^k M\in\mathfrak{su}(2)$ for all $k\ge0$, and
$\mathfrak{su}(2)$ is a \emph{Lie algebra} (closed under commutators), every nested
commutator of the form
\begin{equation}
    \bigl[M,\bigl[M,\cdots\bigl[M,\mathrm{ad}_{H_Q}^k M\bigr]\cdots\bigr]\bigr]
    \;\in\;\mathfrak{su}(2)
    \label{eq:app_general_closure}
\end{equation}
by induction on the nesting depth.  No new operator basis elements are generated
at any order.  This is condition (C3) of Sec.~\ref{sec:closure}, established here
not merely by appeal to the associativity of $\mathcal{A}$ but by explicit Pauli
algebra.

\subsection*{Step 4: The BCH series resums to a $2\times2$ matrix logarithm}

The closure established above implies that $-\beta H_{\mathrm{MF}}=\log(e^Ae^B)$
is an element of $\mathbb{C}\cdot\mathbb{I}\oplus\mathfrak{su}(2)$ at every
truncation order.  But since we can compute $e^A$ and $e^B$ in closed form, we
need not sum the BCH series at all: the product $e^A e^B=e^{-\beta H_Q}e^\Delta$
is a concrete $2\times2$ matrix, and its logarithm is the exact sum of the entire
BCH tower.

Setting $a\equiv\beta\omega_q/2$ and $\varphi\equiv\sinh\chi/\chi$, the matrix
product reads
\begin{equation}
    e^{-\beta H_Q}e^\Delta
    = e^{\Delta_0}
    \begin{pmatrix}
        e^{-a}(\cosh\chi + \varphi\Delta_z) & e^{-a}\varphi\Sigma_+ \\
        e^{a}\varphi\Sigma_-                 & e^{a}(\cosh\chi - \varphi\Delta_z)
    \end{pmatrix}.
    \label{eq:app_product_matrix}
\end{equation}
After normalisation, the Bloch vector of $\rho_Q=\bar\rho_Q/\mathrm{Tr}\,\bar\rho_Q$
is read from the off-diagonal and diagonal entries.  The exact matrix logarithm
then gives $H_{\mathrm{MF}}=-\beta^{-1}\log\rho_Q$.  For any qubit state
$\rho_Q=\tfrac12(\mathbb{I}+\mathbf{r}\cdot\boldsymbol\sigma)$ with Bloch radius
$r=|\mathbf{r}|$, this logarithm is, by elementary $2\times2$ matrix calculus,
\begin{equation}
    H_{\mathrm{MF}} = c_0\,\mathbb{I}
    - \frac{1}{\beta}\frac{\operatorname{arctanh}r}{r}\,\mathbf{r}\cdot\boldsymbol\sigma,
    \label{eq:app_HMF_bloch_log}
\end{equation}
recovering Eq.~\eqref{eq:HMF_bloch_log_v5} of the main text.

\subsection*{Summary: what the BCH tower contributes}

Table~\ref{tab:bch_tower} collects the BCH commutator hierarchy for the qubit.
Every row stays within $\mathcal U(2)$; the resummation of the infinite tower is
achieved by the single matrix computation \eqref{eq:app_product_matrix}.

\begin{table}[h]
\centering
\renewcommand{\arraystretch}{1.4}
\begin{tabular}{lll}
    \hline
    BCH term & Explicit result & Resides in \\
    \hline
    $\mathrm{ad}_{H_Q}^k M$ & $\omega_q^k(\Sigma_+\sigma_+ + (-1)^k\Sigma_-\sigma_-)$ &
        $\mathrm{span}\{\sigma_+,\sigma_-\}$ \\
    $[M,\mathrm{ad}_{H_Q}M]$ & $2\omega_q(\Delta_z M - \chi^2\sigma_z)$ &
        $\mathrm{span}\{\sigma_z,\sigma_+,\sigma_-\}$ \\
    All higher nesting & Lie algebra closure & $\mathfrak{su}(2)$ \\
    Full BCH sum & $2\times2$ matrix log \eqref{eq:app_HMF_bloch_log} &
        $\mathbb{C}\mathbb{I}\oplus\mathfrak{su}(2)$ \\
    \hline
\end{tabular}
\caption{The BCH commutator tower for the qubit.
All terms remain in $\mathcal{U}(2)$; the infinite tower resumms to the exact
Bloch-log formula \eqref{eq:app_HMF_bloch_log}.}
\label{tab:bch_tower}
\end{table}

The connection to the generating-function result of Sec.~\ref{sec:closure} is now
transparent.  The linear-in-$\Delta$ piece of the BCH logarithm is
$-\beta^{-1}\Phi(\beta\,\mathrm{ad}_{H_Q})\Delta$ with $\Phi(x)=x/(1-e^{-x})$,
which, restricted to the qubit's $(0,\pm\omega_q)$ eigenfrequencies, gives
\begin{equation}
    \Phi(\beta\,\mathrm{ad}_{H_Q})M
    = \Delta_z\sigma_z
      + \Phi(\beta\omega_q)\,\Sigma_+\sigma_+
      + \Phi(-\beta\omega_q)\,\Sigma_-\sigma_-.
    \label{eq:app_Phi_on_M}
\end{equation}
The nonlinear BCH corrections ($O(\Delta^2)$ and beyond) shift the effective
arguments of the $\cosh$/$\sinh$ functions in \eqref{eq:app_product_matrix} away
from the linear-in-$\Delta$ values.  In the weak-coupling limit ($g\to0$,
$\chi\to0$), $\cosh\chi\to1$ and $\sinh\chi/\chi\to1$, so the nonlinear
corrections vanish and \eqref{eq:app_Phi_on_M} is exact.  At finite coupling
the full expression \eqref{eq:app_HMF_bloch_log} is required, with $\chi$
depending nonlinearly on the channel amplitudes.



\bibliography{../../literature/references_new}

\end{document}
