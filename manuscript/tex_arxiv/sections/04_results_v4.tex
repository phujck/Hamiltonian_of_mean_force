\section{Exact Solution of the Spin-Boson Model}

We now turn to applying the results developed in the previous section. A particularly instructive example concerns spins, as the $\mathfrak{su}(2)$ algebra provides a particularly compact analytic solution. We demonstrate this by applying it to a transverse-coupling spin-boson model. Following  Ref.~\cite{cresserWeakUltrastrongCoupling2021a}, let
%
\begin{equation}
    H_Q = \frac{\omega_q}{2}\sigma_z, 
    \qquad 
    f = \cos\theta\,\sigma_z - \sin\theta\,\sigma_x,
    \label{eq:qubit_setup}
\end{equation}
%
with $c \equiv \cos\theta$, $s \equiv \sin\theta$ throughout. The 
coupling mixes a commuting part $c\sigma_z$ with a transverse part 
$-s\sigma_x$.

Using $[H_Q,\sigma_x] = i\omega_q\sigma_y$ and 
$[H_Q,\sigma_y] = -i\omega_q\sigma_x$, the $\sigma_z$ component of 
$f$ is annihilated by $\mathrm{ad}_{H_Q}$ while the transverse part 
precesses. A direct induction gives
%
\begin{equation}
    f_n \equiv \mathrm{ad}_{H_Q}^n(f) = \begin{cases}
        c\,\sigma_z - s\,\sigma_x & n = 0, \\[4pt]
        -s\,\omega_q^n\,\sigma_x  & n \geq 1,\; n\;\text{even}, \\[4pt]
        -is\,\omega_q^n\,\sigma_y & n \geq 1,\; n\;\text{odd}.
    \end{cases}
    \label{eq:fn_qubit}
\end{equation}
%
The adjoint chain therefore closes on $\mathrm{span}\{\sigma_x,\sigma_y,
\sigma_z\}$, satisfying condition (C1).

We use $f_n$ to evaluate the influence exponent $\Delta(\beta)$. To streamline the algebra, we treat each $f_n$ as a Pauli vector:
\begin{equation}
    f_n = \mathbf v_n \cdot \boldsymbol{\sigma},
    \qquad 
    \boldsymbol{\sigma}=(\sigma_x,\sigma_y,\sigma_z),
    \label{eq:vn_def}
\end{equation}
with
\begin{equation}
    \mathbf v_n =
    \begin{cases}
        \mathbf v_0 = (-s,\,0,\,c), & n=0, \\
        (-s\,\omega_q^{n},\,0,\,0), & n\ge 1,\; n\ \text{even},\\[2pt]
        (0,\,-is\,\omega_q^{n},\,0), & n\ge 1,\; n\ \text{odd}.
    \end{cases}
    \label{eq:vn_qubit}
\end{equation}
The Pauli multiplication rule then becomes the single identity
\begin{equation}
    (\mathbf a\cdot\boldsymbol{\sigma})(\mathbf b\cdot\boldsymbol{\sigma})
    = (\mathbf a\cdot\mathbf b)\,\mathbb I + i(\mathbf a\times\mathbf b)\cdot\boldsymbol{\sigma},
    \label{eq:pauli_vector_product}
\end{equation}
so that
\begin{equation}
    f_n f_m
    = (\mathbf v_n\cdot\mathbf v_m)\,\mathbb I
    + i(\mathbf v_n\times\mathbf v_m)\cdot\boldsymbol{\sigma}.
    \label{eq:fnfm_vector}
\end{equation}
The scalar part contributes only to the overall normalization (free-energy shift) and may be discarded when determining the operator structure.  The non-trivial Pauli sector is therefore controlled entirely by the cross products $\mathbf v_n\times \mathbf v_m$.

Given this, we introduce the symmetric and antisymmetric parts of the ordered kernel moments:
\begin{equation}
    S_{nm}:=\frac{1}{2}\big(C^{>}_{nm}+C^{>}_{mn}\big),
    \quad 
    A_{nm}:=\frac{1}{2}\big(C^{>}_{nm}-C^{>}_{mn}\big),
    \label{eq:Snm_Anm_def}
\end{equation}
so that $C^{>}_{nm}=S_{nm}+A_{nm}$ with $S_{nm}=S_{mn}$ and $A_{nm}=-A_{mn}$.
Since $\mathbf v_n\times\mathbf v_m$ is antisymmetric under $n\leftrightarrow m$, only the antisymmetric sector of the coefficients contributes to the Pauli part of $\Delta$. We may therefore restrict the sum to ordered indices $n>m$ to avoid double counting:
\begin{equation}
\begin{split}
    \Delta(\beta) 
    &\cong i\sum_{n,m\ge 0} A_{nm}\,(\mathbf v_n\times\mathbf v_m)\cdot\boldsymbol{\sigma} \\
    &= 2i\sum_{n>m} A_{nm}\,(\mathbf v_n\times\mathbf v_m)\cdot\boldsymbol{\sigma},
\end{split}
\label{eq:Delta_Pauli_Anm}
\end{equation}
where $\cong$ indicates equality up to an irrelevant $\mathbb I$ contribution. This makes explicit that the non-commuting structure in $\Delta$ is entirely inherited from the ordered nature of $C^{>}_{nm}$.

For $n,m\ge 1$, only even-odd index pairs contribute and the cross product is always parallel to $\hat{\mathbf z}$. This yields a purely $\sigma_z$ contribution from the $n,m\ge 1$ block. The special $n=0$ layer contributes additional terms. In particular, for $\ell\ge 0$,
\begin{equation}
    \begin{split}
    \mathbf v_0\times \mathbf v_{2\ell+1} &= \big(i c s\,\omega_q^{2\ell+1},\,0,\, i s^2\,\omega_q^{2\ell+1}\big), \\
    \mathbf v_0\times \mathbf v_{2\ell} &= \big(0,\,-c s\,\omega_q^{2\ell},\,0\big),
    \end{split}
    \label{eq:v0_cross_parity}
\end{equation}
so that both $\sigma_x$ and $\sigma_z$ arise from the $0$--odd sector, while a $\sigma_y$ sector is generated by the $0$--even sector whenever the ordered coefficients have a non-vanishing antisymmetric part $A_{0,2\ell}$.
Collecting these contributions, the Pauli sector of $\Delta$ may be written in the general Bloch form
\begin{equation}
    \Delta(\beta)\cong \Delta_x(\beta)\,\sigma_x+\Delta_y(\beta)\,\sigma_y+\Delta_z(\beta)\,\sigma_z,
    \label{eq:Delta_xyz_form}
\end{equation}

\begin{align}
    \Delta_x(\beta)
    &= -2cs\sum_{\ell\ge 0} A_{2\ell+1,\, 0}(\beta)\,\omega_q^{2\ell+1},
    \label{eq:Deltax_series_corr}\\[4pt]
    \Delta_y(\beta)
    &= -2ics\sum_{\ell\ge 1} A_{2\ell,\, 0}(\beta)\,\omega_q^{2\ell},
    \label{eq:Deltay_series_corr}\\[4pt]
    \Delta_z(\beta)
    &= -2s^2\sum_{k\ge 1,\,\ell\ge 0} A_{2k,\,2\ell+1}(\beta)\,\omega_q^{2k+2\ell+1} \notag\\
    &\quad\, -2s^2\sum_{\ell\ge 0} A_{2\ell+1,\, 0}(\beta)\,\omega_q^{2\ell+1}.
    \label{eq:Deltaz_series_corr}
\end{align}
The first term in \eqref{eq:Deltaz_series_corr} comes from the $n,m\ge 1$ even--odd sector, and the remaining terms from the $n=0$ cross layer.

\subsection{Generating function for the influence exponent}

While technically complete, this series representation of $\Delta$ is not particularly illuminating. A more transparent expression for the components - with a correspondingly more robust physical interpretation - is obtained by recognising that the ordered moments $C^{>}_{nm}$ can be interpreted as Laplace transforms of the imaginary-time kernel. To this end, we introduce the bivariate generating function
\begin{equation}
    \mathcal{G}^>(x,y)
    =
    \int_0^\beta d\tau \int_0^\tau d\tau'\,
    K(\tau-\tau')\,e^{x\tau+y\tau'},
    \label{eq:Gxy_def}
\end{equation}
and its anti-ordered counterpart $\mathcal{G}^<(x,y) = \mathcal{G}^>(y,x)$, where the equality follows from $K(-u)=K(u)$. Structurally, these objects are the time-ordered and anti-ordered Green's functions of the bath projected onto the Laplace domain. They directly sample the bath's correlations at the frequencies $x$ and $y$ dictated by the system's eigenoperators. This identification immediately suggests an implicit fluctuation theorem, readily uncovered by application of the KMS condition $K(\beta-u)=K(u)$:
\begin{equation}
    \mathcal{G}^>(x,y) = e^{\beta(x+y)}\,\mathcal{G}^>(-y,-x).
    \label{eq:KMS_Gxy}
\end{equation}
Changing variables to $u=\tau-\tau'$, $v=\tau'$ and using the KMS symmetry yields the closed form
\begin{equation}
    \mathcal{G}^>(x,y) = \frac{e^{x\beta}\mathcal{K}(y) - \mathcal{K}(x)}{x+y},
    \label{eq:Gxy_closed}
\end{equation}
where we have defined the bare Laplace transform of the kernel as
\begin{equation}
    \mathcal{K}(\omega) \equiv \int_0^\beta du\, K(u)\,e^{\omega u}.
    \label{eq:Ktilde_def}
\end{equation}

The resonant case $x+y \to 0$ is recovered as the limit $\mathcal{G}^>(x,-x) = e^{x\beta}\mathcal{K}'(-x)$, where pure derivatives with respect to frequency appear:
\begin{equation}
    \mathcal{K}'(-x) \equiv \partial_{\omega}\mathcal{K}(\omega)\big|_{\omega=-x} = \int_0^\beta du\, u\, K(u)\, e^{-xu}.
\end{equation}
Using the KMS symmetry, the resonant limit takes the explicit integral form
\begin{equation}
    \mathcal{G}^>(x,-x) = \int_0^\beta du\,(\beta-u)\,K(u)\,e^{xu}.
    \label{eq:Gxy_resonant_integral}
\end{equation}

The KMS reflection symmetry $K(\beta-u)=K(u)$ implies a corresponding constraint on the Laplace transform \eqref{eq:Ktilde_def}. Indeed,
\begin{align}
    \mathcal{K}(-\omega)
    &= \int_0^\beta du\,K(u)e^{-\omega u}
     = \int_0^\beta du\,K(\beta-u)e^{-\omega u} \notag\\
    &= \int_0^\beta dv\,K(v)\,e^{-\omega(\beta-v)}
     = e^{-\beta\omega}\int_0^\beta dv\,K(v)e^{\omega v} \notag \\
     &= e^{-\beta\omega}\mathcal{K}(\omega).
    \label{eq:Ktilde_detailed_balance}
\end{align}
If we then define the symmetric and antisymmetric pairs
\begin{equation}
    \mathcal{K}^\pm \equiv \frac{1}{2}\bigl(\mathcal{K}(\omega_q)\pm\mathcal{K}(-\omega_q)\bigr),
    \label{eq:Ktilde_pm}
\end{equation}
then the even/odd combinations \eqref{eq:Ktilde_pm} satisfy the fluctuation-dissipation relation
\begin{equation}
    \frac{\mathcal{K}^+}{\mathcal{K}^-}
    = \frac{1+e^{-\beta\omega_q}}{1-e^{-\beta\omega_q}}
    = \coth\!\left(\frac{\beta\omega_q}{2}\right),
    \label{eq:FDT_Ktilde_pm}
\end{equation}
Because the qubit adjoint chain \eqref{eq:fn_qubit} has a strict even/odd parity structure, only the values $x,y\in\{0,\pm\omega_q\}$ enter the sum. The influence components reduce to
\begin{align}
    \Delta_x(\beta)
        &= \frac{cs}{\omega_q}\,\mathcal{K}^-,
    \label{eq:Deltax_final}\\[4pt]
    \Delta_y(\beta)
        &= \frac{ics}{\omega_q}\,\mathcal{K}^+,
    \label{eq:Deltay_final}\\[4pt]
    \Delta_z(\beta)
        &= \frac{s^2}{2\omega_q}\,\mathcal{K}^-.
    \label{eq:Deltaz_final}
\end{align}



\subsection{Recombination to Mean-Force Hamiltonian}

The $\mathfrak{su}(2)$ algebra allows the influence components to be recombined into a closed-form Mean-Force Hamiltonian. We work directly with the physical parameters $\theta, \beta, \omega_q$ and the odd kernel component $\mathcal{K}^-$. The influence magnitude $\delta \equiv \sqrt{\boldsymbol{\Delta}\cdot\boldsymbol{\Delta}}$ evaluates to
\begin{equation}
    \delta = \frac{\mathcal{K}^-\sin^2\theta}{2\omega_q} \chi,
    \qquad
    \chi(\beta,\theta) = \sqrt{1 - \frac{4\cot^2\theta}{\sinh^2(\beta\omega_q/2)}}.
\end{equation}
The reduced density matrix $\bar{\rho}_Q$ is Hermitian, and the Mean-Force Hamiltonian $H_{\mathrm{MF}} = -\beta^{-1} \log \bar{\rho}_Q$ is given in the real gauge by
\begin{equation}
    H_{\mathrm{MF}}^{(\mathrm{real})}(\beta) = -\frac{\Delta_0}{\beta}\mathbb{I} + \frac{\gamma}{\beta\sinh\gamma}\left( \mathcal{H}_\perp \sigma_x + \mathcal{H}_z \sigma_z \right),
\end{equation}
where the common prefactor is determined by the closure angle
\begin{align}
    \gamma &= \operatorname{arcosh}\!\left[ \cosh\left(\tfrac{\beta\omega_q}{2}\right)\cosh\delta - \frac{\sinh(\beta\omega_q/2)}{\chi}\sinh\delta \right],
\end{align}
and the unscaled field components are
\begin{align}
    \mathcal{H}_\perp &= \frac{2\cot\theta}{\chi\sinh(\beta\omega_q/2)}\sinh\delta, \\
    \mathcal{H}_z &= \frac{1}{\chi}\cosh\left(\tfrac{\beta\omega_q}{2}\right)\sinh\delta - \sinh\left(\tfrac{\beta\omega_q}{2}\right)\cosh\delta.
\end{align}
This result is exact for the spin-boson model with transverse coupling, expressing the equilibrium state purely in terms of the bath's on-shell response $\mathcal{K}^-$, the system frequency $\omega_q$, the mixing angle $\theta$, and the inverse temperature $\beta$.
