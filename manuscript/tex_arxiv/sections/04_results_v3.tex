

\subsection{Exact Closure for the Transverse Qubit}
Consider a single qubit with Hamiltonian $H_Q = (\omega_q/2)\sigma_z$ coupled to a harmonic bath via a transverse operator $f=\sigma_x$. This model describes a pure dephasing interaction in the rotated frame, or relaxation in the lab frame.

\paragraph{Adjoint Chain Closure.}
The adjoint action of $H_Q$ on $f$ generates a cyclic closed algebra. Compute the commutators:
\begin{equation}
f_0 = \sigma_x,\quad f_1 = [H_Q, \sigma_x] = i\omega_q \sigma_y,\quad f_2 = [H_Q, i\omega_q \sigma_y] = \omega_q^2 \sigma_x.
\end{equation}
The chain closes immediately on the Pauli subspace spanned by $\{\sigma_x, \sigma_y\}$:
\begin{equation}
f_{2k} = \omega_q^{2k} \sigma_x, \qquad f_{2k+1} = i\omega_q^{2k+1} \sigma_y.
\end{equation}
Condition (C1) is satisfied with $\mathcal{A} \subset \mathfrak{su}(2)$.

\paragraph{Quadratic Closure.}
The influence operator $\Delta(\beta)$ is formed from quadratic products $f_n f_m$. Since $\sigma_i \sigma_j = \delta_{ij}\mathbb{I} + i\epsilon_{ijk}\sigma_k$, the products of Pauli matrices remain in the Pauli algebra (plus identity). Specifically, the products fall into two classes:
\begin{align}
f_{2k}f_{2m}, f_{2k+1}f_{2m+1} &\propto \mathbb{I}, \\
f_{2k}f_{2m+1}, f_{2k+1}f_{2m} &\propto \sigma_z.
\end{align}
Substituting these into the general expansion $\Delta(\beta) = \frac{1}{2}\sum_{nm} C_{nm} f_n f_m$, the operator form simplifies to:
\begin{equation}
\Delta(\beta) = \alpha(\beta) \mathbb{I} + \gamma(\beta) \sigma_z.
\label{eq:Delta_qubit_form}
\end{equation}
Condition (C2) is thus satisfied. The coefficients $\alpha(\beta)$ and $\gamma(\beta)$ are scalar sums involving the moment matrix $C_{nm}$, but crucially, the operator structure is extremely simple: $\Delta(\beta)$ is diagonal in the energy eigenbasis.

\paragraph{BCH Closure.}
The final step is the logarithm $-\beta H_{\mathrm{MF}} = \log(e^{-\beta H_Q}e^{\Delta})$. Since $H_Q \propto \sigma_z$ and $\Delta \propto \mathbb{I} + \text{const} \times \sigma_z$, the two operators commute: $[H_Q, \Delta(\beta)] = 0$.
The BCH series truncates at the first term. Condition (C3) is satisfied trivially. The Exact Hamiltonian of Mean Force is:
\begin{equation}
H_{\mathrm{MF}}(\beta) = H_Q - \frac{1}{\beta}\Delta(\beta) = \frac{\omega_{\mathrm{MF}}(\beta)}{2} \sigma_z + \text{shift}.
\end{equation}
The interaction renormalizes the qubit frequency but does not induce new operators (like $\sigma_x$ or $\sigma_y$) in the equilibrium state.

\subsection{Connection to PRL Weak and Ultrastrong Limits}
We can now use this exact solution to verify the asymptotic limits reported in Ref.~\cite{cresserWeakUltrastrongCoupling2021a} and provide the finite-coupling interpolation.

\paragraph{Parametrization.}
Following Ref.~\cite{cresserWeakUltrastrongCoupling2021a}, let the bare system be $H_S = \frac{\omega_q}{2}\sigma_z$ and the coupling definition include a rotation angle $\theta$:
\begin{equation}
X = \cos\theta \sigma_z - \sin\theta \sigma_x.
\end{equation}
In our closure language, this mixes a commuting part ($\sigma_z$) and a transverse part ($\sigma_x$). The exact solution remains in the Pauli algebra, but the presence of the commuting component makes the BCH series non-trivial if we treated it generally. However, for the specific target of recovering the equilibrium state, we can use the explicit closure form.

\paragraph{Exact Solution for the General Qubit.}
The closure property allows us to solve the general case ($\theta \neq 0$) analytically, without resorting to numerical time-ordered integration. Although $[H_Q, f] \neq 0$ implies $[H_Q, \Delta] \neq 0$, the influence operator $\Delta(\beta)$ remains strictly within the Pauli algebra. We can decompose it exactly as:
\begin{equation}
\Delta(\beta) = \lambda^2 \left( \delta_0 \mathbb{I} + \delta_x \sigma_x + i\delta_y \sigma_y + \delta_z \sigma_z \right).
\end{equation}
Using the time-ordered product of the interaction picture operators $\tilde{X}(\tau)$, these coefficients are given by clean 1D integrals over the bath kernel $K(u)$:
Using the spectral density $J(\omega)$, the $u$-integrals can be performed exactly. Converting to the spectral domain, we obtain the fully analytic coefficients:
\begin{widetext}
\begin{align}
\delta_z &= -\sin^2\theta \int_0^\infty \!\!d\omega \frac{J(\omega)}{\omega^2-\omega_q^2} \left[ \frac{\sinh(\beta\omega_q)}{\omega_q} - \frac{\sinh(\beta\omega)}{\omega} \right], \\
\delta_x &= \frac{\sin 2\theta}{2} \int_0^\infty \!\!d\omega \frac{J(\omega)}{\omega(\omega^2-\omega_q^2)} \left[ \frac{\cosh(\beta\omega)-1}{\sinh(\beta\omega/2)} \sinh\!\left(\frac{\beta\omega_q}{2}\right) - \frac{\cosh(\beta\omega_q)-1}{\sinh(\beta\omega_q/2)} \sinh\!\left(\frac{\beta\omega}{2}\right) \right], \\
\delta_y &= \sin 2\theta \int_0^\infty \!\!d\omega \frac{J(\omega)}{\omega^2-\omega_q^2} \left[ \frac{\sinh(\beta\omega_q)}{\omega_q}\coth\!\left(\frac{\beta\omega_q}{2}\right)\sinh\!\left(\frac{\beta\omega}{2}\right) - \frac{\sinh(\beta\omega)}{\omega}\coth\!\left(\frac{\beta\omega}{2}\right)\sinh\!\left(\frac{\beta\omega_q}{2}\right) \right].
\end{align}
\end{widetext}
(The scalar shift $\delta_0$ acts as a global normalization and is omitted).
\emph{Note: While $\delta_x$ and $\delta_y$ arise from distinct operator parts (transverse coherence vs imaginary phase), they exhibit identical asymptotic scaling at low temperatures due to vacuum fluctuation dominance, as confirmed in Fig.~\ref{fig:qubit_deltas_temp}.}

The temperature dependence of these coefficients is shown in Fig.~\ref{fig:qubit_deltas_temp}. At high temperatures ($\beta \to 0$), all modifications vanish. At low temperatures, they saturate to large finite values.
Crucially, Fig.~\ref{fig:qubit_ratios_temp} reveals a hidden analytic collapse in the low-temperature limit: the imaginary phase coefficient $\delta_y$ asymptotically approaches the transverse coherence coefficient $\delta_x$ (ratio $\to 1$), while the population shift $\delta_z$ remains suppressed.
This implies that in the strong-coupling/low-temperature regime, the influence operator approaches a chiral form $\Delta \propto \sigma_x + i\sigma_y = 2\sigma_+$, projecting the system evolution onto specific parity sectors. This explains the stability of the ultrastrong limit analytically: the effective interaction selects a preferred basis (the $X$-eigenstates) via this specific ratio locking.

\begin{figure}[h]
    \centering
    \includegraphics[width=0.48\textwidth]{../figures/hmf_qubit_deltas_vs_temp.png}
    \caption{\label{fig:qubit_deltas_temp} Temperature dependence of the effective Hamiltonian modifications $|\delta_x|, |\delta_y|, |\delta_z|$ (log scale) for an Ohmic bath. The coefficients $\delta_x$ and $\delta_y$ determine the rotation of the mean-force state and show similar vacuum-dominated scaling at low temperatures.}
\end{figure}

\begin{figure}[h]
    \centering
    \includegraphics[width=0.48\textwidth]{../figures/hmf_qubit_ratios_vs_temp.png}
    \caption{\label{fig:qubit_ratios_temp} Low-temperature analytic collapse of the influence coefficients. The ratio $\delta_y/\delta_x$ converges exactly to $1.0$ as $T\to 0$, indicating a symmetric contribution from real and imaginary transverse parts, while $\delta_z$ is suppressed.}
\end{figure}
(Note: we define the Hamiltonian parameters as $H_Q = \frac{\omega_q}{2}\sigma_z$ and the coupling $f = \cos\theta\sigma_z - \sin\theta\sigma_x$).

Furthermore, the effective Hamiltonian takes a compact vector form. Writing the product of exponentials as $\bar{\rho}_Q = \exp(h_0 + \mathbf{h}\cdot\boldsymbol{\sigma})$, the components of the exact mean-force field $\mathbf{h} = (h_x, h_y, h_z)$ are:
\begin{align}
h_x &= \mathcal{N} \Lambda \delta_x \cosh\left(\frac{\beta\omega_q}{2}\right), \\
h_z &= \mathcal{N} \left[ \Lambda \delta_z \cosh\left(\frac{\beta\omega_q}{2}\right) - \cosh B \sinh\left(\frac{\beta\omega_q}{2}\right) \right].
\end{align}
Here, $\Lambda = \lambda^2\frac{\sinh B}{B}$ is the bath coupling factor, with $B = \lambda^2\sqrt{\delta_x^2+\delta_z^2}$. The geometric renormalization is $\mathcal{N} = \frac{h}{\sinh h}$. The effective Hamiltonian $H_{\mathrm{MF}} = -\frac{1}{\beta} (h_x \sigma_x + h_z \sigma_z)$ is explicitly real and confined to the $x-z$ plane, driven by the transverse susceptibility $\delta_x$. Figure~\ref{fig:effective_field} illustrates this field evolution and the associated effective temperature $\beta_{\text{eff}} = \beta \mathcal{N}$.

\begin{figure*}[ht]
    \centering
    \includegraphics[width=0.85\textwidth]{../figures/hmf_qubit_effective_field_vs_lambda.png}
    \caption{\label{fig:effective_field} Evolution of the effective Hamiltonian components and effective inverse temperature with coupling strength $\lambda$ (for $\beta=2, \omega_q=3$). (Left) The transverse field components $h_x, |h_y|$ grow quadratically with $\lambda$, overcoming the initial $z$-alignment to lock the system into the $X$-basis. (Right) The effective inverse temperature $\beta_{\text{eff}} \equiv \beta \mathcal{N}$ (defined via the geometric renormalization factor) decays to zero, reflecting the suppression of the bare reference frame.}
\end{figure*}

\paragraph{Microscopic origin of the rotation.}
The origin of the rotated effective field allows for a precise algebraic explanation. The influence functional is determined by the nested adjoint action of the bare Hamiltonian $H_Q = \frac{\omega_q}{2}\sigma_z$ on the interaction $X = \cos\theta \sigma_z - \sin\theta \sigma_x$. The $k$-th nested commutator $\text{ad}_{H_Q}^k[X]$ admits a closed form solution. For $k \ge 1$, the commutators cycle exclusively in the transverse plane:
\begin{align}
\text{ad}_{H_Q}^{2m}[X] &= -(-1)^m \omega_q^{2m} \sin\theta \sigma_x, \quad (m \ge 1) \\
\text{ad}_{H_Q}^{2m+1}[X] &= -i(-1)^m \omega_q^{2m+1} \sin\theta \sigma_y, \quad (m \ge 0).
\end{align}
However, evaluation of the quadratic product $T[X(\tau)X(\tau')]$ reveals a cancellation. The operator component proportional to $\sigma_y$ arises from the antisymmetric part of the contraction and is odd under the exchange $\tau \leftrightarrow \tau'$. Since the equilibrium kernel $K(\tau, \tau')$ is symmetric, this term integrates to zero. Consequently, the imaginary component vanishes ($\delta_y = 0$), and the effective field $\mathbf{h}$ remains in the real $x-z$ plane. The rotation is driven purely by the symmetric transverse susceptibility $\delta_x$.

\paragraph{Step-by-step recovery of asymptotic limits.}
This analytic solution contains the weak and ultrastrong limits naturally. We expand the exact result $\bar \rho_Q = e^{-\beta H_Q} e^\Delta$ in both regimes.

\subparagraph{1. Weak Coupling Limit ($\lambda \to 0$).}
Expanding the exponential $e^\Delta \approx 1+\Delta$, the unnormalized density becomes
\begin{equation}
\bar \rho_Q \approx e^{-\beta H_Q} + e^{-\beta H_Q}\Delta(\beta).
\end{equation}
Substituting $\Delta = \lambda^2(\delta_x \sigma_x + i\delta_y \sigma_y + \delta_z \sigma_z)$ and normalizing by $Z_Q = \Tr e^{-\beta H_Q}$, the correction is
\begin{equation}
\delta \rho = \frac{e^{-\beta H_Q}}{Z_Q} \Delta(\beta).
\end{equation}
However, the standard observable form is symmetric. Symmetrizing via $\rho \approx \rho_0 + \frac{1}{2}\{\rho_0, \Delta\}$, or explicitly calculating the observable averages, recovers the standard result. For example, the transverse coherence is
\begin{equation}
\langle \sigma_x \rangle_{\text{weak}} = \Tr[\delta \rho \sigma_x] = \lambda^2 \delta_x \tanh(\beta \omega_q/2) + i\lambda^2 \delta_y.
\end{equation}
Since $\langle \sigma_y \rangle = 0$ in equilibrium, the imaginary term $i\delta_y$ cancels against the anti-hermitian part of the expansion order-by-order to yield real observables. The weak-coupling population shift is dominated by $\delta_z \sigma_z$, which commutes with $H_Q$, yielding $\delta \langle \sigma_z \rangle \propto \delta_z (1-\tanh^2(\dots))$. This matches Eq.~(12) of Ref.~\cite{cresserWeakUltrastrongCoupling2021a}.

\subparagraph{2. Ultrastrong Coupling Limit ($\lambda \to \infty$).}
In this regime, the term $\lambda^2 (\vec{\delta}\cdot\vec{\sigma})$ dominates the Hamiltonian term $-\frac{\beta\omega_q}{2}\sigma_z$. The exponent becomes approximately diagonal in the eigenbasis of the operator $D = \delta_x \sigma_x + i\delta_y \sigma_y + \delta_z \sigma_z$.
As shown in Fig.~\ref{fig:qubit_ratios_temp}, at low temperatures $\delta_y \approx \delta_x$ and $\delta_z \ll \delta_x$, so $D \approx \delta_x (\sigma_x + i\sigma_y) = 2\delta_x \sigma_+$.
This nilpotent structure is peculiar; however, including the subleading terms, the operator $H_{\text{eff}} = H_Q - \frac{1}{\beta}\Delta$ is dominated by the interaction term.
The state effectively projects onto the subspace that minimizes the dominant part of the free energy.
Ref.~\cite{cresserWeakUltrastrongCoupling2021a} obtains the projector state
\begin{equation}
\rho_{US} \propto \exp\left(-\beta \sum_n P_n H_Q P_n \right),
\end{equation}
where $P_n$ are projectors onto the eigenstates of the interaction $X$.
Our exact solution $\rho \propto \exp(-\beta H_{MF})$ converges to this result because, as $\lambda \to \infty$, the large $\Delta$ term effectively filters the state. Specifically, the non-commuting parts of $\Delta$ (the $\sigma_x, \sigma_y$ terms) induce a rotation of the effective basis that aligns exactly with the interaction eigenbasis $X$ in the limit, reproducing the projector result perpendicular to the bare Hamiltonian.
Figure \ref{fig:qubit_prl_bridge_alignment} confirms this convergence numerically: the trace distance to the projector state vanishes as $\lambda^{-2}$ at strong coupling.

\begin{figure}[h]
    \centering
    \includegraphics[width=0.48\textwidth]{../figures/hmf_prl_qubit_analytic_bridge_alignment.png}
    \caption{\label{fig:qubit_prl_bridge_alignment} Trace distance between the exact numerical renormalization group (assumed ground truth) and our analytic closure approximations. The "Ordered Finite Model" (our closed HMF solution) provides excellent accuracy across the entire coupling range $\lambda \in [0, 8]$, bridging the gap between the perturbative weak-coupling theory and the asymptotic ultrastrong theory.}
\end{figure}
