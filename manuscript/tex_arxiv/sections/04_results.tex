% \section{Results: Analytic Examples}
% \label{sec:results}

% We provide three minimal analytic examples that illustrate the closure
% criterion and its consequences. These examples are not approximations; they
% serve only to show when closure is exact.

% \subsection{Commuting coupling}
% If $[H_Q,f]=0$, then $\mathrm{ad}_{H_Q}^n(f)=0$ for all $n\ge 1$, so
% $\tilde{f}(\tau)=f$ and the time ordering becomes trivial. The bilocal exponent
% reduces to a scalar multiple of $f^2$, and the HMF is exactly local. This is the
% simplest solvable case and is consistent with known exactly solvable strong
% coupling models\cite{campisiTalknerHanggi2009Solvable}.

% \subsection{Quadratic/Gaussian system}
% Consider a harmonic system with
% $H_Q = p^2/2m + (1/2)m\omega^2 q^2$ and linear coupling $f=q$. The adjoint
% action closes on the finite set $\{q,p,\mathbb{I}\}$ since
% $[H_Q,q] \propto p$ and $[H_Q,p] \propto q$. Consequently, the associative
% algebra generated by $\mathcal{A}_f$ is finite dimensional and the HMF is a
% quadratic operator. This reproduces the known Gaussian character of the
% reduced equilibrium state and its mean-force Hamiltonian in damped harmonic
% models\cite{grabertQuantumBrownianMotion1988,hiltHamiltonianMeanForce2011}.

% \subsection{Single qubit (Pauli algebra)}
% Let $H_Q = (\omega/2)\sigma_z$ and $f=\sigma_x$. Then
% $\mathrm{ad}_{H_Q}(f) = \omega i\sigma_y$ and
% $\mathrm{ad}_{H_Q}^2(f) = -\omega^2 \sigma_x$, so the adjoint chain closes on the
% Pauli algebra $\{\sigma_x,\sigma_y,\sigma_z,\mathbb{I}\}$. Hence the closure
% criterion is satisfied and $H_{\mathrm{MF}}$ lies in the same finite operator
% class. This is consistent with standard spin-boson constructions\cite{leggettDynamicsDissipativeTwostate1987}.

% ==========================================================
% Qubit (Pauli-closed algebra): exact H_MF in closed form
% ==========================================================

\subsection{Closed-algebra example: qubit system (Pauli closure)}
\label{sec:qubit_closed_algebra}

Assume the system is a qubit with Hamiltonian
\begin{equation}
    H_Q=\frac12\,\mathbf h\cdot\boldsymbol\sigma,
\end{equation}
and the CL coupling operator is a Hermitian Pauli-direction
\begin{equation}
    f=\mathbf n\cdot\boldsymbol\sigma,
\end{equation}
with $\mathbf h,\mathbf n\in\mathbb R^3$. The Pauli product identity
\begin{equation}
    (\mathbf a\cdot\boldsymbol\sigma)(\mathbf b\cdot\boldsymbol\sigma)
    =
    (\mathbf a\cdot\mathbf b)\,\mathbb I
    +
    i(\mathbf a\times\mathbf b)\cdot\boldsymbol\sigma
    \label{eq:pauli_product}
\end{equation}
implies the operator algebra closes on $\mathrm{span}\{\mathbb I,\boldsymbol\sigma\}$.
Consequently, any reduced equilibrium operator can be written in Bloch form
\begin{equation}
    \bar\rho_S(\beta)= r_0(\beta)\,\mathbb I + \mathbf r(\beta)\cdot\boldsymbol\sigma,
    \qquad r_0>|\mathbf r|.
    \label{eq:bar_rho_bloch}
\end{equation}

\paragraph{Step 1: adjoint-chain vectors.}
Define the nested-commutator chain
\begin{equation}
    f^{(n)} \equiv \mathrm{ad}_{H_Q}^n(f) = [H_Q,[H_Q,\dots[H_Q,f]\dots]],
\end{equation}
and write it in Pauli-vector form
\begin{equation}
    f^{(n)} = \mathbf u_n\cdot\boldsymbol\sigma,
    \qquad \mathbf u_n\in\mathbb R^3.
    \label{eq:fn_vector}
\end{equation}
For $H_Q=\frac12\mathbf h\cdot\boldsymbol\sigma$ one has the recursion
\begin{equation}
    \mathbf u_{n+1} = i\,\mathbf h\times \mathbf u_n,
    \qquad \mathbf u_0=\mathbf n.
    \label{eq:un_recursion}
\end{equation}
Equivalently, decomposing $\mathbf n=\mathbf n_\parallel+\mathbf n_\perp$ with respect to
$\hat{\mathbf h}=\mathbf h/|\mathbf h|$,
the chain closes on the three real directions
$\{\mathbf n_\parallel,\mathbf n_\perp,\hat{\mathbf h}\times\mathbf n\}$.

\paragraph{Step 2: kernel moments and the quadratic influence operator $\Delta$.}
Define the scalar kernel moments (bath data)
\begin{equation}
    M_{nm}(\beta)\equiv
    \int_0^\beta d\tau\int_0^\beta d\tau'\;
    K(\tau-\tau')\,\frac{\tau^n}{n!}\frac{(\tau')^m}{m!}.
    \label{eq:Mnm_def_qubit}
\end{equation}
Then the (exact) Gaussian influence operator may be written in adjoint-chain form
\begin{equation}
    \Delta(\beta)=\frac12\sum_{n,m\ge 0} M_{nm}(\beta)\, f^{(n)} f^{(m)}.
    \label{eq:Delta_adj_chain_qubit}
\end{equation}
Using Eq.~\eqref{eq:pauli_product} and $f^{(n)}=\mathbf u_n\cdot\boldsymbol\sigma$ gives a \emph{closed}
decomposition
\begin{equation}
    \Delta(\beta)=\alpha(\beta)\,\mathbb I + \boldsymbol\delta(\beta)\cdot\boldsymbol\sigma,
    \label{eq:Delta_bloch}
\end{equation}
with coefficients
\begin{align}
    \alpha(\beta)
    &=
    \frac12\sum_{n,m\ge 0} M_{nm}(\beta)\,\mathbf u_n\cdot\mathbf u_m,
    \label{eq:alpha_def}\\
    \boldsymbol\delta(\beta)
    &=
    \frac{i}{2}\sum_{n,m\ge 0} M_{nm}(\beta)\,\big(\mathbf u_n\times\mathbf u_m\big).
    \label{eq:delta_def}
\end{align}
(Hermiticity of $\bar\rho_S$ enforces that the final $\alpha$ and $\boldsymbol\delta$ entering $\bar\rho_S$
are real; in practice this is ensured by the symmetry properties of $M_{nm}$ and the structure of the $\mathbf u_n$.)

\paragraph{Step 3: reduced equilibrium operator in Bloch form.}
From the quenched construction one obtains
\begin{equation}
    \bar\rho_S(\beta)
    =
    e^{-\beta H_Q}\,e^{\Delta(\beta)}
    =
    e^{\alpha(\beta)}\,
    e^{-\frac{\beta}{2}\mathbf h\cdot\boldsymbol\sigma}\,
    e^{\boldsymbol\delta(\beta)\cdot\boldsymbol\sigma}.
    \label{eq:bar_rho_factor_qubit}
\end{equation}
For any Pauli vector $\mathbf x$,
\begin{equation}
    e^{\mathbf x\cdot\boldsymbol\sigma}
    =
    \cosh|\mathbf x|\;\mathbb I
    +
    \sinh|\mathbf x|\;\hat{\mathbf x}\cdot\boldsymbol\sigma.
    \label{eq:pauli_exp}
\end{equation}
Therefore the product in Eq.~\eqref{eq:bar_rho_factor_qubit} can be multiplied explicitly to yield
\begin{equation}
    \bar\rho_S(\beta)=r_0(\beta)\,\mathbb I+\mathbf r(\beta)\cdot\boldsymbol\sigma,
\end{equation}
where, defining
\begin{equation}
    \mathbf x\equiv -\frac{\beta}{2}\mathbf h,
    \qquad
    \mathbf y\equiv \boldsymbol\delta(\beta),
\end{equation}
one finds
\begin{align}
    r_0(\beta)
    &=
    e^{\alpha(\beta)}\Big(\cosh|\mathbf x|\cosh|\mathbf y|
    +(\hat{\mathbf x}\cdot\hat{\mathbf y})\,\sinh|\mathbf x|\,\sinh|\mathbf y|\Big),
    \label{eq:r0_formula}\\
    \mathbf r(\beta)
    &=
    e^{\alpha(\beta)}\Big(
    \hat{\mathbf x}\,\sinh|\mathbf x|\cosh|\mathbf y|
    +\hat{\mathbf y}\,\sinh|\mathbf y|\cosh|\mathbf x|
    + i(\hat{\mathbf x}\times\hat{\mathbf y})\,\sinh|\mathbf x|\,\sinh|\mathbf y|
    \Big).
    \label{eq:rvec_formula}
\end{align}
(For the physical parameter regime yielding Hermitian $\bar\rho_S$, $\mathbf r(\beta)$ is real.)

\paragraph{Step 4: exact logarithm and the Hamiltonian of mean force.}
Any positive $2\times 2$ operator in Bloch form
$R=r_0\mathbb I+\mathbf r\cdot\boldsymbol\sigma$ has eigenvalues
$\lambda_\pm=r_0\pm|\mathbf r|$ and
\begin{equation}
    \log R
    =
    \frac12\log(\lambda_+\lambda_-)\,\mathbb I
    +
    \frac12\log\!\left(\frac{\lambda_+}{\lambda_-}\right)\,
    \hat{\mathbf r}\cdot\boldsymbol\sigma.
    \label{eq:log_bloch}
\end{equation}
Applying Eq.~\eqref{eq:log_bloch} to $\bar\rho_S(\beta)$ gives
\begin{equation}
    H_{\mathrm{MF}}(\beta)
    =
    -\frac{1}{\beta}\log\!\left[\frac{\bar\rho_S(\beta)}{Z_X(\beta)}\right]
    =
    c(\beta)\,\mathbb I
    +
    \frac12\,\mathbf h_{\mathrm{MF}}(\beta)\cdot\boldsymbol\sigma,
    \label{eq:HMF_qubit_form}
\end{equation}
with
\begin{align}
    c(\beta)
    &=
    -\frac{1}{2\beta}\log\!\big(\lambda_+(\beta)\lambda_-(\beta)\big)
    +\frac{1}{\beta}\log Z_X(\beta),
    \label{eq:c_beta_def}\\
    \mathbf h_{\mathrm{MF}}(\beta)
    &=
    -\frac{1}{\beta}\log\!\left(\frac{\lambda_+(\beta)}{\lambda_-(\beta)}\right)\,
    \hat{\mathbf r}(\beta),
    \qquad
    \lambda_\pm(\beta)=r_0(\beta)\pm|\mathbf r(\beta)|.
    \label{eq:hMF_beta_def}
\end{align}
Equations \eqref{eq:alpha_def}--\eqref{eq:hMF_beta_def} constitute an exact closed-algebra construction of
$H_{\mathrm{MF}}(\beta)$ for a qubit at arbitrary coupling strength: all bath dependence enters through the scalar moments
$M_{nm}(\beta)$ (equivalently through $K$ or $J(\omega)$), while all system dependence enters through the finite adjoint-chain
vectors $\{\mathbf u_n\}$ determined by $(\mathbf h,\mathbf n)$.

% ==========================================================
% Add-on: make H_MF fully explicit for a qubit (Pauli closure)
% in terms of a finite set of scalar kernel-mode sums.
% ==========================================================

\subsection{Add-on: explicit $H_{\rm MF}$ for a qubit}
\label{sec:addon_qubit_explicit}

We specialise to a qubit with
\begin{equation}
    H_Q=\frac{1}{2}\,\mathbf h\cdot\boldsymbol\sigma,
    \qquad
    f=\mathbf n\cdot\boldsymbol\sigma,
    \qquad
    \mathbf h,\mathbf n\in\mathbb R^3,
\end{equation}
and define $\omega\equiv|\mathbf h|$ and $\hat{\mathbf h}\equiv\mathbf h/\omega$ (for $\omega\neq 0$).
Decompose $\mathbf n$ into components parallel/perpendicular to $\hat{\mathbf h}$:
\begin{equation}
    \mathbf n_\parallel \equiv (\mathbf n\cdot\hat{\mathbf h})\,\hat{\mathbf h},
    \qquad
    \mathbf n_\perp \equiv \mathbf n-\mathbf n_\parallel,
    \qquad
    \mathbf e_2 \equiv \hat{\mathbf h}\times \mathbf n_\perp.
\end{equation}

\paragraph{Interaction-picture coupling in closed form.}
Using the adjoint action of $\mathfrak{su}(2)$ (equivalently analytic continuation of the real-time rotation),
the similarity transform $\tilde f(\tau)=e^{\tau H_Q} f e^{-\tau H_Q}$ evaluates to
\begin{equation}
    \tilde f(\tau)
    =
    \Big(
        \mathbf n_\parallel
        + \cosh(\omega\tau)\,\mathbf n_\perp
        + i\sinh(\omega\tau)\,\mathbf e_2
    \Big)\cdot\boldsymbol\sigma.
    \label{eq:ftilde_qubit_closed}
\end{equation}
(For $\omega=0$, one has $\tilde f(\tau)=f$.)

\paragraph{Mode projections.}
Recall the translation-invariant Matsubara representation
\begin{equation}
    K(u)=\frac{1}{\beta}\sum_{\ell\in\mathbb Z}\kappa_\ell\,e^{i\nu_\ell u},
    \qquad
    \nu_\ell=\frac{2\pi\ell}{\beta},
    \qquad
    \kappa_{-\ell}=\kappa_\ell\in\mathbb R.
\end{equation}
Define the mode-projected operators
\begin{equation}
    \tilde F_\ell \equiv \int_0^\beta d\tau\,e^{i\nu_\ell\tau}\,\tilde f(\tau).
\end{equation}
Using Eq.~\eqref{eq:ftilde_qubit_closed} and $e^{i\nu_\ell\beta}=1$, we obtain the closed decomposition
\begin{equation}
    \tilde F_\ell
    =
    \Big(
        A_\ell\,\mathbf n_\parallel
        + B_\ell\,\mathbf n_\perp
        + i\,C_\ell\,\mathbf e_2
    \Big)\cdot\boldsymbol\sigma,
    \label{eq:Ftilde_qubit_decomp}
\end{equation}
with scalar coefficients
\begin{align}
    A_\ell
    &\equiv
    \int_0^\beta d\tau\,e^{i\nu_\ell\tau}
    =
    \beta\,\delta_{\ell 0},
    \label{eq:Aell_def}\\[3pt]
    B_\ell
    &\equiv
    \int_0^\beta d\tau\,e^{i\nu_\ell\tau}\cosh(\omega\tau)
    =
    \frac{1}{2}\left(
        \frac{e^{\beta\omega}-1}{\omega+i\nu_\ell}
        +
        \frac{1-e^{-\beta\omega}}{\omega-i\nu_\ell}
    \right),
    \label{eq:Bell_def}\\[3pt]
    C_\ell
    &\equiv
    \int_0^\beta d\tau\,e^{i\nu_\ell\tau}\sinh(\omega\tau)
    =
    \frac{1}{2}\left(
        \frac{e^{\beta\omega}-1}{\omega+i\nu_\ell}
        -
        \frac{1-e^{-\beta\omega}}{\omega-i\nu_\ell}
    \right).
    \label{eq:Cell_def}
\end{align}
For $\ell\neq 0$, $A_\ell=0$ while $B_\ell$ and $C_\ell$ remain nonzero.

\paragraph{Exact influence operator $\Delta=\alpha\,\mathbb I+\boldsymbol\delta\cdot\boldsymbol\sigma$.}
From the sum-of-squares identity,
\begin{equation}
    \Delta(\beta)
    =
    \frac{1}{2\beta}\sum_{\ell\in\mathbb Z}\kappa_\ell\,\tilde F_\ell \tilde F_\ell^\dagger,
    \label{eq:Delta_sumsq_recalled_addon}
\end{equation}
and the Pauli product rule
$(\mathbf a\cdot\boldsymbol\sigma)(\mathbf b\cdot\boldsymbol\sigma)
=(\mathbf a\cdot\mathbf b)\mathbb I+i(\mathbf a\times\mathbf b)\cdot\boldsymbol\sigma$,
one finds
\begin{equation}
    \Delta(\beta)=\alpha(\beta)\,\mathbb I + \boldsymbol\delta(\beta)\cdot\boldsymbol\sigma,
    \label{eq:Delta_alpha_delta}
\end{equation}
with \emph{explicit} coefficients
\begin{align}
    \alpha(\beta)
    &=
    \frac{1}{2\beta}\sum_{\ell\in\mathbb Z}\kappa_\ell
    \Big(
        |A_\ell|^2\,|\mathbf n_\parallel|^2
        + |B_\ell|^2\,|\mathbf n_\perp|^2 \nonumber\\
        &\quad + |C_\ell|^2\,|\mathbf e_2|^2
    \Big),
    \label{eq:alpha_explicit}\\[4pt]
    \boldsymbol\delta(\beta)
    &=
    \frac{1}{2\beta}\sum_{\ell\in\mathbb Z}\kappa_\ell
    \Big(
        2\,\mathrm{Re}[B_\ell C_\ell^*]\,
        |\mathbf n_\perp|^2\,\hat{\mathbf h}
    \Big).
    \label{eq:delta_explicit}
\end{align}
Here we used $\mathbf n_\parallel\times \mathbf e_2 = -(\mathbf n\cdot\hat{\mathbf h})\,\mathbf n_\perp$ and
$\mathbf n_\perp\times \mathbf e_2 = |\mathbf n_\perp|^2\hat{\mathbf h}$, and the fact that the $\kappa_\ell$ are real and even in $\ell$,
so the non-Hermitian transverse components cancel in the $\ell$-sum, leaving $\boldsymbol\delta(\beta)$ aligned with $\hat{\mathbf h}$.
(For $\mathbf n_\perp=\mathbf 0$ the coupling commutes with $H_Q$ and $\boldsymbol\delta=0$.)

It is convenient to define the two scalar bath-dressed qubit coefficients
\begin{align}
    \Xi_0(\beta)&\equiv \frac{1}{2\beta}\sum_{\ell}\kappa_\ell |A_\ell|^2, \nonumber\\
    \Xi_c(\beta)&\equiv \frac{1}{2\beta}\sum_{\ell}\kappa_\ell |B_\ell|^2, \nonumber\\
    \Xi_s(\beta)&\equiv \frac{1}{2\beta}\sum_{\ell}\kappa_\ell |C_\ell|^2,
\end{align}
and
\begin{equation}
    \Xi_{cs}(\beta)\equiv \frac{1}{\beta}\sum_{\ell}\kappa_\ell\,\mathrm{Re}[B_\ell C_\ell^*],
\end{equation}
so that Eqs.~\eqref{eq:alpha_explicit}--\eqref{eq:delta_explicit} become
\begin{align}
    \alpha(\beta)
    &= \Xi_0(\beta)\,|\mathbf n_\parallel|^2
       + \Xi_c(\beta)\,|\mathbf n_\perp|^2
       + \Xi_s(\beta)\,|\mathbf e_2|^2,
       \label{eq:alpha_compact}\\
    \boldsymbol\delta(\beta)
    &= \frac{1}{2}\,\Xi_{cs}(\beta)\,|\mathbf n_\perp|^2\,\hat{\mathbf h}.
       \label{eq:delta_compact}
\end{align}
Since $|\mathbf e_2|^2=|\mathbf n_\perp|^2$, the scalar part may be written as
$\alpha(\beta)=\Xi_0|\mathbf n_\parallel|^2+(\Xi_c+\Xi_s)|\mathbf n_\perp|^2$.

\paragraph{Collapse to a single exponent and take the logarithm.}
The reduced equilibrium operator is
\begin{align}
    \bar\rho_S(\beta)&=e^{-\beta H_Q}e^{\Delta(\beta)}
    = e^{\alpha(\beta)}\,e^{\mathbf x\cdot\boldsymbol\sigma}\,e^{\mathbf y\cdot\boldsymbol\sigma}, \nonumber\\
    \mathbf x&\equiv -\frac{\beta}{2}\mathbf h,
    \qquad
    \mathbf y\equiv \boldsymbol\delta(\beta).
    \label{eq:bar_rho_qubit_xy}
\end{align}
In $\mathfrak{su}(2)$ one has the exact composition rule
\begin{equation}
    e^{\mathbf x\cdot\boldsymbol\sigma}\,e^{\mathbf y\cdot\boldsymbol\sigma}
    = e^{\mathbf z\cdot\boldsymbol\sigma},
    \label{eq:su2_comp_rule}
\end{equation}
where $\mathbf z$ is determined by
\begin{align}
    \cosh|\mathbf z|
    &=
    \cosh|\mathbf x|\cosh|\mathbf y|
    + (\hat{\mathbf x}\cdot\hat{\mathbf y})\,\sinh|\mathbf x|\sinh|\mathbf y|,
    \label{eq:z_norm}\\
    \hat{\mathbf z}\,\sinh|\mathbf z|
    &=
    \hat{\mathbf x}\,\sinh|\mathbf x|\cosh|\mathbf y|
    + \hat{\mathbf y}\,\sinh|\mathbf y|\cosh|\mathbf x| \nonumber\\
    &\quad + i(\hat{\mathbf x}\times\hat{\mathbf y})\,\sinh|\mathbf x|\sinh|\mathbf y|.
    \label{eq:z_dir}
\end{align}
Thus
\begin{equation}
    \bar\rho_S(\beta)=e^{\alpha(\beta)}\,e^{\mathbf z(\beta)\cdot\boldsymbol\sigma},
\end{equation}
and taking the logarithm is immediate:
\begin{equation}
    \log \bar\rho_S(\beta) = \alpha(\beta)\,\mathbb I + \mathbf z(\beta)\cdot\boldsymbol\sigma.
\end{equation}

\paragraph{Final explicit $H_{\rm MF}$.}
Using $H_{\rm MF}(\beta)=-(1/\beta)\log[\bar\rho_S(\beta)/Z_X(\beta)]$, we obtain
\begin{equation}
    \boxed{
    H_{\rm MF}(\beta)
    =
    \left(
        \frac{1}{\beta}\log Z_X(\beta) - \frac{\alpha(\beta)}{\beta}
    \right)\mathbb I
    -\frac{1}{\beta}\,\mathbf z(\beta)\cdot\boldsymbol\sigma,
    }
    \label{eq:HMF_qubit_explicit_final}
\end{equation}
where $\alpha(\beta)$ is given explicitly by Eqs.~\eqref{eq:alpha_explicit} or \eqref{eq:alpha_compact},
$\boldsymbol\delta(\beta)$ by Eqs.~\eqref{eq:delta_explicit} or \eqref{eq:delta_compact},
and $\mathbf z(\beta)$ is obtained from the closed composition relations \eqref{eq:z_norm}--\eqref{eq:z_dir}
with $\mathbf x=-\frac{\beta}{2}\mathbf h$ and $\mathbf y=\boldsymbol\delta(\beta)$.

\medskip
In the commuting limit $\mathbf n_\perp=\mathbf 0$ one has $\boldsymbol\delta(\beta)=\mathbf 0$ and $\mathbf z=\mathbf x$,
so Eq.~\eqref{eq:HMF_qubit_explicit_final} reduces to
$H_{\rm MF}=H_Q-(\kappa_0/2)f^2+(1/\beta)\log Z_X\,\mathbb I$, as expected.
