\section{Quenched representation of the reduced equilibrium operator\label{sec:quenched}}

The Hamiltonian of mean force is defined implicitly by the reduced equilibrium
operator $\bar\rho_S(\beta)=\Tr_X e^{-\beta H_{\mathrm{tot}}}$. To obtain a
constructive handle on $H_{\mathrm{MF}}(\beta)$, it is useful to reorganise the
problem in two steps.

First, we treat $\bar\rho_S(\beta)$ as an imaginary-time propagator on the system
Hilbert space: it is a positive operator obtained by evolving over an interval
of length $\beta$, and any such operator may be written in the form
\begin{equation}
    \bar\rho_S(\beta)
    =
    \mathcal T_\tau \exp\!\left[-\int_0^\beta d\tau\, H_{\mathrm{eff}}(\tau)\right],
    \label{eq:bar_rho_as_tau_propagator}
\end{equation}
for some (in general non-unique) generator $H_{\mathrm{eff}}(\tau)$.

Second, we explicitly allow the generator to depend on $\tau$. This is not a technical luxury but a structural necessity: after eliminating the bath, the system acquires a non-local memory-kernel self-coupling. As emphasized in recent partition-free approaches~\cite{mccaulPartitionFreeApproach2018,mccaulHowToWin2021} and the development of stochastic Liouville-von Neumann methods~\cite{stockburgerExactNumberRepresentation2002,stockburgerSimulatingSpinbosonDynamics2004,stockburgerVarianceReduction2016,stockburgerStochasticLiouvillevon2018}, this non-local object can be mapped \emph{exactly} to a local, stochastic evolution via the Hubbard-Stratonovich transformation. This transformation trades the temporal non-locality for an average over an auxiliary Gaussian field, ensuring that the full non-Markovian character of the bath is retained without approximation.

A full derivation of this result is provided in Appendix~\ref{app:influence_derivation}.

For a CL environment the bath can be eliminated exactly, yielding an imaginary-time
influence functional with a memory kernel fixed by the spectral density $J(\omega)$.
Rather than working with the resulting bilocal object directly, we adopt an equivalent
representation in which the bath enters through an auxiliary Gaussian field and the
system evolves with a \emph{local-in-$\tau$} generator. Specifically, we introduce a real stochastic field $\xi(\tau)$ on $\tau\in[0,\beta]$ with
\begin{equation}
    \langle \xi(\tau)\rangle_\xi = 0,
    \qquad
    \langle \xi(\tau)\xi(\tau')\rangle_\xi = K(\tau-\tau'),
    \label{eq:xi_cov_main}
\end{equation}
where $K$ is the Euclidean bath correlation kernel determined by $J(\omega)$. A convenient
explicit form is
\begin{equation}
    K(\tau)=\frac{1}{\pi}\int_0^\infty d\omega\,J(\omega)\,
    \frac{\cosh\!\big(\omega(\beta/2-|\tau|)\big)}{\sinh(\beta\omega/2)} ,
    \qquad \tau\in[-\beta,\beta].
    \label{eq:K_of_tau_main}
\end{equation}

For each realisation $\xi$, we define the quenched effective Hamiltonian
\begin{equation}
    H_{\mathrm{eff}}[\xi](\tau) \equiv H_Q + \xi(\tau)\,f,
    \label{eq:Heff_xi_main}
\end{equation}
and the corresponding quenched propagator
\begin{equation}
    U_\xi(\beta)\equiv \mathcal T_\tau \exp\!\left[-\int_0^\beta d\tau\,H_{\mathrm{eff}}[\xi](\tau)\right].
    \label{eq:Uxi_def_main}
\end{equation}
The reduced equilibrium operator admits the quenched average representation
\begin{equation}
    \bar\rho_S(\beta)\equiv \Tr_X e^{-\beta H_{\mathrm{tot}}}
    \;=\;
    \big\langle U_\xi(\beta)\big\rangle_\xi,
    \label{eq:quenched_identity_main}
\end{equation}
with the Gaussian measure on $\xi$ fixed by Eqs.\eqref{eq:xi_cov_main}--\eqref{eq:K_of_tau_main}.
Consequently, performing the average in \eqref{eq:quenched_identity_main} yields the
Hamiltonian of mean force identically via
\begin{equation}
    H_{\mathrm{MF}}(\beta)
    = -\frac{1}{\beta}\log\!\left[\frac{\bar\rho_S(\beta)}{Z_X(\beta)}\right],
    \qquad Z_X(\beta)=\Tr_X e^{-\beta H_X},
    \label{eq:HMF_from_quenched_identity_main}
\end{equation}
with the scalar fixed by the chosen normalisation. Consequently, calculating $H_{\mathrm{MF}}(\beta)$ is equivalent to determining $\bar\rho_S(\beta)$, which in turn requires performing the average in \eqref{eq:quenched_identity_main} over the Gaussian field $\xi$. 

To perform this trick, it is necessary first to introduce the imaginary-time interaction picture with respect to $H_Q$ by writing
\begin{align}
    U_\xi(\beta)&=e^{-\beta H_Q}\,W_\xi(\beta), \nonumber\\
    W_\xi(\beta)&\equiv \mathcal T_\tau
    \exp\!\left[-\int_0^\beta d\tau\,\xi(\tau)\,\tilde f(\tau)\right],
    \label{eq:Uxi_interaction_picture}
\end{align}
where
\begin{equation}
    \tilde f(\tau)\equiv e^{\tau H_Q}\,f\,e^{-\tau H_Q}.
    \label{eq:f_tilde_def}
\end{equation}
This factors out the $\int H_Q$ piece from the ordered exponential, leaving $\int \xi(\tau)\tilde f(\tau)$ as the remaining term to be averaged over. As a final step before averaging, it is convenient to introduce the adjoint action of $H_Q$ on $f$, defined as $\mathrm{ad}_{H_Q}(f) = [H_Q, f]$ and $\mathrm{ad}_{H_Q}^n(f) = [H_Q, \mathrm{ad}_{H_Q}^{n-1}(f)]$. Then the interaction-picture coupling obeys
\begin{equation}
    \tilde f(\tau)=e^{\tau H_Q} f e^{-\tau H_Q}
    = e^{\tau\,{\rm ad}_{H_Q}}(f)
    = \sum_{n=0}^\infty \frac{\tau^n}{n!}\,{\rm ad}_{H_Q}^n(f).
    \label{eq:adjoint_series}
\end{equation}
All operator growth induced by eliminating the bath is therefore controlled by the nested-commutator chain $\{{\rm ad}_{H_Q}^n(f)\}_{n\ge 0}$.

We now compute the averaged interaction-picture propagator and actor out the free evolution:
\begin{equation}
    U_\xi(\beta)=e^{-\beta H_Q}\,W_\xi(\beta),
    \qquad
    W_\xi(\beta)\equiv
    \mathcal T_\tau
    \exp\!\left[
        -\int_0^\beta d\tau\,\xi(\tau)\,\tilde f(\tau)
    \right].
    \label{eq:Uxi_factor_Wxi}
\end{equation}
Taking the Gaussian average yields
\begin{equation}
    \bar\rho_S(\beta)=e^{-\beta H_Q}\,\bar W(\beta),
    \qquad
    \bar W(\beta)\equiv \big\langle W_\xi(\beta)\big\rangle_\xi.
    \label{eq:bar_rho_as_eH_barW}
\end{equation}
Thus the task is to evaluate the averaged interaction-picture propagator $\bar W(\beta)$. To do so, we expand the time-ordered exponential in Eq.~\eqref{eq:Uxi_factor_Wxi} in the standard Dyson series form,
\begin{equation}
\begin{split}
    W_\xi(\beta)
    =
    \sum_{k=0}^\infty (-1)^k
    \int_{\Delta_k} d\tau_1\cdots d\tau_k\;
    &\xi(\tau_1)\cdots\xi(\tau_k) \\
    &\times \tilde f(\tau_1)\cdots \tilde f(\tau_k),
\end{split}
    \label{eq:Wxi_Dyson_simplex}
\end{equation}
where $\Delta_k$ denotes the time-ordered integration domain
\begin{equation}
    \Delta_k:\qquad
    0\le \tau_k\le\cdots\le \tau_2\le\tau_1\le\beta,
    \label{eq:simplex_domain_def}
\end{equation}
and $d\tau_1\cdots d\tau_k$ is shorthand for the product measure over that domain.

Averaging term-by-term gives
\begin{equation}
\begin{split}
    \bar W(\beta)
    =
    \sum_{k=0}^\infty (-1)^k
    &\int_{\Delta_k} d\tau_1\cdots d\tau_k\,
    \big\langle \xi(\tau_1)\cdots\xi(\tau_k)\big\rangle_\xi \\
    &\times \tilde f(\tau_1)\cdots \tilde f(\tau_k).
\end{split}
    \label{eq:barW_Dyson_avg}
\end{equation}

\paragraph{Gaussian moments and Wick's theorem.}
Since $\xi$ is Gaussian with zero mean, all odd moments vanish:
\begin{equation}
    \big\langle \xi(\tau_1)\cdots\xi(\tau_{2n+1})\big\rangle_\xi = 0.
    \label{eq:odd_moments_vanish}
\end{equation}
For even moments, Wick's theorem (Isserlis' theorem) yields a sum over all pairings:
\begin{equation}
    \big\langle \xi(\tau_1)\cdots\xi(\tau_{2n})\big\rangle_\xi
    =
    \sum_{P\in \mathcal P_{2n}}
    \prod_{(i,j)\in P}
    K(\tau_i-\tau_j),
    \label{eq:wick_pairings_general}
\end{equation}
where $\mathcal P_{2n}$ is the set of perfect matchings (pair partitions) of $\{1,\dots,2n\}$.
Substituting Eq.~\eqref{eq:wick_pairings_general} into Eq.~\eqref{eq:barW_Dyson_avg} gives the explicit
all-orders expansion
\begin{equation}
\begin{split}
    \bar W(\beta)
    =
    \sum_{n=0}^\infty
    \int_{\Delta_{2n}} &d\tau_1\cdots d\tau_{2n}\,
    \left(
        \sum_{P\in \mathcal P_{2n}}
        \prod_{(i,j)\in P}
        K(\tau_i-\tau_j)
    \right) \\
    &\times \tilde f(\tau_1)\cdots \tilde f(\tau_{2n}).
\end{split}
    \label{eq:barW_all_orders_pairings}
\end{equation}
This series is precisely the Wick expansion of a time-ordered exponential
with a quadratic (pairwise) contraction. Since the exponent in Eq.~\eqref{eq:Uxi_factor_Wxi} is linear in the
Gaussian field, the average can be resummed exactly to yield the bilocal influence-functional form
\begin{equation}
    \boxed{
    \bar W(\beta)
    =
    \mathcal T_\tau
    \exp\!\left[
        \frac{1}{2}
        \int_0^\beta d\tau
        \int_0^\beta d\tau'\;
        K(\tau-\tau')\,\tilde f(\tau)\tilde f(\tau')
    \right].
    }
    \label{eq:barW_bilocal_exact}
\end{equation}
For later convenience we define the bilocal quadratic operator
\begin{equation}
    \Delta
    \;\equiv\;
    \frac{1}{2}
    \int_0^\beta d\tau
    \int_0^\beta d\tau'\;
    K(\tau-\tau')\,\tilde f(\tau)\tilde f(\tau'),
    \label{eq:Delta_def_bilocal}
\end{equation}
so that
\begin{equation}
    \bar W(\beta)=\mathcal T_\tau e^{\Delta}.
    \label{eq:barW_as_Texp_Delta}
\end{equation}
At this stage, the Gaussian field has been eliminated exactly, and all bath dependence is contained in the
translation-invariant kernel $K(\tau-\tau')$. In the next step we exploit this translation invariance to diagonalise $K$ in Matsubara modes and rewrite $\Delta$ as a sum of squares of mode-projected operators.

\subsection{Diagonalising the translation-invariant kernel and a sum-of-squares form}
\label{sec:kernel_matsubara_diagonalisation}

We now exploit the key structural property of the Euclidean correlation kernel:
it is translation invariant on the thermal circle, i.e.\ it depends only on the time difference.
Equivalently, $K$ is $\beta$-periodic and admits a bosonic Matsubara Fourier series.
This allows us to rewrite the bilocal quadratic operator $\Delta$ in Eq.~\eqref{eq:Delta_def_bilocal}
in a form that isolates all bath dependence into scalar mode weights.

\paragraph{Matsubara expansion of the kernel.}
On $\tau\in[0,\beta]$, any $\beta$-periodic translation-invariant kernel may be expanded as
\begin{equation}
    K(u)
    =
    \frac{1}{\beta}\sum_{\ell\in\mathbb Z}\kappa_\ell\,e^{i\nu_\ell u},
    \qquad
    \nu_\ell\equiv \frac{2\pi\ell}{\beta},
    \label{eq:K_matsubara_series}
\end{equation}
with Fourier coefficients
\begin{equation}
    \kappa_\ell
    \equiv
    \int_0^\beta du\;K(u)\,e^{-i\nu_\ell u}.
    \label{eq:kappa_l_def}
\end{equation}
For a real, even kernel $K(u)=K(-u)$ one has $\kappa_\ell\in\mathbb R$ and $\kappa_{-\ell}=\kappa_\ell$.
(For the Caldeira--Leggett class one further has $\kappa_\ell\ge 0$.)

Substituting Eq.~\eqref{eq:K_matsubara_series} into Eq.~\eqref{eq:Delta_def_bilocal} gives
\begin{align}
    \Delta
    &=
    \frac{1}{2}\int_0^\beta d\tau\int_0^\beta d\tau'\;
    \left[
        \frac{1}{\beta}\sum_{\ell\in\mathbb Z}\kappa_\ell\,e^{i\nu_\ell(\tau-\tau')}
    \right]
    \tilde f(\tau)\tilde f(\tau')
    \nonumber\\
    &=
    \frac{1}{2\beta}\sum_{\ell\in\mathbb Z}\kappa_\ell\;
    \left(\int_0^\beta d\tau\,e^{i\nu_\ell\tau}\,\tilde f(\tau)\right)
    \left(\int_0^\beta d\tau'\,e^{-i\nu_\ell\tau'}\,\tilde f(\tau')\right).
    \label{eq:Delta_matsubara_inserted}
\end{align}

\paragraph{Mode-projected operators and sum-of-squares form.}
Define the Matsubara-projected interaction-picture operators
\begin{equation}
    \tilde F_\ell
    \;\equiv\;
    \int_0^\beta d\tau\,e^{i\nu_\ell\tau}\,\tilde f(\tau).
    \label{eq:Ftilde_l_def}
\end{equation}
Since $\tilde f(\tau)$ is Hermitian for Hermitian $f$, one has $\tilde F_{-\ell}=\tilde F_\ell^\dagger$.
Equation~\eqref{eq:Delta_matsubara_inserted} then becomes the exact quadratic sum
\begin{equation}
    \boxed{
    \Delta
    =
    \frac{1}{2\beta}\sum_{\ell\in\mathbb Z}\kappa_\ell\,
    \tilde F_\ell\,\tilde F_\ell^\dagger.
    }
    \label{eq:Delta_sum_of_squares_complex}
\end{equation}
This form cleanly separates bath and system structure:
all information about the bath enters through the scalar mode weights $\{\kappa_\ell\}$,
while the system algebra is carried by the family of operators $\{\tilde F_\ell\}$.

\paragraph{Manifestly real form (optional).}
When $K$ is real and even (so $\kappa_\ell=\kappa_{-\ell}\in\mathbb R$), it is often convenient to pair $\pm\ell$
and write $\Delta$ in terms of cosine and sine projections. Using
\begin{equation}
    K(u)=\frac{\kappa_0}{\beta}+\frac{2}{\beta}\sum_{\ell\ge 1}\kappa_\ell\cos(\nu_\ell u),
    \label{eq:K_cos_series}
\end{equation}
define the real mode operators
\begin{align}
    \tilde F_{\ell,c}&\equiv \int_0^\beta d\tau\,\cos(\nu_\ell\tau)\,\tilde f(\tau), \nonumber\\
    \tilde F_{\ell,s}&\equiv \int_0^\beta d\tau\,\sin(\nu_\ell\tau)\,\tilde f(\tau),
    \quad (\ell\ge 1),
    \label{eq:Ftilde_cos_sin_def}
\end{align}
and $\tilde F_{0,c}\equiv \int_0^\beta d\tau\,\tilde f(\tau)$.
Then Eq.~\eqref{eq:Delta_sum_of_squares_complex} is equivalent to
\begin{equation}
    \Delta
    =
    \frac{\kappa_0}{2\beta}\,\tilde F_{0,c}^2
    +
    \frac{1}{2\beta}\sum_{\ell\ge 1}\kappa_\ell\,
    \Big(\tilde F_{\ell,c}^2+\tilde F_{\ell,s}^2\Big),
    \label{eq:Delta_sum_of_squares_real}
\end{equation}
making explicit that the quadratic form is real.

In the next step we express the projected operators $\tilde F_\ell$ (or equivalently $\tilde F_{\ell,c/s}$)
in terms of the nested-commutator chain generated by $H_Q$ acting on $f$. This yields a generating-function
representation for all kernel moments and will allow us to collect the bath dependence into scalar prefactors
multiplying products of $\mathrm{ad}_{H_Q}^n(f)$.

% =========================
% Step 3: Express mode operators in the adjoint chain; generating functions for moments
% (keeps factorial/binomial structure explicit)
% =========================

\subsection{Adjoint-chain expansion and generating functions for Matsubara moments}
\label{sec:adjoint_chain_generating_functions}

We now rewrite the mode-projected operators $\tilde F_\ell$ in Eq.~\eqref{eq:Ftilde_l_def}
in a basis that makes the operator growth induced by the bath completely explicit.
The key observation is that the entire $\tau$-dependence of $\tilde f(\tau)$ is generated by the
adjoint action of $H_Q$.

\paragraph{Adjoint chain and interaction-picture coupling.}
Introduce the adjoint superoperator
\begin{equation}
\begin{split}
    &\mathrm{ad}_{H_Q}(A)\equiv [H_Q,A],
    \quad
    \mathrm{ad}_{H_Q}^{0}(A)\equiv A, \\
    &\mathrm{ad}_{H_Q}^{n}(A)\equiv [H_Q,\mathrm{ad}_{H_Q}^{n-1}(A)].
\end{split}
    \label{eq:ad_def_recalled}
\end{equation}
Define the nested-commutator chain generated from $f$,
\begin{equation}
    f_n \;\equiv\; \mathrm{ad}_{H_Q}^{n}(f),
    \qquad n\ge 0.
    \label{eq:fn_def}
\end{equation}
Then the interaction-picture operator is
\begin{equation}
    \tilde f(\tau)
    =
    e^{\tau H_Q} f e^{-\tau H_Q}
    =
    e^{\tau\,\mathrm{ad}_{H_Q}}(f)
    =
    \sum_{n=0}^\infty \frac{\tau^n}{n!}\,f_n.
    \label{eq:ftilde_adjoint_series_step3}
\end{equation}
Equation~\eqref{eq:ftilde_adjoint_series_step3} cleanly separates
(i) the scalar monomials $\tau^n/n!$ from
(ii) the operator content $f_n$, and the latter is the sole source of operator growth.

\paragraph{Matsubara moments and operator-valued mode expansion.}
Substituting Eq.~\eqref{eq:ftilde_adjoint_series_step3} into the mode definition \eqref{eq:Ftilde_l_def} yields
\begin{equation}
    \tilde F_\ell
    =
    \int_0^\beta d\tau\,e^{i\nu_\ell\tau}\sum_{n=0}^\infty \frac{\tau^n}{n!}\,f_n
    =
    \sum_{n=0}^\infty I_n(\nu_\ell)\,f_n,
    \label{eq:Ftilde_l_as_sum_fn}
\end{equation}
where we have defined the factorial-weighted Matsubara moments
\begin{equation}
    I_n(\nu)
    \;\equiv\;
    \int_0^\beta d\tau\,\frac{\tau^n}{n!}\,e^{i\nu\tau}.
    \label{eq:In_def}
\end{equation}
Since $\nu_{-\ell}=-\nu_\ell$, one has $I_n(\nu_{-\ell})=I_n(\nu_\ell)^\ast$ and hence $\tilde F_{-\ell}=\tilde F_\ell^\dagger$
for Hermitian $f$.

\paragraph{Generating function for all moments $I_n(\nu_\ell)$.}
The full family $\{I_n(\nu_\ell)\}_{n\ge 0}$ is generated by the scalar function
\begin{equation}
\begin{split}
    I(\nu;s)
    \;\equiv\;
    \sum_{n=0}^\infty s^n\,I_n(\nu)
    &=
    \int_0^\beta d\tau\,\exp\!\big[(s+i\nu)\tau\big] \\
    &=
    \frac{e^{(s+i\nu)\beta}-1}{s+i\nu}.
\end{split}
    \label{eq:I_gen_general}
\end{equation}
For bosonic Matsubara frequencies $\nu=\nu_\ell=2\pi\ell/\beta$ one has $e^{i\nu_\ell\beta}=1$, and therefore
\begin{equation}
    I(\nu_\ell;s)
    =
    \frac{e^{s\beta}-1}{s+i\nu_\ell}.
    \label{eq:I_gen_matsubara}
\end{equation}
The individual coefficients are recovered by differentiation at $s=0$:
\begin{equation}
    I_n(\nu_\ell)
    =
    \frac{1}{n!}\,\partial_s^n I(\nu_\ell;s)\Big|_{s=0}.
    \label{eq:In_from_gen}
\end{equation}
Equation~\eqref{eq:I_gen_matsubara} provides a compact representation for all prefactors produced by the time integrals.

\paragraph{Factorisation of the quadratic form.}
Combining Eq.~\eqref{eq:Delta_sum_of_squares_complex} with Eq.~\eqref{eq:Ftilde_l_as_sum_fn} yields
\begin{align}
    \Delta
    &=
    \frac{1}{2\beta}\sum_{\ell\in\mathbb Z}\kappa_\ell\,
    \left(\sum_{n=0}^\infty I_n(\nu_\ell)\,f_n\right)
    \left(\sum_{m=0}^\infty I_m(\nu_\ell)^\ast\,f_m\right)
    \nonumber\\
    &=
    \frac{1}{2}
    \sum_{n,m=0}^\infty
    C_{nm}(\beta)\,f_n f_m,
    \label{eq:Delta_as_Cnm_fnfm}
\end{align}
where the moment matrix is
\begin{equation}
    C_{nm}(\beta)
    \;\equiv\;
    \frac{1}{\beta}\sum_{\ell\in\mathbb Z}\kappa_\ell\,
    I_n(\nu_\ell)\,I_m(\nu_\ell)^\ast.
    \label{eq:Cnm_factorised}
\end{equation}
This makes the desired decoupling explicit: $C_{nm}$ factorises as a sum of outer products over Matsubara modes.
Equivalently, defining mode coefficients
\begin{equation}
    a_{n\ell}\;\equiv\;\sqrt{\frac{\kappa_\ell}{\beta}}\,I_n(\nu_\ell),
    \label{eq:anl_def}
\end{equation}
one has
\begin{equation}
    C_{nm}(\beta)
    =
    \sum_{\ell\in\mathbb Z} a_{n\ell}\,a_{m\ell}^\ast,
    \label{eq:Cnm_outer_product}
\end{equation}
and therefore
\begin{equation}
    \Delta=\frac{1}{2}\sum_{\ell\in\mathbb Z}
    \left(\sum_{n\ge 0} a_{n\ell}\,f_n\right)
    \left(\sum_{m\ge 0} a_{m\ell}^\ast\,f_m\right).
    \label{eq:Delta_as_sum_mode_squares}
\end{equation}
All temperature and spectral-density dependence is confined to the scalar mode weights $\kappa_\ell$
and the generating function \eqref{eq:I_gen_matsubara}, while all operator complexity resides in the adjoint chain $\{f_n\}$.

\medskip
In the next step we return from the interaction picture to the Schr\"odinger picture, combining
$e^{-\beta H_Q}$ with the averaged propagator $\bar W(\beta)=\mathcal T_\tau e^{\Delta}$ to obtain a single-exponent
representation and hence a systematic expansion of the mean-force Hamiltonian in commutator order.

% =========================
% Step 4: Drop time ordering (Delta is tau-independent), de-interaction-picture,
% and collapse to a single exponent via BCH / Bernoulli (systematic commutator order).
% Keep binomial factors explicit.
% =========================

\subsection{De-interaction picture and single-exponent form}
\label{sec:collapse_single_exponent}

At this point the Gaussian average has been performed exactly, and the averaged interaction-picture propagator
is
\begin{align}
    \bar W(\beta)&=\mathcal T_\tau e^{\Delta}, \nonumber\\
    \Delta&=\frac{1}{2\beta}\sum_{\ell\in\mathbb Z}\kappa_\ell\,\tilde F_\ell\tilde F_\ell^\dagger
    =\frac12\sum_{n,m\ge 0} C_{nm}(\beta)\,f_n f_m,
    \label{eq:barW_Texp_Delta_recalled_step4}
\end{align}
with $\tilde F_\ell=\sum_{n\ge0}I_n(\nu_\ell)f_n$ and $C_{nm}$ given by Eq.~\eqref{eq:Cnm_factorised}.

\paragraph{Time ordering becomes redundant after mode reduction.}
Although Eq.~\eqref{eq:barW_Texp_Delta_recalled_step4} is written as a time-ordered exponential,
the operator $\Delta$ itself contains no remaining $\tau$-dependence: all $\tau$-integrals have been carried out
explicitly into scalar coefficients ($\kappa_\ell$, $I_n$, and hence $C_{nm}$), while the operator content is
carried by the time-independent adjoint-chain elements $\{f_n\}$.
Therefore $\mathcal T_\tau$ acts trivially and we may write
\begin{equation}
    \bar W(\beta)=e^{\Delta}.
    \label{eq:barW_equals_expDelta}
\end{equation}
Consequently, the reduced equilibrium operator takes the nonperturbative product form
\begin{equation}
    \bar\rho_S(\beta)=e^{-\beta H_Q}\,e^{\Delta}.
    \label{eq:bar_rho_product_form}
\end{equation}

\paragraph{Definition of the effective (mean-force) Hamiltonian.}
We now define an effective system Hamiltonian $H_{\rm eff}(\beta)$ by the single-exponent representation
\begin{equation}
    \bar\rho_S(\beta) \equiv e^{-\beta H_{\rm eff}(\beta)},
    \label{eq:Heff_def_by_single_exponent}
\end{equation}
up to the overall scalar normalisation fixed by $Z_X(\beta)$ in Eq.~\eqref{eq:HMF_from_quenched_identity_main}.
Combining Eqs.~\eqref{eq:bar_rho_product_form} and \eqref{eq:Heff_def_by_single_exponent} gives
\begin{equation}
    -\beta H_{\rm eff}(\beta) = \log\!\left(e^{-\beta H_Q}e^{\Delta}\right),
    \label{eq:Heff_log_product}
\end{equation}
so $H_{\rm eff}$ is obtained by the Baker--Campbell--Hausdorff (BCH) series with
\begin{equation}
\begin{split}
    &A\equiv -\beta H_Q,
    \qquad
    B\equiv \Delta, \\
    &\log(e^{A}e^{B}) = A + B + \frac12[A,B] \\
    &\quad + \frac{1}{12}[A,[A,B]] - \frac{1}{12}[B,[A,B]] + \cdots.
\end{split}
    \label{eq:BCH_recalled}
\end{equation}

\paragraph{Systematic commutator-order expansion at fixed quadratic content.}
A key advantage of the adjoint-chain representation is that commutators with $H_Q$ simply shift the chain index.
Indeed, by definition $[H_Q,f_n]=f_{n+1}$, and the adjoint action satisfies the Leibniz rule. Iterating yields
the explicit binomial decomposition
\begin{equation}
    \mathrm{ad}_{H_Q}^{r}(f_n f_m)
    =
    \sum_{k=0}^{r}\binom{r}{k}\,f_{n+k}\,f_{m+r-k},
    \qquad r\ge 0,
    \label{eq:ad_r_on_product_binomial}
\end{equation}
which isolates the combinatorics (the binomial factors) from the operator content.

Using Eq.~\eqref{eq:Delta_as_Cnm_fnfm}, the $r$-fold adjoint action of $H_Q$ on $\Delta$ is therefore
\begin{equation}
\begin{split}
    \mathrm{ad}_{H_Q}^{r}(\Delta)
    &=
    \frac12\sum_{n,m\ge 0} C_{nm}(\beta)\;
    \mathrm{ad}_{H_Q}^{r}(f_n f_m) \\
    &=
    \frac12\sum_{n,m\ge 0} C_{nm}(\beta)\;
    \sum_{k=0}^{r}\binom{r}{k}\,f_{n+k}\,f_{m+r-k}.
\end{split}
    \label{eq:ad_r_on_Delta_binomial}
\end{equation}

\paragraph{Linear-in-$\Delta$ resummation (Bernoulli/BCH kernel).}
If we retain only terms \emph{linear} in $\Delta$ in the BCH series (i.e.\ we drop terms of order $\Delta^2$ and higher,
such as $[B,[A,B]]$ in Eq.~\eqref{eq:BCH_recalled}), then the remaining commutator tower
$B+\frac12[A,B]+\frac{1}{12}[A,[A,B]]+\cdots$ can be resummed to all commutator orders via a Bernoulli series.
Specifically,
\begin{align}
    \log(e^{A}e^{B})
    &=
    A + \varphi(\mathrm{ad}_{A})\,B + \mathcal O(B^2), \nonumber\\
    \varphi(x)&\equiv \frac{x}{1-e^{-x}}
    =
    \sum_{r=0}^\infty \frac{B_r}{r!}\,x^r,
    \label{eq:Bernoulli_resummation_phi}
\end{align}
where $B_r$ are Bernoulli numbers with the convention $B_0=1$ and $B_1=+1/2$.
Since $A=-\beta H_Q$, one has $\mathrm{ad}_{A}=-\beta\,\mathrm{ad}_{H_Q}$, and hence
\begin{equation}
    -\beta H_{\rm eff}(\beta)
    =
    -\beta H_Q + \varphi\!\big(-\beta\,\mathrm{ad}_{H_Q}\big)\Delta + \mathcal O(\Delta^2).
    \label{eq:Heff_phi_adHQ}
\end{equation}
Equivalently,
\begin{equation}
    \boxed{
    \begin{aligned}
    H_{\rm eff}(\beta)
    &=
    H_Q
    -\frac{1}{\beta}\,
    \varphi\!\big(-\beta\,\mathrm{ad}_{H_Q}\big)\Delta \\
    &\quad +\mathcal O(\Delta^2).
    \end{aligned}
    }
    \label{eq:Heff_linear_in_Delta}
\end{equation}
Expanding $\varphi$ explicitly gives the commutator-order series
\begin{equation}
    H_{\rm eff}(\beta)
    =
    H_Q
    -\frac{1}{\beta}\sum_{r=0}^\infty \frac{B_r}{r!}\,(-\beta)^r\,
    \mathrm{ad}_{H_Q}^{r}(\Delta)
    +\mathcal O(\Delta^2),
    \label{eq:Heff_Bernoulli_commutator_series}
\end{equation}
and substituting Eq.~\eqref{eq:ad_r_on_Delta_binomial} yields the fully explicit adjoint-chain form
\begin{equation}
\begin{split}
    H_{\rm eff}(\beta)
    =
    H_Q
    &-\frac{1}{2\beta}\sum_{r=0}^\infty \frac{B_r}{r!}\,(-\beta)^r
    \sum_{n,m\ge 0} C_{nm}(\beta) \\
    &\times \sum_{k=0}^{r}\binom{r}{k}\,f_{n+k}\,f_{m+r-k} + \mathcal O(\Delta^2).
\end{split}
    \label{eq:Heff_explicit_binomial}
\end{equation}
All bath dependence is contained in the scalar matrix $C_{nm}(\beta)$ (or equivalently $\kappa_\ell$ and $I_n(\nu_\ell)$),
while the full operator content is controlled by the adjoint chain $\{f_n\}$ and the binomial factors in
Eq.~\eqref{eq:ad_r_on_product_binomial}.

% =========================
% Step 5: Insert an explicit "commuting check / baseline" subsection
% and then append the normalisation to obtain H_MF from H_eff.
% =========================

\subsection{Commuting baseline and commutator expansion of $H_{\rm eff}(\beta)$}
\label{sec:commuting_baseline}

It is useful to isolate the simplest limiting case in which the coupling operator $f$ commutes with the system
Hamiltonian $H_Q$. This provides both a consistency check and a natural ``baseline'' around which to organise
the general noncommuting corrections in terms of nested commutators with $H_Q$.

\paragraph{Commuting check: $[H_Q,f]=0$.}
If $[H_Q,f]=0$, then the interaction-picture coupling is $\tau$-independent:
$\tilde f(\tau)=f$. The Matsubara-projected operators defined in Eq.~\eqref{eq:Ftilde_l_def} reduce to
\begin{equation}
    \tilde F_\ell
    =
    \int_0^\beta d\tau\,e^{i\nu_\ell\tau}\,f
    =
    f\int_0^\beta d\tau\,e^{i\nu_\ell\tau}
    =
    \beta f\,\delta_{\ell 0},
    \label{eq:Ftilde_l_commuting_limit}
\end{equation}
so only the $\ell=0$ mode contributes to the sum-of-squares form \eqref{eq:Delta_sum_of_squares_complex}.
Consequently,
\begin{equation}
    \Delta
    =
    \frac{1}{2\beta}\kappa_0\,(\beta f)(\beta f)
    =
    \frac{\beta\kappa_0}{2}\,f^2.
    \label{eq:Delta_commuting_limit}
\end{equation}
Since $\Delta$ then commutes with $H_Q$, the reduced equilibrium operator
$\bar\rho_S(\beta)=e^{-\beta H_Q}e^{\Delta}$ collapses to a single exponential,
\begin{equation}
    \bar\rho_S(\beta)
    =
    \exp\!\left[-\beta\left(H_Q-\frac{\kappa_0}{2}f^2\right)\right].
    \label{eq:bar_rho_commuting_single_exponent}
\end{equation}
Thus in the commuting limit the effective Hamiltonian acquires a purely local quadratic correction,
\begin{equation}
    H_{\rm eff}(\beta)
    =
    H_Q-\frac{\kappa_0}{2}f^2,
    \qquad\text{when }[H_Q,f]=0,
    \label{eq:Heff_commuting_limit}
\end{equation}
consistent with interpreting $\kappa_0$ as the zero-frequency weight of the Euclidean bath correlations.

\paragraph{General case: organise $H_{\rm eff}$ in $f$ and commutators with $H_Q$.}
Away from the commuting limit, we retain the exact product form
\begin{equation}
    \bar\rho_S(\beta)=e^{-\beta H_Q}e^{\Delta},
    \label{eq:bar_rho_product_recalled_step5}
\end{equation}
with $\Delta$ given explicitly by Eqs.~\eqref{eq:Delta_sum_of_squares_complex}--\eqref{eq:Delta_as_Cnm_fnfm}.
We then define $H_{\rm eff}(\beta)$ by
\begin{equation}
    \bar\rho_S(\beta)\equiv e^{-\beta H_{\rm eff}(\beta)},
    \label{eq:Heff_def_recalled_step5}
\end{equation}
so that $-\beta H_{\rm eff}(\beta)=\log(e^{-\beta H_Q}e^{\Delta})$.

To express $H_{\rm eff}$ in a form that depends only on $f$ and nested commutators with $H_Q$, we use the
adjoint-chain representation $f_n\equiv \mathrm{ad}_{H_Q}^n(f)$ in Eq.~\eqref{eq:fn_def}.
In particular, $[H_Q,f_n]=f_{n+1}$ and for products the $r$-fold adjoint action obeys the binomial identity
\begin{equation}
    \mathrm{ad}_{H_Q}^{r}(f_n f_m)
    =
    \sum_{k=0}^{r}\binom{r}{k}\,f_{n+k}\,f_{m+r-k}.
    \label{eq:ad_r_product_binomial_recalled_step5}
\end{equation}
Using Eq.~\eqref{eq:Delta_as_Cnm_fnfm}, the commutator tower $\mathrm{ad}_{H_Q}^r(\Delta)$ is therefore
\begin{equation}
    \mathrm{ad}_{H_Q}^{r}(\Delta)
    =
    \frac12\sum_{n,m\ge 0} C_{nm}(\beta)
    \sum_{k=0}^{r}\binom{r}{k}\,f_{n+k}\,f_{m+r-k}.
    \label{eq:ad_r_Delta_explicit_step5}
\end{equation}
This exhibits the desired separation: all $\beta$- and bath-dependence resides in the scalar coefficients $C_{nm}(\beta)$,
while the operator basis is generated entirely by $f$ and successive commutators with $H_Q$.

\paragraph{Linear-in-$\Delta$ commutator resummation.}
Retaining only terms linear in $\Delta$ in the BCH series for $\log(e^{-\beta H_Q}e^{\Delta})$,
the commutator tower can be resummed to all orders in $\mathrm{ad}_{H_Q}$ using Bernoulli numbers:
\begin{equation}
    H_{\rm eff}(\beta)
    =
    H_Q
    -\frac{1}{\beta}\sum_{r=0}^\infty \frac{B_r}{r!}\,(-\beta)^r\,
    \mathrm{ad}_{H_Q}^{r}(\Delta)
    +\mathcal O(\Delta^2),
    \label{eq:Heff_Bernoulli_recalled_step5}
\end{equation}
with $B_0=1$ and $B_1=+1/2$.
Substituting Eq.~\eqref{eq:ad_r_Delta_explicit_step5} yields an explicit expansion in products of commutator-chain elements:
\begin{equation}
    \boxed{
    \begin{aligned}
    H_{\rm eff}(\beta)
    =
    H_Q
    &-\frac{1}{2\beta}\sum_{r=0}^\infty \frac{B_r}{r!}\,(-\beta)^r \\
    &\times \sum_{n,m\ge 0} C_{nm}(\beta)
    \sum_{k=0}^{r}\binom{r}{k}\,f_{n+k}\,f_{m+r-k} \\
    &+\mathcal O(\Delta^2).
    \end{aligned}
    }
    \label{eq:Heff_commutator_chain_explicit_step5}
\end{equation}
Equation~\eqref{eq:Heff_commutator_chain_explicit_step5} is a systematic commutator-order construction of the effective
Hamiltonian: for any truncation in the adjoint chain $\{f_n\}$ one obtains a closed operator approximation, with all
scalar prefactors determined exactly from the translation-invariant kernel via Eqs.~\eqref{eq:Cnm_factorised} and
\eqref{eq:I_gen_matsubara}.

\subsection{Normalisation and the Hamiltonian of mean force}
\label{sec:normalisation_HMF}

The Hamiltonian of mean force is defined (up to an additive scalar multiple of the identity) by
\begin{equation}
    H_{\mathrm{MF}}(\beta)
    =
    -\frac{1}{\beta}\log\!\left[\frac{\bar\rho_S(\beta)}{Z_X(\beta)}\right],
    \qquad
    Z_X(\beta)=\Tr_X e^{-\beta H_X}.
    \label{eq:HMF_def_recalled_step5}
\end{equation}
Using the definition of $H_{\rm eff}$ in Eq.~\eqref{eq:Heff_def_recalled_step5},
\begin{equation}
    \bar\rho_S(\beta)=e^{-\beta H_{\rm eff}(\beta)},
    \label{eq:bar_rho_as_Heff_again}
\end{equation}
we obtain the compact relation
\begin{equation}
    \boxed{
    H_{\mathrm{MF}}(\beta)
    =
    H_{\rm eff}(\beta)
    +\frac{1}{\beta}\log Z_X(\beta)\;\mathbb I_S.
    }
    \label{eq:HMF_from_Heff}
\end{equation}
Equivalently, if one chooses the conventional normalisation $\rho_S(\beta)\equiv \bar\rho_S(\beta)/Z_X(\beta)$,
then $\rho_S(\beta)=e^{-\beta H_{\rm MF}(\beta)}$ exactly, and the scalar shift in Eq.~\eqref{eq:HMF_from_Heff}
ensures $\Tr_S \rho_S(\beta)=1$.
