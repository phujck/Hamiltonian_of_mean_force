\section{Introduction}
\label{sec:intro}

In closed quantum statistical mechanics, equilibrium is generated by a
Hamiltonian: $\rho \propto e^{-\beta H}$. For an open system with finite
coupling, the operationally defined equilibrium state of the subsystem is the
reduced state of the global Gibbs ensemble,
\begin{align}
    \bar{\rho}_S(\beta)
    &= \mathrm{Tr}_B\, e^{-\beta H_{\mathrm{tot}}},
    \label{eq:intro_rho_unnorm} \\
    e^{-\beta H_{\mathrm{MF}}(\beta)}
    &\propto \frac{\mathrm{Tr}_B e^{-\beta H_{\mathrm{tot}}}}{Z_B(\beta)},
    \quad Z_B(\beta) = \mathrm{Tr}_B e^{-\beta H_X}.
    \label{eq:intro_hmf_def}
\end{align}
This object is generally not $e^{-\beta H_Q}$ for the bare system Hamiltonian
$H_Q$. The resulting representational question is precise: what operator, if
any, plays the role of an equilibrium generator for the subsystem once the
coupling is non-negligible? The Hamiltonian of mean force (HMF) answers this by
construction and is the standard starting point in strong-coupling
thermodynamics\cite{campisiFluctuationTheoremArbitrary2009,talknerColloquiumStatisticalMechanics2020,trushechkinOpenQuantumSystem2022,seifertFirstSecondLaw2016}.

Historically, much of open-system theory emphasized weak-coupling and Markovian
regimes, where reduced equilibrium states are well approximated by Gibbs states
of renormalized system Hamiltonians. At finite coupling, equilibrium
consistency is anchored in the correlated global Gibbs state, and the reduced
operator inherits coupling-dependent structure that need not be captured by a
simple renormalization. The HMF can become temperature dependent and can encode
interaction-induced terms that are absent in $H_Q$, including effective
many-body or nonlocal operator content\cite{talknerColloquiumStatisticalMechanics2020,seifertFirstSecondLaw2016,espositoNatureHeatStrongly2015,hanggiFiniteQuantumDissipation2008,ingoldSpecificHeatAnomalies2009,correaPotentialRenormalisationLamb2025}.
This is the conceptual tension: the reduced equilibrium state is well defined,
but its generator is not generally simple.

The literature on $H_{\mathrm{MF}}$ has developed along several lines.
Canonical definitions appear in strong-coupling thermodynamics and fluctuation
relations, where $H_{\mathrm{MF}}$ is tied to partition functions, free
energies, and thermodynamic identities\cite{campisiFluctuationTheoremArbitrary2009,jarzynskiNonequilibriumWorkTheorem2004,jarzynskiStochasticMacroscopicThermodynamics2017,talknerColloquiumStatisticalMechanics2020,seifertFirstSecondLaw2016}.
Related work examines measurability and operational meaning in the presence of
finite coupling\cite{strasbergMeasurabilityNonequilibriumThermodynamics2020,rivasStrongCouplingThermodynamics2020,trushechkinOpenQuantumSystem2022}.
Exact or controlled computations are available in special models: commuting
(QND-type) couplings where the operator algebra closes trivially
\cite{campisiTalknerHanggi2009Solvable}, quadratic/Gaussian systems such as the
damped harmonic oscillator where Gaussianity is preserved
\cite{caldeiraQuantumTunnellingDissipative1983a,grabertQuantumBrownianMotion1988,hiltHamiltonianMeanForce2011},
and finite-dimensional closures such as spin-boson or single-qubit settings
\cite{leggettDynamicsDissipativeTwostate1987}.
Recent work also investigates structural aspects and generalized definitions of
$H_{\mathrm{MF}}$ beyond these solvable cases\cite{burkeStructureHamiltonianMean2024,duGeneralizedHamiltonianMeanforce2025a}.

Outside solvable models, most approaches are perturbative or numerical.
Weak-coupling and high-temperature expansions yield controlled but limited
series\cite{cresserWeakUltrastrongCoupling2021a}, while semiclassical limits
require additional structure that is model dependent\cite{talknerColloquiumStatisticalMechanics2020}.
Operator expansions proliferate under nested commutators, and truncations can be
difficult to justify a priori. Numerical approaches based on imaginary-time path
integrals and HEOM compute $\rho_S(\beta)$ directly but do not typically yield a
compact operator expression for $H_{\mathrm{MF}}$\cite{moixEquilibriumreducedDensityMatrix2012,chenRigorousStochasticMatrix2014,tanimuraReducedHierarchicalEquations2014,songCalculationCorrelatedInitial2015,makriExploitingClassicalDecoherence2014}.
These methods are essential for quantitative results, but they do not resolve
the representational question: when is a closed-form local generator available?

The influence-functional formalism is a natural language for this problem.
For Gaussian baths with linear coupling, it provides an exact route to
integrating out bath degrees of freedom\cite{feynmanTheoryGeneralQuantum1963a,caldeiraQuantumTunnellingDissipative1983a,grabertQuantumBrownianMotion1988}.
In equilibrium this becomes a Euclidean (imaginary-time) influence functional
that is bilocal in $\tau$ and can be rewritten as a quenched Gaussian-field
average via Hubbard--Stratonovich transformations
\cite{hubbardCalculationPartitionFunctions1959a,stratonovich1957QDistro,stockburgerExactNumberRepresentation2002,moixEquilibriumreducedDensityMatrix2012,chenRigorousStochasticMatrix2014}.
This formulation clarifies an important distinction: ``nonlocal'' initially
means nonlocal in imaginary time through the kernel, whereas the locality
question for $H_{\mathrm{MF}}$ concerns the operator structure on the system
Hilbert space.

The existence of $H_{\mathrm{MF}}$ is not the issue---it is defined by a
logarithm of a traced exponential. The real obstruction is representability:
when does $H_{\mathrm{MF}}$ admit a closed-form expression within a restricted
operator family (for example, few-body or spatially local Hamiltonians)? Most of
the literature either (i) solves special models, (ii) expands perturbatively, or
(iii) computes $\rho_S(\beta)$ numerically and studies its properties
\cite{burkeStructureHamiltonianMean2024,duGeneralizedHamiltonianMeanforce2025a,cresserWeakUltrastrongCoupling2021a}.
What is missing is an exact structural reformulation that makes the obstruction
explicit and checkable.

This paper provides an exact structural reformulation of the reduced equilibrium
operator for Gaussian baths with linear coupling, written both as an
imaginary-time influence functional and as a quenched Gaussian-field average.
We then organize the operator content by the adjoint-action hierarchy and
Magnus/BCH/Lie-factorization language to state an explicit closure criterion: a
closed-form local HMF exists only when the operator algebra generated by
repeated adjoint action of $H_Q$ on the coupling operators closes inside the
target ansatz. The analysis is anchored by minimal solvable examples (commuting
coupling, quadratic/Gaussian models, single qubit), and broader implications are
deferred.

