% \section{Results: Analytic Examples}
% \label{sec:results}

% We provide three minimal analytic examples that illustrate the closure
% criterion and its consequences. These examples are not approximations; they
% serve only to show when closure is exact.

% \subsection{Commuting coupling}
% If $[H_Q,f]=0$, then $\mathrm{ad}_{H_Q}^n(f)=0$ for all $n\ge 1$, so
% $\tilde{f}(\tau)=f$ and the time ordering becomes trivial. The bilocal exponent
% reduces to a scalar multiple of $f^2$, and the HMF is exactly local. This is the
% simplest solvable case and is consistent with known exactly solvable strong
% coupling models\cite{campisiTalknerHanggi2009Solvable}.

% \subsection{Quadratic/Gaussian system}
% Consider a harmonic system with
% $H_Q = p^2/2m + (1/2)m\omega^2 q^2$ and linear coupling $f=q$. The adjoint
% action closes on the finite set $\{q,p,\mathbb{I}\}$ since
% $[H_Q,q] \propto p$ and $[H_Q,p] \propto q$. Consequently, the associative
% algebra generated by $\mathcal{A}_f$ is finite dimensional and the HMF is a
% quadratic operator. This reproduces the known Gaussian character of the
% reduced equilibrium state and its mean-force Hamiltonian in damped harmonic
% models\cite{grabertQuantumBrownianMotion1988,hiltHamiltonianMeanForce2011}.

% \subsection{Single qubit (Pauli algebra)}
% Let $H_Q = (\omega/2)\sigma_z$ and $f=\sigma_x$. Then
% $\mathrm{ad}_{H_Q}(f) = \omega i\sigma_y$ and
% $\mathrm{ad}_{H_Q}^2(f) = -\omega^2 \sigma_x$, so the adjoint chain closes on the
% Pauli algebra $\{\sigma_x,\sigma_y,\sigma_z,\mathbb{I}\}$. Hence the closure
% criterion is satisfied and $H_{\mathrm{MF}}$ lies in the same finite operator
% class. This is consistent with standard spin-boson constructions\cite{leggettDynamicsDissipativeTwostate1987}.

% ==========================================================
% Qubit (Pauli-closed algebra): exact H_MF in closed form
% ==========================================================

\section{Mean Force for qubits}
\label{sec:qubit}
One of the simplest nontrivial examples where the closure criterion are satisfied is a single qubit with a transverse Pauli coupling. Let the system be a single qubit with Hamiltonian
\begin{equation}
H_Q=\frac{\omega}{2}\,\hat n\cdot\boldsymbol\sigma,\qquad \|\hat n\|=1,
\end{equation}
and consider a transverse Pauli coupling
\begin{equation}
f=\hat m\cdot\boldsymbol\sigma,\qquad \hat m\cdot\hat n=0.
\end{equation}
Choose a unitary $U$ such that $U(\hat n\cdot\boldsymbol\sigma)U^\dagger=\sigma_z$ and
$U(\hat m\cdot\boldsymbol\sigma)U^\dagger=\sigma_x$. In this rotated frame we have
\begin{equation}
H'_Q=\frac{\omega}{2}\sigma_z,\qquad f'=\sigma_x,
\end{equation}
and the adjoint chain closes immediately:
\begin{equation}
\label{eq:qubit_chain}
\begin{split}
f'_0&=\sigma_x,\qquad f'_1=[H'_Q,f'_0]=i\omega\sigma_y, \\
f'_{2k}&=\omega^{2k}\sigma_x,\qquad f'_{2k+1}=i\omega^{2k+1}\sigma_y.
\end{split}
\end{equation}
Consequently, the quadratic products collapse onto $\mathrm{span}\{\mathbb I,\sigma_z\}$:
\begin{equation}
\label{eq:qubit_products}
\begin{split}
f'_{2k}f'_{2\ell}&=\omega^{2(k+\ell)}\mathbb I,\qquad
f'_{2k+1}f'_{2\ell+1}=-\omega^{2(k+\ell)+2}\mathbb I, \\
f'_{2k}f'_{2\ell+1}&=-\omega^{2(k+\ell)+1}\sigma_z,\quad
f'_{2k+1}f'_{2\ell}=+\omega^{2(k+\ell)+1}\sigma_z.
\end{split}
\end{equation}
Inserting \eqref{eq:qubit_products} into the exact quadratic form
$\Delta(\beta)=\frac12\sum_{n,m\ge0}C_{nm}(\beta)f_nf_m$ yields the compact decomposition
\begin{equation}
\label{eq:Delta_qubit_decomp}
\Delta(\beta)=\alpha(\beta)\,\mathbb I+\gamma(\beta)\,\sigma_z,
\end{equation}
where
\begin{align}
\label{eq:alpha_gamma_qubit_sums}
\alpha(\beta)
&=\frac12\sum_{k,\ell\ge0}\omega^{2(k+\ell)}
\Big(C_{2k,2\ell}(\beta)-\omega^2\,C_{2k+1,2\ell+1}(\beta)\Big), \\
\gamma(\beta)
&=\frac{\omega}{2}\sum_{k,\ell\ge0}\omega^{2(k+\ell)}
\Big(C_{2k+1,2\ell}(\beta)-C_{2k,2\ell+1}(\beta)\Big).
\end{align}
Equivalently, introduce the parity-projected generating functions
\begin{equation}
\label{eq:qubit_generating_functions}
\begin{split}
G_{ee}(x)&=\sum_{k,\ell\ge0}C_{2k,2\ell}(\beta)\,x^{k+\ell}, \\
G_{oo}(x)&=\sum_{k,\ell\ge0}C_{2k+1,2\ell+1}(\beta)\,x^{k+\ell}, \\
G_{eo}(x)&=\sum_{k,\ell\ge0}C_{2k,2\ell+1}(\beta)\,x^{k+\ell}, \\
G_{oe}(x)&=\sum_{k,\ell\ge0}C_{2k+1,2\ell}(\beta)\,x^{k+\ell},
\end{split}
\end{equation}
in terms of which
\begin{equation}
\label{eq:Delta_qubit_generating}
\begin{split}
\Delta(\beta)&=\frac12\Big(G_{ee}(\omega^2)-\omega^2G_{oo}(\omega^2)\Big)\mathbb I \\
&\quad +\frac{\omega}{2}\Big(G_{oe}(\omega^2)-G_{eo}(\omega^2)\Big)\sigma_z.
\end{split}
\end{equation}
Since $\Delta(\beta)\in\mathrm{span}\{\mathbb I,\sigma_z\}$ commutes with $H'_Q\propto\sigma_z$,
the BCH logarithm collapses exactly:
\begin{equation}
\label{eq:qubit_BCH_collapse}
\begin{split}
-\beta H'_{\mathrm{MF}}(\beta)&=\log\!\left(e^{-\beta H'_Q}e^{\Delta(\beta)}\right)
=-\beta H'_Q+\Delta(\beta), \\
H'_{\mathrm{MF}}(\beta)&=H'_Q-\frac{1}{\beta}\Delta(\beta).
\end{split}
\end{equation}
Discarding the additive scalar (fixed by normalisation), the mean-force Hamiltonian is therefore a
renormalised splitting along the Hamiltonian axis:
\begin{equation}
\label{eq:qubit_final_HMF}
\begin{split}
H_{\mathrm{MF}}(\beta)&=\frac{\omega_{\mathrm{MF}}(\beta)}{2}\,\hat n\cdot\boldsymbol\sigma+\mathrm{const}, \\
\omega_{\mathrm{MF}}(\beta)&=\omega-\frac{2\gamma(\beta)}{\beta}
=\omega\left[1-\frac{G_{oe}(\omega^2)-G_{eo}(\omega^2)}{\beta}\right].
\end{split}
\end{equation}

\subsection{Recovering PRL Weak/Ultrastrong Qubit Limits and Finite-Coupling Extension}
\label{sec:qubit_prl_recovery}
To make contact with Ref.~\cite{cresserWeakUltrastrongCoupling2021a}, take
\begin{equation}
\label{eq:prl_qubit_param}
H_S=\frac{\omega_q}{2}\sigma_z,\qquad
X=\cos\theta\,\sigma_z-\sin\theta\,\sigma_x\equiv \sigma_{\hat r}.
\end{equation}
We expose coupling strength explicitly by writing
\begin{equation}
\label{eq:prl_kernel_scaling}
J_\lambda(\omega)=\lambda^2 J_0(\omega),\qquad
\kappa_\ell(\lambda)=\lambda^2\kappa_\ell^{(0)}.
\end{equation}
Using the direct Gaussian influence operator,
\begin{equation}
\label{eq:Delta_direct_prl}
\Delta(\beta,\lambda)=\frac12\int_0^\beta d\tau\int_0^\beta d\tau'\,
K_\lambda(\tau-\tau')\,\mathcal T_\tau\!\left[\tilde X(\tau)\tilde X(\tau')\right],
\end{equation}
with
\begin{equation}
\label{eq:Xtilde_prl}
\tilde X(\tau)=\cos\theta\,\sigma_z
-\sin\theta\!\left[\cosh(\omega_q\tau)\sigma_x+i\sinh(\omega_q\tau)\sigma_y\right],
\end{equation}
direct multiplication gives (App.~\ref{app:prl_qubit_delta})
\begin{equation}
\label{eq:Xtilde_product_prl}
\begin{split}
\tilde X(\tau)\tilde X(\tau')
&=\Big[\cos^2\theta+\sin^2\theta\cosh\!\big(\omega_q(\tau-\tau')\big)\Big]\mathbb I \\
&\quad +\sin^2\theta\sinh\!\big(\omega_q(\tau-\tau')\big)\sigma_z \\
&\quad +\sin\theta\cos\theta\Big(\sinh(\omega_q\tau)-\sinh(\omega_q\tau')\Big)\sigma_x \\
&\quad +i\sin\theta\cos\theta\Big(\cosh(\omega_q\tau)-\cosh(\omega_q\tau')\Big)\sigma_y.
\end{split}
\end{equation}
Write $c\equiv\cos\theta$, $s\equiv\sin\theta$ and $K_\lambda=\lambda^2K_0$.
Time ordering contributes the commutator sector with $\mathrm{sgn}(\tau-\tau')$,
so the antisymmetric pieces do \emph{not} cancel. The result is
\begin{equation}
\label{eq:Delta_qubit_general_pauli}
\Delta(\beta,\lambda)=\lambda^2\!\left[\alpha_0(\beta)\,\mathbb I
+\delta_x(\beta)\,\sigma_x+i\delta_y(\beta)\,\sigma_y+\delta_z(\beta)\,\sigma_z\right],
\end{equation}
with one-dimensional direct-kernel coefficients
\begin{equation}
\label{eq:alpha0_direct_kernel}
\alpha_0(\beta)=\int_{0}^{\beta}\!du\,(\beta-u)\,K_0(u)
\left[c^2+s^2\cosh(\omega_q u)\right],
\end{equation}
\begin{equation}
\label{eq:deltax_direct_kernel}
\delta_x(\beta)=\frac{cs}{\omega_q}\int_{0}^{\beta}\!du\,K_0(u)\!
\left[\cosh(\beta\omega_q)+1-\cosh(\omega_q u)-\cosh\!\big(\omega_q(\beta-u)\big)\right],
\end{equation}
\begin{equation}
\label{eq:deltay_direct_kernel}
\delta_y(\beta)=\frac{cs}{\omega_q}\int_{0}^{\beta}\!du\,K_0(u)\!
\left[\sinh(\beta\omega_q)-\sinh(\omega_q u)-\sinh\!\big(\omega_q(\beta-u)\big)\right],
\end{equation}
\begin{equation}
\label{eq:deltaz_direct_kernel}
\delta_z(\beta)=s^2\int_{0}^{\beta}\!du\,(\beta-u)\,K_0(u)\sinh(\omega_q u).
\end{equation}
Hence
\begin{equation}
\label{eq:rho_weak_prl}
\begin{split}
\rho_S(\beta,\lambda)&=\tau_S+\lambda^2\,\delta\rho_{\mathrm{weak}}+O(\lambda^4),\\
\tau_S&=\frac12\!\left(\mathbb I-\tanh\!\frac{\beta\omega_q}{2}\,\sigma_z\right),
\end{split}
\end{equation}
with
\begin{equation}
\label{eq:delta_rho_weak_prl}
\delta\rho_{\mathrm{weak}}
=\frac{\delta_x(\beta)}{2}\sigma_x
+\frac{1-\tanh^2(\beta\omega_q/2)}{2}\,\delta_z(\beta)\sigma_z.
\end{equation}
The energetic-coherence channel is therefore explicit in the direct-kernel route:
$\delta_x\propto\sin\theta\cos\theta$ and vanishes only at the commuting/transverse endpoints.
The full derivation, including the time-ordering decomposition, is given in
App.~\ref{app:prl_qubit_delta}.

In the ultrastrong limit, PRL Eq.~(7) gives
\begin{equation}
\label{eq:rho_us_projector_prl}
\rho_{US}
=\frac{\exp\!\left[-\beta\sum_n P_n H_S P_n\right]}
{\Tr\!\left[\exp\!\left(-\beta\sum_n P_n H_S P_n\right)\right]},
\end{equation}
with $P_\pm=(\mathbb I\pm \sigma_{\hat r})/2$. For the qubit this evaluates to
\begin{equation}
\label{eq:rho_us_qubit_prl}
\rho_{US}
=\frac12\!\left[\mathbb I-\sigma_{\hat r}\tanh\!\left(\frac{\beta\omega_q\cos\theta}{2}\right)\right],
\end{equation}
which is exactly the PRL qubit result (their Eq.~(8)); the projector reduction is shown in
App.~\ref{app:prl_qubit_delta}.

The finite-coupling extension in the present framework is therefore
\begin{equation}
\label{eq:HMF_bridge_general}
\rho_{\mathrm{ord}}(\beta,\lambda)
=\frac{e^{-\beta H_S}\,
\mathcal T_\tau\exp\!\left[
\lambda^2\!\int_0^\beta d\tau\!\int_0^\tau d\tau'\,
K_0(\tau-\tau')\,\tilde X(\tau)\tilde X(\tau')
\right]}
{\Tr\!\left[e^{-\beta H_S}\,
\mathcal T_\tau\exp\!\left(
\lambda^2\!\int_0^\beta d\tau\!\int_0^\tau d\tau'\,
K_0(\tau-\tau')\,\tilde X(\tau)\tilde X(\tau')
\right)\right]},
\end{equation}
with Eqs.~\eqref{eq:alpha0_direct_kernel}--\eqref{eq:deltaz_direct_kernel}
fixing the direct-kernel coefficients that control both the weak expansion and the
ordered finite-coupling propagator. The ultrastrong asymptote is fixed by
\eqref{eq:rho_us_qubit_prl}, recovered from
Ref.~\cite{cresserWeakUltrastrongCoupling2021a}. This exhibits the PRL formulas as asymptotic
faces of a single finite-coupling qubit benchmark construction.

To visualise this bridge, we run a direct finite-bath benchmark at fixed
$(\beta,\omega_q,\theta)=(1,3,0.25)$ over $\lambda\in[0,8]$.
For each coupling value we compute the exact reduced equilibrium state
$\rho_S^{\mathrm{exact}}(\lambda)$ by global Gibbs diagonalisation and bath
trace-out, then compare it to three analytic targets:
(i) weak-order $\rho_{\mathrm{weak}}$ from Eq.~\eqref{eq:rho_weak_prl},
(ii) the ordered finite-coupling state $\rho_{\mathrm{ord}}$ from Eq.~\eqref{eq:HMF_bridge_general},
and (iii) the ultrastrong projector state \eqref{eq:rho_us_qubit_prl}.
For item (ii), the ordered exponential is evaluated numerically by deterministic
Karhunen--Lo\`eve + Gauss--Hermite quadrature over the Gaussian kernel
(App.~\ref{app:prl_qubit_delta}); no commuting or product-collapse shortcut is used.
Figure~\ref{fig:qubit_prl_bridge_alignment} reports trace distances and Bloch
components across coupling.
As expected, the weak expansion is confined to small $\lambda$, while the ultrastrong
projector state aligns at large coupling:
$D(\rho_S^{\mathrm{exact}},\rho_{US})$ decreases from $0.1124$ at $\lambda=0$
to $1.99\times 10^{-3}$ at $\lambda=8$.
The ordered finite model tracks the full scan with maximum error
$\max_\lambda D(\rho_S^{\mathrm{exact}},\rho_{\mathrm{ord}})=1.91\times10^{-2}$ and
reaches $1.81\times10^{-3}$ at $\lambda=8$.
For comparison, the collapsed product shortcut
$\rho\propto e^{-\beta H_S}e^{\Delta}$ remains poor at strong coupling
($D_{\mathrm{collapsed}}=0.1719$ at $\lambda=8$).

\begin{figure*}[t]
    \centering
    \includegraphics[width=0.95\textwidth]{../figures/hmf_prl_qubit_analytic_bridge_alignment.png}
    \caption{
    Analytic-bridge alignment for the PRL qubit benchmark.
    Left: trace distances between the exact finite-bath reduced state and the
    weak-order, ordered finite-model, collapsed-product shortcut, and ultrastrong
    analytic targets.
    Middle/right: Bloch components comparing exact, ordered finite-model, and
    ultrastrong states.
    The ordered finite model yields close agreement across the full coupling scan,
    while the ultrastrong projector prediction (PRL Eq.~(8)) is recovered at large coupling.
    }
    \label{fig:qubit_prl_bridge_alignment}
\end{figure*}
