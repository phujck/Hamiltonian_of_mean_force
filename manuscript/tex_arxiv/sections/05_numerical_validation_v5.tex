\section{Numerical Regime Benchmark: ED versus Ordered-Kernel Theory}
\label{sec:numerical_v5}

We benchmark the v5 qubit predictions against exact diagonalisation (ED) of a finite-mode
spin-boson model across weak-to-strong dressing regimes.
The objective is state-level validation, not only observable matching.
In practice, the numerically accurate comparator is the \emph{ordered-kernel finite model}
(the same construction that produced the strong agreement in the earlier simulation pipeline);
by contrast, the naive compact left-factorised closure over-polarises the state at finite coupling
and is not used for the main comparison figure.

We fix
\(
H_Q=(\omega_q/2)\sigma_z
\),
\(
f=\cos\theta\,\sigma_z-\sin\theta\,\sigma_x
\),
with
\(
\theta=\pi/4
\),
\(
\omega_q=1
\),
and temperatures
\(
\beta\omega_q\in\{0.5,2,8\}
\).
The bath discretisation matches the simulation pipeline:
\(
J(\omega)=Q\tau_c\omega e^{-\tau_c\omega}
\)
on
\(
\omega\in[0.5,8]
\),
with
\(
Q=10
\),
\(
\tau_c=1
\),
\(
N_\omega=2
\)
and per-mode cutoff
\(
n_{\max}=4
\)
(thread-limited ``safe'' profile).

For each coupling \(g\), we construct
\begin{equation}
    H_{\mathrm{tot}}(g)=H_Q+H_B+g\,f\otimes B+g^2\,Q_{\mathrm{reorg}}\,f^2,
\end{equation}
evaluate
\(
\rho_{\mathrm{tot}}=e^{-\beta H_{\mathrm{tot}}}/\Tr e^{-\beta H_{\mathrm{tot}}}
\),
and trace out the bath:
\(
\rho_Q^{\mathrm{ED}}=\Tr_B\rho_{\mathrm{tot}}
\).
The analytic comparator \(\rho_Q^{\mathrm{th}}\) is built from the ordered-kernel finite model,
which directly evaluates the time-ordered Gaussian propagator and therefore resums the
noncommuting operator structure without further truncation.
We scan \(g\) on a temperature-dependent interval set by Eq.~\eqref{eq:chi_crossover_condition_v5},
\begin{equation}
    g_\star(\beta)=\chi_0(\beta)^{-1/2},
\end{equation}
and use \(g\in[0,\,2.5\,g_\star]\) (no cap reached for these runs).
The resulting crossover scales are
\(
g_\star\simeq 0.608,\;0.143,\;0.0106
\)
for
\(
\beta\omega_q=0.5,\;2,\;8
\),
respectively.

To compare theory and ED on equal footing, both states are phase-gauge rotated by a \(z\)-axis unitary
so that \(\rho_{01}\) is real and nonnegative. We then evaluate:
\begin{equation}
\phi(g)=\arctan2\!\big(m_x(g),m_z(g)\big),
\end{equation}
\begin{equation}
r(g)=\sqrt{m_x^2+m_y^2+m_z^2},
\end{equation}
\begin{equation}
C(g)=2|\rho_{01}(g)|,
\qquad
\Xi_\phi(g)=\partial_g\phi(g),
\end{equation}
\begin{equation}
D(g)=\frac{1}{2}\left\|\rho_Q^{\mathrm{ED}}-\rho_Q^{\mathrm{th}}\right\|_1.
\end{equation}
Here \(m_\alpha=\Tr(\rho_Q\sigma_\alpha)\).

Figure~\ref{fig:v5_regime_panels} summarises the regime scan.
Panel (a) tracks the Bloch-geometry flow through \(\phi(g)\) and \(r(g)\): the basis-rotation signal
shifts systematically to smaller \(g\) as temperature is lowered, consistent with the \(g_\star(\beta)\)
scaling. Panel (b) shows the bare-basis coherence \(C(g)\), which grows continuously away from \(g=0\),
confirming that the reduced mean-force state is generically not diagonal in the bare energy basis.
Panel (c) shows the susceptibility \(\Xi_\phi\): the largest response is concentrated in the crossover region
and shifts with \(g_\star\), providing a clean ``sharpness'' diagnostic. Panel (d) reports the state-level
distance \(D(g)\), verifying exact agreement at \(g=0\) (numerically \(D(0)\sim 10^{-16}\)) and quantifying
the finite-coupling discrepancy across temperatures.

For this deliberately conservative finite-bath setup (\(N_\omega=2\), \(n_{\max}=4\)),
the agreement remains strong across all three temperatures:
\(
\max_g D(g)\approx 2.91\times10^{-2},\,2.10\times10^{-2},\,1.80\times10^{-3}
\)
for
\(
\beta\omega_q=0.5,\,2,\,8
\),
respectively.
Residual discrepancy is consistent with finite discretisation/truncation error in ED
and sets a concrete target for systematic convergence studies
(larger \(N_\omega\), larger \(n_{\max}\), bath-window refinement).
For panel (c), \(\Xi_\phi\) is computed from an unwrapped \(\pi\)-periodic orientation angle;
pointwise derivatives near the smallest \(g\) values remain step-size sensitive, but
peak locations and crossover trends are robust.

\subsection*{Cutoff and mode-count sensitivity}

To make the truncation dependence explicit, we performed a dedicated convergence probe at
representative couplings close to the observed crossover-response maxima:
\(g=0.329\) (\(\beta\omega_q=0.5\)),
\(g=0.143\) (\(\beta\omega_q=2\)),
and \(g=0.0264\) (\(\beta\omega_q=8\)).
Figure~\ref{fig:v5_cutoff_modes} reports
\(D(\rho_Q^{\mathrm{ED}},\rho_Q^{\mathrm{ord}})\)
under two scans:
\(n_{\max}\) at fixed \(N_\omega=2\), and \(N_\omega\) at fixed \(n_{\max}\in\{4,6\}\).

The strongest sensitivity is to the Fock cutoff in the hotter/intermediate regimes.
At \(\beta\omega_q=0.5\), the distance drops from
\(2.91\times10^{-2}\) at \(n_{\max}=4\) to
\(6.19\times10^{-3}\) at \(n_{\max}=12\).
At \(\beta\omega_q=2\), it falls from
\(2.10\times10^{-2}\) at \(n_{\max}=4\) to a minimum
\(2.79\times10^{-3}\) near \(n_{\max}=6\).
By contrast, the low-temperature point \(\beta\omega_q=8\) is nearly cutoff-insensitive
over the same range (\(\sim 1.80\times10^{-3}\rightarrow1.79\times10^{-3}\)).

The mode-count scan shows that increasing \(N_\omega\) from the two-mode baseline changes the
distance at the \(10^{-3}\) to \(10^{-2}\) level for these probes, with weak non-monotonicity
at finite truncation. Operationally, this indicates the dominant numerical limitation in the
current setup is \(n_{\max}\) (especially away from the cold regime), while \(N_\omega=2\) already
captures the qualitative crossover structure used in Fig.~\ref{fig:v5_regime_panels}.

\begin{figure*}[t]
    \centering
    \includegraphics[width=0.98\textwidth]{../figures/hmf_v5_regime_panels.png}
    \caption{
    v5 regime benchmark for \(\theta=\pi/4\), \(\omega_q=1\), and
    \(\beta\omega_q\in\{0.5,2,8\}\), comparing ED (markers) with the ordered-kernel
    finite-theory state (lines).
    (a) Bloch angle \(\phi(g)\) (left axis) and radius \(r(g)\) (right axis).
    (b) Bare-energy-basis coherence \(C(g)=2|\rho_{01}|\).
    (c) Angle susceptibility \(\Xi_\phi(g)\), with dotted vertical lines at
    \(g_\star(\beta)\) defined by \(\chi(\beta,g_\star)=1\).
    (d) State-level trace distance
    \(D(g)=\frac{1}{2}\|\rho_Q^{\mathrm{ED}}-\rho_Q^{\mathrm{th}}\|_1\).
    }
    \label{fig:v5_regime_panels}
\end{figure*}

\begin{figure*}[t]
    \centering
    \includegraphics[width=0.95\textwidth]{../figures/hmf_v5_cutoff_modes_convergence.png}
    \caption{
    Cutoff/mode convergence probe at representative couplings
    \(g=\{0.329,\,0.143,\,0.0264\}\) for
    \(\beta\omega_q=\{0.5,\,2,\,8\}\).
    (a) Trace-distance sensitivity to Fock cutoff \(n_{\max}\) at fixed \(N_\omega=2\).
    (b) Sensitivity to bath mode count \(N_\omega\) at fixed \(n_{\max}=4,6\).
    The dominant finite-size effect in the hot/intermediate points is cutoff truncation,
    while low-temperature agreement is comparatively stable across truncations.
    }
    \label{fig:v5_cutoff_modes}
\end{figure*}
