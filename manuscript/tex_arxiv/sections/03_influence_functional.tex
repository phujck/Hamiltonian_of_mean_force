\section{The Influence Functional and Algebraic Structure}
\label{sec:influence_functional}

The path integral formulation of the reduced density matrix provides the starting point for our construction. As derived in Appendix~\ref{app:influence_derivation}, integrating out the Gaussian bath degrees of freedom yields an exact expression for the unnormalized reduced state $\bar{\rho}_S(\beta)$. In the interaction picture with respect to $H_0 = H_Q + H_X$, the result is the time-ordered exponential of a bilocal influence functional:
\begin{equation}
    \bar{\rho}_S(\beta) = e^{-\beta H_Q} \mathcal{T}_\tau \exp\left( \frac{1}{2} \int_0^\beta d\tau \int_0^\beta d\tau' K(\tau-\tau') \tilde{f}(\tau) \tilde{f}(\tau') \right).
    \label{eq:influence_functional_bilocal}
\end{equation}
Here, $K(\tau-\tau')$ is the bath force autocorrelation function (the kernel), and $\tilde{f}(\tau)$ is the system coupling operator evolved in the interaction picture:
\begin{equation}
    \tilde{f}(\tau) = e^{\tau H_Q} f e^{-\tau H_Q}.
    \label{eq:f_interaction_pic}
\end{equation}
Crucially, Eq.~\eqref{eq:influence_functional_bilocal} is exact. It describes the system evolving under a self-interaction mediated by the bath. The non-locality in imaginary time arises entirely from the non-commutativity of the system Hamiltonian $H_Q$ and the coupling operator $f$. If $[H_Q, f] = 0$, then $\tilde{f}(\tau) = f$ is constant, and the time-ordering becomes trivial, reducing the problem to a simple potential renormalization. In the general case, however, $\tilde{f}(\tau)$ is a non-trivial operator function of $\tau$.

\subsection{Adjoint Action Expansion}
To characterize the structure of this non-locality, we expand the time-evolved operator using the adjoint action of the system Hamiltonian. Let $\mathrm{ad}_{H_Q}(O) = [H_Q, O]$ and $\mathrm{ad}_{H_Q}^n(O) = [H_Q, \mathrm{ad}_{H_Q}^{n-1}(O)]$. The imaginary-time evolution is given by the exact Lie series
\begin{equation}
    \tilde{f}(\tau) = \sum_{n=0}^\infty \frac{\tau^n}{n!} \mathrm{ad}_{H_Q}^n(f).
    \label{eq:adjoint_expansion}
\end{equation}
Substituting this expansion into the influence functional exponent in Eq.~\eqref{eq:influence_functional_bilocal} allows us to separate the temporal integrals over the kernel from the operator algebra. The bilocal interaction term becomes
\begin{equation}
    \frac{1}{2} \int_0^\beta d\tau \int_0^\beta d\tau' K(\tau-\tau') \tilde{f}(\tau) \tilde{f}(\tau') = \sum_{n,m=0}^\infty \mu_{nm} \, \mathrm{ad}_{H_Q}^n(f) \mathrm{ad}_{H_Q}^m(f),
    \label{eq:bilocal_expansion}
\end{equation}
where the coefficients $\mu_{nm}$ are the \emph{kernel moments}:
\begin{equation}
    \mu_{nm} = \frac{1}{2 n! m!} \int_0^\beta d\tau \int_0^\beta d\tau' \tau^n (\tau')^m K(\tau-\tau').
    \label{eq:kernel_moments}
\end{equation}
This representation transforms the non-local integral operator into an algebraic series of local operators. The physics of the bath is entirely contained in the scalar moments $\mu_{nm}$, while the operator structure is determined solely by the nested commutators of $f$ with $H_Q$.

\subsection{Algebraic Closure and Exactness}
The expansion in Eq.~\eqref{eq:bilocal_expansion} suggests a natural criterion for the tractability of the Hamiltonian of mean force. We define the \emph{dynamical algebra generated by the coupling} as the vector space spanned by all iterated commutators:
\begin{equation}
    A_f = \mathrm{span} \left\{ \mathrm{ad}_{H_Q}^n(f) \right\}_{n=0}^\infty.
    \label{eq:algebra_Af}
\end{equation}
The structure of the HMF is determined by whether this space is finite-dimensional and closed under multiplication.

\textbf{Criterion (Exact Local Existence):} A strictly local Hamiltonian of mean force $H_{\mathrm{MF}}$ exists as a finite polynomial of operators from $A_f$ (plus the identity) if and only if $A_f$ generates a finite-dimensional Lie algebra.

If $A_f$ is finite, the infinite sum in Eq.~\eqref{eq:bilocal_expansion} collapses to a finite sum of operators. Consequently, the ordered exponential can be disentangled using a finite number of Baker-Campbell-Hausdorff (BCH) terms, leading to a local effective Hamiltonian $H_{\mathrm{MF}}$. Conversely, if the commutators generate new linearly independent operators ad infinitum, the influence functional generates an infinite operator series, and no finite, local $H_{\mathrm{MF}}$ exists. In such cases, one must resort to truncations, where validity is determined by the decay of the moments $\mu_{nm}$ or the magnitude of higher-order commutators.
